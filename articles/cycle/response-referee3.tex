% response-editor.tex
% ----------------------------------------------------------------------
% response-header.tex
% ----------------------------------------------------------------------

% Base class and packages
\documentclass[11pt]{article}

% Included in online comment header
\usepackage[pdftex]{graphicx}
\usepackage[pdftex]{color}
\usepackage{amssymb}
%\usepackage{times}

% Additional packages
\usepackage[T1]{fontenc}
\usepackage{geometry}
\usepackage[hidelinks]{hyperref}
\usepackage{natbib}

% Graphic path of main manuscript
\graphicspath{{../../figures/}}

% Replacements for Copernicus commands
\newcommand{\unit}[1]{\ensuremath{\mathrm{#1}}}
\newcommand{\chem}[1]{\ensuremath{\mathrm{#1}}}
\newcommand{\urlprefix}[0]{}

% Default font and spacing
\renewcommand\familydefault{\sfdefault}
\setlength{\parskip}{1.2ex}
\setlength{\parindent}{0em}
\linespread{1.0}

% Color defined in comment template
\definecolor{journalname}{rgb}{0.34,0.59,0.82}

% Personal colours
\definecolor{darkblue}{cmyk}{0.9,0.3,0.0,0.0}
\definecolor{darkgreen}{cmyk}{0.8,0.0,1.0,0.0}
\definecolor{darkred}{cmyk}{0.1,0.9,0.8,0.0}
\definecolor{darkorange}{cmyk}{0.0,0.5,1.0,0.0}
\definecolor{darkpurple}{cmyk}{0.6,0.7,0.0,0.0}
\definecolor{darkbrown}{cmyk}{0.23,0.73,0.98,0.12}

% Personal commands not used in final version
\newcommand{\todo}[1]{\textcolor{darkred}{\emph{[\textbf{TODO:} #1]}}}
\newcommand{\idea}[1]{\textcolor{darkgreen}{\emph{[\textbf{IDEA:} #1]}}}
\newcommand{\note}[1]{\textcolor{darkblue}{\emph{[\textbf{NOTE:} #1]}}}
\newcommand{\aref}[0]{\textcolor{darkblue}{\textbf{[REF.]}}}

% Redefine title and section heads
\makeatletter
\renewcommand{\familydefault}{\sfdefault}
\renewcommand{\maketitle}{\noindent\textbf{\@title}\\\@author\\\@date\\[3ex]}
\renewcommand\section{\@startsection{section}{1}{\z@}{-3ex}{2ex}%
                                    {\normalfont\large\bfseries}}
\renewcommand\subsection{\@startsection{subsection}{2}{\z@}{-3ex}{2ex}%
                                       {\normalfont\bfseries}}
\makeatother


\title{Authors' response to Anonymous Referee \#3}
\author{J.~Seguinot et al.}
%\date{}

\begin{document}
\maketitle
\bigskip

% ----------------------------------------------------------------------
% Interactive comment text begins
% ----------------------------------------------------------------------

\newcommand{\sechead}[1]{\bigskip\noindent\textbf{#1}}
\newcommand{\referee}[1]{\bigskip\noindent\textcolor{darkblue}{#1}}
\newcommand{\msquote}[1]{\begin{quote}\textit{#1}\end{quote}}
\newcommand{\doi}[1]{doi:\allowbreak\href{http://dx.doi.org/#1}{#1}}

Dear Anonymous Referee \#3,

Thank you very much for your detailed review of our manuscript.

    \referee{%
        This is an excellent manuscript and I highly recommend publication
        after a few minor changes. The authors present a well thought out
        modelling experiment which they combine (albeit in a qualitative
        manner) with extensive palaeo-glaciological data and cumulative work. I
        can see this work being extended into more extensive and rigorous work
        (RCM forcing, ice dynamic sensitivity, quantitative fitting to
        geomorphological record etc), but this is an important leap forward.}

% ----------------------------------------------------------------------

\sechead{Abstract}

    \referee{\textbf{p.~1, l.~2:}
        ``pioneer'' should be ``pioneering''.}

        Corrected.

    \referee{\textbf{p.~1, l.~16:}
        I think the finding that
        you get asynchronous glaciation extents with a uniform climate offset
        is due to glacier hypsometry and setting should be stated here. i.e.
        that the timing of maximum glaciation and recession isn't purely a
        function of climate. This finding needs to be highlighted better in the
        abstract.}

        Thank you for pointing this out. We have reworked the last sentence in
        the abstract:

        \msquote{%
            Finally, despite the uniform climate forcing, differences in
            glacier catchment hypsometry cause the Last Glacial Maximum advance
            to be modelled as a time-transgressive event, with some glaciers
            reaching their maximum as early as 26\,ka, and others as late as
            20\,ka.}


% ----------------------------------------------------------------------

\sechead{Introduction}

    \referee{\textbf{p.~2, l.~28:}
        Ballantyne and Stone (2015) should be added to this list.}

        This reference is relevant to the discussion on trimlines. We have
        added the reference in the introduction and in the corresponding
        section in the discussion.

    \referee{\textbf{p.~2, l.~34:}
        It should be stated that it could be a consequence of both
        glaioclimatic interactions and uncertainties in dating methods.}

        We have added ``or both''.

    \referee{\textbf{p.~3, l.~2:}
        These points serve the literature well to highlight gaps for future
        research. However, I would argue that you do not get very far here on 1
        and 5 and do not completely solve the other 3 points. Your text
        reflects these shortcomings very well, for which you should be
        applauded. Though I think at this stage of the manuscript, your
        statement of intent, you should state that you do not claim to solve
        these questions, but rather push forward on all of them using your new
        approach of ice sheet modelling.}

        To clarify our statement of intent, the following text was added:

        \msquote{%
            Additional geological research will be needed to complete our
            knowledge. But here, we intend to explore these open questions from
            a new angle and [use the Parallel Ice Sheet Model ...].}


% ----------------------------------------------------------------------

\sechead{Section 2.6}

    \referee{\textbf{p.~7:}
        The spatial distribution of your modern climate variables (precip,
        temp) will be massively influenced by elevation. Though there is a
        lapse rate, does this pattern of high precip and low temp over
        mountains remain throughout the simulation despite ice surface
        topography, and if so, how does this influence your results?}

% ----------------------------------------------------------------------

\sechead{Section 3}

    \referee{\textbf{p.~7, l.~14:}
        You keep mentioning the number of processors. I find this information
        slightly irrelevant, and it will soon become outdated as processing
        speed and models increase (GPUs for example). The only way it will
        serve the community is if there is a full description of the computer
        set up. For example, it could be that the simulations took 4 days on
        144 processors, but the processors were slow. Suggest removing these
        references.}

    We removed mentions of computing times and numbers of processors.

    \referee{\textbf{p.~10, l.~14:}
        This is a more general point. You eventually choose the EPICA record,
        and for justifiable reasons based on comparison to reconstructed ice
        extent and timing.  However, this is likely coincidence. EPICA is
        likely a complex record containing global and local antarctic
        influences upon climate. The ``real'' climate over the alps during
        glaciation is like decoupled from that of Antarctica to an extent.
        Therefore, the match you find is not an inference about climate, as
        different combinations of offsets may have made the same result
        (smoothed GRIP to remove some of the D-O scale noise?). You should make
        this explicit somewhere in the manuscript.}

    \referee{\textbf{p.~11, l.~2:}
        First sentence needs reconsidering as it is slightly broken in its
        current form. Perhaps ``Figure 3a shows the cumulative extent of
        glaciated area during MIS2''.}

    We reworked the first sentence. It now reads:

    \msquote{%
        During MIS 2, all six simulations yield comparative cumulative
        glaciated areas modelled maximum extents (Fig.~3a).}

    \referee{\textbf{p.~11, l.~10:}
        Is it possible some ice is missing from the geological reconstruction
        in some instances? I guess some outlets are well constrained, whilst
        others areas could be ``filled in'' by this modelling experiment.}

% ----------------------------------------------------------------------

\sechead{Section 4}

    \referee{\textbf{p.~14, l.~15--31:}
        I find this description of sites and timing of glacier extent compared
        to dates difficult to follow. I suggest a new figure to convey this
        important comparison: Have the reconstructed and modelled ice extents
        at key times for each of the mentioned glaciers on several smaller
        maps, including geochronological constraints.}

    \referee{\textbf{p.~17, l.~17:}
        This finding is important and should be highlighted better in abstract
        and conclusion.}

        This finding was better highlighted in the abstract (see our previous
        comment). However, we feel it is already well expressed by the
        following bullet point in the conclusion:

        \msquote{%
            The LGM (maximum) extent was a transient stage in which glaciers
            were out of balance with the contemporary climate. Its timing
            potentially varied across the range due to inherent glacier
            dynamics (Sect. 4.3).}


    \referee{\textbf{p.~17, l.~22:}
        Is the model recreating possible surging? Or would this be an
        over-interpretation given the uncertainty in climate and physics. Seems
        to fit with the enthalpy model of Benn and others for surging. As
        reads, it suggests that these areas were possible palaeo-surges, please
        clarify.}

    \referee{\textbf{p.~19, l.~9--12:}
        On trimlines: I think you justify well why you haven't yet modelled the
        sensitivity to your trimline result - previous work backs this up. But,
        I think you should consider the following: Adding a plus/minus to your
        results to reflect the uncertainty.  The importance of resolution -
        many trimlines will be below the resolution of your model, so perhaps
        aren't resolved enough for model-data comparison. They may have acted
        to deflect ice flow around mountain peaks for example. This resolution
        caveat should be mentioned. I would be surprised if all trimlines are
        subglacial transitions as this paper suggests - perhaps you need to
        directly challenge the geochronological community to find better
        trimline constraints (sub/supra) as a statement in this paper. Your
        mean value of 861 m is unrepresentative of your sample. Your sample is
        highly skewed, so a modal value (1050 m ish from Fig 6) is more
        appropriate. A similar finding with a similar approach has already been
        found for the British Isles, with the added constraint of GIA
        observations. I suggest referencing Kuchar et al. 2011 for this
        reason.}

    \referee{%
        A philosophical but important point is that your discussion throughout
        is written from the standpoint of the geochronological/geomorphological
        data and reconstructions as being ``truth''. It should consider
        somewhere that perhaps data is missing as it is hard-won in the places
        it exists and interpretations may be slightly wrong. The model is also
        not the ``truth'' and there is probably a blurred line in-between upon
        which we can proceed.}

    \referee{\textbf{p.~21, l.~11--19:}
        Be really clear here that these are modelled, and perhaps not
        geologically recorded, advances of the ice sheet. If there is no data,
        it might be correct, might be just a modelled result.}

        Actually we mean geologically recorded asymmetry. This clearly needed
        clarification. The corresponding sentence now reads:

        \msquote{%
            The observed asymmetric extent of ice north and south of the Alps
            can be explained by the modelled transient nature of the LGM extent
            without involving north-south gradients in temperature and
            precipitation change (Sect.~4.1).}

    \referee{%
        Additional references: Ballantyne, C.K. and Stone, J.O., 2015.
        Trimlines, blockfields and the vertical extent of the last ice sheet in
        southern Ireland. Boreas, 44(2), pp.277- 287. Kuchar, J., Milne, G.,
        Hubbard, A., Patton, H., Bradley, S., Shennan, I. and Edwards, R.,
        2012. Evaluation of a numerical model of the British–Irish ice sheet
        using relative sea-level data: implications for the interpretation of
        trimline observations.  Journal of Quaternary Science, 27(6),
        pp.597-605.}

\end{document}
