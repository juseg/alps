% response-editor.tex
% ----------------------------------------------------------------------
% response-header.tex
% ----------------------------------------------------------------------

% Base class and packages
\documentclass[11pt]{article}

% Included in online comment header
\usepackage[pdftex]{graphicx}
\usepackage[pdftex]{color}
\usepackage{amssymb}
%\usepackage{times}

% Additional packages
\usepackage[T1]{fontenc}
\usepackage{geometry}
\usepackage[hidelinks]{hyperref}
\usepackage{natbib}

% Graphic path of main manuscript
\graphicspath{{../../figures/}}

% Replacements for Copernicus commands
\newcommand{\unit}[1]{\ensuremath{\mathrm{#1}}}
\newcommand{\chem}[1]{\ensuremath{\mathrm{#1}}}
\newcommand{\urlprefix}[0]{}

% Default font and spacing
\renewcommand\familydefault{\sfdefault}
\setlength{\parskip}{1.2ex}
\setlength{\parindent}{0em}
\linespread{1.0}

% Color defined in comment template
\definecolor{journalname}{rgb}{0.34,0.59,0.82}

% Personal colours
\definecolor{darkblue}{cmyk}{0.9,0.3,0.0,0.0}
\definecolor{darkgreen}{cmyk}{0.8,0.0,1.0,0.0}
\definecolor{darkred}{cmyk}{0.1,0.9,0.8,0.0}
\definecolor{darkorange}{cmyk}{0.0,0.5,1.0,0.0}
\definecolor{darkpurple}{cmyk}{0.6,0.7,0.0,0.0}
\definecolor{darkbrown}{cmyk}{0.23,0.73,0.98,0.12}

% Personal commands not used in final version
\newcommand{\todo}[1]{\textcolor{darkred}{\emph{[\textbf{TODO:} #1]}}}
\newcommand{\idea}[1]{\textcolor{darkgreen}{\emph{[\textbf{IDEA:} #1]}}}
\newcommand{\note}[1]{\textcolor{darkblue}{\emph{[\textbf{NOTE:} #1]}}}
\newcommand{\aref}[0]{\textcolor{darkblue}{\textbf{[REF.]}}}

% Redefine title and section heads
\makeatletter
\renewcommand{\familydefault}{\sfdefault}
\renewcommand{\maketitle}{\noindent\textbf{\@title}\\\@author\\\@date\\[3ex]}
\renewcommand\section{\@startsection{section}{1}{\z@}{-3ex}{2ex}%
                                    {\normalfont\large\bfseries}}
\renewcommand\subsection{\@startsection{subsection}{2}{\z@}{-3ex}{2ex}%
                                       {\normalfont\bfseries}}
\makeatother


\title{Authors' response to Anonymous Referee \#2}
\author{J.~Seguinot et al.}
%\date{}

\begin{document}
\maketitle
\bigskip

% ----------------------------------------------------------------------
% Interactive comment text begins
% ----------------------------------------------------------------------

\newcommand{\sechead}[1]{\bigskip\noindent\textbf{#1}}
\newcommand{\referee}[1]{\bigskip\noindent\textcolor{darkblue}{#1}}
\newcommand{\msquote}[1]{\begin{quote}\textit{#1}\end{quote}}
\newcommand{\doi}[1]{doi:\allowbreak\href{http://dx.doi.org/#1}{#1}}

Dear Anonymous Referee \#2,

Thank you very much for your detailed review of our manuscript.



% ----------------------------------------------------------------------

\sechead{General comments}

    \referee{%
        This manuscript describes a model study with the Parallel Ice Sheet
        Model applied to the last glacier cycle in the Alps. The climate
        forcing is derived from present day climate of WorldClim and the
        ERA-Interim reanalysis and time-dependent temperature offsets derived
        from the Greenland Ice core (GRIP), from Antarctic ice core (EPICA) and
        Marine sediment core from the Iberian margin (MD01-2444). The study is
        split in two, in the first part analysis of six model simulations (with
        and without precipitation scaling) made on a 2 km resolution grid is
        presented and concluded that out of the three climate forcing records
        used, the EPICA record gives the most realistic ice volume history
        during MIS 4 and 2. The second part analyses simulation made with the
        EPICA forcing record on a 1km grid. The authors draw conclusions about
        the ice cover, ice flow pattern, ice thickness, LGM ice extent, which
        in their model is a transient stage with varying timing across their
        model domain due to glacier dynamics. The manuscript is well organized
        and clearly written, the missing thing in this study is a discussion of
        and preferably a sensitivity study of the ice dynamic-model assumptions
        made. Is the sliding of the ice age ice sheet realistically modelled
        with pseudo-plastic assumption using the Shallow Shelf approximation?
        How sensitive are the results to the selected model parameters? It is
        briefly mentioned once on page 19, line 8, but a thorough analysis of
        the model sensitivity would strengthen the paper.}

% ----------------------------------------------------------------------

\sechead{Specific comments}

    \referee{%
        I find missing something that indicates that the times are before
        present, as in line 5, line 8 and line 16 on page 1 -- and elsewhere in
        the paper it is written (120-0 ka) should you add before present or BP
        to indicate the time interval?}

    \referee{%
        Could the reason for too large ice volumes when using the GRIP record,
        as mentioned in lines 13-15 on page 9 be due to Arctic Amplification?
        Could that have an effect then as it has now? This is also mentioned in
        line 11 on page 10.}


% ----------------------------------------------------------------------

\sechead{Minor comments}

    \referee{\textbf{p.~1, l.~24:}
        suggest: ``have extended well outside their current margins''.}

    \referee{\textbf{p.~2, l.~22:}
        could you add a reference and a timing for LGM?}

    \referee{\textbf{p.~3, l.~1:}
        suggest to replace ``thus'' with ``still''.}

    \referee{\textbf{p.~3, l.~3:}
        add s to responses.}

    \referee{\textbf{p.~3, l.~18:}
        something missing in the sentence, suggest ``formulation'' after
        ``creep''.}

    \referee{\textbf{p.~4, l.~4:}
        suggest to replace ``field'' with ``sheet''.}

    \referee{\textbf{p.~5, Fig.~1:}
        c) can you add a scale and maybe indication with a box in b) where this
        extract is from?  ``from the estimate'' (not plural in line 4 of
        caption), Line 6 (PDD) is not acronym for surface mass balance, some
        more explanation is needed here, indicate also, like in the other
        figures that h) is January and i) is July figuresPage 6 line 6 suggest
        to replace ``of'' with ``with''.}

    \referee{\textbf{p.~6, l.~14:}
        suggest ``The climate forcing driving the ice sheet simulations consist
        spatially of a present-day monthly mean climatology...``}

    \referee{\textbf{p.~6, l.~15:}
        suggest to delete ``s'' on corrections.}

    \referee{\textbf{p.~6, l.~19:}
        add ``mean'' after monthly.}

    \referee{\textbf{p.~6, l.~22:}
        note, if true (clarify in Figure caption, see comment above) then the
        reference should be to Fig. 1 i) for summer precipitation.}

    \referee{\textbf{p.~6, l.~27:}
        replace ``is'' with ``are''.}

    \referee{\textbf{p.~6, footnote:}
        add a reference for the correct formula for the rigidity and clarify
        (add something like, the consequence of this error is that the
        simulations effectively use....) Delete ``in'' before ``a small'' and
        add a quantification of the small change in the length scale – how
        small, is it a few percentage?}

    \referee{\textbf{p.~7, l.~2:}
        ``shipped with WorldClim'' is not clear, please edit, also suggest not
        to use b for topography, s surf and s bed , or h would be better.}

    \referee{\textbf{p.~7, l.~19:}
        add ``the'' before ``oxygen''.}

    \referee{\textbf{p.~7, l.~24:}
        ``and within a rectangular region ..'' is not clear, edit this text.}

    \referee{\textbf{p.~8:}
        add ``acceleration of'' before gravity, add a reference for the ideal
        gas constant?}

    \referee{\textbf{p.~9, l.~14--15:}
        see comment above, could this be an example of Arctic amplification?}

    \referee{\textbf{p.~9, l.~19:}
        add ``a'' before ``very'' and suggest to turn the sentence around, the
        EPICA simulations are in a good agreement with the data.}

    \referee{\textbf{p.~9, l.~20:}
        followed by first a retreat and then a standstill? replace ``blue''
        with ``red''.}

    \referee{\textbf{p.~9, l.~21:}
        suggest : The simulations forced by the GRIP palaeo-temperature forcing
        yield ... (blue curves)''.}

    \referee{\textbf{p.~10, l.~2:}
        suggest to add ``followed by a rapid retreat''.}

    \referee{\textbf{p.~10, l.~3:}
        why state ``two or three'' in intro the suggestion is either 4 major or
        15, is there a reference for 2 or 3?}

    \referee{\textbf{p.~10, l.~10:}
        suggest to replace ``lower'' with ``smaller''.}

    \referee{\textbf{p.~10, l.~11:}
        is this due to Arctic Amplification?}

    \referee{\textbf{p.~10, l.~12:}
        suggest to replace ``least'' with ``the smallest''.}

    \referee{\textbf{p.~11, Fig.~3 caption:}
        what does ``cumulative'' mean here? Do you mean maximum in each
        location?  clarify what the black line indicates. No solid red line is
        visible in figures. What is meant with ``reasonable'', maybe replace
        with ``realistic''? suggest to replace ``cover'' with ``extent''.}

    \referee{\textbf{p.~11, l.~2:}
        suggest to replace ``cumulative'' with ``maximum extent in each area''
        or something similar.}

    \referee{\textbf{p.~11, l.~1--3:}
        how sensitive is the model to different parameters in the applied
        sliding formulation?}

    \referee{\textbf{p.~11, l.~4--5:}
        ``outside this benchmark'' clarify if you mean spatially or temporally.}

    \referee{\textbf{p.~11, l.~6--11:}
        is this text better fitted in a discussion section?}

    \referee{\textbf{p.~12, l.~2:}
        suggest to replace ``higher'' with ``larger''.}

    \referee{\textbf{p.~12, l.~4:}
        do you mean to refer to Fig. 3?}

    \referee{\textbf{p.~12, l.~29--32:}
        does this text fit better in method section?}

    \referee{\textbf{p.~13, Fig.~4 caption:}
        the 200 m surface contours are not clearly visible in the figure, can
        they be made sharper, or just skipped? Line 3, something like ``are
        shown'' is missing. Suggest to replace``background'' with ``bedrock''.
        This is the first time ``Natural Earth Data'' is mentioned, should that
        be in the section on the data? Suggest to replace ``Gray fields'' with
        ``shaded gray area'' and ``boundaries'' with ``timing''.}

    \referee{\textbf{p.~15, l.~8:}
        replace ``was'' with ``is''.}

    \referee{\textbf{p.~15, l.~25:}
        ``occurred''.}

    \referee{\textbf{p.~16, Fig.~5:}
        in figure the color bar is written to indicated maximum surface
        elevation, but in figure caption the maximum ice thickness, which is
        correct? The surface elevation contours are not clearly visible. The
        dark orange color in the figure, that covers the central part of the
        ice sheet is not (?) in the bar on the left (or is it before 27 ka
        BP?).}

    \referee{\textbf{p.~17, l.~30:}
        suggest to replace ``have'' with ``could''.}

    \referee{\textbf{p.~18, Fig.~6:}
        the 200 m contours are not clearly visible.}

    \referee{\textbf{p.~19, l.~7:}
        could you add what the modelled regional ice thickness is, for
        comparison?}

    \referee{\textbf{p.~20, Fig.~7:}
        add info about the gray areas indicating MIS 4 and 2 ``Isolated patches
        indicate periodic surges from tributary glaciers'' needs more
        explanation and it is not clear what is meant. Does the model simulate
        periodic surges?}

    \referee{\textbf{p.~21, l.~21:}
        here is a reference for 2 or 3 glaciations (see comment above) but in
        the intro is only mentioned 4 or 15, suggest to change text to
        harmonize.}

    \referee{\textbf{p.~21, l.~27:}
        suggest to edit ``the study consists of'' or start for example like
        ``In this study the model has been applied...''}

    \referee{\textbf{p.~21, l.~28:}
        how important is it that the model has been validated for the
        Cordilleran ice sheet? Will that support the choices of the sliding
        model applied in the Alps? I think that it would be valuable for this
        study to do a sensitivity runs for at least some of the model parameter
        choices.}

    \referee{\textbf{p.~22, l.~6:}
        add ``records'' after forcing.}

    \referee{\textbf{p.~22, l.~19:}
        why do you add ``potentially'' here? Isn't this a firm conclusion from
        you study?}

    \referee{\textbf{p.~22, l.~20:}
        suggest to replace ``higher'' with ``larger''.}

    \referee{\textbf{p.~22, l.~22:}
        same as above, why ``potentially'' here?}

    \referee{\textbf{p.~22, l.~24:}
        suggest to replace ``nevertheless'' with ``however'' or edit the
        sentence.}

    \referee{\textbf{p.~22, l.~26--28:}
        the paper would be able to give stronger conclusions with sensitivity
        study, suggest to edit the sentence by replace ``statements'' with
        ``limitations'' or ``drawbacks'' and ``last glacier cycle ice dynamics
        in the Alps'' is not easy to read.}

    \referee{\textbf{p.~23, l.~2:}
        replace ``mode'' with ``more''.}

    \referee{%
        Figures are generally clear and well set up.  The surface contours in
        Figs 4,5 and 6 is not clearly visible in my printout and could maybe
        become clearer?}

\end{document}
