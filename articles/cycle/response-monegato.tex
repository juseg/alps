% response-editor.tex
% ----------------------------------------------------------------------
% response-header.tex
% ----------------------------------------------------------------------

% Base class and packages
\documentclass[11pt]{article}

% Included in online comment header
\usepackage[pdftex]{graphicx}
\usepackage[pdftex]{color}
\usepackage{amssymb}
%\usepackage{times}

% Additional packages
\usepackage[T1]{fontenc}
\usepackage{geometry}
\usepackage[hidelinks]{hyperref}
\usepackage{natbib}

% Graphic path of main manuscript
\graphicspath{{../../figures/}}

% Replacements for Copernicus commands
\newcommand{\unit}[1]{\ensuremath{\mathrm{#1}}}
\newcommand{\chem}[1]{\ensuremath{\mathrm{#1}}}
\newcommand{\urlprefix}[0]{}

% Default font and spacing
\renewcommand\familydefault{\sfdefault}
\setlength{\parskip}{1.2ex}
\setlength{\parindent}{0em}
\linespread{1.0}

% Color defined in comment template
\definecolor{journalname}{rgb}{0.34,0.59,0.82}

% Personal colours
\definecolor{darkblue}{cmyk}{0.9,0.3,0.0,0.0}
\definecolor{darkgreen}{cmyk}{0.8,0.0,1.0,0.0}
\definecolor{darkred}{cmyk}{0.1,0.9,0.8,0.0}
\definecolor{darkorange}{cmyk}{0.0,0.5,1.0,0.0}
\definecolor{darkpurple}{cmyk}{0.6,0.7,0.0,0.0}
\definecolor{darkbrown}{cmyk}{0.23,0.73,0.98,0.12}

% Personal commands not used in final version
\newcommand{\todo}[1]{\textcolor{darkred}{\emph{[\textbf{TODO:} #1]}}}
\newcommand{\idea}[1]{\textcolor{darkgreen}{\emph{[\textbf{IDEA:} #1]}}}
\newcommand{\note}[1]{\textcolor{darkblue}{\emph{[\textbf{NOTE:} #1]}}}
\newcommand{\aref}[0]{\textcolor{darkblue}{\textbf{[REF.]}}}

% Redefine title and section heads
\makeatletter
\renewcommand{\familydefault}{\sfdefault}
\renewcommand{\maketitle}{\noindent\textbf{\@title}\\\@author\\\@date\\[3ex]}
\renewcommand\section{\@startsection{section}{1}{\z@}{-3ex}{2ex}%
                                    {\normalfont\large\bfseries}}
\renewcommand\subsection{\@startsection{subsection}{2}{\z@}{-3ex}{2ex}%
                                       {\normalfont\bfseries}}
\makeatother


\title{Authors' response to Giovanni Monegato}
\author{J.~Seguinot, on behalf of all authors.}
%\date{}

\begin{document}
\maketitle
\bigskip

% ----------------------------------------------------------------------
% Interactive comment text begins
% ----------------------------------------------------------------------

\newcommand{\sechead}[1]{\bigskip\noindent\textbf{#1}}
\newcommand{\referee}[1]{\bigskip\noindent\textcolor{darkblue}{#1}}
\newcommand{\msquote}[1]{\begin{quote}\textit{#1}\end{quote}}
\newcommand{\doi}[1]{doi:\allowbreak\href{http://dx.doi.org/#1}{#1}}

    Dear Giovanni Monegato,

    Thank you very much for your public comment on our manuscript.

    \referee{%
        The manuscript is very interesting as approach on the Late Pleistocene
        Alpine glaciation, for which good data are available for the LGM
        onwards, but few is known about pre-MIS2. The present article does not
        solve the problem of what how the glaciers behave before the LGM, but
        casts new light on this topic suggesting interesting details and field
        to investigate. I would like to post some general comments on the
        manuscript especially regarding the Italian side of the Alps.}

    Thank you very much for your supportive comments!

    \referee{\textbf{1 --}
        Reading the manuscript and watching the supplemental file, it is
        remarkable how few systems in the model result to have the same
        extension compared to the geomorphological/geological evidence. For the
        Italian side roughly the Riparia, Baltea, Ossola/Ticino and Tagliamento
        looks right. While other systems are much over-(western Alps) or
        under-estimated (from Adda to Piave). The overestimation of the western
        Alps can be related to the scarcity of updated chronological data; for
        example the glacier reconstruction in Gesso Valley of the Maritime Alps
        (Federici et al. 2017) shows a much larger extent than Ehlers and
        Gibbard (2004) compilation. Anyhow the sentence of line 30 page 14 is
        reliable and suggests that more data are needed about. Also the Eastern
        Alpine glaciers resulted very overestimated, for these and for the
        not-matching glaciers in the Italian side I think that paleoclimate
        forcing and especially precipitation models are one of the keys
        factors. The WorldClim model that is considered and showed in Figure 1
        seems to be inconsistent respect to other models focused on the Alps. I
        suggest to take into consideration Isotta et al. (Int. J. Climatol.
        2013) where distribution of the precipitations shows areas of high
        precipitation rates (both as annual mean and daily peak). This
        distribution would point to high precipitation rates also in the Piave,
        Adda and Oglio catchments, while the Tagliamento and Ticino systems fit
        well in the model as well. The knot of the Valais-Ticino-upper Rhine
        has also modern high precipitations, and this is one of the key areas
        for large ice accumulation and for the southerly component during the
        LGM according to Luetscher et al. (2015). Actually, considering the
        modern precipitation rates and comparing to your model, the
        underestimation for the Adige system sounds reasonable but not in
        agreement with the chronology and geomorphology found in the Garda. So
        other causes, and not only precipitation, have to be considered.}

    Thank you for this detailed analysis. We want to point out that WorldClim
    is not a model, but is also based on observations. However we agree, that
    the regional dataset by \citep{Isotta.etal.2013}, tailored to the dense
    station network and steep topography of the Alps, may be a more robust
    reference precitation forcing that needs to be considered in future
    studies. A reference was added:

    \msquote{%
        Finally, modern precipitation data from WorldClim also bear
        uncertainties and exhibit local disagreement with other regional data
        \citep{Isotta.etal.2013}.}

    Besides, the following sentence was added to the discussion of glacier
    extent in the south-western Alps,

    \msquote{%
    However, geochronological data from the south-western Alps are sparse, and
    the LGM extent compilation \citep{Ehlers.Gibbard.2004} is inconsistent with
    more recent regional reconstructions \citep{Federici.etal.2017}.}

    \referee{\textbf{2 --}
        Concerning the ice-transfluence. I agree that it has to be much more
        considered. I wonder if the Adige had a great contribution from the
        Austrian Tauern (Toblach area) this would have increased the ice in
        Adige and deplete the Drava, or in Winschgau valley where the catchment
        upstream Mustar saddle is more related to Adige than the Inn. The same
        could have happened for the Piave catchment, which is very
        underestimated in the model. Again, ice flowing to the south did not
        flow to the east. But I think that this may be not enough for
        justifying the overestimation of the Eastern Alps. Concerning the
        Straninger saddle I am a bit skeptic because it is not the lowermost
        saddle (Plockenpass and Nassfeldpass are lower in elevation). The
        central Adda glacier (and not the Ticino as at page 14 line 26) is much
        underestimated and here the contribution from transfluence in St.
        Moritz area could have been remarkable. I attach a figure separately
        with transfluences (yellow stars and red arrows).}

    The amount of ice flowing through transfluences is certainly dependent on
    model resolution and basal sliding parameters. However, we agree that this
    will have little effect on the ice volume overestimation in the eastern
    Alps in comparison to potential shortcomings in the climate forcing.

    The tranfluence at the Straniger Saddle may be a byproduct of bedrock
    topography aggregation to the 1\,km grid.  As a follow-up study, we have
    begun to use higher model resolution to study the location of transfluences
    in more detail \citep{Seguinot.etal.2018a}. In these results, the
    transfluence at the Straniger Saddle looses importance relative to other
    transfluences in the region including, in fact, Plöckenpass and
    Nassfeldpass. Because the choice of transfluences plotted on Fig.~4 is
    somewhat arbitrary, we have decided to update it.

    \referee{\textbf{3 --}
        The basal sliding of glaciers and its overall velocity is another
        interesting factor to discuss. For Adige for example we know that at
        about 28 cal BP the Adige glacier was damming a tributary valley north
        of Trento around the same age (Avanzini et al 2009), while the same
        glacier arrived at Garda at 24.6 cal BP (Monegato et al., 2017).  This
        means that the glacier front advanced of about 100 km in around 3.5 ka
        so 280 m/y. The same could have been for the Adda and Oglio glaciers,
        even if a such robust chronology is lacking. Is it possible that
        overestimated velocities produced large eastern glaciers?}

    Thank you. In fact basal sliding is one of the major uncertainties in our
    results which is now better acknowledged for in our manuscript following
    other reviewer's comments. However, basal sliding velocity are somewhat
    decoupled from glacier front advance velocities and glacier front maximum
    extent. Although this is not shown in our manuscript, increased basal
    velocities tend to result in thinner ice tongues. Due to the
    mass-balance-elevation feedback, this causes more surface melt and, in
    turn, lesser extent. However this effect is small in comparison to changes
    due to climate forcing.

    \referee{\textbf{4 --}
        About the self-sustained ice domes, their importance is for me unclear
        and not well explained in the text. Why they are only two? Why the
        Adamello or the Tauern massifs, as an example, were not considered as
        self-sustained ice domes?}

    By self-sustained ice domes, we mean ice domes that are not sitting on top
    of basal topographic highs, but instead shift away from the modern
    topographic divides and ``sustain'' themselves by reaching higher
    elevation than the local mountains. We indicate these as a sign that ice
    begins to behave as in an ice sheet, and flow in directions independent
    from the local basal topography.

    However, we agree that this is somewhat arbitrary, and again, as we found
    out, dependent on model resolution \citep{Seguinot.etal.2018a}. Therefore,
    we decided to remove the ``ice domes'' from Fig.~4a and the main text, and
    save this discussion for future studies.

    \referee{\textbf{5 --}
        Global circulation models (e.g., Löfverström et al., 2014, Beghin et
        al., 2015) suggest that the Polar front was at different latitudes
        during each cold phases of the Late Pleistocene. This could have effect
        of different impact on the Alps. For example during MIS4 if the
        westerlies were dominant this could have driven more effective
        precipitations in the western and northern Alps in respect to the
        southern and the eastern Alps. If this is true, and can be applied to
        the major cold phases. How was the behavior of glaciers during MIS 5
        and 3?}


% ----------------------------------------------------------------------
% References

\bibliographystyle{copernicus}
\bibliography{../../references/references}

\end{document}
