% response-editor.tex
% ----------------------------------------------------------------------
% response-header.tex
% ----------------------------------------------------------------------

% Base class and packages
\documentclass[11pt]{article}

% Included in online comment header
\usepackage[pdftex]{graphicx}
\usepackage[pdftex]{color}
\usepackage{amssymb}
%\usepackage{times}

% Additional packages
\usepackage[T1]{fontenc}
\usepackage{geometry}
\usepackage[hidelinks]{hyperref}
\usepackage{natbib}

% Graphic path of main manuscript
\graphicspath{{../../figures/}}

% Replacements for Copernicus commands
\newcommand{\unit}[1]{\ensuremath{\mathrm{#1}}}
\newcommand{\chem}[1]{\ensuremath{\mathrm{#1}}}
\newcommand{\urlprefix}[0]{}

% Default font and spacing
\renewcommand\familydefault{\sfdefault}
\setlength{\parskip}{1.2ex}
\setlength{\parindent}{0em}
\linespread{1.0}

% Color defined in comment template
\definecolor{journalname}{rgb}{0.34,0.59,0.82}

% Personal colours
\definecolor{darkblue}{cmyk}{0.9,0.3,0.0,0.0}
\definecolor{darkgreen}{cmyk}{0.8,0.0,1.0,0.0}
\definecolor{darkred}{cmyk}{0.1,0.9,0.8,0.0}
\definecolor{darkorange}{cmyk}{0.0,0.5,1.0,0.0}
\definecolor{darkpurple}{cmyk}{0.6,0.7,0.0,0.0}
\definecolor{darkbrown}{cmyk}{0.23,0.73,0.98,0.12}

% Personal commands not used in final version
\newcommand{\todo}[1]{\textcolor{darkred}{\emph{[\textbf{TODO:} #1]}}}
\newcommand{\idea}[1]{\textcolor{darkgreen}{\emph{[\textbf{IDEA:} #1]}}}
\newcommand{\note}[1]{\textcolor{darkblue}{\emph{[\textbf{NOTE:} #1]}}}
\newcommand{\aref}[0]{\textcolor{darkblue}{\textbf{[REF.]}}}

% Redefine title and section heads
\makeatletter
\renewcommand{\familydefault}{\sfdefault}
\renewcommand{\maketitle}{\noindent\textbf{\@title}\\\@author\\\@date\\[3ex]}
\renewcommand\section{\@startsection{section}{1}{\z@}{-3ex}{2ex}%
                                    {\normalfont\large\bfseries}}
\renewcommand\subsection{\@startsection{subsection}{2}{\z@}{-3ex}{2ex}%
                                       {\normalfont\bfseries}}
\makeatother


\title{Authors' response to Anonymous Referee \#1}
\author{J.~Seguinot et al.}
%\date{}

\begin{document}
\maketitle
\bigskip

% ----------------------------------------------------------------------
% Interactive comment text begins
% ----------------------------------------------------------------------

\newcommand{\sechead}[1]{\bigskip\noindent\textbf{#1}}
\newcommand{\referee}[1]{\bigskip\noindent\textcolor{darkblue}{#1}}
\newcommand{\msquote}[1]{\begin{quote}\textit{#1}\end{quote}}
\newcommand{\doi}[1]{doi:\allowbreak\href{http://dx.doi.org/#1}{#1}}

    Dear Anonymous Referee \#1,

    Thank you very much for your detailed review of our manuscript.

    \referee{%
        This paper is a landmark advance in modelling European Alpine ice
        cover, applying a high-resolution (1\,km) ice model to the entire Alps
        through the last glacial cycle, for the first time to my knowledge.
        Results are compared with diverse geological data, and several
        important findings are presented, including time-transgressive ice
        marginal extents at LGM. The climate forcing is simple, applying
        uniform perturbations to modern observed datasets, which leads to some
        uncertainty in the results, but does not detract from them too much
        given the advances made in the ice modelling alone.}

    \referee{%
        The introduction gives an elegant summary of Alpine glacial science
        since the 1700's, including many historical references. The paper is
        well organized, with well-chosen sensitivities described first that
        calibrate the climate forcing, followed by detailed analysis of one
        best-fit high-resolution (1\,km) simulation through the last 120\,kyrs.
        Detailed comparisons to a variety of geological data are made,
        constituting a thorough assessment of model performance. An impressive
        animation of the whole cycle is included as supplementary material.}

    Thank you very much for this elogious summary.


% ----------------------------------------------------------------------

\sechead{Specific comments}

    \referee{\textbf{p.~4, l.~9--10:}
        Can the physical basis of englacial water fraction and sensitivity of
        results be summarized briefly? This is not a usual component in
        ice-sheet models. Is the cap value ("capped at 0.01") well constrained,
        and does it have a significant effect on results?}

    Laboratory experiments have demonstrated that the rheology of temperate,
    polycrystalline ice depends on its content in liquid water
    \citep[p.~65--66]{Cuffey.Paterson.2010}. Unfortunately, the only
    measurements available to date \citep{Duval.1977}, used to quantify the
    effect of liquid water on ice softness, the creep parameter $A$ in Glen's
    flow law \citep{Lliboutry.Duval.1985}, only extend to fractions of liquid
    water content between 0 and 0.8\%. They show a three-fold increase of ice
    softness over this range \citep[Fig.~1]{Duval.1977}.

    Although ice sheet models have previously ignored this effect, it has now
    become a typical component of polythermal models such as SICOPOLIS
    \citep{Greve.1997}, COMICE \citep{Ruckamp.etal.2010}, PISM
    \citep{Aschwanden.etal.2012}, ISSM \citep{Seroussi.etal.2013}, and
    TIM-FD$\unit{^{3}}$ \citep{Kleiner.Humbert.2014}.

    Water fractions between 1 and 5\,\% have repreatedly been observed in
    temperate glaciers \citep{Murray.etal.2000, Murray.etal.2007,
    Bradford.Harper.2005, Bradford.etal.2009}, and also occur in model results
    \citep[e.g.,][]{Blatter.Greve.2015}, but it is not known whether ice
    viscosity continues to decrease substancially for values above 0.8\,\%.
    Previous modelling studies have commonly assumed constant ice viscosity
    above 1\,\%. This arbitrary threshold is not constrained at all, and the
    urgent need for new ice deformation experiments has been pointed out
    \citep{Kleiner.etal.2015}.

    The sensitivity of our results to the 1\,\% threshold was not tested.
    However, in our model results, liquid water fractions above 1\,\% typically
    only occur within the basal temperate layer of the fastest-moving glaciers
    where ice movement is dominated by basal sliding. We suspect that a local
    increase of ice deformation there is negligible in comparaison to
    uncertainties caused by basal sliding and more importantly climate forcing.

    We prefer to avoid including the above discussion in the manuscript, but we
    have reworked the sentence on water content:

    \msquote{%
        [Ice softness] increases with liquid water fractions up to 0.01
        \citep[p.~65--66]{Duval.1977, Lliboutry.Duval.1985,
        Cuffey.Paterson.2010}, an arbitrary threshold above which new
        measurements are critically needed \citep{Kleiner.etal.2015}.}

    The uncertainty to unknown rheology of water-rich temperate ice was also
    mentioned in the conclusions:

    \msquote{%
        A constant rheology was used for temperate ice containing more than
        1\,\% of liquid water.}

    \referee{\textbf{p.~4, l.~19--20:}
        The sub-glacial hydrologic component should be described more (even if
        exactly as in Bueler and van Pelt, 2015). Is basal water transported
        horizontally down the hydropotential gradient? This is usually a highly
        uncertain component of ice-sheet models, but can have a large effect on
        results through its influence on basal sliding, and basal frozen vs.
        thawed areas, which is relevant to section 4.4 regarding trimlines.}

    We refer to \citet{Bueler.Pelt.2015} as their paper contain the most
    up-to-date description of PISM till effective pressure physics used in our
    simulations \citet[Eqs.~18, 23, and 24]{Bueler.Pelt.2015}. However,
    subglacial water is not routed down the hydropotential gradient. We have
    clarified this:

    \msquote{%
        Effective pressure is related to the ice overburden stress and the
        modelled amount of subglacial water, using a~formula derived from
        laboratory experiments with till extracted from the base of Ice Stream
        B in West Antarctica \citep[Table~1;][Eqs~23 and
        24]{Tulaczyk.etal.2000, Bueler.Pelt.2015}. Basal meltwater is
        accumulated locally without transportation. When the till becomes
        saturated, additional meltwater assumed to drain off instantaneously
        outside the glacier margins, i.e. it is removed from the system in an
        accountable way.}

    \referee{\textbf{Somewhat related:}
        Little information is given on the choices of basal sliding parameter
        values in Table 1. This could be discussed briefly. Presumably no
        inversion or optimization was performed for these values beforehand,
        and they do vary spatially. Are they appropriate for Alpine bedrock
        overall?} 

    In general, we chosed to not repeat parameter values from Table~1 in the
    main text. The following text was added in the methods:

    \msquote{%
        [a constant basal friction angle] corresponding to the average of
        available measurements \citep[p.~268]{Cuffey.Paterson.2010}. [...]
        Other parameters (Table~1) follow simulations of the Greenland ice
        sheet \citep{Aschwanden.etal.2013}, or benchmarks when other data is
        missing \citep{Bueler.Pelt.2015}.}

    Inversion of specific basal sliding parameters for past Alpine glaciers
    is difficult because the altitude and age of maximum ice surface elevation
    is discussed (cf. introduction and discussion on trimlines), and also
    depends on the even more uncertain regional climate history.  Therefore,
    basal sliding was also mentioned as one of the major sources of
    uncertainty in the conclusions.

    \msquote{%
        The till deformation model used here does not hold for sliding over
        bedrock surfaces. On the other hand, the constant friction angle used
        is representative of wet till but weaker basal conditions may have
        applied over saturated lake sediments where they occured.}
    
    \referee{\textbf{p.~7, l.~7:}
        Are there any data to support this atmospheric lapse rate value
        (6\,K\,km$^{-1}$), and do other values have the potential to
        significantly affect ice temperatures? In particular, could they change
        the basal areas of frozen/unfrozen ice and so the comparisons with
        trimlines in section 4.4?}

% ----------------------------------------------------------------------

\sechead{Climate forcing}

    \referee{%
        The method of spatially uniform shifts to modern climate forcing is
        common in paleo-modeling of large ice sheets, and in my opinion is
        acceptable as a starting point in this work, with coupling to regional
        climate models (RCMs) left to follow-on work. There are good
        discussions on possible shortcomings of this method, for instance as a
        cause of anomalous east-west marginal ice extents at LGM (pg. 11, line
        6-8). However, I suggest changing the sentence on pg. 9, line 5, which
        mentions some RCMs applied to LGM Europe, but also says "...over the
        Alps during the last glacial cycle, of which little is known apart from
        the LGM". There are several other RCM modeling studies over Europe
        during the last 120\,kyrs, e.g., for MIS Stage 3, Kjellstrom et al.,
        Boreas, 2010; Barron and Pollard, Quat. Res., 2002; Alfano et al.,
        Quat. Res., 2002; and for 6\,ka, Strandberg et al., Clim. Past, 2014.
        Perhaps there is little useful material there for the Alps, but such
        papers exist.}

    \referee{%
        The past climate variations are prescribed following 3 quite distal
        core records, and the most distal (EPICA) is chosen as yielding the
        best fit to Alpine glacial evidence. The basis for preferring EPICA
        seems reasonable (matching some higher-frequency amplitudes of ice
        variability, section 3.3). However, this agreement is in a sense
        coincidental, in that there is no direct meteorological link between
        Antarctic and Alpine regional climate variations. Are there any
        proximal proxy records of Alpine climate at all, perhaps lacustrian
        varves, that could be used to assess the EPICA-based shifts in air
        temperatures and precipitation, even over limited periods of the last
        120\,kyrs?}

    \referee{%
        A PDD scheme is used based only on seasonal air temperatures. van de
        Berg et al. (Nature Geosc., 2011) showed that for long-term variations
        including the Eemian, orbital changes in insolation are important and
        should be considered explicitly. This could be particularly relevant
        here, because the EPICA core does not reflect changes in insolation
        over the Alps. In further work, an insolation-change term (summer,
        local) could be combined with EPICA in the climate temperature
        paleo-forcing.}

% ----------------------------------------------------------------------

\sechead{Trimlines}

    \referee{%
        The point is well taken that trimlines do not necessarily indicate past
        ice surface elevations, but the upper limit of temperate ice with cold
        ice above (pg. 18, line 5 to pg. 19, line 4). It is an important point,
        because model LGM ice surfaces are far above most trimline elevations
        (as noted and in Fig.~6a,b). The pertinent results are shown in
        Fig.~6c, and I agree there is good support for the cold vs. temperate
        (basal) ice hypothesis.}

    \referee{%
        It might help general readers to spell out the interpretation even more
        in the text. That is, as I understand it, the observed trimlines should
        coincide with the boundaries between model areas of frozen vs.
        temperate beds, so the dots in Fig.~6c should all lie on the borders
        between the hatched and white areas. The sentence on p.~19, l.~20-21,
        is confusing in this regard. Incidentally, it would also help to add
        the word "basal" to the last sentence of the Fig.~6c caption:
        "...experienced temperate basal ice for ...".}

    \referee{%
        One reason for the remaining discrepancies in Fig.~6c could be temporal
        variations in the model boundaries, that are aggregated in time by the
        "< 1\,kyr" criterion for the hatching and the grouping of all trimline
        data. To go into this in more detail, in principle Fig.~6c could be
        expanded to show the model basal frozen-temperate boundaries at
        particular times (21.5, 22.5, etc, ka), with only the dots for each
        time period superimposed. But that may not be worth it unless there are
        large temporal variations in the model boundaries.}

    \referee{%
        A slight concern is that the majority of the trimline data seems to be
        orange dots i.e., older that 27\,ka in the timescale of Fig.~6c. The
        period for the model hatching extends back only to 29\,ka. Hopefully,
        most of the orange-dotted data are within that period and are not older
        than 29\,ka(?).}

    \referee{%
        The text could briefly mention (and hopefully rule out) the issue of
        very fine-scale topographic features on which the trimlines are
        located, not resolved by the 1-km model topography. If data sites are
        on small-scale highs or lows significantly different from their
        ~km-scale surroundings, that could contribute to the discrepancies in
        Fig.~6c.}

% ----------------------------------------------------------------------

\sechead{Technical comments}

    \referee{\textbf{p.~3, l.~2:}
        Perhaps change ``lead'' to ``led'', ``to which'' to ``to what''.}

        Done.

    \referee{\textbf{Fig.~1 caption, l.~4:}
        Perhaps change ``estimated'' to ``estimate'', or ``estimated of'' to
        ``estimated''.}

        Done.

% ----------------------------------------------------------------------

\sechead{References}

\bibliographystyle{copernicus}
\bibliography{../../references/references}

\end{document}
