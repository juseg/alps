% response-editor.tex
% ----------------------------------------------------------------------
% response-header.tex
% ----------------------------------------------------------------------

% Base class and packages
\documentclass[11pt]{article}

% Included in online comment header
\usepackage[pdftex]{graphicx}
\usepackage[pdftex]{color}
\usepackage{amssymb}
%\usepackage{times}

% Additional packages
\usepackage[T1]{fontenc}
\usepackage{geometry}
\usepackage[hidelinks]{hyperref}
\usepackage{natbib}

% Graphic path of main manuscript
\graphicspath{{../../figures/}}

% Replacements for Copernicus commands
\newcommand{\unit}[1]{\ensuremath{\mathrm{#1}}}
\newcommand{\chem}[1]{\ensuremath{\mathrm{#1}}}
\newcommand{\urlprefix}[0]{}

% Default font and spacing
\renewcommand\familydefault{\sfdefault}
\setlength{\parskip}{1.2ex}
\setlength{\parindent}{0em}
\linespread{1.0}

% Color defined in comment template
\definecolor{journalname}{rgb}{0.34,0.59,0.82}

% Personal colours
\definecolor{darkblue}{cmyk}{0.9,0.3,0.0,0.0}
\definecolor{darkgreen}{cmyk}{0.8,0.0,1.0,0.0}
\definecolor{darkred}{cmyk}{0.1,0.9,0.8,0.0}
\definecolor{darkorange}{cmyk}{0.0,0.5,1.0,0.0}
\definecolor{darkpurple}{cmyk}{0.6,0.7,0.0,0.0}
\definecolor{darkbrown}{cmyk}{0.23,0.73,0.98,0.12}

% Personal commands not used in final version
\newcommand{\todo}[1]{\textcolor{darkred}{\emph{[\textbf{TODO:} #1]}}}
\newcommand{\idea}[1]{\textcolor{darkgreen}{\emph{[\textbf{IDEA:} #1]}}}
\newcommand{\note}[1]{\textcolor{darkblue}{\emph{[\textbf{NOTE:} #1]}}}
\newcommand{\aref}[0]{\textcolor{darkblue}{\textbf{[REF.]}}}

% Redefine title and section heads
\makeatletter
\renewcommand{\familydefault}{\sfdefault}
\renewcommand{\maketitle}{\noindent\textbf{\@title}\\\@author\\\@date\\[3ex]}
\renewcommand\section{\@startsection{section}{1}{\z@}{-3ex}{2ex}%
                                    {\normalfont\large\bfseries}}
\renewcommand\subsection{\@startsection{subsection}{2}{\z@}{-3ex}{2ex}%
                                       {\normalfont\bfseries}}
\makeatother


\title{Authors' response to Anonymous Referee \#1}
\author{J.~Seguinot, on behalf of all authors.}
%\date{}

\begin{document}
\maketitle
\bigskip

% ----------------------------------------------------------------------
% Interactive comment text begins
% ----------------------------------------------------------------------

\newcommand{\sechead}[1]{\bigskip\noindent\textbf{#1}}
\newcommand{\referee}[1]{\bigskip\noindent\textcolor{darkblue}{#1}}
\newcommand{\msquote}[1]{\begin{quote}\textit{#1}\end{quote}}
\newcommand{\doi}[1]{doi:\allowbreak\href{http://dx.doi.org/#1}{#1}}

    Dear Anonymous Referee \#1,

    Thank you very much for your detailed review of our manuscript.

    \referee{%
        This paper is a landmark advance in modelling European Alpine ice
        cover, applying a high-resolution (1\,km) ice model to the entire Alps
        through the last glacial cycle, for the first time to my knowledge.
        Results are compared with diverse geological data, and several
        important findings are presented, including time-transgressive ice
        marginal extents at LGM. The climate forcing is simple, applying
        uniform perturbations to modern observed datasets, which leads to some
        uncertainty in the results, but does not detract from them too much
        given the advances made in the ice modelling alone.}

    \referee{%
        The introduction gives an elegant summary of Alpine glacial science
        since the 1700's, including many historical references. The paper is
        well organized, with well-chosen sensitivities described first that
        calibrate the climate forcing, followed by detailed analysis of one
        best-fit high-resolution (1\,km) simulation through the last 120\,kyrs.
        Detailed comparisons to a variety of geological data are made,
        constituting a thorough assessment of model performance. An impressive
        animation of the whole cycle is included as supplementary material.}

    Thank you very much for these elogious and supportive comments!


% ----------------------------------------------------------------------

\sechead{Specific comments}

    \referee{\textbf{p.~4, l.~9--10:}
        Can the physical basis of englacial water fraction and sensitivity of
        results be summarized briefly? This is not a usual component in
        ice-sheet models. Is the cap value ("capped at 0.01") well constrained,
        and does it have a significant effect on results?}

    Thank you for bringing this up.
    Laboratory experiments have demonstrated that the rheology of temperate,
    polycrystalline ice depends on its content in liquid water
    \citep[p.~65--66]{Cuffey.Paterson.2010}. Unfortunately, the only
    measurements available to date \citep{Duval.1977}, used to quantify the
    effect of liquid water on ice softness, the creep parameter $A$ in Glen's
    flow law \citep{Lliboutry.Duval.1985}, only extend to fractions of liquid
    water content between 0 and 0.8\%. They show a three-fold increase of ice
    softness over this range \citep[Fig.~1]{Duval.1977}.

    Ice sheet models have previously ignored this effect, but it has now
    become a typical component of polythermal models such as SICOPOLIS
    \citep{Greve.1997}, COMICE \citep{Ruckamp.etal.2010}, PISM
    \citep{Aschwanden.etal.2012}, ISSM \citep{Seroussi.etal.2013}, and
    TIM-FD$\unit{^{3}}$ \citep{Kleiner.Humbert.2014}.

    However water fractions between 1 and 5\,\% have repeatedly been observed
    in temperate glaciers \citep{Murray.etal.2000, Murray.etal.2007,
    Bradford.Harper.2005, Bradford.etal.2009}, and also occur in model results
    \citep[e.g.,][]{Blatter.Greve.2015}, but it is not known whether ice
    viscosity continues to decrease substancially for values above 0.8\,\%.
    Previous modelling studies have commonly assumed constant ice viscosity
    above 1\,\%. This arbitrary threshold is not constrained at all, and the
    urgent need for new ice deformation experiments has already been pointed
    out \citep{Kleiner.etal.2015}.

    The sensitivity of our results to the 1\,\% threshold was not tested.
    However, in our model results, liquid water fractions above 1\,\% typically
    only occur within the basal temperate layer of the fastest-moving glaciers
    where ice movement is largely dominated by basal sliding. We suspect that
    increased ice deformation in these regions is negligible in comparaison to
    uncertainties related to basal sliding and, more importantly, to climate 
    forcing.

    Thus, we prefer to avoid including the above discussion in the manuscript,
    but have reworked the sentence on water content:

    \msquote{%
        [Ice softness] increases with liquid water fractions up to 0.01
        \citep[p.~65--66]{Duval.1977, Lliboutry.Duval.1985,
        Cuffey.Paterson.2010}, an arbitrary threshold above which new
        ice deformation measurements are critically needed
        \citep{Kleiner.etal.2015}.}

    The uncertainty to unknown rheology of water-rich temperate ice was also
    mentioned in the conclusions:

    \msquote{%
        In the absence of ice deformation measurements, a constant rheology was
        used for temperate ice containing more than 1\,\% of liquid water.}

    \referee{\textbf{p.~4, l.~19--20:}
        The sub-glacial hydrologic component should be described more (even if
        exactly as in Bueler and van Pelt, 2015). Is basal water transported
        horizontally down the hydropotential gradient? This is usually a highly
        uncertain component of ice-sheet models, but can have a large effect on
        results through its influence on basal sliding, and basal frozen vs.
        thawed areas, which is relevant to section 4.4 regarding trimlines.}

    We refer to \citet{Bueler.Pelt.2015} as their paper contain the most
    up-to-date description of PISM till effective pressure physics used in our
    simulations \citep[Eqs.~18, 23, and 24]{Bueler.Pelt.2015}. However,
    subglacial water is not routed down the hydropotential gradient. We have
    clarified this:

    \msquote{%
        Effective pressure is related to the ice overburden stress and the
        modelled amount of subglacial water, using a~formula derived from
        laboratory experiments with till extracted from the base of Ice Stream
        B in West Antarctica \citep[Table~1;][Eqs~23 and
        24]{Tulaczyk.etal.2000, Bueler.Pelt.2015}. Basal meltwater is
        accumulated locally without transportation. When the till becomes
        saturated, additional meltwater assumed to drain off instantaneously
        outside the glacier margins, i.e. it is removed from the system in an
        accountable way.}

    \referee{\textbf{Somewhat related:}
        Little information is given on the choices of basal sliding parameter
        values in Table 1. This could be discussed briefly. Presumably no
        inversion or optimization was performed for these values beforehand,
        and they do vary spatially. Are they appropriate for Alpine bedrock
        overall?} 

    Although we refrain from repeating parameter values given in Table~1 in the
    main text, the following text was added in the methods:

    \msquote{%
        [a constant basal friction angle] corresponding to the average of
        available measurements \citep[p.~268]{Cuffey.Paterson.2010}. [...]
        Other parameters (Table~1) follow simulations of the Greenland ice
        sheet \citep{Aschwanden.etal.2013}, or benchmarks when other data is
        missing \citep{Bueler.Pelt.2015}.}

    Inversion of specific basal sliding parameters for past Alpine glaciers
    is difficult because the altitude and age of maximum ice surface elevation
    is discussed (cf. introduction and discussion on trimlines), and also
    depends on the even more uncertain regional climate history.  Therefore,
    basal sliding was also mentioned as one of the major sources of
    uncertainty in the conclusions.

    \msquote{%
        The till deformation model used here does not hold for sliding over
        bedrock surfaces. On the other hand, the constant friction angle used
        is representative of wet till but weaker basal conditions may have
        applied over saturated lake sediments where they occured.}
 
    \referee{\textbf{p.~7, l.~7:}
        Are there any data to support this atmospheric lapse rate value
        (6\,K\,km$^{-1}$), and do other values have the potential to
        significantly affect ice temperatures? In particular, could they change
        the basal areas of frozen/unfrozen ice and so the comparisons with
        trimlines in section 4.4?}

    In the European Alps, monthly temperature lapse rates vary between
    approximately 4\,K\,km$^{-1}$ in winter and 7\,K\,km$^{-1}$ in summer, and
    annual temperature lapse rates vary spatially between 5.4 and
    5.8\,K\,km$^{-1}$ \citep{Rolland.2003}. A reference to the study by
    \citet{Rolland.2003} was added in Table~1, and as we have now explicited,
    our constant value of 6\,K\,km$^{-1}$ is thus:

    \msquote{%
        slightly above average but more representative of summer months when
        surface melt occurs \citep[Fig.~3]{Rolland.2003}.}

    Although no sensitivity tests were conducted, we argue here that the effect
    of atmospheric temperature lapse rate variations on ice temperature is
    negligible. First, seasonal variations do not penetrate ice or bedrock
    below a few metres. Second, even above 2\,km of ice, spatial variations of
    0.4\,K\,km$^{-1}$ would translate into surface temperature variations of
    only 0.8\,K. But during the Last Glacial Maximum, ice surface temperatures
    in the trimline region are typically between 15 and 20\,K below freezing.
    Thus the effect on the temperature gradient would be small.  Actually, the
    effect would even nearly disappear near the glacier base, where the
    temperature gradient is much steeper and primarily controlled by geothermal
    heat flux and shear heating.


% ----------------------------------------------------------------------

\sechead{Climate forcing}

    \referee{%
        The method of spatially uniform shifts to modern climate forcing is
        common in paleo-modeling of large ice sheets, and in my opinion is
        acceptable as a starting point in this work, with coupling to regional
        climate models (RCMs) left to follow-on work. There are good
        discussions on possible shortcomings of this method, for instance as a
        cause of anomalous east-west marginal ice extents at LGM (pg. 11, line
        6-8). However, I suggest changing the sentence on pg. 9, line 5, which
        mentions some RCMs applied to LGM Europe, but also says "...over the
        Alps during the last glacial cycle, of which little is known apart from
        the LGM". There are several other RCM modeling studies over Europe
        during the last 120\,kyrs, e.g., for MIS Stage 3, Kjellstrom et al.,
        Boreas, 2010; Barron and Pollard, Quat. Res., 2002; Alfano et al.,
        Quat. Res., 2002; and for 6\,ka, Strandberg et al., Clim. Past, 2014.
        Perhaps there is little useful material there for the Alps, but such
        papers exist.}

    Thank you.
    We did not know about references on MIS~3 and have developped our sentence
    to better reflect the current state of knowledge on palaeo-precipitation:

    \msquote{%
        Palaeoclimate proxies indicate slightly reduced LGM precipitation in
        western Europe with anomalies diminishing eastwards
        \citep{Wu.etal.2007}. Regional circulation models indicate generally
        dryer conditions during MIS~3 \citep{Barron.Pollard.2002,
        Kjellstrom.etal.2010} but more precipitation south of the Alps during
        MIS~2 \citep{Strandberg.etal.2011, Ludwig.etal.2016}.}

    \referee{%
        The past climate variations are prescribed following 3 quite distal
        core records, and the most distal (EPICA) is chosen as yielding the
        best fit to Alpine glacial evidence. The basis for preferring EPICA
        seems reasonable (matching some higher-frequency amplitudes of ice
        variability, section 3.3). However, this agreement is in a sense
        coincidental, in that there is no direct meteorological link between
        Antarctic and Alpine regional climate variations. Are there any
        proximal proxy records of Alpine climate at all, perhaps lacustrian
        varves, that could be used to assess the EPICA-based shifts in air
        temperatures and precipitation, even over limited periods of the last
        120\,kyrs?}

    A short review of available proximal proxy records was added here:

    \msquote{%
        Only few regional proxy records exist that extend over periods when the
        Alps were glaciated \citep{Heiri.etal.2014}. These include lake
        sediment records in north and west of the Alps
        \citep[e.g.,]{Beaulieu.Reille.1992, Wohlfarth.etal.2008,
        Duprat-Oualid.etal.2017}, and cave speleothems in the Eastern
        \citep[e.g.,][]{Spotl.Mangini.2002, Boch.etal.2011} and Western
        \citep{Luetscher.etal.2015} Alps. Due to the scarcity of vegetation
        north of the Alps during glacial periods, varying sources for moisture
        advection, and the limited duration of the records, quantitative
        palaeoclimatic interpretation will require combining multiple proxies
        in space and time, and comparing them against regional circulation
        model output \citep{Heiri.etal.2014}.}

    In the review paper by \citet{Heiri.etal.2014}, the latter has been
    specifically identified as a neeeded future development, but to our
    knowledge, no quantitative reconstruction has been made available since
    then. The following word of caution was also added in the conclusions:

    \msquote{%
        This suprising result is likely coincidental as there is no direct link
        between European and Antarctic climate. This highlights the need for
        more quantitative reconstructions of European palaeoclimate.}

    \referee{%
        A PDD scheme is used based only on seasonal air temperatures. van de
        Berg et al. (Nature Geosc., 2011) showed that for long-term variations
        including the Eemian, orbital changes in insolation are important and
        should be considered explicitly. This could be particularly relevant
        here, because the EPICA core does not reflect changes in insolation
        over the Alps. In further work, an insolation-change term (summer,
        local) could be combined with EPICA in the climate temperature
        paleo-forcing.}

    We agree. This is one of many possible improvements that could be tested
    upon the climate forcing used here. In the conclusions we now advocate for:

    \msquote{%
        a more realistic climate forcing based on regional circulation model
        output or including the effect of long-term term changes in incoming
        solar radiation.}


% ----------------------------------------------------------------------

\sechead{Trimlines}

    \referee{%
        The point is well taken that trimlines do not necessarily indicate past
        ice surface elevations, but the upper limit of temperate ice with cold
        ice above (pg. 18, line 5 to pg. 19, line 4). It is an important point,
        because model LGM ice surfaces are far above most trimline elevations
        (as noted and in Fig.~6a,b). The pertinent results are shown in
        Fig.~6c, and I agree there is good support for the cold vs. temperate
        (basal) ice hypothesis.}

    Thank you for emphasizing this point.

    \referee{%
        It might help general readers to spell out the interpretation even more
        in the text. That is, as I understand it, the observed trimlines should
        coincide with the boundaries between model areas of frozen vs.
        temperate beds, so the dots in Fig.~6c should all lie on the borders
        between the hatched and white areas. The sentence on p.~19, l.~20-21,
        is confusing in this regard. Incidentally, it would also help to add
        the word "basal" to the last sentence of the Fig.~6c caption:
        "...experienced temperate basal ice for ...".}

    We realise that Fig.~6 is confusing. We had chosen the 1\,ka limit because temperate
    basal ice tends to occur above the trimlines during short periods of warmer
    climate. To avoid confusion, Fig.~6c has been simplified to show the basal
    thermal boundary at the age of maximum ice thickness. The last sentence in
    the caption now reads:

    \msquote{%
        Hatches mark the LGM cold-based ares (basal temperature above $1 \times
        10^{-3}$\,K below freezing at the age of maximum ice thickness).}

    Unfortunately this also results in a worse fit to trimline locations, as
    was noted in the main text:

    \msquote{%
        In the upper Rhone Valley, observed trimlines are often located near
        the LGM cold-temperate basal thermal transition, or in cold-based areas
        (Fig.~6c).}
 
    \referee{%
        One reason for the remaining discrepancies in Fig.~6c could be temporal
        variations in the model boundaries, that are aggregated in time by the
        "< 1\,kyr" criterion for the hatching and the grouping of all trimline
        data. To go into this in more detail, in principle Fig.~6c could be
        expanded to show the model basal frozen-temperate boundaries at
        particular times (21.5, 22.5, etc, ka), with only the dots for each
        time period superimposed. But that may not be worth it unless there are
        large temporal variations in the model boundaries.}

    Fig.~6c has been simplified to show the basal thermal boundary at the age
    of maximum ice thickness. Although this results in a time-transgressive
    picture, it means that the frozen areas indicated on the new figure are
    contemporaneous with the plotted ages and surface topography contours, and
    data shown on panels~a and b.

    Although the model output allows to study the migration of the basal
    thermal boundary over time into more detail, the basal velocity, which also
    controls erosion, should probably be considered as well. We leave this for
    future studies. The following sentence was added in the main text:

    \msquote{%
        The remaining discrepancies may relate to temporal migrations of the
        basal thermal boundary, an absence of sliding in warm-based areas [...]}

    \referee{%
        A slight concern is that the majority of the trimline data seems to be
        orange dots i.e., older that 27\,ka in the timescale of Fig.~6c. The
        period for the model hatching extends back only to 29\,ka. Hopefully,
        most of the orange-dotted data are within that period and are not older
        than 29\,ka(?).}

    In fact, much of the mountainous areas reach maximum ice thickness during
    early MIS~2 when ice is colder and stiffer. This is also visible on
    Fig.~5. Some areas even reach maximum ice thickness during MIS~4 when the
    bedrock is slightly less depressed. There is certainly very much scope to
    discuss the age of the trimlines, which may differ significantly from that of
    the maximum ice extent on the lowland, but we are not aware of any data
    that could be used for validation here. However, frozen areas plotted on
    Fig.~6c are now consistent with the modelled ages used in panels a and c.

    \referee{%
        The text could briefly mention (and hopefully rule out) the issue of
        very fine-scale topographic features on which the trimlines are
        located, not resolved by the 1-km model topography. If data sites are
        on small-scale highs or lows significantly different from their
        \~km-scale surroundings, that could contribute to the discrepancies in
        Fig.~6c.}

    This is a valid point which we can unfortunately not rule out. Despite
    PISM's enthalpy scheme, which ensures a seamless transition between cold
    and temperate ice physics, resolving the basal thermal boundary on the
    valley sides is delicate, because it implies resolving a steep (vertical)
    enthalpy gradient over a steep bedrock slope. This requires both high
    vertical and high horizontal grid resolutions. This limitation was mentioned
    in the section on trimlines:

    \msquote{%
        [The remaining discrepancies may relate to] levelling of small-scale
        topographic features in the 1\,km horizontal grid. They call for more
        detailed comparisons [...]}


% ----------------------------------------------------------------------

\sechead{Technical comments}

    \referee{\textbf{p.~3, l.~2:}
        Perhaps change ``lead'' to ``led'', ``to which'' to ``to what''.}

        Done.

    \referee{\textbf{Fig.~1 caption, l.~4:}
        Perhaps change ``estimated'' to ``estimate'', or ``estimated of'' to
        ``estimated''.}

        Done. We thank you very much again for the time and effort you put into
        our manuscript.


% ----------------------------------------------------------------------
% References

\bibliographystyle{copernicus}
\bibliography{../../references/references}

\end{document}
