% response-editor.tex
% ----------------------------------------------------------------------
% response-header.tex
% ----------------------------------------------------------------------

% Base class and packages
\documentclass[11pt]{article}

% Included in online comment header
\usepackage[pdftex]{graphicx}
\usepackage[pdftex]{color}
\usepackage{amssymb}
%\usepackage{times}

% Additional packages
\usepackage[T1]{fontenc}
\usepackage{geometry}
\usepackage[hidelinks]{hyperref}
\usepackage{natbib}

% Graphic path of main manuscript
\graphicspath{{../../figures/}}

% Replacements for Copernicus commands
\newcommand{\unit}[1]{\ensuremath{\mathrm{#1}}}
\newcommand{\chem}[1]{\ensuremath{\mathrm{#1}}}
\newcommand{\urlprefix}[0]{}

% Default font and spacing
\renewcommand\familydefault{\sfdefault}
\setlength{\parskip}{1.2ex}
\setlength{\parindent}{0em}
\linespread{1.0}

% Color defined in comment template
\definecolor{journalname}{rgb}{0.34,0.59,0.82}

% Personal colours
\definecolor{darkblue}{cmyk}{0.9,0.3,0.0,0.0}
\definecolor{darkgreen}{cmyk}{0.8,0.0,1.0,0.0}
\definecolor{darkred}{cmyk}{0.1,0.9,0.8,0.0}
\definecolor{darkorange}{cmyk}{0.0,0.5,1.0,0.0}
\definecolor{darkpurple}{cmyk}{0.6,0.7,0.0,0.0}
\definecolor{darkbrown}{cmyk}{0.23,0.73,0.98,0.12}

% Personal commands not used in final version
\newcommand{\todo}[1]{\textcolor{darkred}{\emph{[\textbf{TODO:} #1]}}}
\newcommand{\idea}[1]{\textcolor{darkgreen}{\emph{[\textbf{IDEA:} #1]}}}
\newcommand{\note}[1]{\textcolor{darkblue}{\emph{[\textbf{NOTE:} #1]}}}
\newcommand{\aref}[0]{\textcolor{darkblue}{\textbf{[REF.]}}}

% Redefine title and section heads
\makeatletter
\renewcommand{\familydefault}{\sfdefault}
\renewcommand{\maketitle}{\noindent\textbf{\@title}\\\@author\\\@date\\[3ex]}
\renewcommand\section{\@startsection{section}{1}{\z@}{-3ex}{2ex}%
                                    {\normalfont\large\bfseries}}
\renewcommand\subsection{\@startsection{subsection}{2}{\z@}{-3ex}{2ex}%
                                       {\normalfont\bfseries}}
\makeatother


\title{Authors' response to the Editor}
\author{J. Seguinot, on behalf of all authors.}
%\date{}

\begin{document}
\thispagestyle{empty}
\maketitle
\bigskip

% ----------------------------------------------------------------------
% Interactive comment text begins
% ----------------------------------------------------------------------

    Dear Andreas Vieli,

    \newcommand{\referee}[1]{\bigskip\noindent\textcolor{journalname}{#1}}

    \referee{\textbf{p.~1, l.~14-17:}
        somewhat odd phrasing: `...cause ...to be modelled...', maybe
        `...produces in the model...' or similar would be better.}

    \referee{\textbf{p.~3, l.~4:}
        `it still remains incompletely known' is awkward phrasing, maybe better
        say: `...knowledge remains incomplete...'}

    \referee{\textbf{p.~3, l.~7-8:}
        the added sentence starting with `Additional...' seems slightly out of
        place here (or the flow is someaht disrupted). Maybe better move it to
        end of nextz paragraph (p.~3, l.~14).}

    \referee{\textbf{p.~3, l.~9:}
        starting a paragraph with a `But...' is awkward, maybe change back to
        `Here we...' in particular when you move the sentence before.}

    \referee{\textbf{p.~3, l.~32:}
        would `different' instead of `distinct' not be more appropriate.}

    \referee{\textbf{p.~4, l.~8:}
        `In the SIA, topographic roughness using a bed smoother range of 5 km
        (Schoof, 2003).' I really do not understand this sentence (what is
        message), nor its connection to the sentence before.}

    \referee{\textbf{p.~4, l.~24:}
        `...without horizontal transport...'?}

    \referee{\textbf{p.~4, l.~25:}
        maybe simplify to `...is assumed to drain instantaneously and is
        removed....'}

    \referee{\textbf{p.~7, l.~3:}
        `above average' of what? What is reference?}

    \referee{\textbf{p.~14, fig.~4:}
        are the surface contours really in 200m intervals?, that would make the
        ice surface only a bit over 1200m high. Or do you leave some out?  You
        should add some labels to the surface contour lines so one can judge
        the surface elevation better.}

    \referee{\textbf{p.~13, l.~30 -- p.~14, l.~3:}
        but jouvet (TC) claims (based on tracking erratics) that north south
        contrast in in precip is needed. Is this still consistent with this
        work? Maybe should be included. This is mentioned on next page but
        would be relevant here as well.}

    \referee{\textbf{p.~14, fig.~4 caption:}
        for `(b)' it should say what the vertical black line (time of max
        extent in (a) at 24.57 ka???) refers to.}

    \referee{\textbf{p.~15, l.~31:}
        `simulation' (not sumulation)}

    \referee{\textbf{p.~16, l.~12:}
        `Fig. 5' has nothing to do with `reviews', what do you mean with `Fig.
        5 for reviews'???}

    \referee{\textbf{p.~19, l.~15-19:}
        in this context of sensitivity to basal sliding parameter (and lacking
        exploration) maybe some reference to the Phd-thesis of Becker which
        includes at least some preliminary (mostly steady state) sliding
        investigation could be made.}

    \referee{\textbf{p.~21, l.~19:}
        `last major Alpine reliefs' ...? Do you mean the `last foothills' of the
        alps? Clarify.}

    \referee{\textbf{p.~23, l.~1:}
        is it not rather `glacier physics' as the till is not ice (or ice and
        glacier physics).}

    We thank you again very much for your editorial work on our manuscript.

\end{document}
