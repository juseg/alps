\documentclass{article}

\usepackage{authblk}
\usepackage[T1]{fontenc}
\usepackage[utf8]{inputenc}
\usepackage[pdftex]{xcolor}
\usepackage[pdftex]{graphicx}
\graphicspath{{../../figures/}}

\definecolor{darkblue}{cmyk}{0.9,0.3,0.0,0.0}
\definecolor{darkgreen}{cmyk}{0.8,0.0,1.0,0.0}
\definecolor{darkred}{cmyk}{0.1,0.9,0.8,0.0}

\newcommand{\idea}[1]{\textcolor{darkgreen}{\emph{[\textbf{IDEA:} #1]}}}
\newcommand{\note}[1]{\textcolor{darkblue}{\emph{[\textbf{NOTE:} #1]}}}
\newcommand{\todo}[1]{\textcolor{darkred}{\emph{[\textbf{TODO:} #1]}}}
\newcommand{\aref}[0]{\textcolor{darkblue}{\textbf{[REF.]}}}

\title{Modelling last glacial cycle ice dynamics in the Alps}

\author[1]{Julien Seguinot%
           \thanks{Correspondence to seguinot@vaw.baug.ethz.ch}}
\author[1]{Guillaume Jouvet}
\author[1]{Matthias Huss}
\author[1]{Martin Funk}
\author[2]{Frank Preusser}

\affil[1]{Laboratory of Hydraulics, Hydrology and Glaciology,
          ETH Zürich, Switzerland}
\affil[2]{Institute of Earth and Environmental Sciences,
          University of Freiburg, Germany}

% ======================================================================
\begin{document}
% ======================================================================

\maketitle

\begin{abstract}

    The European Alps, cradle of pioneer glacial studies, are one of the
    regions where geological markers of past glaciations are most abundant and
    well-studied. Such conditions make the region ideal for testing numerical
    glacier models based on approximated ice flow physics against field-based
    reconstructions, and vice-versa.

    Here, we use the Parallel Ice Sheet Model (PISM) to model the entire last
    glacial cycle (120--0\,ka) in the Alps, with a horizontal resolution of
    1\,km. Climate forcing is derived using present-day climate data from
    WorldClim and the ERA-Interim reanalysis, and time-dependent temperature
    offsets from multiple paleo-climate proxies, among which only the EPICA ice
    core record yields glacial extent during marine oxygen isotope stages~4
    (69--62\,ka) and~2 (34--18\,ka) in agreement to geological reconstructions.

    Despite the low variability of this Antarctic-based climate forcing, our
    simulation depicts a highly dynamic ice cap, showing that alpine glaciers
    may have advanced many times over the foreland during the last glacial
    cycle. Cumulative basal sliding, a proxy for glacial erosion, is modelled
    to be highest in the deep valleys of the western Alps. Finally, the Last
    Glacial Maximum advance, often considered synchronous, is here modelled as
    a time-transgressive event, with some glacier lobes reaching their maximum
    as early as 27\,ka, and some as late as 21\,ka. Modelled ice thickness is
    about 900\,m higher than observed trimline elevations, yet our simulation
    predicts little erosion at high elevation due to cold ice conditions.

\end{abstract}

\section{Main}
\subsection{Introduction}
\subsection{Ice cap evolution}
    Fig.~\ref{fig:lgmvel} -- Snapshot at 21\,ka and volume time series.
\subsection{Timing of the LGM}
    Fig.~\ref{fig:timing} -- Timing of the LGM and area time series.
\subsection{Ice thickness}
    Fig.~\ref{fig:trimlines} -- Ice thickness comparison to trimlines.
\subsection{Erosion potential}
    Fig.~\ref{fig:erosion} -- Integrated erosion potential.

\section{Methods}
\subsection{Overview}
    Tab. -- Model parameters
\subsection{Ice rheology}
\subsection{Basal sliding}
\subsection{Bedrock deformation}
\subsection{Surface mass balance}
\subsection{Climate forcing}
    Fig.~\ref{fig:inputs} -- Climate and geothermal model inputs.\\
    Tab. -- Records\\
    Fig.~\ref{fig:timeseries} -- Low-resolution time series.\\
    Fig.~\ref{fig:footprints} -- Low-resolution ice cover.

\section{Conclusion}

% ----------------------------------------------------------------------
% References
% ----------------------------------------------------------------------

\bibliographystyle{abbrvnat}
\bibliography{../../references/references}

% ----------------------------------------------------------------------
% Figures
% ----------------------------------------------------------------------

    \begin{figure}
      \centerline{\includegraphics{alpcyc_hr_lgmvel}}
      \caption{Model output at 21\,ka.}
      \label{fig:lgmvel}
    \end{figure}

    \begin{figure}
      \centerline{\includegraphics{alpcyc_hr_timing}}
      \caption{Timing of the LGM and area time series.}
      \label{fig:timing}
    \end{figure}

    \begin{figure}
      \centerline{\includegraphics{alpcyc_hr_trimlines}}
      \caption{Ice thickness comparison to trimlines.}
      \label{fig:trimlines}
    \end{figure}

    \begin{figure}
      \centerline{\includegraphics{alpcyc_hr_erosion}}
      \caption{Integrated erosion potential.}
      \label{fig:erosion}
    \end{figure}

% ----------------------------------------------------------------------
% Supplementary tables
\setcounter{table}{0}
\renewcommand{\thetable}{S\arabic{table}}%
% ----------------------------------------------------------------------

% ----------------------------------------------------------------------
% Supplementary figures
\setcounter{figure}{0}
\renewcommand{\thefigure}{S\arabic{figure}}%
% ----------------------------------------------------------------------

    \begin{figure}
      \centerline{\includegraphics{alpcyc_hr_inputs}}
      \caption{Climate and geothermal model inputs.}
      \label{fig:inputs}
    \end{figure}

    \begin{figure}
      \centerline{\includegraphics{alpcyc_lr_timeseries}}
      \caption{Low-resolution time series.}
      \label{fig:timeseries}
    \end{figure}

    \begin{figure}
      \centerline{\includegraphics{alpcyc_lr_footprints}}
      \caption{Low-resolution ice cover.}
      \label{fig:footprints}
    \end{figure}

% ======================================================================
\end{document}
% ======================================================================
