\documentclass{article}

\usepackage{doi}
\usepackage{amsmath}
\usepackage{authblk}
\usepackage{booktabs}
\usepackage{multirow}
\usepackage[T1]{fontenc}
\usepackage[utf8]{inputenc}
\usepackage[pdftex]{xcolor}
\usepackage[pdftex]{graphicx}
\usepackage[authoryear,round]{natbib}

\graphicspath{{../../figures/}}

\definecolor{darkblue}{cmyk}{0.9,0.3,0.0,0.0}
\definecolor{darkgreen}{cmyk}{0.8,0.0,1.0,0.0}
\definecolor{darkred}{cmyk}{0.1,0.9,0.8,0.0}
\definecolor{darkorange}{cmyk}{0.0,0.5,1.0,0.0}
\definecolor{darkpurple}{cmyk}{0.6,0.7,0.0,0.0}
\definecolor{darkbrown}{cmyk}{0.23,0.73,0.98,0.12}

\newcommand{\idea}[1]{\textcolor{darkgreen}{\emph{[\textbf{IDEA:} #1]}}}
\newcommand{\note}[1]{\textcolor{darkblue}{\emph{[\textbf{NOTE:} #1]}}}
\newcommand{\todo}[1]{\textcolor{darkred}{\emph{[\textbf{TODO:} #1]}}}
\newcommand{\aref}[0]{\textcolor{darkblue}{\textbf{[REF.]}}}

\hypersetup{colorlinks, linkcolor=darkorange,
            urlcolor=darkbrown, citecolor=darkpurple}

\title{Modelling last glacial cycle ice dynamics in the Alps}

\author[1]{Julien Seguinot%
           \thanks{Correspondence to seguinot@vaw.baug.ethz.ch}}
\author[1]{Guillaume Jouvet}
\author[1]{Matthias Huss}
\author[1]{Martin Funk}
\author[2]{Frank Preusser}

\affil[1]{Laboratory of Hydraulics, Hydrology and Glaciology,
          ETH Zürich, Switzerland}
\affil[2]{Institute of Earth and Environmental Sciences,
          University of Freiburg, Germany}

% Common units
\newcommand{\e}[1]{\ensuremath{\times 10^{#1}}}
\newcommand{\chem}[1]{\ensuremath{\mathrm{#1}}}
\newcommand{\unit}[1]{\ensuremath{\mathrm{#1}}}
\newcommand{\degree}[0]{\ensuremath{^{\circ}}}
\newcommand{\degC}[0]{\unit{{\degree}C}}


% ======================================================================
\begin{document}
% ======================================================================

\maketitle

\begin{abstract}

    The European Alps, cradle of pioneer glacial studies, are one of the
    regions where geological markers of past glaciations are most abundant and
    well-studied. Such conditions make the region ideal for testing numerical
    glacier models based on approximated ice flow physics against field-based
    reconstructions, and vice-versa.

    Here, we use the Parallel Ice Sheet Model (PISM) to model the entire last
    glacial cycle (120--0\,ka) in the Alps, with a horizontal resolution of
    1\,km. Climate forcing is derived using present-day climate data from
    WorldClim and the ERA-Interim reanalysis, and time-dependent temperature
    offsets from multiple paleo-climate proxies, among which only the EPICA ice
    core record yields glacial extent during marine oxygen isotope stages~4
    (69--62\,ka) and~2 (34--18\,ka) in agreement to geological reconstructions.

    Despite the low variability of this Antarctic-based climate forcing, our
    simulation depicts a highly dynamic ice cap, showing that alpine glaciers
    may have advanced many times over the foreland during the last glacial
    cycle. Cumulative basal sliding, a proxy for glacial erosion, is modelled
    to be highest in the deep valleys of the western Alps. Finally, the Last
    Glacial Maximum advance, often considered synchronous, is here modelled as
    a time-transgressive event, with some glacier lobes reaching their maximum
    as early as 27\,ka, and some as late as 21\,ka. Modelled ice thickness is
    about 900\,m higher than observed trimline elevations, yet our simulation
    predicts little erosion at high elevation due to cold ice conditions.

\end{abstract}


% ----------------------------------------------------------------------
\section{Introduction}
\label{sec:intro}
% ----------------------------------------------------------------------

    Glaciers move by a combination of meltwater-induced sliding at the base
    \citep[\S532]{Saussure.1779}, and viscous deformation within the ice body
    \citep{Forbes.1846b}. As glaciers flow and slide across their bed, they
    transport rock debris and erode the landscape, thereby leaving
    geomorphological traces of their former presence.

    For nearly 300~years, montane people and early explorers of the European
    Alps learned to read the geomorphological imprint left by glaciers in the
    landscape, and to understand that glaciers had once been more extensive
    than today \citep[e.g.,][p.~21]{Windham.Martel.1744}. In the mid-nineteenth
    century, more systematic studies of glacial features showed that alpine
    glaciers extended well below their contemporary margins \citep{Venetz.1821}
    and even onto the alpine foreland \citep{Charpentier.1841}, yielding the
    idea that, under colder temperatures, expansive ice sheets had once covered
    much of Europe and North America \citep{Agassiz.1840}.

    However, this glacial theory did not gain general acceptance until the
    discovery and exploration of the two present-day ice sheets on Earth, the
    Greenland and Antarctic ice sheets, provided a modern analogue for the
    proposed European and North American ice sheets. Although it was long
    unclear whether there had been a single or multiple glaciations, this
    controversy ended with the large-scale mapping of two distinct moraine
    systems in North America \citep{Chamberlin.1894}. In the European Alps, the
    systematic classification of the glaciofluvial terraces on the northern
    foreland indicated that landscape, that there had been at least four major
    glaciations in the Alps \citep{Penck.Bruckner.1909}. More recent studies
    of glaciofluvial stratigraphy in the foreland indicate up to eight
    glaciations \citep{Ivy-Ochs.etal.2008, Preusser.etal.2011}.

    More recently, palaeoclimate records extracted from deep sea sediments and
    ice cores have provided a much more detailed picture of the Earth
    environmental history \citep[e.g.,][]{Emiliani.1955,
    Shackleton.Opdyke.1973, Dansgaard.etal.1993, Augustin.etal.2004},
    indicating neither one, nor four, but rather tens of glacial cycles. During
    the last 800000 years (800\,ka), glacial and interglacial periods have
    succeeded each other with a 100\,ka periodicity \citep{Hays.etal.1976,
    Augustin.etal.2004}. However, much of the traces left by glaciers on the
    landscape date from the most recent of these cycles, the \emph{last glacial
    cycle}.

    Nevertheless, the landform record is sparse in time and, most often,
    spatially incomplete. Palaeo-ice sheets did not leave a continuous imprint
    on the lanscape, and much of this evidence has been overprinted by
    subsequent glacier re-advances and other geomorphological processes
    \citep{Kleman.1994, Kleman.etal.2006, Kleman.etal.2010}. The landscape
    imprint of previous glaciations has been largely overprinted by the more
    recent ones; therefore most of the glacial features currently left on the
    foreland present a record of the last major glaciation of the Alps, dating
    from the Last Glacial Maximum (LGM).

    Because this global signal is largely governed by the North American and
    Eurasian ice sheet complexes, it is unclear wether glacier advances and
    retreats in the Alps were in pac(71--57\,ka)e with global sea-level
    fluctuations. The timing of the LGM in the Alps (Ivy-Ochs et al., 2008;
    Monegato et al., 2007) is in good agreement with the maximum expansion of
    continental ice sheets recorded by the Marine Oxygen Isotope Stage (MIS) 2
    between 29 000 and 14 000 years before present (BP) (Lisiecki and Raymo,
    2005). There are nonetheless regional variations between different piedmont
    lobes, that are of the same order of magnitude with the dating
    uncertainties. Because recent glaciation overprinted older ones, and
    because dating uncertainties typically increase with age, dated
    reconstructions are strongly biased towards earlier glaciations
    \citep{Heyman.etal.2011}, and thus much of the glacial records dates to the
    Last Glacial Maximum (LGM). The Alps have been studied in more detail than
    any other glacial region. There is evidence for glaciation during MIS 4
    (71--57\,ka) but potential foreland glaciations during the rest of the last
    glacial cycle are still discussed.

    Here, we use the Parallel Ice Sheet Model
    \citep[PISM,][]{PISM-authors.2017}, a numerical ice sheet model that
    approximates glacier sliding and deformation (Sect.~\ref{sec:model}), to
    model alpine glacier dynamics through the last glacial cycle (120--0\,ka),
    a period for which palaeo-temperature proxies are available, albeit non
    regional. We test the model sensitivity to multiple palaeo-climate forcing
    (Sect.~\ref{sec:climate}), and then explore the modelled glacier dynamics
    at high resolution for the optimal forcing (Sect.~\ref{sec:results}), and
    the resulting glacial erosion pattern (Sect.~\ref{sec:erosion}).

    \idea{Matthias: I think it is very important to emphasize the knowledge gap
          that you fill in with your work in the Intro. (1) it is incompletely
          known how many advances occurred during the Last glacial cycle, (2)
          what drove the different response of the individual lobes (you show
          that it is glacier response time, i.e. ice dynamics), (3) how the
          temporal dynamics of subglacial erosion was structured.}


% ----------------------------------------------------------------------
\section{Ice sheet model set-up}
\label{sec:model}
% ----------------------------------------------------------------------

    Table~\ref{tab:params} -- Model parameters.\\
    Fig.~\ref{fig:inputs} -- Climate and geothermal model inputs.\\


% -- -- -- -- -- -- -- -- -- -- -- -- -- -- -- -- -- -- -- -- -- -- -- -
\subsection{Overview}
\label{sec:overview}
% -- -- -- -- -- -- -- -- -- -- -- -- -- -- -- -- -- -- -- -- -- -- -- -

    We use the Parallel Ice Sheet Model (PISM, development version~e9d2d1f), an
    open source, finite difference, shallow ice sheet model
    \citep{PISM-authors.2017}. The model requires input on basal
    topography, geothermal heat flux and climate forcing. It computes the
    evolution of ice extent and thickness over time, the thermal and dynamic
    states of the ice sheet, and the associated lithospheric response. The
    set-up used here is largely based on that used in previous studies of the
    former Cordilleran ice sheet \citep{Seguinot.2014, Seguinot.etal.2014,
    Seguinot.etal.2016}.

    \idea{Remove next paragraph?}

    Ice deformation follows temperature and water-content dependent creep
    (Sect.~\ref{sec:icedyn}). Basal sliding follows a~pseudo-plastic law where
    the yield stress accounts for till deconsolidation under high water
    pressure (Sect.~\ref{sec:sliding}). Bedrock topography is deflected
    under the ice load (Sect.~\ref{sec:bedrock}). Surface mass balance is
    computed using a~positive degree-day (PDD) model (Sect.~\ref{sec:surface}).
    Climate forcing is provided by a~monthly climatology from interpolated
    observational data \citep[WorldClim;][]{Hijmans.etal.2005} and the European
    Centre for Medium-Range Weather Forecasts Reanalysis Interim
    \citep[ERA-Interim;][]{Dee.etal.2011}, perturbed by time-dependent
    temperature offsets, temperature lapse-rate corrections, and in some cases,
    time-dependent paleo-precipitation reductions (Sect.~\ref{sec:atm},
    Table~\ref{tab:records}).

    \idea{Remove next paragraph?}

    Each simulation starts from assumed present-day ice thickness and
    equilibrium temperature distribution at 120\,\unit{ka}, and runs to the
    present. Our modelling domain of 900 by 600\,\unit{km} encompasses the
    entire Alpine range (Fig.~\ref{fig:inputs}). The simulations were run on
    two distinct grids, using a~lower horizontal resolution of 2\,\unit{km},
    and a~higher horizontal resolution of 1\,\unit{km}.

    \idea{In the following, I repeat all parameters in the text and table. Is
          this necessary? Equations are the same as in my Cordillera paper.
          Should I remove them?}


% -- -- -- -- -- -- -- -- -- -- -- -- -- -- -- -- -- -- -- -- -- -- -- -
\subsection{Ice rheology}
\label{sec:icedyn}
% -- -- -- -- -- -- -- -- -- -- -- -- -- -- -- -- -- -- -- -- -- -- -- -

    Ice sheet dynamics are typically modelled using a~combination of internal
    deformation and basal sliding. PISM is a~shallow ice sheet model, which
    implies that the balance of stresses is approximated based on their
    predominant components. The Shallow Shelf Approximation (SSA) is combined
    with the Shallow Ice Approximation (SIA) by adding velocity solutions of
    the two approximations \citep[Eqs.~7--9 and 15]{Winkelmann.etal.2011}.
    Although this heuristic approach implies errors in the transition zone
    where gravitational stresses intervene both in the SIA and SSA velocity
    computation, this hybrid scheme is computationally much more efficient than
    a fully three-dimensional model with which the simulations presented here
    would not be feasible.

    Ice deformation is governed by the constitutive law for ice
    \citep{Glen.1952, Nye.1953},
    %
    \begin{align}
      \vec{\dot{\epsilon}} = A\,\tau_{\mathrm{e}}^{n-1}\,\vec{\tau} \,.
    \end{align}
    %
    where $\vec{\dot{\epsilon}}$ is the strain-rate tensor, $\vec{\tau}$ the
    deviatoric stress tensor, and $\tau_{\mathrm{e}}$ the effective stress
    defined in our case by
    ${\tau_{\mathrm{e}}}^2=\frac{1}{2}\mathrm{tr}(\vec{\tau}^2)$. The ice
    softness coefficient, $A$, depends on ice temperature, $T$, pressure, $p$,
    and water content, $\omega$, through a~piece-wise Arrhenius-type law,
    %
    \begin{align}
    &A = E\cdot
      \begin{cases}
        A_{\mathrm{c}} \,e^\frac{-Q_{\mathrm{c}}}{RT_{\text{pa}}}
            & \text{if}\ T_   {\text{pa}}  <  T_{\mathrm{c}} \,, \\
        A_{\mathrm{w}} (1+f\omega)\,e^\frac{-Q_{\mathrm{w}}}{RT_{\text{pa}}}
            & \text{if}\ T_   {\text{pa}} \ge T_{\mathrm{c}} \,,
      \end{cases}
    \end{align}
    %
    where $T_{\text{pa}}$ is the pressure-adjusted ice temperature calculated
    using the Clapeyron relation, $T_{\text{pa}}=T-{\beta}p$.
    $R=8.31441$\,\unit{J\,mol^{-1}\,K^{-1}} is the ideal gas constant, and
    $A_{\mathrm{c}}=2.847\e{-13}$\,\unit{Pa^{-3}\,s^{-1}},
    $A_{\mathrm{w}}=2.356\e{-2}$\,\unit{Pa^{-3}\,s^{-1}},
    $Q_{\mathrm{c}}=6.0\e4$\,\unit{J\,mol^{-1}}, and
    $Q_{\mathrm{w}}=11.5\e4$\,\unit{J\,mol^{-1}} are constant parameters
    corresponding to values measured below and above a~critical temperature
    threshold $T_{\mathrm{c}}=-10$\,\unit{{\degree}C}
    \citep[p.~72]{Cuffey.Paterson.2010}. The water fraction, $\omega$, is
    capped at a~maximum value of 0.01, above which no measurements are
    available \citep[Eq.~5.7]{Lliboutry.Duval.1985, Greve.1997}. Finally, $E$
    is a~non-dimensional enhancement factor which can take different values,
    $E_{\text{SIA}}=2$, in the SIA and $E_{\text{SSA}}=1$, in the SSA, as
    recommended for Holocene polar ice \citep[p.~77]{Cuffey.Paterson.2010}.

    In all our simulations, we set constant the power-law exponent, $n=3$,
    according to \citet[p.~55--57]{Cuffey.Paterson.2010}, the Clapeyron
    constant, $\beta=7.9\times 10^{-8}$\,\unit{K\,Pa^{-1}}, according to
    \citet{Luthi.etal.2002}, and the water fraction coefficient, $f=181.25$,
    according to \citet{Lliboutry.Duval.1985}. These fixed parameter values are
    summarized in Table~\ref{tab:params}.

    Surface air temperature derived from the climate forcing
    (Sect.~\ref{sec:atm}) provides the upper boundary condition to the ice
    enthalpy model. Temperature is computed in the ice and in the bedrock to
    a~depth of 3\,\unit{km} below the ice-bedrock interface, where it is
    conditioned by a~lower boundary geothermal heat flux estimate from multiple
    geothermal proxies \citep[similarity method]{Goutorbe.etal.2011}. In the
    2\,km-resolution simulations, the vertical grid consists of 31~temperature
    layers in the bedrock and up to 126~enthalpy layers in the ice,
    corresponding to vertical resolutions of 100 and 40\,\unit{m},
    respectively. The 1\,km-resolution simulation uses 61~bedrock layers and up
    to 251~ice layers with a~vertical resolutions of 50 and 20\,\unit{m}.


% -- -- -- -- -- -- -- -- -- -- -- -- -- -- -- -- -- -- -- -- -- -- -- -
\subsection{Basal sliding}
\label{sec:sliding}
% -- -- -- -- -- -- -- -- -- -- -- -- -- -- -- -- -- -- -- -- -- -- -- -

    A~pseudo-plastic sliding law,
    %
    \begin{align}
      \vec{\tau}_{\mathrm{b}} = -\tau_{\mathrm{c}}
        \frac{\vec{v}_{\mathrm{b}}}
             {{v_{\text  {th}}}^q\,|\vec{v}_{\mathrm{b}}|^{1-q}} \,,
    \end{align}
    %
    relates the bed-parallel shear stresses, $\vec{\tau}_{\mathrm{b}}$, to the
    sliding velocity, $\vec{v}_{\mathrm{b}}$. The pseudo-plastic sliding
    exponent, $q=0.25$, and the threshold velocity,
    $v_{\text{th}}=100$\,\unit{m\,a^{-1}}, are set to values successfully used
    to model the Greenland ice sheet \citet{Aschwanden.etal.2013} The yield
    stress, $\tau_{\mathrm{c}}$, is modelled using the Mohr--Coulomb criterion,
    %
    \begin{align}
      \tau_{\mathrm{c}} = c_0 + N\,\tan{\phi} \,,
    \end{align}
    %
    where the till cohesion, $c_0=0$, was consistently measured to be
    negligible \citep[p.~268]{Tulaczyk.etal.2000, Cuffey.Paterson.2010}. We use
    a~constant till friction angle, $\phi=30$\unit{\degree}, corresponding to
    the average of values presented in \citet[p.~268]{Cuffey.Paterson.2010}.

    Effective pressure, $N$, is related to the ice overburden stress, $P_0=\rho
    gh$, and the modelled amount of subglacial water, using a~formula derived
    from laboratory experiments with till extracted from the base of Ice Stream
    B in West Antarctica \citep{Tulaczyk.etal.2000, Bueler.Pelt.2015},
    %
    \begin{align}
      N = \delta P_0 \, 10^{(e_0/C_{\mathrm{c}}) (1 - (W/W_{\text{max}}))} \,,
    \end{align}
    %
    where $\delta=0.02$ sets the minimum ratio between the effective and
    overburden pressures. Parameter values for the till reference void ratio,
    $e_0=0.69$, and the till compressibility coefficient,
    $C_{\mathrm{c}}=0.12$, were set to the only measurements available to our
    knowledge \citep{Tulaczyk.etal.2000}. The amount of water at the base, $W$,
    varies from zero to $W_{\text{max}}=2$\,m, a~threshold above which
    additional melt water is assumed to drain off instantaneously. These fixed
    parameter values are summarized in Table~\ref{tab:params}.


% -- -- -- -- -- -- -- -- -- -- -- -- -- -- -- -- -- -- -- -- -- -- -- -
\subsection{Basal topography}
\label{sec:bedrock}
% -- -- -- -- -- -- -- -- -- -- -- -- -- -- -- -- -- -- -- -- -- -- -- -

    The initial basal topography is bilinearly interpolated from the
    hole-filled, Shuttle Radar Topography Mission (SRTM) data with a~resolution
    of 30\,arc-sec \citep{Jarvis.etal.2008}. These data include post-glacial
    sediment fills and lakes surface topography. However, they were corrected
    for an estimation of present-day ice thickness according from modern
    glacier outlines, surface topography and simplified ice physics
    \citep{Huss.Farinotti.2012}.

    Basal topography responds to ice load following a~bedrock deformation model
    that includes local isostasy, elastic lithosphere flexure and viscous
    astenosphere deformation in an infinite half-space
    \citep{Lingle.Clark.1985,Bueler.etal.2007}. The astenosphere viscosity,
    $\nu_{\mathrm{m}}=2.2\times10^{20}$\,\unit{Pa\,s}, the astenosphere
    density, $\rho_{\mathrm{m}}=3300$\,\unit{kg\,m^{-3}}, and the lithosphere
    elastic rigidity, $D=1.389\e{24}$\,\unit{N\,m}, were set according to
    results from glacial isostatic adjustment modelling of deglacial rebound in
    the Alps most closely reproducing observed modern uplift rates
    \citep[Table~\ref{tab:params};][Supplementary Fig.~7]{Mey.etal.2016}. The
    latter was computed as $D=YE^3/12(1-\nu)$, where $Y=100$\,GPa is Young's
    modulus, $\nu=0.25$ is the Poisson ratio, and $E=50$\,km is the average
    effective elastic thickness of the lithosphere
    \citep[Table~\ref{tab:params};][]{Mey.etal.2016}.


% -- -- -- -- -- -- -- -- -- -- -- -- -- -- -- -- -- -- -- -- -- -- -- -
\subsection{Surface mass balance}
\label{sec:surface}
% -- -- -- -- -- -- -- -- -- -- -- -- -- -- -- -- -- -- -- -- -- -- -- -

    Ice surface accumulation and ablation are computed from monthly mean
    near-surface air temperature, $T_{\mathrm{m}}$, monthly standard deviation
    of near-surface air temperature, $\sigma$, and monthly precipitation,
    $P_{\mathrm{m}}$, using a~temperature-index model
    \citep[e.g.,][]{Hock.2003}. Accumulation is equal to precipitation when air
    temperatures are below 0\,\unit{{\degree}C}, and decreases to zero linearly
    with temperatures between 0 and 2\,\unit{{\degree}C}. Ablation is computed
    from PDD, defined as an integral of temperatures above 0\,\unit{{\degree}C}
    in one year.

    The PDD computation accounts for stochastic temperature variations by
    assuming a~normal temperature distribution of standard deviation $\sigma$
    around the expected value $T_{\mathrm{m}}$. It is expressed by an
    error-function formulation \citep{Calov.Greve.2005},
    %
    \begin{align}
      {\text{PDD}} = \int_{t_1}^{t_2} \mathrm{d}t
        \left[\frac{\sigma}{\sqrt{2\pi}}
                \exp\left({-\frac{T_{\mathrm{m}}^2}{2\sigma^2}}\right)
              + \frac{T_{\mathrm{m}}}{2} \, {\text{erfc}}
                \left(-\frac{T_{\mathrm{m}}}{\sqrt{2}\sigma}\right)\right] \,,
    \end{align}
    %
    which is numerically approximated using week-long sub-intervals. In
    order to account for the effects of spatial and seasonal variations of
    temperature variability \citep{Seguinot.2013}, $\sigma$ is computed
    from ERA-Interim daily temperature values from 1979 to 2012
    \citep{Mesinger.etal.2006}, including variability associated with the
    seasonal cycle \citep{Seguinot.2013}, and bilinearly interpolated to the
    model grids (Fig.~\ref{fig:inputs}). Degree-day factors for snow and ice
    melt are set to values used in the European Ice Sheet Modelling INiTiative
    \citep[Table~\ref{tab:params}; EISMINT,][]{Huybrechts.1998}.


% -- -- -- -- -- -- -- -- -- -- -- -- -- -- -- -- -- -- -- -- -- -- -- -
\subsection{Reference climate forcing}
\label{sec:atm}
% -- -- -- -- -- -- -- -- -- -- -- -- -- -- -- -- -- -- -- -- -- -- -- -

    Climate forcing driving ice sheet simulations consists of a~present-day
    monthly climatology, $\{T_{\mathrm{m}0}, P_{\mathrm{m}0}\}$, modified by
    temperature lapse-rate corrections, ${\Delta}T_{\text{LR}}$, temperature
    offset time series, ${\Delta}T_{\text{TS}}$, and time-dependent
    palaeo-precipitation corrections, $\Psi_{\text{PP}}$:
    %
    \begin{align}
      &T_{\mathrm{m}}(t, x, y) = T_{\mathrm{m}0}(x, y) +
                                 {\Delta}T_{\text{LR}}(t, x, y) +
                                 {\Delta}T_{\text{TS}}(t) \,, \\
      &P_{\mathrm{m}}(t, x, y) = P_{\mathrm{m}0}(x, y) \cdot
                                 {\Psi}_{\text{PP}}(t) \,, \\
    \end{align}
    %
    The present-day monthly climatology was bilinearly interpolated from
    near-surface air temperature and precipitation rate fields from
    \citep[WorldClim;][]{Hijmans.etal.2005}, representative of the period 1960
    to 1990. Modern climate of the European Alps is characterised by a
    latitudinal gradient in summer air temperatures, and a longitudinal
    gradient in winter precipitation (Fig.~\ref{fig:inputs}). WorldClim data
    were selected as an input to the ice sheet model because they incorporate
    observations from the dense weather station network of central Europe.
    Besides, last glacial cycle alpine glaciers did not extend over marine
    areas where WorldClim data is missing. Finally, WorldClim data were
    previously used as climate forcing for PISM to model the LGM extent of the
    former Cordilleran ice sheet in good agreement with geological evidence
    along the southern margin \citep{Seguinot.etal.2014} were weather station
    density is lower than in the Alps.

    The temperature lapse-rate corrections, ${\Delta}T_{\text{LR}}$, are
    computed as a~function of ice surface elevation, $s$, using the SRTM
    topography shipped with WorldClimas a~reference, $b_{\text{ref}}$:
    %
    \begin{align}
      {\Delta}T_{\text{LR}}(t, x, y) &= -\gamma [s(t, x, y)-b_{\text{ref}}] \\
                                     &= -\gamma [h(t, x, y)+
                                                 b(t, x, y)-b_{\text{ref}}],
    \end{align}
    %
    thus accounting for the evolution of ice thickness, ${h=s-b}$, on the one
    hand, and for differences between the basal topography of the ice flow
    model, $b$, and the NARR reference topography, $b_{\text{ref}}$, on the
    other hand. All simulations use an annual temperature lapse rate of
    $\gamma=6\,\unit{K\,km^{-1}}$.


% ----------------------------------------------------------------------
\section{Palaeo-climate forcing}
\label{sec:climate}
% ----------------------------------------------------------------------

    Table~\ref{tab:records} -- Palaeo-climate records.\\
    Fig.~\ref{fig:timeseries} -- Low-resolution time series.\\
    Fig.~\ref{fig:footprints} -- Low-resolution ice cover.\\

    In this section, we analyze the model sensitivity to palaeo-climate forcing
    through the last glacial cycle, using three palaeo-temperature records
    (Sect.~\ref{sec:paltemp}) and two parametrizations of palaeo-precipitation
    (Sect.~\ref{sec:palprec}), in terms of modelled evolution of total ice
    volume (Sect.~\ref{sec:timeseries}) and glaciated area during MIS~2 and~4
    (Sect.~\ref{sec:footprints}).

    These simulations use an horizontal resolution of 2\,km. The vertical grid
    consists of 31~temperature layers in the bedrock and up to 126~enthalpy
    layers in the ice, corresponding to vertical resolutions of 100 and
    40\,\unit{m}, respectively.


% -- -- -- -- -- -- -- -- -- -- -- -- -- -- -- -- -- -- -- -- -- -- -- -
\subsection{Palaeo-temperature forcing}
\label{sec:paltemp}
% -- -- -- -- -- -- -- -- -- -- -- -- -- -- -- -- -- -- -- -- -- -- -- -

    Temperature offset time-series, ${\Delta}T_{\text{TS}}$, are derived from
    palaeo-temperature proxy records from the Greenland Ice Core Project
    \citep[GRIP,][]{Dansgaard.etal.1993}, the European Project for Ice Coring
    in Antarctica \citep[EPICA,][] {Jouzel.etal.2007}, and an oceanic sediment
    core from the Iberian margin \citep[MD01-2444,][]{Martrat.etal.2007}.
    Palaeo-temperature anomalies from the GRIP record were calculated from
    oxygen isotope (\chem{\delta^{18}O}) measurements using a~quadratic
    equation \citep{Johnsen.etal.1995},
    %
    \begin{align}
      {\Delta}T_{\text{TS}}(t) ={~}&-11.88 [\chem{\delta^{18}O}(t)
                                    -\chem{\delta^{18}O}(0)] \nonumber \\
                                   &-0.1925 [\chem{\delta^{18}O}(t)^2
                                    -\chem{\delta^{18}O}(0)^2] \,,
    \end{align}
    %
    while temperature reconstructions from the EPICA and MD01-2444 records were
    provided as such. For each proxy record used and each of the parameter
    setup used in the sensitivity tests, palaeo-temperature anomalies are
    scaled linearly (Table~\ref{tab:records}) so that the cumulative glaciated
    area of the Rhine glacier piedmont lobe during Oxygen Marine Isotope Stage
    (MIS)~2 (29--14\,ka) is modelled comparably to the reconstructions
    (Fig.~\ref{fig:footprints}).


% -- -- -- -- -- -- -- -- -- -- -- -- -- -- -- -- -- -- -- -- -- -- -- -
\subsection{Palaeo-precipitation forcing}
\label{sec:palprec}
% -- -- -- -- -- -- -- -- -- -- -- -- -- -- -- -- -- -- -- -- -- -- -- -

    Finally, in some simulations, precipitation was reduced with temperature in
    order to simulate the potential rarification of atmospheric moisture in
    colder climates. This was done using an empirical relationship derived from
    observed accumulation rates and oxygen isotopes concentrations in the GRIP
    ice core \citep{Dahl-Jensen.etal.1993},
    %
    \begin{align}
      {\Psi}_{\text{PP}}(t) = \exp[\psi{\Delta}T_{\text{TS}}(t)] \,.
    \end{align}
    %
    with $\psi=0.169/2.4=0.0704$ \citep{Huybrechts.2002}. This simple
    relationship likely does not reflect the complexity of atmospheric
    circulation changes that governed moisture availability over the Alps
    during the last glacial cycle. Thus other simulations use constant
    precipitation, corresponding to $\psi=0$. In the rest of this paper, we
    refer to different model runs by the name of the proxy record used for the
    palaeo-temperature forcing. Model runs using paleo-precipitation
    corrections are labelled PP.


% -- -- -- -- -- -- -- -- -- -- -- -- -- -- -- -- -- -- -- -- -- -- -- -
\subsection{Sensitivity of ice volume evolution}
\label{sec:timeseries}
% -- -- -- -- -- -- -- -- -- -- -- -- -- -- -- -- -- -- -- -- -- -- -- -

    For the three palaeo-temperature records and the two palaeo-precipitation
    parametrizations used, the model yield significant ice volume build-up
    during MIS~4 and 2 (Fig.~\ref{fig:timeseries}). All simulations also
    yield important glaciations during MIS~5 and 3, but timing and amplitude
    varies a lot between the different forcing. All simulations overestimate
    ice cover during the Younger Dryas. This might be due partly to the 2\,km
    resolution, except for GRIP which largely overestimates Younger Dryas ice
    cover.

    Even more importantly, all simulations yield very strong ice volume
    variability which is the result of multiple glacier advance and retreats
    over the foreland, many more than two. Palaeo-precipitation reductions
    partly results in a smoother modelled ice volume time series and less
    extreme variations, but variability remains high. Ice volume variability
    is the smallest for the EPICA palaeo-temperature record, which has the
    least temperature variability.


% -- -- -- -- -- -- -- -- -- -- -- -- -- -- -- -- -- -- -- -- -- -- -- -
\subsection{Sensitivity of glaciated area}
\label{sec:footprints}
% -- -- -- -- -- -- -- -- -- -- -- -- -- -- -- -- -- -- -- -- -- -- -- -

    To select an optimal climate forcing, we look at the best known alpine
    glaciations during MIS~2 and 4. During MIS~2, all forcing yield ice extent
    of the same order of magnitude (Fig.~\ref{fig:footprints}). This is
    because the temperature records have been scaled to resul in similar
    ice covered area for the Rhine Glacier piedmont lobe. Looking at ice cover
    on other parts, all simulations tend to overestimate ice extent in the
    eastern part of the model domain, and to underestimate it in the western
    part, as compared to the reconstructed LGM margin. Regarding the western
    part, there are discussions though, as if the Lyon Lobe actually dates from
    MIS~2 or from an older glaciation. Regarding the eastern part, the MIS~2
    glaciation has been mapped in detail and dated (?), so that there is no
    doubt that our model overestimate ice cover in this region. This could
    indicate that during LGM the east-west precipitation gradient over the Alps
    was lower than today, or that temperature depression in the east was lower
    than in the west.

    There are spatial differences though: MD01-2444, and to a greater extent
    GRIP, forcings, tend to overestimate ice cover in all peripherical ranges:
    the Vosges, Black Forest, Bavarian Forest, Dinaric Alps and Pennine. This
    is because these records have a higher temperature variability so that
    after scaling they comport brief periods of cold climate, not long enough
    to develop the alpine ice cap, but long enough to build up ice on the
    peripherical ranges.

    The extent of ice cover during MIS~4 shows more sensitivity to the choice
    of palaeo-climate forcing. Using the GRIP and MD01-2444, glaciers extend
    well beyound the reach of documented moraines and glacial erratic terrain.
    Because this extent corresponds to a short-term cold climate event,
    palaeo-precipitation reductions help to greatly reduce this excessive
    modelled ice cover, yet in both cases, modelled ice extent and volume
    during MIS~4 remains higher than during MIS~2. EPICA forcing, on the other
    hand, yield a MIS~4 glaciation that is only slightly less expensive than
    the MIS~2, with only little sensitivity of the glaciated area to
    palaeo-precipitation reductions.

    Based on the above consideration on timing of the LGM ice extent and
    ice extent during MIS~4, we chose EPICA as our optimal temperature record
    for the rest of this paper. As a conservative approach in regard to quick
    ice volume fluctuations, we choose to include palaeo-precipitation
    correction in the following 1\,km-resolution simulation.

    \note{An important assumption in the method is that the MIS~4 glaciation
          was less (or equally) extensive than the MIS~2, i.e. that the LGM
          outline from \citet{Ehlers.etal.2011} dates indeed from the LGM
          (MIS~2). A different assumption could be that this outline represents
          the most extensive glaciation in the last 120\,ka. If we allow the
          MIS~2 glaciation to be a bit smaller than MIS~4, simulations driven
          by GRIP and MD01-2444 also give reasonable results.}


% ----------------------------------------------------------------------
\section{Glacier dynamics}
\label{sec:results}
% ----------------------------------------------------------------------

    Fig.~\ref{fig:lgmvel} -- Snapshot at 21\,ka and volume time series.\\
    Fig.~\ref{fig:timing} -- Timing of the LGM and area time series.\\
    Fig.~\ref{fig:profiles} -- Individual glacier extent profiles.

    In this section, we compare the model output to geological evidence from
    the last glacial cycle, in terms of LGM extent (Sect.~\ref{sec:extent}),
    ice flow patterns (Sect.~\ref{sec:flow}), ice thickness
    (Sect.~\ref{sec:thickness}), timing (Sect.~\ref{sec:timing}), and the
    number of glaciations (Sect.~\ref{sec:glaciations}).

    This simulation is forced by the optimal EPICA palaeo-temperature record
    (Sect.~\ref{sec:paltemp}) and includes palaeo-precipitation reduction
    (Sect.~\ref{sec:palprec}). It uses an horizontal resolution of 1\,km. The
    vertical grid consists of 61~temperature layers in the bedrock and up to
    251~enthalpy layers in the ice, corresponding to vertical resolutions of 50
    and 20\,\unit{m}, respectively.


% -- -- -- -- -- -- -- -- -- -- -- -- -- -- -- -- -- -- -- -- -- -- -- -
\subsection{LGM extent}
\label{sec:extent}
% -- -- -- -- -- -- -- -- -- -- -- -- -- -- -- -- -- -- -- -- -- -- -- -

    \begin{itemize}
    \item Rapid variations of total ice volume (Fig.~\ref{fig:lgmvel}b).
    \item Two major glaciations during MIS 4 and 2 (Fig.~\ref{fig:lgmvel}b).
    \item Maximum extent is farily well reproduced (Fig.~\ref{fig:lgmvel}a).
    \end{itemize}


% -- -- -- -- -- -- -- -- -- -- -- -- -- -- -- -- -- -- -- -- -- -- -- -
\subsection{LGM flow pattern}
\label{sec:flow}
% -- -- -- -- -- -- -- -- -- -- -- -- -- -- -- -- -- -- -- -- -- -- -- -

    \begin{itemize}
    \item Maximum stages flow pattern is complex (Fig.~\ref{fig:lgmvel}a).
    \end{itemize}

    \idea{Add trimline figure in the main text now.}


% -- -- -- -- -- -- -- -- -- -- -- -- -- -- -- -- -- -- -- -- -- -- -- -
\subsection{LGM ice thickness}
\label{sec:thickness}
% -- -- -- -- -- -- -- -- -- -- -- -- -- -- -- -- -- -- -- -- -- -- -- -

    \begin{itemize}
    \item Ice surface is well above the trimlines (Fig.~\ref{fig:lgmvel}a).
    \end{itemize}


% -- -- -- -- -- -- -- -- -- -- -- -- -- -- -- -- -- -- -- -- -- -- -- -
\subsection{Timing of the LGM}
\label{sec:timing}
% -- -- -- -- -- -- -- -- -- -- -- -- -- -- -- -- -- -- -- -- -- -- -- -

    \begin{itemize}
    \item Maximum extent is time-transgressive (Fig.~\ref{fig:timing}).
    \item Glaciers have different response times (Fig.~\ref{fig:timing}).
    \end{itemize}


% -- -- -- -- -- -- -- -- -- -- -- -- -- -- -- -- -- -- -- -- -- -- -- -
\subsection{Number of glaciations}
\label{sec:glaciations}
% -- -- -- -- -- -- -- -- -- -- -- -- -- -- -- -- -- -- -- -- -- -- -- -

    \begin{itemize}
    \item Some glaciers advance many times (Fig.~\ref{fig:profiles}).
    \end{itemize}

    \idea{Matthias: The selection of figures in the main text and SOM is good.
          I find Figure 5 quite difficult to understand at first glance. I
          would combine it (already before submission) with Fig. 6. the Figure
          can then focus both on the temporal advance/retreat patterns of the 4
          lobes (four panels on the left side of figure) and the spatial
          differences in max. ice elevation (1 panel on right side of figure,
          with centerlines indicated). I would have the one with trimline
          elevation again in the Supplementary.}


% ----------------------------------------------------------------------
\section{Erosion potential}
\label{sec:erosion}
% ----------------------------------------------------------------------

    Fig.~\ref{fig:erosion} -- Integrated erosion potential.

    \idea{Frank: I would keep this topic for another paper.}

    \begin{itemize}
    \item We calculate erosion after \citet{Herman.etal.2015}
    \item More erosion in valleys of western Alps (Fig.~\ref{fig:erosion}a).
    \item Erosion is constant during 110--15\,ka (Fig.~\ref{fig:erosion}b).
    \end{itemize}


% ----------------------------------------------------------------------
\section{Conclusions}
% ----------------------------------------------------------------------

    \begin{itemize}
    \item Summary of the previous stuff.
    \end{itemize}


% ----------------------------------------------------------------------
% References
% ----------------------------------------------------------------------

\bibliographystyle{abbrvnat}
\bibliography{../../references/references}


% ----------------------------------------------------------------------
% Figures
\clearpage
% ----------------------------------------------------------------------

    \begin{figure}
      \centerline{\includegraphics{alpcyc_hr_inputs}}
      \caption{%
        \textbf{(a)} July mean near-surface air temperature and
        \textbf{(b)} January precipitation from WorldClim
        \citep[1960--1990]{Hijmans.etal.2005}, and
        \textbf{(c)} Mordern July standard deviation of daily mean temperature
        from the ERA-Interim \citep[1979--2012]{Dee.etal.2011} from the
        reference monthly climatology used to force the surface mass balance
        (PDD) component of the ice sheet model.
        \textbf{(d)} Geothermal heat flow from applying the similarity method
        to multiple geophysical proxies \citep{Goutorbe.etal.2011} used as a
        boundary condition to the bedrock thermal model 3\,km below the
        ice-bedrock interface.}
      \label{fig:inputs}
    \end{figure}

    \begin{figure}
      \centerline{\includegraphics{alpcyc_lr_timeseries}}
      \caption{%
        \textbf{(a)} Temperature offset time-series from ice core and ocean
        records (Table~\ref{tab:records}) used as palaeo-climate forcing for
        the ice sheet model.
        \textbf{(a)} Modelled total ice volume through the last 120\,ka,
        expressed in meters of sea level equivalent (m~s.l.e.). Gray fields
        indicate Marine Oxygen Isotope Stage (MIS) boundaries for MIS~2 and
        MIS~4 according to a~global compilation of benthic \chem{\delta^{18}O}
        records \citep{Lisiecki.Raymo.2005}.}
      \label{fig:timeseries}
    \end{figure}

    \begin{figure}
      \centerline{\includegraphics{alpcyc_lr_footprints}}
      \caption{%
        \textbf{(a--c)} Cumulative extent of modelled ice cover during MIS~2
        (29--14\,ka) using temperature time-series scaling factors
        (Table~\ref{tab:records}) adjusted to obtain model results in agreement
        with the Last Glacial Maximum (LGM) geomorphological reconstruction
        \citep[solid red line,][]{Ehlers.etal.2011}.
        \textbf{(d--f)} Cumulative extent of modelled ice cover during MIS~4
        (71--57\,ka). Only the simulation driven by the EPICA temperature
        time-series yields reasonable MIS~4 ice cover.
        \todo{Try to make reduced precipitation outlines more visible.}}
      \label{fig:footprints}
    \end{figure}

    \begin{figure}
      \centerline{\includegraphics{alpcyc_hr_lgmvel}}
      \caption{%
        \textbf{(a)} Modelled bedrock topography (grey) ice surface topography
        (200\,m contours) and ice surface velocity (blue) in the Alps
        21~thousand years (ka) before present. Modelled Last Glacial Maximum
        (LGM) ice extent (dashed orange line) and geomorphological
        reconstruction \citep[solid red line,][]{Ehlers.etal.2011}. The
        background map consists of depressed SRTM \citep{Jarvis.etal.2008}
        topography and Natural Earth Data \citep{Patterson.Kelso.2017}.
        \textbf{(b)} Temperature offset time-series from the EPICA ice core
        used as palaeo-climate forcing for the ice flow model \citep[black
        curve,][]{Jouzel.etal.2007}, and modelled total ice volume through the
        last glacial cycle (120--0\,ka), expressed in meters of sea level
        equivalent (m~s.l.e., blue curve). Gray fields indicate Marine
        Oxygen Isotope Stage (MIS) boundaries for MIS~2 and MIS~4 according to
        a~global compilation of benthic \chem{\delta^{18}O} records
        \citep{Lisiecki.Raymo.2005}.}
      \label{fig:lgmvel}
    \end{figure}

    \begin{figure}
      \centerline{\includegraphics{alpcyc_hr_timing}}
      \caption{%
        \textbf{(a)} Timing of the Last Glacial Maximum (LGM) given by the
        modelled age (colour mapping) and value (200\,m contours) of maximum
        surface elevation throughout the entire simulation.
        \textbf{(b)} Temperature offset time-series from the EPICA ice core
        used as palaeo-climate forcing for the ice flow model (black curve),
        and modelled glaciated area around the LGM (coloured curve). The LGM
        is here modelled as a time-transgressive event.}
      \label{fig:timing}
    \end{figure}

    \begin{figure}
      \centerline{\includegraphics{alpcyc_hr_trimlines}}
      \caption{%
        \textbf{(a)} Modelled age of maximum ice thickness (colour) and maximum
        ice surface elevation (200\,m contours) above trimline locations
        \citep[Table~1]{Kelly.etal.2004} in the upper Rhone valley.
        \textbf{(b)} Modelled maximum ice thickness against observed trimline
        elevations \citep[Table~1]{Kelly.etal.2004}.
        \textbf{(c)} Histogram of modelled maximum ice thickness above trimline
        locations (500\,m bands). The average maximum ice thickness above the
        trimline is 979\,m.}
      \label{fig:trimlines}
    \end{figure}

    \begin{figure}
      \centerline{\includegraphics{alpcyc_hr_profiles}}
      \caption{%
        Modelled extent of glaciation along selected profiles for the Lyon,
        Solothurn, Rhine and Ivrea glaciers.
        \todo{Implement extraction of ice thickness along a more detailed
              profile, and add the other glaciers.}
        \idea{We can also move this as panel b on Fig.~\ref{fig:timing}, with
              profiles drawn on panel a. But then we should probably limit it
              to two glaciers instead of four.}}
      \label{fig:profiles}
    \end{figure}

    \begin{figure}
      \centerline{\includegraphics{alpcyc_hr_erosion}}
      \caption{%
        \textbf{(a)} Modelled total erosion integrated in time over the entire
        simulation (120--0\,ka) is highest in the deep valleys and cirques of
        the western Alps.
        \textbf{(b)} Temperature offset time-series from the EPICA ice core
        used as palaeo-climate forcing for the ice flow model (black curve),
        and modelled erosion rate integrated in space over the entire
        ice-covered area. The local erosion rate, $\dot{e}$, is computed from
        the sliding velocity, $\vec{v}_{\mathrm{b}}$, through
        $\dot{e} = K_g \cdot |\vec{v}_{\mathrm{b}}|^{l}$, with
        $l = 2.02$ and $K_g = 2.7\e{-7}\,m^{1-l}\,a^{l-1}$
        \citep{Herman.etal.2015}.}
      \label{fig:erosion}
    \end{figure}


% ----------------------------------------------------------------------
% Tables
\clearpage
% ----------------------------------------------------------------------

    \begin{table*}
      \caption{%
        Parameter values used in the ice sheet model.
        \todo{Update parameters to new configuration.}}
      \label{tab:params}
      \noindent\small\makebox[\textwidth]
      {\begin{tabular}{llrll}
        \toprule

        Not.    & Name & Value & Unit & Source \\

        \midrule
        \multicolumn{2}{l}{{Ice rheology}} \\
        \midrule

        $\rho$  & Ice density
                & 910
                & \unit{kg\,m^{-3}}
                & \citet{Aschwanden.etal.2012} \\

        $g$     & Standard gravity
                & 9.81
                & \unit{m\,s^{-2}}
                & \citet{Aschwanden.etal.2012} \\

        $n$     & Glen exponent
                & 3
                & --
                & \citet{Cuffey.Paterson.2010} \\

        $A_{\mathrm{c}}$   & Ice hardness coefficient cold
                & $2.847\e{-13}$
                & \unit{Pa^{-3}\,s^{-1}}
                & \citet{Cuffey.Paterson.2010} \\

        $A_{\mathrm{w}}$   & Ice hardness coefficient warm
                & $2.356\e{-2}$
                & \unit{Pa^{-3}\,s^{-1}}
                & \citet{Cuffey.Paterson.2010} \\

        $Q_{\mathrm{c}}$   & Flow law activation energy cold
                & $6.0\e4$
                & \unit{J\,mol^{-1}}
                & \citet{Cuffey.Paterson.2010} \\

        $Q_{\mathrm{w}}$   & Flow law activation energy warm
                & $11.5\e4$
                & \unit{J\,mol^{-1}}
                & \citet{Cuffey.Paterson.2010} \\

        $E_{\text{SIA}}$   & SIA enhancement factor
                & 2
                & --
                & \citet{Cuffey.Paterson.2010} \\

        $E_{\text{SSA}}$   & SSA enhancement factor
                & 1
                & --
                & \citet{Cuffey.Paterson.2010} \\

        $T_{\mathrm{c}}$   & Flow law critical temperature
                & 263.15
                & \unit{K}
                & \citet{Paterson.Budd.1982} \\

        $f$     & Flow law water fraction coeff.
                & 181.25
                & --
                & \citet{Lliboutry.Duval.1985} \\

        $R$     & Ideal gas constant
                & 8.31441
                & \unit{J\,mol^{-1}\,K^{-1}}
                & -- \\

        $\beta$ & Clapeyron constant
                & $7.9\e{-8}$
                & \unit{K\,Pa^{-1}}
                & \citet{Luthi.etal.2002} \\

        $c_{\mathrm{i}}$   & Ice specific heat capacity
                & 2009
                & \unit{J\,kg^{-1}\,K^{-1}}
                & \citet{Aschwanden.etal.2012} \\

        $c_{\mathrm{w}}$   & Water specific heat capacity
                & 4170
                & \unit{J\,kg^{-1}\,K^{-1}}
                & \citet{Aschwanden.etal.2012} \\

        $k$     & Ice thermal conductivity
                & 2.10
                & \unit{J\,m^{-1}\,K^{-1}\,s^{-1}}
                & \citet{Aschwanden.etal.2012} \\

        $L$     & Water latent heat of fusion
                & $3.34\e5$
                & \unit{J\,kg^{-1}\,K^{-1}}
                & \citet{Aschwanden.etal.2012} \\

        \midrule
        \multicolumn{2}{l}{{Basal sliding}} \\
        \midrule

        $q$     & Pseudo-plastic sliding exponent
                & 0.25
                & --
                & \citet{Aschwanden.etal.2013} \\

        $v_{\text{th}}$& Pseudo-plastic threshold velocity
                & 100
                & \unit{m\,a^{-1}}
                & \citet{Aschwanden.etal.2013} \\

        $c_0$   & Till cohesion
                & 0
                & Pa
                & \citet{Tulaczyk.etal.2000} \\

        $e_0$   & Till reference void ratio
                & 0.69
                & --
                & \citet{Tulaczyk.etal.2000} \\

        $C_{\mathrm{c}}$   & Till compressibility coefficient
                & 0.12
                & --
                & \citet{Tulaczyk.etal.2000} \\

        $\delta$& Minimum effective pressure ratio
                & 0.02
                & --
                & \citet{Bueler.Pelt.2015} \\

        $\phi$  & Till friction angle
                & 30
                & \degree
                & \citet{Cuffey.Paterson.2010} \\

        $W_{\text{max}}$ & Maximum till water thickness
                & 2
                & m
                & \citet{Bueler.Pelt.2015} \\

        \midrule
        \multicolumn{2}{l}{{Bedrock and lithosphere}} \\
        \midrule

        $\rho_{\mathrm{b}}$& Bedrock density
                & 3300
                & \unit{kg\,m^{-3}}
                & -- \\

        $c_{\mathrm{b}}$   & Bedrock specific heat capacity
                & 1000
                & \unit{J\,kg^{-1}\,K^{-1}}
                & -- \\

        $k_{\mathrm{b}}$   & Bedrock thermal conductivity
                & 3
                & \unit{J\,m^{-1}\,K^{-1}\,s^{-1}}
                & -- \\

        $\nu_{\mathrm{m}}$ & Astenosphere viscosity
                & $2.2\e{20}$
                & \unit{Pa\,s}
                & \citet{Mey.etal.2016} \\

        $\rho_{\mathrm{m}}$& Astenosphere density
                & 3300
                & \unit{kg\,m^{-3}}
                & \citet{Mey.etal.2016} \\

        $D$     & Lithosphere flexural rigidity
                & $1.389\e{24}$
                & \unit{N\,m}
                & \citet{Mey.etal.2016} \\

        \midrule
        \multicolumn{2}{l}{{Surface and atmosphere}} \\
        \midrule

        $T_{\mathrm{s}}$   & Temperature of snow precipitation
                & 273.15
                & \unit{K}
                & -- \\

        $T_{\mathrm{r}}$   & Temperature of rain precipitation
                & 275.15
                & \unit{K}
                & -- \\

        $F_{\mathrm{s}}$   & Degree-day factor for snow
                & $3.297\e{-3}$
                & \unit{m\,K^{-1}\,day^{-1}}
                & \citet{Huybrechts.1998} \\

        $F_{\mathrm{i}}$   & Degree-day factor for ice
                & $8.791\e{-3}$
                & \unit{m\,K^{-1}\,day^{-1}}
                & \citet{Huybrechts.1998} \\

        $R$     & Refreezing fraction
                & 0.0
                & --
                & -- \\

        $\gamma$& Air temperature lapse rate
                & $6\e{-3}$
                & \unit{K\,m{-1}}
                & -- \\

        $\psi$  & Precipitation factor
                & 0.0704
                & --
                & \citet{Huybrechts.2002} \\

        \bottomrule
      \end{tabular}}
    \end{table*}

    \begin{table*}
      \caption{%
        Palaeo-temperature proxy records and scaling factors yielding
        temperature offset time-series used to force the ice sheet model
        through the last glacial cycle (Fig.~\ref{fig:timeseries}). $f$
        corresponds to the scaling factor adopted to yield Last Glacial Maximum
        ice limits in the vicinity of mapped end moraines
        (Fig.~\ref{fig:footprints}a), and $[{\Delta}T_{\textrm{TS}}]_{32}^{22}$
        refers to the resulting mean temperature anomaly during the period 32
        to~22\,\unit{ka} after scaling.}
      \label{tab:records}
      \noindent\small\makebox[\textwidth]
      {\begin{tabular}{lccccccl}
        \toprule

        Forcing   & Latitude & Longitude & Elev. (m~a.s.l.)
                  & Proxy & $f$ & $[{\Delta}\text{TS}]_{32}^{22}$ (K)
                  & Reference\\

        \midrule

        GRIP      & \multirow{2}{*}{ 72{\degree}35$^{\prime}$\,N}   % 72.58 (decimal)
                  & \multirow{2}{*}{ 37{\degree}38$^{\prime}$\,W}   % 37.64 (decimal)
                  & \multirow{2}{*}{3238}
                  & \multirow{2}{*}{\chem{\delta^{18}O}}
                  & 0.50 & $-$8.2  % -16.4126 (before scaling)
                  & \multirow{2}{*}{\citet{Dansgaard.etal.1993}} \\

        GRIP, $\Delta P$ &&&&& 0.63 & $-$10.4 \\

        EPICA     & \multirow{2}{*}{ 75{\degree}06$^{\prime}$\,S}   % 75.1
                  & \multirow{2}{*}{123{\degree}21$^{\prime}$\,E}   % 123.35
                  & \multirow{2}{*}{3233}
                  & \multirow{2}{*}{\chem{\delta^{18}O}}
                  & 1.05 & $-$9.7  % -9.2055
                  & \multirow{2}{*}{\citet{Jouzel.etal.2007}} \\

        EPICA, $\Delta P$ &&&&& 1.33 & $-$12.2 \\

        MD01-2444 & \multirow{2}{*}{ 37{\degree}34$^{\prime}$\,N}   % 37.561
                  & \multirow{2}{*}{ 10{\degree}04$^{\prime}$\,W}   % -10.142
                  & \multirow{2}{*}{$-$2637}
                  & \multirow{2}{*}{\chem{U^{K'}_{37}}}
                  & 1.84 & $-$8.0  % -4.345625
                  & \multirow{2}{*}{\citet{Martrat.etal.2007}} \\

        MD01-2444, $\Delta P$ &&&&& 2.44 & $-$10.6 \\

        \bottomrule
      \end{tabular}}

      %\noindent\small\makebox[\textwidth]
      %{\begin{tabular}{llrlrlr}
      %  \toprule
      %
      %  Record    & \multicolumn{2}{c}{GRIP}
      %            & \multicolumn{2}{c}{EPICA}
      %            & \multicolumn{2}{c}{MD01-2444} \\
      %
      %  Precip.   & cst. & $\Delta P$
      %            & cst. & $\Delta P$
      %            & cst. & $\Delta P$ \\
      %
      %  \midrule
      %
      %  Latitude  & \multicolumn{2}{c}{ 72{\degree}35$^{\prime}$\,N}   % 72.58 (decimal)
      %            & \multicolumn{2}{c}{ 75{\degree}06$^{\prime}$\,S}   % 75.1
      %            & \multicolumn{2}{c}{ 37{\degree}34$^{\prime}$\,N} \\  % 37.561
      %
      %  Longitude & \multicolumn{2}{c}{ 37{\degree}38$^{\prime}$\,W}   % 37.64 (decimal)
      %            & \multicolumn{2}{c}{123{\degree}21$^{\prime}$\,E}   % 123.35
      %            & \multicolumn{2}{c}{ 10{\degree}04$^{\prime}$\,W} \\  % -10.142
      %
      %  Elevation & \multicolumn{2}{c}{3238\,m~a.s.l.}
      %            & \multicolumn{2}{c}{3233\,m~a.s.l.}
      %            & \multicolumn{2}{c}{$-$2637\,m~a.s.l.} \\
      %
      %  Proxy     & \multicolumn{2}{c}{\chem{\delta^{18}O}}
      %            & \multicolumn{2}{c}{\chem{\delta^{18}O}}
      %            & \multicolumn{2}{c}{\chem{U^{K'}_{37}}} \\
      %
      %  $f$       & 0.50 & 0.63
      %            & 1.05 & 1.33
      %            & 1.84 & 2.44 \\
      %
      %  $[{\Delta}\text{TS}]_{32}^{22}$
      %            & $-$8.2\,K & $-$10.4\,K     % -16.4126 (before scaling)
      %            & $-$9.7\,K & $-$12.2\,K     % -9.2055
      %            & $-$8.0\,K & $-$10.6\,K \\  % -4.345625
      %
      %  Reference & \multicolumn{2}{c}{\citet{Dansgaard.etal.1993}}
      %            & \multicolumn{2}{c}{\citet{Jouzel.etal.2007}}
      %            & \multicolumn{2}{c}{\citet{Martrat.etal.2007}} \\
      %
      %  \bottomrule
      %\end{tabular}}

    \end{table*}


% ======================================================================
\end{document}
% ======================================================================
