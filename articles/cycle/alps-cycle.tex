\documentclass[tc, manuscript]{copernicus}

\graphicspath{{../../figures/}}

\definecolor{darkblue}{cmyk}{0.9,0.3,0.0,0.0}
\definecolor{darkgreen}{cmyk}{0.8,0.0,1.0,0.0}
\definecolor{darkred}{cmyk}{0.1,0.9,0.8,0.0}
\definecolor{darkorange}{cmyk}{0.0,0.5,1.0,0.0}
\definecolor{darkpurple}{cmyk}{0.6,0.7,0.0,0.0}
\definecolor{darkbrown}{cmyk}{0.23,0.73,0.98,0.12}

\newcommand{\idea}[1]{\textcolor{darkgreen}{\emph{[\textbf{IDEA:} #1]}}}
\newcommand{\note}[1]{\textcolor{darkblue}{\emph{[\textbf{NOTE:} #1]}}}
\newcommand{\todo}[1]{\textcolor{darkred}{\emph{[\textbf{TODO:} #1]}}}
\newcommand{\aref}[0]{\textcolor{darkblue}{\textbf{[REF.]}}}

\hypersetup{colorlinks, linkcolor=darkorange,
            urlcolor=darkbrown, citecolor=darkpurple}

\title{Modelling last glacial cycle ice dynamics in the Alps}

\Author[1]{Julien}{Seguinot}
\Author[1]{Guillaume}{Jouvet}
\Author[1]{Matthias}{Huss}
\Author[1]{Martin}{Funk}
\Author[2]{Susan}{Ivy-Ochs}
\Author[3]{Frank}{Preusser}

\affil[1]{Laboratory of Hydraulics, Hydrology and Glaciology,
          ETH Zürich, Switzerland}
\affil[2]{Laboratory of Ion Beam Physics, ETH Zürich, Switzerland}
\affil[3]{Institute of Earth and Environmental Sciences,
          University of Freiburg, Germany}

\runningtitle{Modelling last glacial cycle ice dynamics in the Alps}
\runningauthor{J.~Seguinot et al.}
\correspondence{J.~Seguinot (seguinot@vaw.baug.ethz.ch)}

\received{}
\pubdiscuss{}
\revised{}
\accepted{}
\published{}


% ======================================================================
\begin{document}
% ======================================================================

\firstpage{1}

\maketitle

\begin{abstract}

    The European Alps, cradle of pioneer glacial studies, are one of the
    regions where geological markers of past glaciations are most abundant and
    well-studied. Such conditions make the region ideal for testing numerical
    glacier models based on approximated ice flow physics against field-based
    reconstructions, and vice-versa.

    Here, we use the Parallel Ice Sheet Model (PISM) to model the entire last
    glacial cycle (120--0\,ka) in the Alps, with a horizontal resolution of
    1\,km. Climate forcing is derived using present-day climate data from
    WorldClim and the ERA-Interim reanalysis, and time-dependent temperature
    offsets from multiple paleo-climate proxies, among which only the EPICA ice
    core record yields glaciation during marine oxygen isotope stages~4
    (69--62\,ka) and~2 (34--18\,ka) spatially and temporally consistent with
    the geological reconstructions.

    Despite the low variability of this Antarctic-based climate forcing, our
    simulation depicts a highly dynamic ice sheet, showing that alpine glaciers
    may have advanced many times over the foreland during the last glacial
    cycle. Ice flow patterns during peak glaciation are largely governed by
    subglacial topography but include occasional transfluences and
    self-sustained ice domes. Finally, the Last
    Glacial Maximum advance, often considered synchronous, is here modelled as
    a time-transgressive event, with some glacier lobes reaching their maximum
    as early as 27\,ka, and some as late as 21\,ka. Modelled ice thickness is
    about 900\,m higher than observed trimline elevations, yet our simulation
    predicts little erosion at high elevation due to cold ice conditions.

\end{abstract}

%\copyrightstatement{}


% ----------------------------------------------------------------------
\introduction
\label{sec:intro}
% ----------------------------------------------------------------------

    For nearly 300~years, montane people and early explorers of the European
    Alps learned to read the geomorphological imprint left by glaciers in the
    landscape, and to understand that glaciers had once been more extensive
    than today \citep[e.g.,][p.~21]{Windham.Martel.1744}. Contemporaneously,
    it was also observed in the Alps that glaciers move by a combination of
    meltwater-induced \emph{sliding} at the base \citep[\S532]{Saussure.1779},
    and viscous \emph{deformation} within the ice body \citep{Forbes.1846b}. As
    glaciers flow and slide across their bed, they transport rock debris and
    erode the landscape, thereby leaving geomorphological traces of their
    former presence. In the mid-nineteenth
    century, more systematic studies of glacial features showed that alpine
    glaciers extended well below their contemporary margins \citep{Venetz.1821}
    and even onto the alpine foreland \citep{Charpentier.1841}, yielding the
    idea that, under colder temperatures, expansive \emph{ice sheets} had once
    covered much of Europe and North America \citep{Agassiz.1840}.

    However, this glacial theory did not gain general acceptance until the
    discovery and exploration of the two present-day ice sheets on Earth, the
    Greenland and Antarctic ice sheets, provided a modern analogue for the
    proposed European and North American ice sheets. Although it was long
    unclear whether there had been a single or multiple \emph{glaciations},
    this controversy ended with the large-scale mapping of two distinct moraine
    systems in North America \citep{Chamberlin.1894}. In the European Alps, the
    systematic classification of the glaciofluvial terraces on the northern
    foreland later indicated that there had been at least four major
    glaciations in the Alps \citep{Penck.Bruckner.1909}. More recent studies
    of glaciofluvial stratigraphy in the foreland indicate up to eight
    glaciations \citep{Ivy-Ochs.etal.2008, Preusser.etal.2011}.

    From the mid-twentieth
    century, palaeoclimate records extracted from deep sea sediments and
    ice cores have provided a much more detailed picture of the Earth
    environmental history \citep[e.g.,][]{Emiliani.1955,
    Shackleton.Opdyke.1973, Dansgaard.etal.1993, Augustin.etal.2004},
    indicating neither one, nor four, but rather tens of glacial and
    interglacial periods. During the last 800000 years (800\,ka), these
    \emph{glacial cycles}
    have succeeded each other with a 100\,ka periodicity \citep{Hays.etal.1976,
    Augustin.etal.2004}. Nevertheless, this global signal is largely governed
    by the North American and Eurasian ice sheet complexes. It is thus unclear
    whether glacier advances and retreats in the Alps were in pace with global
    sea-level fluctuations.

    Besides, the landform record is typically sparse in time and, most often,
    spatially incomplete. Palaeo-ice sheets did not leave a continuous imprint
    on the lanscape, and much of this evidence has been overprinted by
    subsequent glacier re-advances and other geomorphological processes
    \citep[e.g.,][]{Kleman.1994, Kleman.etal.2006, Kleman.etal.2010}. Dating
    uncertainties typically increase with age, such that dated reconstructions
    are strongly biased towards earlier glaciations \citep{Heyman.etal.2011}.
    The European Alps have been studied in more detail than any other glacial
    region, but they are no exception to these rules. Although sparse
    geological traces indicate that the last glacial cycle may have comprised
    two or three cycles of glacier growth and decay \citep{Preusser.2004,
    Ivy-Ochs.etal.2008}, most glacial features currently left on the foreland
    present a record of the last major glaciation of the Alps, dating from the
    \emph{Last Glacial Maximum} (LGM).

    The spatial extent and thickness of Alpine glaciers during the LGM, an
    integrated footprint of thousands of years of climate history and glacier
    dynamics, have been reconstructed from moraines and trimlines across the
    mountain range \citep[e.g.,][]{Bini.etal.2009, Coutterand.2010,
    Husen.2011}. The assumption that trimlines, the upper limit of glacial
    erosion, represent the maximum ice surface elevation, has
    been repeatedly invalidated in other glaciated regions of the globe
    \citep[e.g.,][]{Kleman.1994, Kleman.etal.2010, Fabel.etal.2012}. Yet no
    evidence for cold-based glaciation has been reported in the Alps to date.
    Alpine ice flow
    patterns were primarily controlled by subglacial topography, but there is
    evidence for flow accross major mountain passes
    \citep[e.g.,][]{Coutterand.2010, Kelly.etal.2004, Husen.2011}, and
    self-sustained ice domes \citep{Bini.etal.2009}. Finally, the timing of the
    LGM in the Alps \citep{Ivy-Ochs.etal.2008, Monegato.etal.2017} is in good
    agreement with the maximum expansion of continental ice sheets recorded by
    the Marine Oxygen Isotope Stage (MIS) 2
    \citep[29--14\,ka;][]{Lisiecki.Raymo.2005}. Regional variation between
    different piedmont
    lobes exist \citep[Fig.~5]{Wirsig.etal.2016}, but it is unclear wether they
    relate to climate or glacier dynamics \citep{Monegato.etal.2017} or
    uncertainties in the dating methods.

    Although the glacial history of the European Alps has been studied for
    nearly three hundred years, it thus remains incompletely known
    \begin{itemize}
      \item what climate evolution lead to the known maximum ice limits,
      \item to which extent ice flow was controlled by subglacial topography,
      \item what drove the different response of the individual lobes,
      \item how far above the trimline was the ice surface located, and
      \item how many advances occurred during the last glacial cycle.
    \end{itemize}

    Here, we use the Parallel Ice Sheet Model
    \citep[PISM,][]{PISM-authors.2017}, a numerical ice sheet model that
    approximates glacier sliding and deformation (Sect.~\ref{sec:model}), to
    model alpine glacier dynamics through the last glacial cycle (120--0\,ka),
    a period for which palaeo-temperature proxies are available, albeit non
    regional. In an attempt to analyse the long-standing questions outline
    above, we test the model sensitivity to multiple palaeo-climate forcing
    (Sect.~\ref{sec:climate}), and then explore the modelled glacier dynamics
    at high resolution for the optimal forcing (Sect.~\ref{sec:results}).


% ----------------------------------------------------------------------
\section{Ice sheet model set-up}
\label{sec:model}
% ----------------------------------------------------------------------

    \note{This section contains\\
        Table~\ref{tab:params} -- Model parameters.\\
        Fig.~\ref{fig:inputs} -- Climate and geothermal model inputs.}


% -- -- -- -- -- -- -- -- -- -- -- -- -- -- -- -- -- -- -- -- -- -- -- -
\subsection{Overview}
\label{sec:overview}
% -- -- -- -- -- -- -- -- -- -- -- -- -- -- -- -- -- -- -- -- -- -- -- -

    We use the Parallel Ice Sheet Model (PISM, development version~e9d2d1f), an
    open source, finite difference, shallow ice sheet model
    \citep{PISM-authors.2017}. The model requires input on basal
    topography, geothermal heat flux and climate forcing. It computes the
    evolution of ice extent and thickness over time, the thermal and dynamic
    states of the ice sheet, and the associated lithospheric response. The
    model set-up used here was previously developed and tested on the former
    Cordilleran ice sheet \citep{Seguinot.2014, Seguinot.etal.2014,
    Seguinot.etal.2016} and subsequently adapted for steady climate
    \citep{Becker.etal.2016} and regional \citep{Jouvet.etal.2017a}
    applications to the European Alps.

    Ice deformation follows temperature and water-content dependent creep
    (Sect.~\ref{sec:icedyn}). Basal sliding follows a~pseudo-plastic law where
    the yield stress accounts for till deconsolidation under high water
    pressure (Sect.~\ref{sec:sliding}). Bedrock topography is deflected
    under the ice load (Sect.~\ref{sec:bedrock}). Surface mass balance is
    computed using a~positive degree-day (PDD) model (Sect.~\ref{sec:surface}).
    Climate forcing is provided by a~monthly climatology from interpolated
    observational data \citep[WorldClim;][]{Hijmans.etal.2005} and the European
    Centre for Medium-Range Weather Forecasts Reanalysis Interim
    \citep[ERA-Interim;][]{Dee.etal.2011} over the period 1979--2012,
    perturbed by temperature lapse-rate
    corrections (Sect.~\ref{sec:atm}), time-dependent temperature offsets, and
    in some cases, time-dependent paleo-precipitation reductions
    (Sect.~\ref{sec:climate}).

    Each simulation starts from assumed present-day ice thickness and
    equilibrium temperature distribution at 120\,\unit{ka}, and runs to the
    present. Our modelling domain of 900 by 600\,\unit{km} encompasses the
    entire Alpine range (Fig.~\ref{fig:inputs}). The simulations were run on
    two distinct grids, using horizontal resolutions of 2 and 1\,\unit{km}.
    All parameter values are summarized in Table~\ref{tab:params}.


% -- -- -- -- -- -- -- -- -- -- -- -- -- -- -- -- -- -- -- -- -- -- -- -
\subsection{Ice rheology}
\label{sec:icedyn}
% -- -- -- -- -- -- -- -- -- -- -- -- -- -- -- -- -- -- -- -- -- -- -- -

    Ice sheet dynamics are typically modelled using a~combination of internal
    deformation and basal sliding. PISM is a~shallow ice sheet model, which
    implies that the balance of stresses is approximated based on their
    predominant components. The Shallow Shelf Approximation (SSA) is combined
    with the Shallow Ice Approximation (SIA) by adding velocity solutions of
    the two approximations \citep[Eqs.~7--9 and 15]{Winkelmann.etal.2011}.
    Although this heuristic approach implies errors in the transition zone
    where gravitational stresses intervene both in the SIA and SSA velocity
    computation, it was shown to effectively approximate complex ice field flow
    dynamics over mountainous topographies similar to that of the European Alps
    \citep{Golledge.etal.2012, Ziemen.etal.2016}. Beside, this hybrid scheme is
    computationally much more efficient than a fully three-dimensional model
    with which simulations of the scale presented here would not be feasible.

    Ice deformation is governed by the constitutive law for ice
    \citep{Glen.1952, Nye.1953},
    %
    \begin{align}
      \vec{\dot{\epsilon}} = A\,\tau_{\mathrm{e}}^{n-1}\,\vec{\tau} \,.
    \end{align}
    %
    where $\vec{\dot{\epsilon}}$ is the strain-rate tensor, $\vec{\tau}$ the
    deviatoric stress tensor, and $\tau_{\mathrm{e}}$ the effective stress
    defined in our case by
    ${\tau_{\mathrm{e}}}^2=\frac{1}{2}\mathrm{tr}(\vec{\tau}^2)$. We set the
    power-law exponent, $n=3$, according to
    \citet[p.~55--57]{Cuffey.Paterson.2010}. The ice
    softness coefficient, $A$, depends on ice temperature, $T$, pressure, $p$,
    and water content, $\omega$, through a~piece-wise Arrhenius-type law,
    %
    \begin{align}
    &A = E\cdot
      \begin{cases}
        A_{\mathrm{c}} \,e^\frac{-Q_{\mathrm{c}}}{RT_{\text{pa}}}
            & \text{if}\ T_   {\text{pa}}  <  T_{\mathrm{c}} \,, \\
        A_{\mathrm{w}} (1+f\omega)\,e^\frac{-Q_{\mathrm{w}}}{RT_{\text{pa}}}
            & \text{if}\ T_   {\text{pa}} \ge T_{\mathrm{c}} \,,
      \end{cases}
    \end{align}
    %
    where $T_{\text{pa}}$ is the pressure-adjusted ice temperature calculated
    using the Clapeyron relation, $T_{\text{pa}}=T-{\beta}p$.
    $R=8.31441$\,\unit{J\,mol^{-1}\,K^{-1}} is the ideal gas constant, and
    $A_{\mathrm{c}}=2.847 \times 10^{-13}$\,\unit{Pa^{-3}\,s^{-1}},
    $A_{\mathrm{w}}=2.356 \times 10^{-2}$\,\unit{Pa^{-3}\,s^{-1}},
    $Q_{\mathrm{c}}=6.0 \times 10^4$\,\unit{J\,mol^{-1}}, and
    $Q_{\mathrm{w}}=11.5 \times 10^4$\,\unit{J\,mol^{-1}} are constant parameters
    corresponding to values measured below and above a~critical temperature
    threshold $T_{\mathrm{c}}=-10$\,\unit{{\degree}C}
    \citep[p.~72]{Cuffey.Paterson.2010}. We set the Clapeyron constant,
    $\beta=7.9\times 10^{-8}$\,\unit{K\,Pa^{-1}}, according to
    \citet{Luthi.etal.2002}, and the water fraction coefficient, $f=181.25$,
    according to \citet{Lliboutry.Duval.1985}. The water fraction, $\omega$, is
    capped at a~maximum value of 0.01, above which no measurements are
    available \citep[Eq.~5.7]{Lliboutry.Duval.1985, Greve.1997}. Finally, $E$
    is a~non-dimensional enhancement factor which take the values,
    $E_{\text{SIA}}=2$, in the SIA and $E_{\text{SSA}}=1$, in the SSA, as
    recommended for Holocene polar ice \citep[p.~77]{Cuffey.Paterson.2010}.

    Surface air temperature derived from the climate forcing
    (Sects.~\ref{sec:atm}, ~\ref{sec:paltemp}) provides the upper boundary
    condition to the ice
    enthalpy model. Temperature is computed in the ice and in the bedrock down
    to a~depth of 3\,\unit{km} below the glacier base, where it is
    conditioned by a~lower boundary geothermal heat flux estimate from multiple
    geothermal proxies (\citealp[similarity method]{Goutorbe.etal.2011};
    Fig.~\ref{fig:inputs}a).


% -- -- -- -- -- -- -- -- -- -- -- -- -- -- -- -- -- -- -- -- -- -- -- -
\subsection{Basal sliding}
\label{sec:sliding}
% -- -- -- -- -- -- -- -- -- -- -- -- -- -- -- -- -- -- -- -- -- -- -- -

    A~pseudo-plastic sliding law,
    %
    \begin{align}
      \vec{\tau}_{\mathrm{b}} = -\tau_{\mathrm{c}}
        \frac{\vec{v}_{\mathrm{b}}}
             {{v_{\text  {th}}}^q\,|\vec{v}_{\mathrm{b}}|^{1-q}} \,,
    \end{align}
    %
    relates the bed-parallel shear stresses, $\vec{\tau}_{\mathrm{b}}$, to the
    sliding velocity, $\vec{v}_{\mathrm{b}}$. The pseudo-plastic sliding
    exponent, $q=0.25$, and the threshold velocity,
    $v_{\text{th}}=100$\,\unit{m\,a^{-1}}, are set to values successfully used
    to model the Greenland ice sheet \citep{Aschwanden.etal.2013}. The yield
    stress, $\tau_{\mathrm{c}}$, is modelled using the Mohr--Coulomb criterion,
    %
    \begin{align}
      \tau_{\mathrm{c}} = c_0 + N\,\tan{\phi} \,,
    \end{align}
    %
    where the till cohesion, $c_0=0$, was oftentimes measured to be
    negligible \citep[p.~268]{Tulaczyk.etal.2000, Cuffey.Paterson.2010}. We use
    a~constant till friction angle, $\phi=30$\unit{\degree}, corresponding to
    the average of values presented in \citet[p.~268]{Cuffey.Paterson.2010}.

    Effective pressure, $N$, is related to the ice overburden stress, $P_0=\rho
    gh$, and the modelled amount of subglacial water, using a~formula derived
    from laboratory experiments with till extracted from the base of Ice Stream
    B in West Antarctica \citep{Tulaczyk.etal.2000, Bueler.Pelt.2015},
    %
    \begin{align}
      N = \delta P_0 \, 10^{(e_0/C_{\mathrm{c}}) (1 - (W/W_{\text{max}}))} \,,
    \end{align}
    %
    where $\delta=0.02$ sets the minimum ratio between the effective and
    overburden pressures. Parameter values for the till reference void ratio,
    $e_0=0.69$, and the till compressibility coefficient,
    $C_{\mathrm{c}}=0.12$, were set to the only measurements available to our
    knowledge \citep{Tulaczyk.etal.2000}. The amount of water at the base, $W$,
    varies from zero to $W_{\text{max}}=2$\,m, a~threshold above which
    additional melt water is assumed to drain off instantaneously.


% -- -- -- -- -- -- -- -- -- -- -- -- -- -- -- -- -- -- -- -- -- -- -- -
\subsection{Basal topography}
\label{sec:bedrock}
% -- -- -- -- -- -- -- -- -- -- -- -- -- -- -- -- -- -- -- -- -- -- -- -

    The initial basal topography is bilinearly interpolated from the
    hole-filled, Shuttle Radar Topography Mission (SRTM) data with a~resolution
    of 30\,arc-sec \citep{Jarvis.etal.2008}. These data include post-glacial
    sediment fills and lakes surface topography. However, they were corrected
    for an estimation of present-day ice thickness according from modern
    glacier outlines, surface topography and simplified ice physics
    \citep{Huss.Farinotti.2012}.

    Basal topography responds to ice load following a~bedrock deformation model
    that includes local isostasy, elastic lithosphere flexure and viscous
    astenosphere deformation in an infinite half-space
    \citep{Lingle.Clark.1985,Bueler.etal.2007}. The astenosphere viscosity,
    $\nu_{\mathrm{m}}=2.2\times10^{20}$\,\unit{Pa\,s}, the astenosphere
    density, $\rho_{\mathrm{m}}=3300$\,\unit{kg\,m^{-3}}, and the lithosphere
    elastic rigidity, $D=1.389 \times 10^{24}$\,\unit{N\,m}, were set according to
    results from glacial isostatic adjustment modelling of deglacial rebound in
    the Alps most closely reproducing observed modern uplift rates
    \citep[Supplementary Fig.~7]{Mey.etal.2016}. Unfortunately, the latter was
    computed using the erroneous formula, ${D=YE^3/12(1-\nu)}$
    \citep{Mey.etal.2016}. The correct formula is ${D=YE^3/12(1-\nu^2)}$.
    Using the Young's modulus, $Y=100$\,GPa, and the Poisson ratio, $\nu=0.25$
    \citep{Mey.etal.2016}, our value for lithosphere rigidity thus correspond
    to an effective elastic thickness of the lithosphere of $E=52.415$\,km,
    instead of the 50\,km value used by \citet{Mey.etal.2016}.

    \note{This is embarrassing, but I realised too late that the formula in
          \citet{Mey.etal.2016} was wrong.}


% -- -- -- -- -- -- -- -- -- -- -- -- -- -- -- -- -- -- -- -- -- -- -- -
\subsection{Surface mass balance}
\label{sec:surface}
% -- -- -- -- -- -- -- -- -- -- -- -- -- -- -- -- -- -- -- -- -- -- -- -

    Ice surface accumulation and ablation are computed from monthly mean
    near-surface air temperature, $T_{\mathrm{m}}$, monthly standard deviation
    of near-surface air temperature, $\sigma$, and monthly precipitation,
    $P_{\mathrm{m}}$, using a~temperature-index model
    \citep[e.g.,][]{Hock.2003}. Accumulation is equal to precipitation when air
    temperatures are below 0\,\unit{{\degree}C}, and decreases to zero linearly
    with temperatures between 0 and 2\,\unit{{\degree}C}. Ablation is computed
    from PDD, the integral of temperatures above 0\,\unit{{\degree}C}.

    The PDD computation accounts for stochastic temperature variations by
    assuming a~normal temperature distribution of standard deviation, $\sigma$,
    around the expected value, $T_{\mathrm{m}}$. It is expressed by an
    error-function formulation \citep{Calov.Greve.2005},
    %
    \begin{align}
      {\text{PDD}} = \int_{t_1}^{t_2} \mathrm{d}t
        \left[\frac{\sigma}{\sqrt{2\pi}}
                \exp\left({-\frac{T_{\mathrm{m}}^2}{2\sigma^2}}\right)
              + \frac{T_{\mathrm{m}}}{2} \, {\text{erfc}}
                \left(-\frac{T_{\mathrm{m}}}{\sqrt{2}\sigma}\right)\right] \,,
    \end{align}
    %
    which is numerically approximated using week-long sub-intervals. In
    order to account for the effects of spatial and seasonal variations of
    temperature variability \citep{Seguinot.2013}, $\sigma$ is computed
    from ERA-Interim daily temperature values from 1979 to 2012
    \citep{Mesinger.etal.2006}, including variability associated with the
    seasonal cycle \citep{Seguinot.2013}, and bilinearly interpolated to the
    model grids (Fig.~\ref{fig:inputs}b). Degree-day factors for snow and ice
    melt are set to values used in the European Ice Sheet Modelling INiTiative
    \citep[Table~\ref{tab:params}; EISMINT,][]{Huybrechts.1998}.


% -- -- -- -- -- -- -- -- -- -- -- -- -- -- -- -- -- -- -- -- -- -- -- -
\subsection{Reference climate forcing}
\label{sec:atm}
% -- -- -- -- -- -- -- -- -- -- -- -- -- -- -- -- -- -- -- -- -- -- -- -

    Climate forcing driving ice sheet simulations consists of a~present-day
    monthly climatology, $\{T_{\mathrm{m}0}, P_{\mathrm{m}0}\}$, modified by
    temperature lapse-rate corrections, ${\Delta}T_{\text{LR}}$, temperature
    offset time series, ${\Delta}T_{\text{TS}}$, and time-dependent
    palaeo-precipitation corrections, $\Psi_{\text{PP}}$:
    %
    \begin{align}
      &T_{\mathrm{m}}(t, x, y) = T_{\mathrm{m}0}(x, y) +
                                 {\Delta}T_{\text{LR}}(t, x, y) +
                                 {\Delta}T_{\text{TS}}(t) \,, \\
      &P_{\mathrm{m}}(t, x, y) = P_{\mathrm{m}0}(x, y) \cdot
                                 {\Psi}_{\text{PP}}(t) \,, \\
    \end{align}
    %
    The present-day monthly climatology was bilinearly interpolated from
    near-surface air temperature and precipitation rate fields from
    WorldClim \citep{Hijmans.etal.2005}, representative of the period~1960
    to~1990. Modern climate of the European Alps is characterised by a
    north-south gradient in summer air temperatures (Fig.~\ref{fig:inputs}c),
    and an east-west gradient in winter precipitation (Fig.~\ref{fig:inputs}d).
    WorldClim data
    were selected as an input to the ice sheet model because they incorporate
    observations from the dense weather station network of central Europe
    \citep[Fig.~1]{Hijmans.etal.2005}. Besides, WorldClim data were
    previously used as climate forcing for PISM to model the LGM extent of the
    former Cordilleran ice sheet in good agreement with geological evidence
    along the southern margin \citep{Seguinot.etal.2014} were weather station
    density is lower than in the Alps. Finally, the last glacial cycle alpine
    glaciers did not extend over marine areas where WorldClim data is missing.
    Although low station density in the Arctic and missing data on marine
    surfaces were major setbacks against accurate representation of Cordilleran
    climate \citep{Seguinot.etal.2014}, these concerns are irrelevant to the
    European Alps.

    The temperature lapse-rate corrections, ${\Delta}T_{\text{LR}}$, are
    computed as a~function of ice surface elevation, $s$, using the SRTM
    topography shipped with WorldClim as a~reference, $b_{\text{ref}}$:
    %
    \begin{align}
      {\Delta}T_{\text{LR}}(t, x, y) &= -\gamma [s(t, x, y)-b_{\text{ref}}] \\
                                     &= -\gamma [h(t, x, y)+
                                                 b(t, x, y)-b_{\text{ref}}],
    \end{align}
    %
    thus accounting for the evolution of ice thickness, ${h=s-b}$, on the one
    hand, and for differences between the basal topography of the ice flow
    model, $b$, and the NARR reference topography, $b_{\text{ref}}$, on the
    other hand. All simulations use an annual temperature lapse rate of
    $\gamma=6\,\unit{K\,km^{-1}}$.


% ----------------------------------------------------------------------
\section{Palaeo-climate forcing}
\label{sec:climate}
% ----------------------------------------------------------------------

    \note{This section contains\\
      Table~\ref{tab:records} -- Palaeo-climate records.\\
      Fig.~\ref{fig:timeseries} -- Low-resolution time series.\\
      Fig.~\ref{fig:footprints} -- Low-resolution ice cover.}

    In this section, we analyze the model sensitivity to palaeo-climate forcing
    through the last glacial cycle, using three palaeo-temperature records
    (Sect.~\ref{sec:paltemp}) and two parametrizations of palaeo-precipitation
    (Sect.~\ref{sec:palprec}), in terms of modelled evolution of total ice
    volume (Sect.~\ref{sec:timeseries}) and glaciated area during MIS~2 and~4
    (Sect.~\ref{sec:footprints}).

    These simulations use an horizontal resolution of 2\,km. The vertical grid
    consists of 31~temperature layers in the bedrock and up to 126~enthalpy
    layers in the ice, corresponding to vertical resolutions of 100 and
    40\,\unit{m}, respectively.


% -- -- -- -- -- -- -- -- -- -- -- -- -- -- -- -- -- -- -- -- -- -- -- -
\subsection{Palaeo-temperature forcing}
\label{sec:paltemp}
% -- -- -- -- -- -- -- -- -- -- -- -- -- -- -- -- -- -- -- -- -- -- -- -

    Temperature offset time-series, ${\Delta}T_{\text{TS}}$, are derived from
    palaeo-temperature proxy records from the Greenland Ice Core Project
    \citep[GRIP;][]{Dansgaard.etal.1993}, the European Project for Ice Coring
    in Antarctica \citep[EPICA;][] {Jouzel.etal.2007}, and an oceanic sediment
    core from the Iberian margin \citep[MD01-2444;][]{Martrat.etal.2007}.
    Palaeo-temperature anomalies from the GRIP record are calculated from
    oxygen isotope (\chem{\delta^{18}O}) measurements using a~quadratic
    equation \citep{Johnsen.etal.1995},
    %
    \begin{align}
      {\Delta}T_{\text{TS}}(t) ={~}&-11.88 [\chem{\delta^{18}O}(t)
                                    -\chem{\delta^{18}O}(0)] \nonumber \\
                                   &-0.1925 [\chem{\delta^{18}O}(t)^2
                                    -\chem{\delta^{18}O}(0)^2] \,,
    \end{align}
    %
    while temperature reconstructions from the EPICA and MD01-2444 records are
    provided as such. For each proxy record used and each of the parameter
    setup used in the sensitivity tests, palaeo-temperature anomalies are
    scaled linearly (Table~\ref{tab:records}, Fig.~\ref{fig:timeseries}a) so
    that the cumulative glaciated area of the Rhine glacier piedmont lobe
    during Oxygen Marine Isotope Stage (MIS)~2 (29--14\,ka) is modelled
    comparably to the reconstructions (Fig.~\ref{fig:footprints}a, black
    rectangle).


% -- -- -- -- -- -- -- -- -- -- -- -- -- -- -- -- -- -- -- -- -- -- -- -
\subsection{Palaeo-precipitation forcing}
\label{sec:palprec}
% -- -- -- -- -- -- -- -- -- -- -- -- -- -- -- -- -- -- -- -- -- -- -- -

    Finally, in some simulations, precipitation was reduced with temperature in
    order to simulate the potential rarefaction of atmospheric moisture in
    colder climates. This was done using an empirical relationship derived from
    observed accumulation rates and oxygen isotopes concentrations in the GRIP
    ice core \citep{Dahl-Jensen.etal.1993},
    %
    \begin{align}
      {\Psi}_{\text{PP}}(t) = \exp[\psi{\Delta}T_{\text{TS}}(t)] \,.
    \end{align}
    %
    with $\psi=0.169/2.4=0.0704$ \citep{Huybrechts.2002}. This simple
    relationship likely does not reflect the complexity of atmospheric
    circulation changes that governed moisture availability over the Alps
    during the last glacial cycle, of which little is known outside from the
    LGM \citep[cf.][]{Wu.etal.2007, Strandberg.etal.2011, Ludwig.etal.2016}.
    Thus another set of simulations presented here use constant
    precipitation, corresponding to $\psi=0$. In the rest of this paper, we
    refer to different model runs by the name of the proxy record used for the
    palaeo-temperature forcing. Model runs using paleo-precipitation
    corrections are labelled PP.


% -- -- -- -- -- -- -- -- -- -- -- -- -- -- -- -- -- -- -- -- -- -- -- -
\subsection{Sensitivity of ice volume evolution}
\label{sec:timeseries}
% -- -- -- -- -- -- -- -- -- -- -- -- -- -- -- -- -- -- -- -- -- -- -- -

    For the three palaeo-temperature records and the two palaeo-precipitation
    parametrizations used, the model yields significant ice volume build-up
    during MIS~4 and 2 (Fig.~\ref{fig:timeseries}b), corresponding to documented
    glaciation periods in the Alps \citep{Preusser.2004, Ivy-Ochs.etal.2008}.
    All simulations also yield important glaciations during MIS~5 and 3, but
    their timing and amplitude varies significantly depending on the climate
    forcing used. All six simulations overestimate
    ice cover during the Younger Dryas \citep[cf. e.g.,][]{Ivy-Ochs.etal.2009}.
    This is partly because the 2\,km resolution is too coarse to resolve
    Younger Dryas Alpine glaciers, but the large total ice volume
    overestimation resulting from the GRIP temperature forcing
    (Fig.~\ref{fig:timeseries}b, blue curves) can not be explained by
    resolution alone.

    Geological data from the best documented Alpine piedmont lobes indicate
    that their maximum extension during MIS~2 occurred at around 25.5--24.0\,ka
    \citep{Monegato.etal.2017}, after which Alpine glaciers remained or
    potentially re-advanced to within close reach of this maximum extent, until
    as late as 22 to 17\,ka \citep[Fig.~5]{Wirsig.etal.2016}. This is in very
    good agreement with the result of the EPICA simulations, which yield an
    early maximum ice volume at 25.2/24.6 (without/with palaeo-precipitation
    reductions) followed by a standstill until 17.3\,ka
    (Fig.~\ref{fig:timeseries}b, blue curves). The GRIP palaeo-temperature
    forcing, on the other hand, yield two distinct total ice volume maxima
    at 27.3/27.0 and 21.8/21.7\,ka followed by an early deglaciation of the
    foreland by 21.4\,ka. The MD01-2444 palaeo-temperature forcing yields
    a late LGM peak ice volume at 16.5/15.7\,ka several thousand years after
    the dated geological evidence.

    Finally, all simulations yield very strong ice volume
    variability, indicative of many more than two or three cycles of glacier
    advance and retreats onto the foreland (Fig.~\ref{fig:timeseries}b).
    Palaeo-precipitation reductions dampen some of the small-scale variability,
    but they have little effect on the millennial scale that characterise those
    cycles (Fig.~\ref{fig:timeseries}b, light colour curves). On the opposite
    EPICA palaeo-temperature record has least variability the among records
    used (Fig.~\ref{fig:timeseries}a, red curves). This results in fewer
    variability in the total ice volume, and more restrictive glaciations
    during MIS~5 and~3, (Fig.~\ref{fig:timeseries}b, red curves).


% -- -- -- -- -- -- -- -- -- -- -- -- -- -- -- -- -- -- -- -- -- -- -- -
\subsection{Sensitivity of glaciated area}
\label{sec:footprints}
% -- -- -- -- -- -- -- -- -- -- -- -- -- -- -- -- -- -- -- -- -- -- -- -

    During MIS~2, all six simulations yield comparative cumulative glaciated
    areas (Fig.~\ref{fig:footprints}a). This result is inherent to the
    palaeo-climate forcing approach used, which involves a linear scaling of
    each palaeo-temperature anomaly record to a geomorphological
    reconstruction of the area glaciated by the Rhine Glacier piedmont lobe
    (Sect.~\ref{sec:paltemp}, Fig.~\ref{fig:footprints}a, black rectangle).
    Outside this benchmark, all simulations underestimate glaciated area in the
    western part of the model domain, and overestimate it in the eastern
    part, relative to the geomorphological ice limits
    (Fig.~\ref{fig:footprints}a). This result might indicate that the LGM
    temperature depression, here considered homogeneous, was actually lower in
    the Eastern Alps and in the Western Alps, as previously shown by positive
    degree-day modelling of the central European palaeo-ice caps
    \citep{Heyman.etal.2013}. Alternatively, the east-west
    gradient in precipitation existing today (Fig.~\ref{fig:inputs}d) was
    perhaps enhanced during the LGM, as indicated by pollen reconstructions
    \citep{Wu.etal.2007}. Finally, the LGM extent of Alpine glaciers might have
    been affected by east-west gradient in variables not accounted for
    by the PDD model, such as cloud cover or dust deposition.

    Beside this general pattern, there exist differences in glaciated area
    depending on the palaeo-climate forcing used. The MD01-2444, and to a
    greater extent, the GRIP, palaeo-temperature records, tend to overestimate
    MIS~2 ice cover on all peripheral ranges that surround the Alps
    (Fig.~\ref{fig:footprints}a and c). This is particularly true when
    precipitation corrections are applied (bright colours). In fact, both
    palaeo-temperature records have a higher temperature variability and
    contain brief spells of cold temperatures, which are too short to develop a
    fully grown Alpine ice sheet, but long enough to build up ice cover on a
    smaller scale on these peripheral ranges (Fig.~\ref{fig:timeseries}). On
    the other hand, peripheral glaciation modelled in the EPICA simulation
    (Fig.~\ref{fig:footprints}b) is in relatively good agreement with the
    geomorphological reconstructions.

    The modelled extent of glaciation during MIS~4 depicts a more pronounced
    sensitivity to the choice of palaeo-climate forcing used. Using the GRIP
    and MD01-2444 palaeo-temperature records, Alpine glaciers are modelled to
    extend well beyond the reach of documented LGM end-moraines and glacial
    erratic terrain (Fig.~\ref{fig:footprints}d and f). Because, in both cases,
    the modelled glaciation extent corresponds to brief cold spells in the
    palaeo-temperature forcing (Fig.~\ref{fig:timeseries}a),
    palaeo-precipitation reductions greatly reduce the excessive modelled ice
    cover (Figs.~\ref{fig:timeseries}b and \ref{fig:footprints}d and f, light
    colours). However, with or without palaeo-precipitation corrections,
    the GRIP and MD01-2444 forcing yield
    modelled ice extents (Fig.~\ref{fig:footprints}d and f) and volumes
    (Fig.~\ref{fig:timeseries}b) higher during MIS~4 than
    during MIS~2. This is in contradiction with geological evidence
    \citep{Preusser.2004, Ivy-Ochs.etal.2008}.

    The EPICA palaeo-temperature forcing, on the other hand, yields a MIS~4
    glaciation that is only slightly less expensive than the MIS~2 glaciation
    (Fig.~\ref{fig:footprints}e). The sensitivity of the glaciated area to
    palaeo-precipitation reductions is mostly limited to the southern terminal
    lobes of central-Alpine glaciers, which are slightly less extensive during
    MIS~2 and 4 (Fig.~\ref{fig:footprints}b and e).

    Based on the above considerations on timing of the LGM during MIS~2, and
    modelled ice
    extent during MIS~4, we chose EPICA as our optimal palaeo-temperature
    record for a more detailed and higher-resolution comparison of modelled
    glacier dynamics to available geological evidence
    (Sect.~\ref{sec:results}). As a conservative approach in regard to the
    rapid ice volume fluctuations, we choose to include palaeo-precipitation
    corrections in the following comparison.


% ----------------------------------------------------------------------
\section{Glacier dynamics}
\label{sec:results}
% ----------------------------------------------------------------------

    \note{This section contains\\
        Fig.~\ref{fig:lgmvel} -- Snapshot at 21\,ka and volume time series.\\
        Fig.~\ref{fig:timing} -- Timing of the LGM and area time series.\\
        Fig.~\ref{fig:trimlines} -- LGM ice thickness compared to trimlines.\\
        Fig.~\ref{fig:profiles} -- Individual glacier extent profiles.}

    In this section, we compare the model output to geological evidence from
    the last glacial cycle, in terms of Last Glacial Maximum extent
    (Sect.~\ref{sec:extent}), ice flow patterns
    (Sect.~\ref{sec:flow}), timing of the Last Glacial Maximum
    (Sect.~\ref{sec:timing}), ice thickness (Sect.~\ref{sec:thickness}), and
    glacial cycle dynamics (Sect.~\ref{sec:glaciations}).

    This simulation is forced by the optimal EPICA palaeo-temperature record
    (Sect.~\ref{sec:paltemp}) and includes palaeo-precipitation reductions
    (Sect.~\ref{sec:palprec}). It uses an horizontal resolution of 1\,km. The
    vertical grid consists of 61~temperature layers in the bedrock and up to
    251~enthalpy layers in the ice, corresponding to vertical resolutions of 50
    and 20\,\unit{m}, respectively.


% -- -- -- -- -- -- -- -- -- -- -- -- -- -- -- -- -- -- -- -- -- -- -- -
\subsection{Last Glacial Maximum ice extent}
\label{sec:extent}
% -- -- -- -- -- -- -- -- -- -- -- -- -- -- -- -- -- -- -- -- -- -- -- -

    The LGM extent of alpine glaciers has been mapped with varying level of
    detail across the mountain range \citep{Penck.Bruckner.1909, Jackli.1962,
    Husen.1987, Bini.etal.2009, Coutterand.2010, Bavec.Verbic.2011,
    Buoncristiani.Campy.2011, Husen.2011}. These maps have been compiled
    into a reconstruction, covering the entire Alps \citep{Ehlers.etal.2011},
    and reproduced here (Fig.~\ref{fig:lgmvel}, red line) for comparison
    against model results.

    The modelled total ice volume reaches a maximum of $122.5 \times
    10^{3}$\,\unit{km^3}, or 300.4\,mm of sea-level equivalent, at 24.56\,ka
    (Fig.~\ref{fig:lgmvel}b). A maximum glacierized area of $163.3 \times
    10^{3}$\,\unit{km^2} is attained shortly aftewards at 24.57\,ka
    (Fig.~\ref{fig:lgmvel}b). Although the modelled timing of the LGM varies
    accross the mountain range (Sect.~\ref{sec:timing}), at 24.57\,ka nearly
    all outlet glaciers extend to within a few km from their modelled maximum
    stage (Fig.~\ref{fig:lgmvel}a, dashed orange line).

    The palaeo-climate forcing was adapted (Sect.~\ref{sec:paltemp})
    to model this maximum configuration in broad agreement with the geological
    reconstruction of the LGM extent (Fig.~\ref{fig:lgmvel}a, solid red line),
    yet local discrepancies remain. As already outlined for low resolution
    runs (Sect.~\ref{sec:footprints}), the extent of glaciation in the
    north-western Alps, including the Rhone Glacier complex, the Jura ice cap,
    and the Lyon Lobe, is underestimated in the model results
    (Fig.~\ref{fig:lgmvel}a). On the other hand, the model yields excessive ice
    cover in the eastern Alps, where the Mur, Drava and Sava glaciers are
    modelled to extend tens of km beyond the mapped ice limits
    (Fig.~\ref{fig:lgmvel}a). As previously discussed
    (Sect.~\ref{sec:footprints}), these discrepancies might indicate that the
    LGM climate was characterized by an east-west gradient in temperature
    \citep[cf.][]{Heyman.etal.2013} or precipitation
    \citep[cf.][]{Wu.etal.2007} anomalies relative to present, or an east-west
    gradient in variables not accounted for by the PDD model.

    On the other hand, while atmospheric circulation models
    \citep{Strandberg.etal.2011, Ludwig.etal.2016} and palaeo-climate proxies
    \citep{Luetscher.etal.2015} both support differential precipitation changes
    north and south of the Alps, our model results show no obvious north-south
    bias as compared to the mapped LGM margins (Fig.~\ref{fig:lgmvel}a) despite
    the homogeneous temperature and precipitation anomalies applied relative to
    present. Although this may appear as a contradiction with previous,
    constant-climate modelling results \citep{Becker.etal.2016}, it is the
    result of introducing a time-dependent palaeo-temperature forcing. Constant
    climate forcing systematically resulted in extraneous
    \citep[Fig.~3]{Becker.etal.2016} and premature
    \citep[Fig.~4]{Becker.etal.2016} glaciation on the north slope of the Alps.
    Using time-dependent palaeo-temperature forcing, progressive climate
    cooling over several ka allow for warmer temperatures, closer to
    the thermodynamic equilibrium, within the ice and the bedrock. This results
    in thinner glaciers and limits overshoot of the equilibrium state
    (discussed further in Sect.~\ref{sec:timing}).

    On a more regional level, the modelled maximum extent of the Ticino, Iseo,
    Garda and Piave glaciers in the central southern Alps is constrained
    several kilometres within the mapped LGM moraines (Fig.~\ref{fig:lgmvel}a).
    This appears to result
    from the paleo-precipitation reduction used to force the model
    (Fig.~\ref{fig:footprints}b), which is likely unrealistic in at least this
    part of the model domain. On the other hand, the Durance glacier in the
    south-western Alps is modelled to extend several kilometers beyond the
    mapped limit (Fig.~\ref{fig:lgmvel}a), indicating that the LGM temperature
    depression was likely
    dampened by Mediterranean climate in this part of the model domain.

    The model reproduces peripheral ice caps where documented by geological
    evidence on the Vercors, Chartreuse and Bauge prealpine reliefs,
    the Jura Mountains, the Vosges Mountain, the Black Forest, the Bohemian
    Forest and the Dinaric Alps (Fig.~\ref{fig:lgmvel}a). An independent ice
    cap also covers the Hochwart massif during most of the simulation, yet it
    is engulfed by alpine glaciers during the LGM to become a peripheral ice
    dome \citep[Fig.~\ref{fig:lgmvel}a; cf.][Fig.~2.5]{Husen.2011}. The LGM
    extent of peripheral ice caps is underestimated in the Vosges and Jura
    mountains, while it is overestimated for the Vercors
    \citep[Fig.~\ref{fig:lgmvel}a; cf.][Figs.~4.28, 4.32, and 4.33,
    p.~322--321]{Coutterand.2010}. Thus, there exists a regional conformity
    between
    the model-data discrepancies obtained for the main Alpine ice sheet and
    those obtained for peripheral ice caps. Because peripheral ice cap have
    very different glacier dynamics than the main ice sheet, this conformity
    most likely indicates a climatic cause, rather than an ice dynamics cause,
    for these discrepancies.


% -- -- -- -- -- -- -- -- -- -- -- -- -- -- -- -- -- -- -- -- -- -- -- -
\subsection{Ice flow patterns}
\label{sec:flow}
% -- -- -- -- -- -- -- -- -- -- -- -- -- -- -- -- -- -- -- -- -- -- -- -

    The LGM Alpine ice flow pattern was traditionally described as that of a
    network of interconnected valley glaciers, primarily controlled by
    subglacial topography. However, the geomorphology shows that ice was thick
    enough to flow accross high mountain passes throughout the mountain range
    \citep[e.g.,][]{Onde.1938, Penck.Bruckner.1909, Jackli.1962, Husen.1985,
    Coutterand.2010, Kelly.etal.2004, Husen.2011}, and even perhaps to form
    self-sustained ice domes \citep{Florineth.1998, Florineth.Schluchter.1998,
    Kelly.etal.2004, Bini.etal.2009}.

    The modelled flow pattern at 24.57\,ka is complex (Fig.~\ref{fig:lgmvel}a,
    blue colour mapping). Fast-flow regions generally occur along the main
    river valleys, while ice domes and ice divides are predominantly located
    over major reliefs (Fig.~\ref{fig:lgmvel}a). Nevertheless, the model
    results depict occasional ice flow across the modern water divides, i.e.
    transfluences.

    In the Western Alps, major transfluences occur for instance across Col de
    Montgenèvre, Col du Mont-Cenis, Simplon Pass and Brünig Pass
    (Fig.~\ref{fig:lgmvel}a). Although a transfluence across col de Montgenèvre
    was previously questioned \citep[Fig.~2]{Cossart.etal.2012}, evidence for
    southerly ice flow across Mont-Cenis has been recognized
    \citep[Fig.~3.18, p.~284]{Onde.1938, Coutterand.2010}. Similarly,
    tranfluences have been previously identified from the geomorphology across
    Simplon Pass \citep{Kelly.etal.2004} and Brünig Pass \citep{Jackli.1962}.

    In the Eastern Alps, major transfluences are modelled to have occur for
    instance across at Fern Pass, the Seefeld Saddle, the Gailberg Saddle, the
    Kreuxberg Saddle, the Straniger Alp, and the Pyhrn Pass
    (Fig.~\ref{fig:lgmvel}a). Transfluences across Fern Pass and the Seefeld
    Saddle are known from the geomorphology
    \citep[Fig.~2.4]{Penck.Bruckner.1909, Husen.2011}. It is also known that
    ice flew across the Gailberg and Kreuzberg saddles \citep{Husen.1985}, and
    the Pyhrn Pass \citep[Fig.~\ref{fig:lgmvel}a; cf.][Fig.~2.5]{Husen.2011}.
    On the other hand, no transfluence has previously been documented at the
    Straniger Alp. The Tagliamento catchment is usually assumed to not have
    received tranfluences from the Drava catchment \citep{Monegato.etal.2007},
    but this would not be incompatible with reconstructed ice surface
    elevations in this area \citep{Husen.1987}.

    Finally, self-sustained ice domes characteristic of ice caps are modelled
    in two locations over Flüelapass and Ötztal (Fig.~\ref{fig:lgmvel}a).
    Except for too extensive ice cover in the easternmost part of the model
    domain (Sect.~\ref{sec:extent}), there is generally a good agreement
    between transfluences observed in the geomorphology and the model results.
    Both depict the LGM Alpine ice flow pattern as an intermediate between that
    of a topography-controlled ice field, and that of a self-sustained ice cap.

    \note{A lot more could be said here but I want to keep a balance with
          other parts of the discussion. Susan and I are currently working on a
          more detailed comparison of modelled transfluences against documented
          evidence across the Alps for a poster at SGM in November.}


% -- -- -- -- -- -- -- -- -- -- -- -- -- -- -- -- -- -- -- -- -- -- -- -
\subsection{Timing of the Last Glacial Maximum}
\label{sec:timing}
% -- -- -- -- -- -- -- -- -- -- -- -- -- -- -- -- -- -- -- -- -- -- -- -

    The timing of the LGM has been documented by radiocarbon and cosmogenic
    isotope dating techniques at multiple locations around the Alps (cf.
    \citealp[Fig.~5]{Wirsig.etal.2016} for a review and the more recent
    publications by \citealp{Monegato.etal.2017, Federici.etal.2017}). These
    data indicate that Alpine glaciers reached their maximum extent between 20
    and 26\,ka (calibrated \chem{^{14}C} and \chem{^{10}Be} ages), but also
    that terminal lobes stayed in the foreland until as late as 22 to 17\,ka,
    when they experienced rapid retreat synchronously to lowering of the ice
    surface in the mountains (\citealp[Fig.~5]{Wirsig.etal.2016};
    \citealp[Fig.~3]{Monegato.etal.2017}).

    In our simulation, the maximum areal cover is reached at 24.57\,ka
    (Fig.~\ref{fig:lgmvel}a), yet individual glacier lobes reach their
    maximum extent at different ages (Fig.~\ref{fig:timing}). For instance, the
    Dora Riparia, Ivrea, and Tagliamento glaciers reach an early maximum before
    27\,ka; the Durance, Rhone, Inn, Enns, Ticino, and Garda glaciers reach
    their maximum thickness in phase with the global Alpine areal maximum
    around 25\,ka; while the Isère, Como and Iseo glaciers reach a late maximum
    after 24\,ka (Fig.~\ref{fig:timing}). Remarkably, peripheral ice caps on
    the Vercors, Jura Mountains, Black Forest and Hochschwab reach their
    maximum extent even later and locally after 21\,ka (Fig.~\ref{fig:timing}).

    These differences in timing of the LGM occur in the model results despite
    the homogeneous temperature and precipitation anomalies supplied as
    palaeo-climate forcing. The LGM timing differences modelled here are
    thus not related to climate, but are inherent to modelled glacier dynamics.
    They result from differences in the subglacial topography, in particular
    catchment sizes and bed roughness, of the different outlet glaciers.

    PISM is a polythermal ice sheet model, i.e. it computes the distribution
    of englacial temperatures resulting from heat advection, vertical
    diffusion in ice and rock, and strain heat release, in three dimensions.
    For large Alpine glaciers flowing in deep valleys, such as the Rhone,
    Rhine and Garda glaciers, several thousand model years are needed to attain
    a thermodynamical equilibrium between cold ice advection from the high
    accumulation areas, upward diffusion of geothermal heat, and heat release
    from strong basal shear strain. Temperature evolution is, in part, limited
    by the slow warming of subglacial bedrock, that has been previously cooled
    downed by subfreezing air temperature before glacier advance.
    For larger glaciers, this thermodynamical equilibrium is typically not yet
    attained around 25\,ka. This is why several Alpine lobes overshoot their
    balanced extent before thinning and receding towards the mountains as they
    warm towards thermodynamical equilibrium.

    Our results indicate that even for a relatively small ice sheet like that
    found in the Alps, the LGM glacier extent corresponds to a transient
    stage, while ka-scale differences in its timing can
    result from complex glacier dynamics. Further differences in LGM timing
    caused by heterogeneous climatic anomalies, such as precipitation
    anomaly gradients, not included in our model set-up, may have
    counterbalanced or emphasize those modelled here.


% -- -- -- -- -- -- -- -- -- -- -- -- -- -- -- -- -- -- -- -- -- -- -- -
\subsection{Ice thickness and trimlines}
\label{sec:thickness}
% -- -- -- -- -- -- -- -- -- -- -- -- -- -- -- -- -- -- -- -- -- -- -- -

    Following the general assumption that trimlines, which mark the transition
    between the steep frost-shattered ridges and the more gentle glacially eroded
    valley troughs, represent the maximum elevation of the LGM ice surface,
    maximum ice thickness in the Alps been reconstructed in several areas
    \citep{Husen.1987, Florineth.1998, Florineth.Schluchter.1998,
    Kelly.etal.2004, Bini.etal.2009, Coutterand.2010, Cossart.etal.2012}.
    However, this assumption is challenged by geological evidence from
    Scandinavia \citep[e.g.,][]{Kleman.1994, Kleman.Borgstrom.1994},
    the British Isles \citep[e.g.,][]{Fabel.etal.2012}, and
    North America \citep[e.g.,][]{Kleman.etal.2010}, that pre-glacial landscapes
    located well above the trimlines have been glaciated and preserved under
    under cold-based ice, sometimes for several glacial cycles
    \citep{Stroeven.etal.2002}. Thus, the trimline could also mark the
    maximum elevation of the transition to cold-based ice, or a late-glacial
    ice surface elevation \citep[Fig.~1, p.~403]{Coutterand.2010}.

    In the upper Rhone valley, the maximum ice surface elevation reached
    during MIS~2 in our simulation, corrected for bedrock deformation, is
    consistently modelled to have occured several hundred metres above the
    observed trimline elevations (Fig.~\ref{fig:trimlines}a and b), with a mean
    difference of 861\,m (Fig.~\ref{fig:trimlines}b). This result is
    somewhat depedent on uncertain basal sliding and ice rheological parameters
    to which the model sensitivity was not tested here. However, a previous
    sensitivity study conducted with PISM and a very similar model set-up on
    the Cordillera ice sheet show that varying ice rheological parameters
    resulted in relative errors in modelled total ice volume of 25\,\unit{\%}
    with regard to ice rheological parameters and 14\,\unit{\%} regarding basal
    sliding parameters \citep[Fig.~7]{Seguinot.etal.2016}.

    Beside, our model results depict a polythermal LGM Alpine ice cover.
    Due to sub-freezing temperatures applied in the climate forcing of the
    model, the entire Alpine ice sheet is capped by an upper layer of cold ice.
    In major Alpine valleys, such as the upper Rhone Valley, important ice
    thickness and strain heating contribute to form a layer of temperate ice
    near the glacier base, allowing basal melt, sliding and potentially
    erosion (Fig.~\ref{fig:trimlines}c, white areas). On the other hand, on the
    highest mountains ice cover is to thin and too static to form temperate
    ice, resulting in cold ice down to the bed, preventing potential erosion
    (Fig.~\ref{fig:trimlines}c, hatched areas).

    In the upper Rhone valley, the geographical location of observed trimlines
    show remarkable agreement with the areal extent of regions that are
    modelled to have experienced warm-based for less than 1\,ka (7\,\%) of the
    MIS~2 (Fig.~\ref{fig:trimlines}c). Although this result calls for more
    detailed comparisons across the
    entire Alpine range and specific sensitivity studies to relevant basal
    sliding and ice rheological parameters, the presence of an upper layer of
    cold ice during the LGM, already found at high altitude in the much warmer
    climate of today \citep{Suter.etal.2001}, is inevitable.
    Thus we want to challenge the assumption that
    Alpine trimlines mark the maximum upper ice surface elevation of the LGM
    ice cover and call for a more accurate estimation of the thickness of the
    upper layer of cold ice in the Alps.


% -- -- -- -- -- -- -- -- -- -- -- -- -- -- -- -- -- -- -- -- -- -- -- -
\subsection{Glacial cycle dynamics}
\label{sec:glaciations}
% -- -- -- -- -- -- -- -- -- -- -- -- -- -- -- -- -- -- -- -- -- -- -- -

    Glacial history of the Alps prior to the LGM remains poorly constrained.
    Although the four glaciations model \citep{Penck.Bruckner.1909} has long
    been used, it is now known that glaciers advanced onto the foreland many
    more times \citep{Schluchter.1991, Preusser.etal.2011}. Sparse geological
    data indicate that the last glacial cycle may have comprised two or even
    three periods of glacier growth and decay \citep{Preusser.2004,Ivy-Ochs.etal.2008}.
    Luminescence dating from two sites in the central northern foreland
    indicate an early last glacial advance of Alpine glaciers onto the near
    foreland around $107\pm9$ to $101\pm5$\,ka \citep{Preusser.etal.2003,
    Preusser.Schluchter.2004}. Evidence for a MIS~4 glaciation is equally
    sparse, and its timing remain uncertain \citep[e.g.,][]{Preusser.etal.2003,
    Link.Preusser.2006}.

    As previously mentioned (Sect.~\ref{sec:timeseries}), independently of the
    palaeo-temperature (GRIP, EPICA, or MD01-2444) and palaeo-precipitation
    (with or without corrections) applied, all simulations presented here
    result in a high time variability in the total modelled ice volume
    (Fig.~\ref{fig:timeseries}). Using the optimal EPICA palaeo-temperature
    record with least variability (Sect.~\ref{sec:paltemp}) and conservatively
    including palaeo-precipitation reductions (Sect.~\ref{sec:palprec}), the
    1-km resolution simulation also results in strong total ice volume
    variability throughout the last glacial cycle (Fig.~\ref{fig:lgmvel}b).

    Two major glaciations occur during MIS~4 and 2 (Fig.~\ref{fig:lgmvel}b).
    However, several minor glaciations occur during MIS~5 and 3, as well as a
    minor late-glacial readvance at the onset of MIS~1
    (Fig.~\ref{fig:lgmvel}b). These episodes are the result of synchronous
    advances of several Alpine glaciers well into the major valleys and sometimes
    even onto the foreland (Fig.~\ref{fig:profiles}, Supplementary animation).

    For instance the Rhine Glacier (Fig.~\ref{fig:profiles}a) reaches a lenght
    exceeding 130\,km, roughly corresponding to location of the last major
    Alpine reliefs and the beginning of the foothills, 6~times during the
    simulation, and sometimes retreats almost completely between advances
    (Fig.~\ref{fig:profiles}b). The Rhone Glacier, fed by several high-altitude
    accumulation areas, advanced 8\,times onto the location of modern Lake
    Geneva (Fig.~\ref{fig:profiles}c and d). The Ivrea Glacier, characterised
    by a very steep catchment, and mutliple tributaries, show even a higher
    variability and reaches close to its maximum position 6\,times throughout
    the simulation (Fig.~\ref{fig:profiles}e and f). Finally, the Isère
    Glacier, with its complex system of confluences and diffluences, needs
    longer time to build up and reaches the foreland only twice during the
    major glaciations of MIS~4 and 2 (Fig.~\ref{fig:profiles}g and h).

    Despite the low temperature variability of the palaeo-climate forcing, and
    reduced precipitation dampening glacier response, our simulation depict
    the Alpine ice complex as highly dynamic, with many more than two or three
    \citep[cf.][]{Preusser.2004,Ivy-Ochs.etal.2008} glaciations, and regional
    glacier dynamic controlled by variations in subglacial topography and
    catchment geometry.


% ----------------------------------------------------------------------
\conclusions
% ----------------------------------------------------------------------

    This study consists in the application of the numerical ice sheet model
    PISM to ice dynamics of the last glacial cycle in the Alps, using a model
    set-up based on that previously developed and validated for the Cordilleran
    ice sheet (Sect.~\ref{sec:model}).
    Using three different palaeo-temperature forcing records (GRIP, EPICA, and
    MD01-2444, Sect.~\ref{sec:paltemp}), scaled to model the Rhine Glacier
    piedmont lobe in agreement with the mapped LGM ice margin, and two
    different palaeo-precipitation parametrizations (with and without
    precipitation reductions, Sect.~\ref{sec:palprec}), we find that only the
    EPICA palaeo-temperature record yields model results in agreement with
    geological findings, in the sense that:

    \begin{itemize}
      \item The EPICA palaeo-temperature forcing yields maximum ice volume at
            25.2/24.6\,ka (without/with palaeo-precipitation reduction),
            followed by a standstill of major piedmont lobe in the forelands
            until 17.3\,ka, both compatible with much of the dating results,
            whereas the GRIP forcing results in early deglaciation of the
            foreland complete by 21.4\,ka, and the MD01-2444 forcing results in
            a late LGM glaciation peaking at 16.5/15.7\,ka
            (Sect.~\ref{sec:timeseries}).
      \item The EPICA palaeo-temperature forcing yields cumulative ice extent
            compatible with geological evidence during MIS~4 and 2, whereas
            both GRIP and MD01-2444 forcings result in MIS~4 glaciation well
            beyong the mapped LGM ice limits  (Sect.~\ref{sec:footprints}).
    \end{itemize}

    We then use this optimal palaeo-temperature forcing and, as a conservative
    approach, palaeo-precipitation reductions, to force a 1-km resolution
    simulation of the last glacial cycle in the Alps. A more detailed analysis
    of its output lead us to the following conclusions.

    \begin{itemize}
      \item Ice cover is generally underestimated in the Western Alps and
            overestimated in the Eastern Alps, indicating that east-west
            gradients in temperature or precipitation change, absent from our
            model forcing, probably controlled the LGM extent of ice cover in
            the Alps. The asymmetric extent north and south of the Alps
            can be explained by the transient nature of the LGM extent without
            involving north-south gradients in temperature and precipitation
            change (Sect.~\ref{sec:extent}).
      \item The LGM ice flow pattern in the Alps was largely controlled by
            subglacial topography, but several transfluences and two
            self-sustained ice domes may have occurred (Sect.~\ref{sec:flow}).
      \item The LGM was a transient stage in which glaciers were not in
            equilibrium with climate. Timing potentially varied across the
            range due to inherent glacier dynamics (Sect.~\ref{sec:timing}).
      \item Ice thickness during the LGM is modelled to be much higher than in
            reconstructions. In average, modelled surface elevation is 861\,m
            above the Rhone Glacier trimlines, which may instead indicate an
            englacial thermal boundary (Sect.~\ref{sec:thickness}).
      \item Alpine glaciers were potentially very dynamic. They quickly
            responded to climate fluctuation and some potentially advanced many
            times over the foreland during the last glacial cycle
            (Sect.~\ref{sec:glaciations}).
    \end{itemize}

    These results are nevertheless limited by the simplicity of the climate
    forcing and surface mass balance model applied, and uncertainties in ice
    flow physics. Through more detailed sensitivity studies, targeted to
    specific aspects of the last glacial cycle ice dynamics in the Alps, future
    modelling studies will certainly be able to quantify uncertainties
    associated with some of the above statements. Nevertheless, we hope that
    these conclusions will also serve as a basis for future studies of glacial
    geology in the Alps, and call for a more systematic aggregation and
    homogenization of glacial geological data to form a basis for model
    validation across the entire Alpine range.


% ----------------------------------------------------------------------
% Acknowledgements
% ----------------------------------------------------------------------

\codedataavailability{%
    PISM is available as open-source software at http://pism-docs.org/.
    Please contact the corresponding author for model output data.
    \idea{Should we make model output available somewhere?}}

\authorcontribution{%
    J.~Seguinot designed the study, ran the simulations, and wrote most of
    the manuscript. M.~Huss provided modern ice thickness data. All authors
    contributed to the text. The idea for this study stems in part from an
    excursion organised by F.~Preusser in the central Alps in 2012.}

\competinginterests{%
    We have no conflict of interest in publishing these results.}

\begin{acknowledgements}

    This manuscript is partly based on a previous study on the Cordilleran Ice
    Sheet which is part of J.~Seguinot's PhD thesis. The experimental design
    used here, as well as the general layout of this study, are in important
    proportions the work of Irina Rogozhina, Arjen~P. Stroeven, Martin Margold and
    Johan Kleman without whom this study would not have happened. As always, we
    are very thankful to Constantine Khroulev, Ed Bueler, and Andy Aschwanden
    for providing constant help and development with PISM, in particular with
    recent issues with the computation of ice temperature (Github issue
    no.~371) and with the computation of bedrock deformation (Github issues
    no.~370 and 377). We are equally thankful to Giovanni Monegato and Marc
    Luetscher for very insightful discussions of preliminary model results during
    the European Geoscience Union General Assembly~2017.

    This work was supported by the Swiss National Foundation grant
    no.~200021-153179/1 to M.~Funk.
    Computer resources were provided by the Swiss National Supercomputing
    Centre (CSCS) allocation no.~s573 to J.~Seguinot.

\end{acknowledgements}


% ----------------------------------------------------------------------
% References
% ----------------------------------------------------------------------

\bibliographystyle{copernicus}
\bibliography{../../references/references}


% ----------------------------------------------------------------------
% Figures
\clearpage
% ----------------------------------------------------------------------

    \begin{figure*}
      \centerline{\includegraphics{alpcyc_hr_inputs}}
      \caption{%
        \textbf{(a)} Geothermal heat flow from applying the similarity method
        to multiple geophysical proxies \citep{Goutorbe.etal.2011} used as a
        boundary condition to the bedrock thermal model 3\,km below the
        ice-bedrock interface.
        \textbf{(b)} Mordern July standard deviation of daily mean temperature
        from the ERA-Interim \citep[1979--2012;][]{Dee.etal.2011, Seguinot.2013}
        from the reference monthly climatology used to force the surface mass
        balance (PDD) component of the ice sheet model.
        \textbf{(c)} Modern July mean near-surface air temperature and
        \textbf{(d)} January precipitation from WorldClim
        \citep[1960--1990;][]{Hijmans.etal.2005}.}
      \label{fig:inputs}
    \end{figure*}

    \begin{figure*}
      \centerline{\includegraphics{alpcyc_lr_timeseries}}
      \caption{%
        \textbf{(a)} Temperature offset time-series from ice core and ocean
        records (Table~\ref{tab:records}) used as palaeo-climate forcing for
        the ice sheet model.
        \textbf{(b)} Modelled total ice volume through the last 120~thousand
        years (ka) expressed in meters of sea level equivalent (m~s.l.e.). Gray
        fields indicate Marine Oxygen Isotope Stage (MIS) boundaries for MIS~2
        and MIS~4 according to a~global compilation of benthic
        \chem{\delta^{18}O} records \citep{Lisiecki.Raymo.2005}.}
      \label{fig:timeseries}
    \end{figure*}

    \begin{figure*}
      \centerline{\includegraphics{alpcyc_lr_footprints}}
      \caption{%
        \textbf{(a--c)} Cumulative extent of modelled ice cover during MIS~2
        (29--14\,ka), without (dark colours) and with (light colours)
        palaeo-precipitation corrections, and using temperature time-series
        scaling factors (Table~\ref{tab:records}) adjusted to model the area
        glaciated by the Rhine glacier piedmont lobe (black rectangle) in
        agreement with the Last Glacial Maximum (LGM) geomorphological
        reconstruction \citep[solid red line,][]{Ehlers.etal.2011}.
        \textbf{(d--f)} Cumulative extent of modelled ice cover during MIS~4
        (71--57\,ka). Only the simulation driven by the EPICA temperature
        time-series yields reasonable MIS~4 ice cover.}
      \label{fig:footprints}
    \end{figure*}

    \begin{figure*}
      \centerline{\includegraphics{alpcyc_hr_lgmvel}}
      \caption{%
        \textbf{(a)} Modelled bedrock topography (grey), ice surface topography
        (200\,m contours), and ice surface velocity (blue) in the Alps
        24.57\,ka before present, corresponding to the maximum modelled ice
        cover. Modelled self-sustained ice domes (triangles) and major
        transfluences (crosses). Modelled Last Glacial Maximum
        (LGM) ice extent (dashed orange line) and geomorphological
        reconstruction \citep[solid red line,][]{Ehlers.etal.2011}. The
        background map consists of depressed SRTM \citep{Jarvis.etal.2008}
        topography and Natural Earth Data \citep{Patterson.Kelso.2017}.
        \textbf{(b)} Temperature offset time-series from the EPICA ice core
        used as palaeo-climate forcing for the ice flow model \citep[black
        curve,][]{Jouzel.etal.2007}, and modelled total ice volume through the
        last glacial cycle (120--0\,ka), expressed in meters of sea level
        equivalent (m~s.l.e., blue curve). Gray fields indicate Marine
        Oxygen Isotope Stage (MIS) boundaries for MIS~2 and MIS~4 according to
        a~global compilation of benthic \chem{\delta^{18}O} records
        \citep{Lisiecki.Raymo.2005}.}
      \label{fig:lgmvel}
    \end{figure*}

    \begin{figure*}
      \centerline{\includegraphics{alpcyc_hr_timing}}
      \caption{%
        \textbf{(a)} Timing of the LGM given by the modelled age of maximum ice
        thickness throughout the entire simulation (colour mapping) and
        corresponding, ice surface elevation (200\,m contours).
        \textbf{(b)} Temperature offset time-series from the EPICA ice core
        used as palaeo-climate forcing for the ice flow model (black curve),
        and modelled glacierized area during the LGM (coloured curve). The LGM
        is here modelled as a time-transgressive event.}
      \label{fig:timing}
    \end{figure*}

    \begin{figure}
      \centerline{\includegraphics{alpcyc_hr_trimlines}}
      \caption{%
        \textbf{(a)} Comparison of modelled ice surface elevation at the LGM
        (time-transgressive, corresponding to maximum ice thickness,
        Fig.~\ref{fig:timing}), compensated for bedrock deformation, against
        observed trimline elevations \citep[Table~1]{Kelly.etal.2004}. Model
        variables were bilinearly interpolated to the trimline locations.
        \textbf{(b)} Histogram of differences between modelled LGM ice surface
        elevation and trimline elevations (500\,m bands). The average
        difference is 861\,m.
        \textbf{(c)} Modelled age of maximum ice thickness at the trimline
        locations \citep[Table~1]{Kelly.etal.2004} in the upper Rhone valley
        (colour) and LGM ice surface elevation (200\,m contours). Hatches
        mark areas modelled to have experienced less than 1\,ka temperate ice
        cover during MIS~2 (29--14\,ka).}
      \label{fig:trimlines}
    \end{figure}

    \begin{figure*}
      \centerline{\includegraphics{alpcyc_hr_profiles}}
      \caption{%
        \textbf{(a, c, e, g)} Approximate glacier flowlines drawn by hand
        rougly following selected glacial valley centerlines.
        \textbf{(b, d, f, h)} Evolution of modelled glacier extent in time,
        bilinearly interpolated along the corresponding profiles, showing
        numerous cycles of advance and retreat over the last glacial cycle
        modulated by subglacial topography and catchment geometry.
        Isolated patches indicate periodic surges from tributary glaciers.}
      \label{fig:profiles}
    \end{figure*}


% ----------------------------------------------------------------------
% Tables
\clearpage
% ----------------------------------------------------------------------

    \begin{table*}
      \caption{%
        Parameter values used in the ice sheet model.}
      \label{tab:params}
      \scalebox{0.8}
      {\begin{tabular}{llrll}
        \tophline

        Not.    & Name & Value & Unit & Source \\

        \middlehline
        \multicolumn{2}{l}{{Ice rheology}} \\
        \middlehline

        $\rho$  & Ice density
                & 910
                & \unit{kg\,m^{-3}}
                & \citet{Aschwanden.etal.2012} \\

        $g$     & Standard gravity
                & 9.81
                & \unit{m\,s^{-2}}
                & \citet{Aschwanden.etal.2012} \\

        $n$     & Glen exponent
                & 3
                & --
                & \citet{Cuffey.Paterson.2010} \\

        $A_{\mathrm{c}}$   & Ice hardness coefficient cold
                & $2.847 \times 10^{-13}$
                & \unit{Pa^{-3}\,s^{-1}}
                & \citet{Cuffey.Paterson.2010} \\

        $A_{\mathrm{w}}$   & Ice hardness coefficient warm
                & $2.356 \times 10^{-2}$
                & \unit{Pa^{-3}\,s^{-1}}
                & \citet{Cuffey.Paterson.2010} \\

        $Q_{\mathrm{c}}$   & Flow law activation energy cold
                & $6.0 \times 10^4$
                & \unit{J\,mol^{-1}}
                & \citet{Cuffey.Paterson.2010} \\

        $Q_{\mathrm{w}}$   & Flow law activation energy warm
                & $11.5 \times 10^4$
                & \unit{J\,mol^{-1}}
                & \citet{Cuffey.Paterson.2010} \\

        $E_{\text{SIA}}$   & SIA enhancement factor
                & 2
                & --
                & \citet{Cuffey.Paterson.2010} \\

        $E_{\text{SSA}}$   & SSA enhancement factor
                & 1
                & --
                & \citet{Cuffey.Paterson.2010} \\

        $T_{\mathrm{c}}$   & Flow law critical temperature
                & 263.15
                & \unit{K}
                & \citet{Paterson.Budd.1982} \\

        $f$     & Flow law water fraction coeff.
                & 181.25
                & --
                & \citet{Lliboutry.Duval.1985} \\

        $R$     & Ideal gas constant
                & 8.31441
                & \unit{J\,mol^{-1}\,K^{-1}}
                & -- \\

        $\beta$ & Clapeyron constant
                & $7.9 \times 10^{-8}$
                & \unit{K\,Pa^{-1}}
                & \citet{Luthi.etal.2002} \\

        $c_{\mathrm{i}}$   & Ice specific heat capacity
                & 2009
                & \unit{J\,kg^{-1}\,K^{-1}}
                & \citet{Aschwanden.etal.2012} \\

        $c_{\mathrm{w}}$   & Water specific heat capacity
                & 4170
                & \unit{J\,kg^{-1}\,K^{-1}}
                & \citet{Aschwanden.etal.2012} \\

        $k$     & Ice thermal conductivity
                & 2.10
                & \unit{J\,m^{-1}\,K^{-1}\,s^{-1}}
                & \citet{Aschwanden.etal.2012} \\

        $L$     & Water latent heat of fusion
                & $3.34\times10^5$
                & \unit{J\,kg^{-1}\,K^{-1}}
                & \citet{Aschwanden.etal.2012} \\

        \middlehline
        \multicolumn{2}{l}{{Basal sliding}} \\
        \middlehline

        $q$     & Pseudo-plastic sliding exponent
                & 0.25
                & --
                & \citet{Aschwanden.etal.2013} \\

        $v_{\text{th}}$& Pseudo-plastic threshold velocity
                & 100
                & \unit{m\,a^{-1}}
                & \citet{Aschwanden.etal.2013} \\

        $c_0$   & Till cohesion
                & 0
                & Pa
                & \citet{Tulaczyk.etal.2000} \\

        $e_0$   & Till reference void ratio
                & 0.69
                & --
                & \citet{Tulaczyk.etal.2000} \\

        $C_{\mathrm{c}}$   & Till compressibility coefficient
                & 0.12
                & --
                & \citet{Tulaczyk.etal.2000} \\

        $\delta$& Minimum effective pressure ratio
                & 0.02
                & --
                & \citet{Bueler.Pelt.2015} \\

        $\phi$  & Till friction angle
                & 30
                & \degree
                & \citet{Cuffey.Paterson.2010} \\

        $W_{\text{max}}$ & Maximum till water thickness
                & 2
                & m
                & \citet{Bueler.Pelt.2015} \\

        \middlehline
        \multicolumn{2}{l}{{Bedrock and lithosphere}} \\
        \middlehline

        $\rho_{\mathrm{b}}$& Bedrock density
                & 3300
                & \unit{kg\,m^{-3}}
                & -- \\

        $c_{\mathrm{b}}$   & Bedrock specific heat capacity
                & 1000
                & \unit{J\,kg^{-1}\,K^{-1}}
                & -- \\

        $k_{\mathrm{b}}$   & Bedrock thermal conductivity
                & 3
                & \unit{J\,m^{-1}\,K^{-1}\,s^{-1}}
                & -- \\

        $\nu_{\mathrm{m}}$ & Astenosphere viscosity
                & $2.2 \times 10^{20}$
                & \unit{Pa\,s}
                & \citet{Mey.etal.2016} \\

        $\rho_{\mathrm{m}}$& Astenosphere density
                & 3300
                & \unit{kg\,m^{-3}}
                & \citet{Mey.etal.2016} \\

        $D$     & Lithosphere flexural rigidity
                & $1.389 \times 10^{24}$
                & \unit{N\,m}
                & \citet{Mey.etal.2016} \\

        \middlehline
        \multicolumn{2}{l}{{Surface and atmosphere}} \\
        \middlehline

        $T_{\mathrm{s}}$   & Temperature of snow precipitation
                & 273.15
                & \unit{K}
                & -- \\

        $T_{\mathrm{r}}$   & Temperature of rain precipitation
                & 275.15
                & \unit{K}
                & -- \\

        $F_{\mathrm{s}}$   & Degree-day factor for snow
                & $3.297 \times 10^{-3}$
                & \unit{m\,K^{-1}\,day^{-1}}
                & \citet{Huybrechts.1998} \\

        $F_{\mathrm{i}}$   & Degree-day factor for ice
                & $8.791 \times 10^{-3}$
                & \unit{m\,K^{-1}\,day^{-1}}
                & \citet{Huybrechts.1998} \\

        $R$     & Refreezing fraction
                & 0.0
                & --
                & -- \\

        $\gamma$& Air temperature lapse rate
                & $6 \times 10^{-3}$
                & \unit{K\,m{-1}}
                & -- \\

        $\psi$  & Precipitation factor
                & 0.0704
                & --
                & \citet{Huybrechts.2002} \\

        \bottomhline
      \end{tabular}}
    \end{table*}

    \begin{table*}
      \caption{%
        Palaeo-temperature proxy records and scaling factors yielding
        temperature offset time-series used to force the ice sheet model
        through the last glacial cycle (Fig.~\ref{fig:timeseries}). $f$
        corresponds to the scaling factor adopted to yield Last Glacial Maximum
        ice limits in the vicinity of mapped end moraines
        (Fig.~\ref{fig:footprints}a), and $[{\Delta}T_{\textrm{TS}}]_{32}^{22}$
        refers to the resulting mean temperature anomaly during the period 32
        to~22\,\unit{ka} after scaling.}
      \label{tab:records}
      {\begin{tabular}{lccccccl}
        \tophline

        Forcing   & Latitude & Longitude & Elev. (m~a.s.l.)
                  & Proxy & $f$ & $[{\Delta}\text{TS}]_{32}^{22}$ (K)
                  & Reference\\

        \middlehline

        GRIP      & \multirow{2}{*}{ 72{\degree}35$^{\prime}$\,N}   % 72.58 (decimal)
                  & \multirow{2}{*}{ 37{\degree}38$^{\prime}$\,W}   % 37.64 (decimal)
                  & \multirow{2}{*}{3238}
                  & \multirow{2}{*}{\chem{\delta^{18}O}}
                  & 0.50 & $-$8.2  % -16.4126 (before scaling)
                  & \multirow{2}{*}{\citet{Dansgaard.etal.1993}} \\

        GRIP, PP &&&&& 0.63 & $-$10.4 \\

        EPICA     & \multirow{2}{*}{ 75{\degree}06$^{\prime}$\,S}   % 75.1
                  & \multirow{2}{*}{123{\degree}21$^{\prime}$\,E}   % 123.35
                  & \multirow{2}{*}{3233}
                  & \multirow{2}{*}{\chem{\delta^{18}O}}
                  & 1.05 & $-$9.7  % -9.2055
                  & \multirow{2}{*}{\citet{Jouzel.etal.2007}} \\

        EPICA, PP &&&&& 1.33 & $-$12.2 \\

        MD01-2444 & \multirow{2}{*}{ 37{\degree}34$^{\prime}$\,N}   % 37.561
                  & \multirow{2}{*}{ 10{\degree}04$^{\prime}$\,W}   % -10.142
                  & \multirow{2}{*}{$-$2637}
                  & \multirow{2}{*}{\chem{U^{K'}_{37}}}
                  & 1.84 & $-$8.0  % -4.345625
                  & \multirow{2}{*}{\citet{Martrat.etal.2007}} \\

        MD01-2444, PP &&&&& 2.44 & $-$10.6 \\

        \bottomhline
      \end{tabular}}

      %\noindent\small\makebox[\textwidth]
      %{\begin{tabular}{llrlrlr}
      %  \tophline
      %
      %  Record    & \multicolumn{2}{c}{GRIP}
      %            & \multicolumn{2}{c}{EPICA}
      %            & \multicolumn{2}{c}{MD01-2444} \\
      %
      %  Precip.   & cst. & PP
      %            & cst. & PP
      %            & cst. & PP \\
      %
      %  \middlehline
      %
      %  Latitude  & \multicolumn{2}{c}{ 72{\degree}35$^{\prime}$\,N}   % 72.58 (decimal)
      %            & \multicolumn{2}{c}{ 75{\degree}06$^{\prime}$\,S}   % 75.1
      %            & \multicolumn{2}{c}{ 37{\degree}34$^{\prime}$\,N} \\  % 37.561
      %
      %  Longitude & \multicolumn{2}{c}{ 37{\degree}38$^{\prime}$\,W}   % 37.64 (decimal)
      %            & \multicolumn{2}{c}{123{\degree}21$^{\prime}$\,E}   % 123.35
      %            & \multicolumn{2}{c}{ 10{\degree}04$^{\prime}$\,W} \\  % -10.142
      %
      %  Elevation & \multicolumn{2}{c}{3238\,m~a.s.l.}
      %            & \multicolumn{2}{c}{3233\,m~a.s.l.}
      %            & \multicolumn{2}{c}{$-$2637\,m~a.s.l.} \\
      %
      %  Proxy     & \multicolumn{2}{c}{\chem{\delta^{18}O}}
      %            & \multicolumn{2}{c}{\chem{\delta^{18}O}}
      %            & \multicolumn{2}{c}{\chem{U^{K'}_{37}}} \\
      %
      %  $f$       & 0.50 & 0.63
      %            & 1.05 & 1.33
      %            & 1.84 & 2.44 \\
      %
      %  $[{\Delta}\text{TS}]_{32}^{22}$
      %            & $-$8.2\,K & $-$10.4\,K     % -16.4126 (before scaling)
      %            & $-$9.7\,K & $-$12.2\,K     % -9.2055
      %            & $-$8.0\,K & $-$10.6\,K \\  % -4.345625
      %
      %  Reference & \multicolumn{2}{c}{\citet{Dansgaard.etal.1993}}
      %            & \multicolumn{2}{c}{\citet{Jouzel.etal.2007}}
      %            & \multicolumn{2}{c}{\citet{Martrat.etal.2007}} \\
      %
      %  \bottomhline
      %\end{tabular}}

    \end{table*}


% ======================================================================
\end{document}
% ======================================================================
