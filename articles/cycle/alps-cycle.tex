\documentclass{article}

\usepackage{authblk}
\usepackage[T1]{fontenc}
\usepackage[utf8]{inputenc}
\usepackage[pdftex]{xcolor}

\definecolor{darkblue}{cmyk}{0.9,0.3,0.0,0.0}
\definecolor{darkgreen}{cmyk}{0.8,0.0,1.0,0.0}
\definecolor{darkred}{cmyk}{0.1,0.9,0.8,0.0}

\newcommand{\idea}[1]{\textcolor{darkgreen}{\emph{[\textbf{IDEA:} #1]}}}
\newcommand{\note}[1]{\textcolor{darkblue}{\emph{[\textbf{NOTE:} #1]}}}
\newcommand{\todo}[1]{\textcolor{darkred}{\emph{[\textbf{TODO:} #1]}}}
\newcommand{\aref}[0]{\textcolor{darkblue}{\textbf{[REF.]}}}

\title{Modelling the last glacial cycle in the Alps}

\author[1]{Julien Seguinot%
           \thanks{Correspondence to seguinot@vaw.baug.ethz.ch}}
\author[1]{Guillaume Jouvet}
\author[1]{Matthias Huss}
\author[1]{Martin Funk}
\author[2]{Frank Preusser}

\affil[1]{Laboratory of Hydraulics, Hydrology and Glaciology,
          ETH Zürich, Switzerland}
\affil[2]{Institute of Earth and Environmental Sciences,
          University of Freiburg, Germany}

% ======================================================================
\begin{document}
% ======================================================================

\maketitle

\begin{abstract}

    The European Alps, cradle of pioneer glacial studies, are one of the the
    regions where geological markers of past glaciations are most abundant and
    well- studied. Such conditions make the region ideal for testing numerical
    glacier models based on approximated ice flow physics against field-based
    reconstructions, and vice-versa.

    Here, we use the Parallel Ice Sheet Model (PISM) to model the entire last
    glacial cycle (120–0 ka) in the Alps, with a horizontal resolution of 1 km.
    Climate forcing is derived using present-day climate data from WorldClim,
    the ERA- Interim reanalysis, and time-dependent temperature offsets from
    the EPICA ice core record in Antarctica, which allows reproduce the two
    major glaciation events documented in the geological record during marine
    oxygen isotope stages 4 (69–62 ka) and 2 (34–18 ka).

    Despite the low variability of this Antarctic-based climate forcing, our
    simulation depict a highly dynamic ice cap, showing that alpine glaciers
    may have advanced many times over the foreland during MIS 5 and 3.
    Cumulative basal sliding, a proxy for glacial erosion, is modelled to be
    highest in the deep valleys of the western Alps. However, relict uplift
    rates at the end of the simulation are most important in the eastern Alps
    where a thick ice cover persists for most of the modelled glacial cycle.
    Finally, the Last Glacial Maximum advance, often considered synchronous, is
    here modelled as a time-transgressive event, with some glacier lobes
    reaching their maximum as early as 27 ka, and some as late as 21 ka, which
    may explain some of the variability in published cosmogenic dates.

\end{abstract}

% ======================================================================
\end{document}
% ======================================================================
