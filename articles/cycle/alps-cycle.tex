\documentclass{article}

\usepackage{amsmath}
\usepackage{authblk}
\usepackage{booktabs}
\usepackage[T1]{fontenc}
\usepackage[utf8]{inputenc}
\usepackage[pdftex]{xcolor}
\usepackage[pdftex]{graphicx}
\usepackage[authoryear,round]{natbib}

\graphicspath{{../../figures/}}

\definecolor{darkblue}{cmyk}{0.9,0.3,0.0,0.0}
\definecolor{darkgreen}{cmyk}{0.8,0.0,1.0,0.0}
\definecolor{darkred}{cmyk}{0.1,0.9,0.8,0.0}

\newcommand{\idea}[1]{\textcolor{darkgreen}{\emph{[\textbf{IDEA:} #1]}}}
\newcommand{\note}[1]{\textcolor{darkblue}{\emph{[\textbf{NOTE:} #1]}}}
\newcommand{\todo}[1]{\textcolor{darkred}{\emph{[\textbf{TODO:} #1]}}}
\newcommand{\aref}[0]{\textcolor{darkblue}{\textbf{[REF.]}}}

\title{Modelling last glacial cycle ice dynamics in the Alps}

\author[1]{Julien Seguinot%
           \thanks{Correspondence to seguinot@vaw.baug.ethz.ch}}
\author[1]{Guillaume Jouvet}
\author[1]{Matthias Huss}
\author[1]{Martin Funk}
\author[2]{Frank Preusser}

\affil[1]{Laboratory of Hydraulics, Hydrology and Glaciology,
          ETH Zürich, Switzerland}
\affil[2]{Institute of Earth and Environmental Sciences,
          University of Freiburg, Germany}

% Common units
\newcommand{\e}[1]{\ensuremath{\times 10^{#1}}}
\newcommand{\chem}[1]{\ensuremath{\mathrm{#1}}}
\newcommand{\unit}[1]{\ensuremath{\mathrm{#1}}}
\newcommand{\degree}[0]{\ensuremath{^{\circ}}}
\newcommand{\degC}[0]{\unit{{\degree}C}}

% ======================================================================
\begin{document}
% ======================================================================

\maketitle

\begin{abstract}

    \note{For the Nature Geoscience format this abstract will need to be
          converted to an ``introductory paragraph''.}

    The European Alps, cradle of pioneer glacial studies, are one of the
    regions where geological markers of past glaciations are most abundant and
    well-studied. Such conditions make the region ideal for testing numerical
    glacier models based on approximated ice flow physics against field-based
    reconstructions, and vice-versa.

    Here, we use the Parallel Ice Sheet Model (PISM) to model the entire last
    glacial cycle (120--0\,ka) in the Alps, with a horizontal resolution of
    1\,km. Climate forcing is derived using present-day climate data from
    WorldClim and the ERA-Interim reanalysis, and time-dependent temperature
    offsets from multiple paleo-climate proxies, among which only the EPICA ice
    core record yields glacial extent during marine oxygen isotope stages~4
    (69--62\,ka) and~2 (34--18\,ka) in agreement to geological reconstructions.

    Despite the low variability of this Antarctic-based climate forcing, our
    simulation depicts a highly dynamic ice cap, showing that alpine glaciers
    may have advanced many times over the foreland during the last glacial
    cycle. Cumulative basal sliding, a proxy for glacial erosion, is modelled
    to be highest in the deep valleys of the western Alps. Finally, the Last
    Glacial Maximum advance, often considered synchronous, is here modelled as
    a time-transgressive event, with some glacier lobes reaching their maximum
    as early as 27\,ka, and some as late as 21\,ka. Modelled ice thickness is
    about 900\,m higher than observed trimline elevations, yet our simulation
    predicts little erosion at high elevation due to cold ice conditions.

\end{abstract}

% ----------------------------------------------------------------------
\section{Main}
% ----------------------------------------------------------------------

\subsection{Introduction}

    \begin{itemize}
    \item The European Alps are the cradle of pioneer glacial studies.
    \item Today, their glacial history is better understood than any other.
    \item The LGM extent is very well known (Fig.~\ref{fig:lgmvel}a).
    \item But modelling was difficult due to the steep topography.
    \item Here, we us PISM to model the last glacial cycle (120--0\,ka).
    \end{itemize}

    \idea{Matthias: I think it is very important to emphasize the knowledge gap
          that you fill in with your work in the Intro. (1) it is incompletely
          known how many advances occurred during the Last glacial cycle, (2)
          what drove the different response of the individual lobes (you show
          that it is glacier response time, i.e. ice dynamics), (3) how the
          temporal dynamics of subglacial erosion was structured.}

\subsection{Glacier dynamics}
    Fig.~\ref{fig:lgmvel} -- Snapshot at 21\,ka and volume time series.\\
    Fig.~\ref{fig:timing} -- Timing of the LGM and area time series.\\
    Fig.~\ref{fig:profiles} -- Individual glacier extent profiles.

    \begin{itemize}
    \item Rapid variations of total ice volume (Fig.~\ref{fig:lgmvel}b).
    \item Two major glaciations during MIS 4 and 2 (Fig.~\ref{fig:lgmvel}b).
    \item Maximum stages flow pattern is complex (Fig.~\ref{fig:lgmvel}a).
    \item Maximum extent is farily well reproduced (Fig.~\ref{fig:lgmvel}a).
    \item Maximum extent is time-transgressive (Fig.~\ref{fig:timing}).
    \item Glaciers have different response times (Fig.~\ref{fig:timing}).
    \item Some glaciers advance many times (Fig.~\ref{fig:profiles}).
    \end{itemize}

    \idea{Matthias: The selection of figures in the main text and SOM is good.
          I find Figure 2 quite difficult to understand at first glance. I
          would combine it (already before submission) with Fig. 3. the Figure
          can then focus both on the temporal advance/retreat patterns of the 4
          lobes (four panels on the left side of figure) and the spatial
          differences in max. ice elevation (1 panel on right side of figure,
          with centerlines indicated). I would have the one with trimline
          elevation again in the Supplementary.}

\subsection{Erosion potential}
    Fig.~\ref{fig:erosion} -- Integrated erosion potential.

    \idea{Frank: I would keep this topic for another paper.}

    \begin{itemize}
    \item We calculate erosion after \citet{Herman.etal.2015}
    \item More erosion in valleys of western Alps (Fig.~\ref{fig:erosion}a).
    \item Erosion is constant during 110--15\,ka (Fig.~\ref{fig:erosion}b).
    \end{itemize}

% ----------------------------------------------------------------------
\section{Methods}
% ----------------------------------------------------------------------

\subsection{Ice sheet model set-up}
    Table~\ref{tab:params} -- Model parameters\\
    Fig.~\ref{fig:inputs} -- Climate and geothermal model inputs.\\

    \begin{itemize}
    \item Ice dynamics are approximated by SSA+SIA.
    \item Ice temperature follows an enthalpy scheme.
    \item Bedrock temperature computed until 3\,km depth
    \item Temperature conditioned by geothermal flux (Fig.~\ref{fig:inputs}d).
    \item Basal sliding follwos a pseudo-plastic law.
    \item Yield stress is determined by a Mohr-Coulomb criterion.
    \item Till friction angle is constant.
    \item Till effective pressure relates to basal melt.
    \item Surface mass balance is computed by a PDD model.
    \item Modern T and P forcing from WorldClim (Fig.~\ref{fig:inputs}a and~b).
    \item Modern PDD SD forcing from ERA-Interim (Fig.~\ref{fig:inputs}c).
    \item SRTM topography with glaciers removed \citep{Huss.Farinotti.2012}.
    \end{itemize}

\subsection{Palaeo-climate forcing}
    Table~\ref{tab:records} -- Palaeo-climate records.\\
    Fig.~\ref{fig:timeseries} -- Low-resolution time series.\\
    Fig.~\ref{fig:footprints} -- Low-resolution ice cover.

    \begin{itemize}
    \item We conducted a 5\,km-resolution sensitivity study.
    \item We used three palaeo-temperature records (Table~\ref{tab:records}).
    \item All records scaled to fit the LGM extent (Table~\ref{tab:records}).
    \item All yield glaciation during MIS~4 and 2 (Fig.~\ref{fig:timeseries}).
    \item But otherwise very different results (Fig.~\ref{fig:timeseries}).
    \item Two yield two large MIS~4 ice extent (Fig.~\ref{fig:footprints}).
    \item Only EPICA yield realistic MIS~4 (Fig.~\ref{fig:footprints}).
    \end{itemize}

    \note{An important assumption in the method is that the MIS~4 glaciation
          was less (or equally) extensive than the MIS~2, i.e. that the LGM
          outline from \citet{Ehlers.etal.2011} dates indeed from the LGM
          (MIS~2). A different assumption could be that this outline represents
          the most extensive glaciation in the last 120\,ka. I we allow the
          MIS~2 glaciation to be a bit smaller than MIS~4, simulations driven
          by GRIP and MD01-2444 also give reasonable results.}

% ----------------------------------------------------------------------
% References
% ----------------------------------------------------------------------

\bibliographystyle{abbrvnat}
\bibliography{../../references/references}

% ----------------------------------------------------------------------
% Figures
\clearpage
% ----------------------------------------------------------------------

    \begin{figure}
      \centerline{\includegraphics{alpcyc_hr_lgmvel}}
      \caption{%
        \textbf{(a)} Modelled bedrock topography (grey) ice surface topography
        (200\,m contours) and ice surface velocity (blue) in the Alps
        21~thousand years (ka) before present. Modelled Last Glacial Maximum
        (LGM) ice extent (dashed orange line) and geomorphological
        reconstruction \citep[solid red line,][]{Ehlers.etal.2011}. The
        background map consists of depressed SRTM \citep{Jarvis.etal.2008}
        topography and Natural Earth Data \citep{Patterson.Kelso.2015}.
        \textbf{(b)} Temperature offset time-series from the EPICA ice core
        used as palaeo-climate forcing for the ice flow model \citep[black
        curve,][]{Jouzel.etal.2007}, and modelled total ice volume through the
        last glacial cycle (120--0\,ka), expressed in meters of sea level
        equivalent (m~s.l.e., blue curve). Gray fields indicate Marine
        Oxygen Isotope Stage (MIS) boundaries for MIS~2 and MIS~4 according to
        a~global compilation of benthic \chem{\delta^{18}O} records
        \citep{Lisiecki.Raymo.2005}.}
      \label{fig:lgmvel}
    \end{figure}

    \begin{figure}
      \centerline{\includegraphics{alpcyc_hr_timing}}
      \caption{%
        \textbf{(a)} Timing of the Last Glacial Maximum (LGM) given by the
        modelled age (colour mapping) and value (200 m contours) of maximum
        surface elevation throughout the entire simulation.
        \textbf{(b)} Temperature offset time-series from the EPICA ice core
        used as palaeo-climate forcing for the ice flow model (black curve),
        and modelled glaciated area around the LGM (coloured curve). The LGM
        is here modelled as a time-transgressive event.
        \note{The Nature Geoscience format is limited to four figures. Now we
              indeed have four figures, albeit large ones.
              Figure~\ref{fig:timing} will interest the dating people but is
              not crucial for the paper. My plan is to include in the main text
              for the first submission and be prepared that the editor may want
              to move it into supplementary material.}}
      \label{fig:timing}
    \end{figure}

    \begin{figure}
      \centerline{\includegraphics{alpcyc_hr_profiles}}
      \caption{%
        Modelled extent of glaciation along selected profiles for the Lyon,
        Solothurn, Rhine and Ivrea glaciers.
        \todo{Implement extraction of ice thickness along a more detailed
              profile, and add the other glaciers.}
        \idea{We can also move this as panel b on Fig.~\ref{fig:timing}, with
              profiles drawn on panel a. But then we should probably limit it
              to two glaciers instead of four.}}
      \label{fig:profiles}
    \end{figure}

    \begin{figure}
      \centerline{\includegraphics{alpcyc_hr_erosion}}
      \caption{%
        \textbf{(a)} Modelled total erosion integrated in time over the entire
        simulation (120--0\,ka) is highest in the deep valleys and cirques of
        the western Alps.
        \textbf{(b)} Temperature offset time-series from the EPICA ice core
        used as palaeo-climate forcing for the ice flow model (black curve),
        and modelled erosion rate integrated in space over the entire
        ice-covered area. The local erosion rate, $\dot{e}$, is computed from
        the sliding velocity, $\vec{v}_{\mathrm{b}}$, through
        $\dot{e} = K_g \cdot |\vec{v}_{\mathrm{b}}|^{l}$, with
        $l = 2.02$ and $K_g = 2.7\e{-7}\,m^{1-l}\,a^{l-1}$
        \citep{Herman.etal.2015}.}
      \label{fig:erosion}
    \end{figure}


% ----------------------------------------------------------------------
% Supplementary tables
\clearpage
\setcounter{table}{0}
\renewcommand{\thetable}{S\arabic{table}}%
% ----------------------------------------------------------------------

    \begin{table*}
      \caption{%
        Parameter values used in the ice sheet model.
        \idea{If I need to re-run all simulations, we could rethink bedrock,
              lithosphere and PDD parameters.}}
      \label{tab:params}
      \noindent\small\makebox[\textwidth]
      {\begin{tabular}{llrll}
        \toprule

        Not.    & Name & Value & Unit & Source \\

        \midrule
        \multicolumn{2}{l}{{Ice rheology}} \\
        \midrule

        $\rho$  & Ice density
                & 910
                & \unit{kg\,m^{-3}}
                & \citet{Aschwanden.etal.2012} \\

        $g$     & Standard gravity
                & 9.81
                & \unit{m\,s^{-2}}
                & \citet{Aschwanden.etal.2012} \\

        $n$     & Glen exponent
                & 3
                & --
                & \citet{Cuffey.Paterson.2010} \\

        $A_{\mathrm{c}}$   & Ice hardness coefficient cold
                & $2.847\e{-13}$
                & \unit{Pa^{-3}\,s^{-1}}
                & \citet{Cuffey.Paterson.2010} \\

        $A_{\mathrm{w}}$   & Ice hardness coefficient warm
                & $2.356\e{-2}$
                & \unit{Pa^{-3}\,s^{-1}}
                & \citet{Cuffey.Paterson.2010} \\

        $Q_{\mathrm{c}}$   & Flow law activation energy cold
                & $6.0\e4$
                & \unit{J\,mol^{-1}}
                & \citet{Cuffey.Paterson.2010} \\

        $Q_{\mathrm{w}}$   & Flow law activation energy warm
                & $11.5\e4$
                & \unit{J\,mol^{-1}}
                & \citet{Cuffey.Paterson.2010} \\

        $E_{\text{SIA}}$   & SIA enhancement factor
                & 5
                & --
                & \citet{Cuffey.Paterson.2010} \\

        $E_{\text{SSA}}$   & SSA enhancement factor
                & 1
                & --
                & \citet{Cuffey.Paterson.2010} \\

        $T_{\mathrm{c}}$   & Flow law critical temperature
                & 263.15
                & \unit{K}
                & \citet{Paterson.Budd.1982} \\

        $f$     & Flow law water fraction coeff.
                & 181.25
                & --
                & \citet{Lliboutry.Duval.1985} \\

        $R$     & Ideal gas constant
                & 8.31441
                & \unit{J\,mol^{-1}\,K^{-1}}
                & -- \\

        $\beta$ & Clapeyron constant
                & $7.9\e{-8}$
                & \unit{K\,Pa^{-1}}
                & \citet{Luthi.etal.2002} \\

        $c_{\mathrm{i}}$   & Ice specific heat capacity
                & 2009
                & \unit{J\,kg^{-1}\,K^{-1}}
                & \citet{Aschwanden.etal.2012} \\

        $c_{\mathrm{w}}$   & Water specific heat capacity
                & 4170
                & \unit{J\,kg^{-1}\,K^{-1}}
                & \citet{Aschwanden.etal.2012} \\

        $k$     & Ice thermal conductivity
                & 2.10
                & \unit{J\,m^{-1}\,K^{-1}\,s^{-1}}
                & \citet{Aschwanden.etal.2012} \\

        $L$     & Water latent heat of fusion
                & $3.34\e5$
                & \unit{J\,kg^{-1}\,K^{-1}}
                & \citet{Aschwanden.etal.2012} \\

        \midrule
        \multicolumn{2}{l}{{Basal sliding}} \\
        \midrule

        $q$     & Pseudo-plastic sliding exponent
                & 0.25
                & --
                & \citet{Aschwanden.etal.2013} \\

        $v_{\text{th}}$& Pseudo-plastic threshold velocity
                & 100
                & \unit{m\,a^{-1}}
                & \citet{Aschwanden.etal.2013} \\

        $c_0$   & Till cohesion
                & 0
                & Pa
                & \citet{Tulaczyk.etal.2000} \\

        $e_0$   & Till reference void ratio
                & 0.69
                & --
                & \citet{Tulaczyk.etal.2000} \\

        $C_{\mathrm{c}}$   & Till compressibility coefficient
                & 0.12
                & --
                & \citet{Tulaczyk.etal.2000} \\

        $\delta$& Minimum effective pressure ratio
                & 0.02
                & --
                & \citet{Bueler.Pelt.2015} \\

        $\phi$  & Till friction angle
                & 30
                & \degree
                & \citet{Cuffey.Paterson.2010} \\

        $W_{\text{max}}$ & Maximum till water thickness
                & 2
                & m
                & \citet{Bueler.Pelt.2015} \\

        \midrule
        \multicolumn{2}{l}{{Bedrock and lithosphere}} \\
        \midrule

        $\rho_{\mathrm{b}}$& Bedrock density
                & 3300
                & \unit{kg\,m^{-3}}
                & -- \\

        $c_{\mathrm{b}}$   & Bedrock specific heat capacity
                & 1000
                & \unit{J\,kg^{-1}\,K^{-1}}
                & -- \\

        $k_{\mathrm{b}}$   & Bedrock thermal conductivity
                & 3
                & \unit{J\,m^{-1}\,K^{-1}\,s^{-1}}
                & -- \\

        $\nu_{\mathrm{m}}$ & Astenosphere viscosity
                & $1\e{19}$
                & \unit{Pa\,s}
                & \citet{James.etal.2009} \\

        $\rho_{\mathrm{l}}$& Lithosphere density
                & 3300
                & \unit{kg\,m^{-3}}
                & \citet{Lingle.Clark.1985} \\

        $D$     & Lithosphere flexural rigidity
                & $5\e{24}$
                & \unit{N}
                & \citet{Lingle.Clark.1985} \\

        \midrule
        \multicolumn{2}{l}{{Surface and atmosphere}} \\
        \midrule

        $T_{\mathrm{s}}$   & Temperature of snow precipitation
                & 273.15
                & \unit{K}
                & -- \\

        $T_{\mathrm{r}}$   & Temperature of rain precipitation
                & 275.15
                & \unit{K}
                & -- \\

        $F_{\mathrm{s}}$   & Degree-day factor for snow
                & $3.297\e{-3}$
                & \unit{m\,K^{-1}\,day^{-1}}
                & \citet{Huybrechts.1998} \\

        $F_{\mathrm{i}}$   & Degree-day factor for ice
                & $8.791\e{-3}$
                & \unit{m\,K^{-1}\,day^{-1}}
                & \citet{Huybrechts.1998} \\

        $R$     & Refreezing fraction
                & 0.0
                & --
                & -- \\

        $\gamma$& Air temperature lapse rate
                & $6\e{-3}$
                & \unit{K\,m{-1}}
                & -- \\

        \bottomrule
      \end{tabular}}
    \end{table*}

    \begin{table*}
      \caption{%
        Palaeo-temperature proxy records and scaling factors yielding
        temperature offset time-series used to force the ice sheet model
        through the last glacial cycle (Fig.~\ref{fig:timeseries}). $f$
        corresponds to the scaling factor adopted to yield Last Glacial Maximum
        ice limits in the vicinity of mapped end moraines
        (Fig.~\ref{fig:footprints}a), and $[{\Delta}T_{\textrm{TS}}]_{32}^{22}$
        refers to the resulting mean temperature anomaly during the period 32
        to~22\,\unit{ka} after scaling.
        \note{The EPICA scaling factor is close to one, and that ended up
              yielding too much ice in the high-resolution simulation
              (Fig.~\ref{fig:lgmvel}), though not so at low resotion
              (Fig.~\ref{fig:footprints}b). If I re-run the whole thing, I will
              probably fix it to one (i.e. use EPICA temperature anomalies
              without applying any scaling).}}
      \label{tab:records}
      \noindent\small\makebox[\textwidth]
      {\begin{tabular}{lccccccl}
        \toprule

        Record    & Latitude & Longitude & Elev. (m~a.s.l.)
                  & Proxy & $f$ & $[{\Delta}\text{TS}]_{32}^{22}$ (K)
                  & Reference\\

        \midrule

        GRIP      &  72{\degree}35$^{\prime}$\,N   % 72.58 (decimal)
                  &  37{\degree}38$^{\prime}$\,W   % 37.64 (decimal)
                  & 3238
                  & \chem{\delta^{18}O}
                  & 0.46 & $-$7.6  % -16.4126 (before scaling)
                  & \citet{Dansgaard.etal.1993} \\

        EPICA     &  75{\degree}06$^{\prime}$\,S   % 75.1
                  & 123{\degree}21$^{\prime}$\,E   % 123.35
                  & 3233
                  & \chem{\delta^{18}O}
                  & 1.03 & $-$9.5  % -9.2055
                  & \citet{Jouzel.etal.2007} \\

        MD01-2444 &  37{\degree}34$^{\prime}$\,N   % 37.561
                  &  10{\degree}04$^{\prime}$\,W   % -10.142
                  & $-$2637
                  & \chem{U^{K'}_{37}}
                  & 1.84 & $-$8.0  % -4.345625
                  & \citet{Martrat.etal.2007} \\

        \bottomrule
      \end{tabular}}
    \end{table*}


% ----------------------------------------------------------------------
% Supplementary figures
\clearpage
\setcounter{figure}{0}
\renewcommand{\thefigure}{S\arabic{figure}}%
% ----------------------------------------------------------------------

    \begin{figure}
      \centerline{\includegraphics{alpcyc_hr_inputs}}
      \caption{%
        \textbf{(a)} July mean near-surface air temperature and
        \textbf{(b)} January precipitation from WorldClim
        \citep[1960--1990]{Hijmans.etal.2005}, and
        \textbf{(c)} Mordern July standard deviation of daily mean temperature
        from the ERA-Interim \citep[1979--2012]{Dee.etal.2011} from the
        reference monthly climatology used to force the surface mass balance
        (PDD) component of the ice sheet model.
        \textbf{(d)} Geothermal heat flow from applying the similarity method
        to multiple geophysical proxies \citep{Goutorbe.etal.2011} used as a
        boundary condition to the bedrock thermal model 3\,km below the
        ice-bedrock interface.
        \idea{Since this will be in the supplement, I can also add additional
              information such as winter temperature, summer precipitation,
              winter PDD SD, and/or basal topography with modern glaciers
              removed, e.g. in the Bernese Alps.}}
      \label{fig:inputs}
    \end{figure}

    \begin{figure}
      \centerline{\includegraphics{alpcyc_lr_timeseries}}
      \caption{%
        \textbf{(a)} Temperature offset time-series from ice core and ocean
        records (Table~\ref{tab:records}) used as palaeo-climate forcing for
        the ice sheet model.
        \textbf{(a)} Modelled total ice volume through the last 120\,ka,
        expressed in meters of sea level equivalent (m~s.l.e.). Gray fields
        indicate Marine Oxygen Isotope Stage (MIS) boundaries for MIS~2 and
        MIS~4 according to a~global compilation of benthic \chem{\delta^{18}O}
        records \citep{Lisiecki.Raymo.2005}.}
      \label{fig:timeseries}
    \end{figure}

    \begin{figure}
      \centerline{\includegraphics{alpcyc_lr_footprints}}
      \caption{%
        \textbf{(a--c)} Cumulative extent of modelled ice cover during MIS~2
        (29--14\,ka) using temperature time-series scaling factors
        (Table~\ref{tab:records}) adjusted to obtain model results in agreement
        with the Last Glacial Maximum (LGM) geomorphological reconstruction
        \citep[solid red line,][]{Ehlers.etal.2011}.
        \textbf{(d--f)} Cumulative extent of modelled ice cover during MIS~4
        (71--57\,ka). Only the simulation driven by the EPICA temperature
        time-series yields reasonable MIS~4 ice cover.}
      \label{fig:footprints}
    \end{figure}

% ======================================================================
\end{document}
% ======================================================================
