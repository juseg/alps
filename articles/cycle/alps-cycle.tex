\documentclass{article}

\usepackage{doi}
\usepackage{amsmath}
\usepackage{authblk}
\usepackage{booktabs}
\usepackage{multirow}
\usepackage[T1]{fontenc}
\usepackage[utf8]{inputenc}
\usepackage[pdftex]{xcolor}
\usepackage[pdftex]{graphicx}
\usepackage[authoryear,round]{natbib}

\graphicspath{{../../figures/}}

\definecolor{darkblue}{cmyk}{0.9,0.3,0.0,0.0}
\definecolor{darkgreen}{cmyk}{0.8,0.0,1.0,0.0}
\definecolor{darkred}{cmyk}{0.1,0.9,0.8,0.0}

\newcommand{\idea}[1]{\textcolor{darkgreen}{\emph{[\textbf{IDEA:} #1]}}}
\newcommand{\note}[1]{\textcolor{darkblue}{\emph{[\textbf{NOTE:} #1]}}}
\newcommand{\todo}[1]{\textcolor{darkred}{\emph{[\textbf{TODO:} #1]}}}
\newcommand{\aref}[0]{\textcolor{darkblue}{\textbf{[REF.]}}}

\title{Modelling last glacial cycle ice dynamics in the Alps}

\author[1]{Julien Seguinot%
           \thanks{Correspondence to seguinot@vaw.baug.ethz.ch}}
\author[1]{Guillaume Jouvet}
\author[1]{Matthias Huss}
\author[1]{Martin Funk}
\author[2]{Frank Preusser}

\affil[1]{Laboratory of Hydraulics, Hydrology and Glaciology,
          ETH Zürich, Switzerland}
\affil[2]{Institute of Earth and Environmental Sciences,
          University of Freiburg, Germany}

% Common units
\newcommand{\e}[1]{\ensuremath{\times 10^{#1}}}
\newcommand{\chem}[1]{\ensuremath{\mathrm{#1}}}
\newcommand{\unit}[1]{\ensuremath{\mathrm{#1}}}
\newcommand{\degree}[0]{\ensuremath{^{\circ}}}
\newcommand{\degC}[0]{\unit{{\degree}C}}


% ======================================================================
\begin{document}
% ======================================================================

\maketitle

\begin{abstract}

    The European Alps, cradle of pioneer glacial studies, are one of the
    regions where geological markers of past glaciations are most abundant and
    well-studied. Such conditions make the region ideal for testing numerical
    glacier models based on approximated ice flow physics against field-based
    reconstructions, and vice-versa.

    Here, we use the Parallel Ice Sheet Model (PISM) to model the entire last
    glacial cycle (120--0\,ka) in the Alps, with a horizontal resolution of
    1\,km. Climate forcing is derived using present-day climate data from
    WorldClim and the ERA-Interim reanalysis, and time-dependent temperature
    offsets from multiple paleo-climate proxies, among which only the EPICA ice
    core record yields glacial extent during marine oxygen isotope stages~4
    (69--62\,ka) and~2 (34--18\,ka) in agreement to geological reconstructions.

    Despite the low variability of this Antarctic-based climate forcing, our
    simulation depicts a highly dynamic ice cap, showing that alpine glaciers
    may have advanced many times over the foreland during the last glacial
    cycle. Cumulative basal sliding, a proxy for glacial erosion, is modelled
    to be highest in the deep valleys of the western Alps. Finally, the Last
    Glacial Maximum advance, often considered synchronous, is here modelled as
    a time-transgressive event, with some glacier lobes reaching their maximum
    as early as 27\,ka, and some as late as 21\,ka. Modelled ice thickness is
    about 900\,m higher than observed trimline elevations, yet our simulation
    predicts little erosion at high elevation due to cold ice conditions.

\end{abstract}


% ----------------------------------------------------------------------
\section{Introduction}
% ----------------------------------------------------------------------

    \begin{itemize}
    \item The European Alps are the cradle of pioneer glacial studies.
    \item Today, their glacial history is better understood than any other.
    \item The LGM extent is very well known (Fig.~\ref{fig:lgmvel}a).
    \item But modelling was difficult due to the steep topography.
    \item Here, we us PISM to model the last glacial cycle (120--0\,ka).
    \end{itemize}

    \idea{Matthias: I think it is very important to emphasize the knowledge gap
          that you fill in with your work in the Intro. (1) it is incompletely
          known how many advances occurred during the Last glacial cycle, (2)
          what drove the different response of the individual lobes (you show
          that it is glacier response time, i.e. ice dynamics), (3) how the
          temporal dynamics of subglacial erosion was structured.}


% ----------------------------------------------------------------------
\section{Ice sheet model set-up}
% ----------------------------------------------------------------------

    Table~\ref{tab:params} -- Model parameters\\
    Fig.~\ref{fig:inputs} -- Climate and geothermal model inputs.\\

    \begin{itemize}
    \item Ice dynamics are approximated by SSA+SIA.
    \item Ice temperature follows an enthalpy scheme.
    \item Bedrock temperature computed until 3\,km depth
    \item Temperature conditioned by geothermal flux (Fig.~\ref{fig:inputs}d).
    \item Basal sliding follwos a pseudo-plastic law.
    \item Yield stress is determined by a Mohr-Coulomb criterion.
    \item Till friction angle is constant.
    \item Till effective pressure relates to basal melt.
    \item Surface mass balance is computed by a PDD model.
    \item Modern T and P forcing from WorldClim (Fig.~\ref{fig:inputs}a and~b).
    \item Modern PDD SD forcing from ERA-Interim (Fig.~\ref{fig:inputs}c).
    \item SRTM topography with glaciers removed \citep{Huss.Farinotti.2012}.
    \end{itemize}


\subsection{Overview}
\label{sec:overview}%

    We use the Parallel Ice Sheet Model (PISM, development version~e9d2d1f), an
    open source, finite difference, shallow ice sheet model
    \citep{PISM-authors.2017}. The model requires input on basal
    topography, geothermal heat flux and climate forcing. It computes the
    evolution of ice extent and thickness over time, the thermal and dynamic
    states of the ice sheet, and the associated lithospheric response. The
    set-up used here is largely based on that used in previous studies of the
    former Cordilleran ice sheet \citep{Seguinot.2014, Seguinot.etal.2014,
    Seguinot.etal.2016}.

    \idea{Remove next paragraph?}

    Ice deformation follows temperature and water-content dependent creep
    (Sect.~\ref{sec:icedyn}). Basal sliding follows a~pseudo-plastic law where
    the yield stress accounts for till deconsolidation under high water
    pressure (Sect.~\ref{sec:sliding}). Bedrock topography is deflected
    under the ice load (Sect.~\ref{sec:bedrock}). Surface mass balance is
    computed using a~positive degree-day (PDD) model (Sect.~\ref{sec:surface}).
    Climate forcing is provided by a~monthly climatology from interpolated
    observational data \citep[WorldClim;][]{Hijmans.etal.2005} and the European
    Centre for Medium-Range Weather Forecasts Reanalysis Interim
    \citep[ERA-Interim;][]{Dee.etal.2011}, perturbed by time-dependent
    temperature offsets, temperature lapse-rate corrections, and in some cases,
    time-dependent paleo-precipitation reductions (Sect.~\ref{sec:atm},
    Table~\ref{tab:records}).

    \idea{Remove next paragraph?}

    Each simulation starts from assumed present-day ice thickness and
    equilibrium temperature distribution at 120\,\unit{ka}, and runs to the
    present. Our modelling domain of 900 by 600\,\unit{km} encompasses the
    entire Alpine range (Fig.~\ref{fig:inputs}). The simulations were run on
    two distinct grids, using a~lower horizontal resolution of 2\,\unit{km},
    and a~higher horizontal resolution of 1\,\unit{km}.

\subsection{Ice rheology}
\label{sec:icedyn}

      Ice sheet dynamics are typically modelled using a~combination of
      internal deformation and basal sliding. PISM is a~shallow ice sheet
      model, which implies that the balance of stresses is approximated
      based on their predominant components. The Shallow Shelf Approximation
      (SSA) is combined with the Shallow Ice Approximation
      (SIA) by adding velocity solutions of the two approximations
      \citep[Eqs.~7--9 and 15]{Winkelmann.etal.2011}. Although this
      heuristic approach implies errors in the transition zone where gravitational
      stresses intervene both in the SIA and SSA velocity computation, this
      hybrid scheme is computationally much more efficient than a fully
      three-dimensional model with which the simulations presented here
      would not be feasible.

      Ice deformation is governed by the constitutive law for ice
      \citep{Glen.1952, Nye.1953},
%
\begin{align}
&\label{eqn:glenslaw}
&\vec{\dot{\epsilon}} = A\,\tau_{\mathrm{e}}^{n-1}\,\vec{\tau} \,.
\end{align}
%
      where $\vec{\dot{\epsilon}}$ is the strain-rate tensor,
      $\vec{\tau}$ the deviatoric stress tensor, and $\tau_{\mathrm{e}}$ the
      effective stress defined in our case by ${\tau_{\mathrm{e}}}^2 = \frac{1}{2}
      \mathrm{tr}(\vec{\tau}^2)$. The ice softness coefficient, $A$, depends
      on ice temperature, $T$, pressure, $p$, and water content, $\omega$,
      through a~piece-wise Arrhenius-type law,
%
\begin{align}
&\label{eqn:softness}
&A = E\cdot
\begin{cases}
A_{\mathrm{c}} \,e^\frac{-Q_{\mathrm{c}}}{RT_{\text{pa}}}            & \text{if}\ T_{\text{pa}}  <  T_{\mathrm{c}} \,, \\
A_{\mathrm{w}} (1+f\omega)\,e^\frac{-Q_{\mathrm{w}}}{RT_{\text{pa}}} & \text{if}\ T_{\text{pa}} \ge T_{\mathrm{c}} \,,
\end{cases}
\end{align}
%
      where $T_{\text{pa}}$ is the pressure-adjusted ice temperature
      calculated using the Clapeyron relation, ${T_{\text{pa}} = T - \beta
      p}$. $R=8.31441$\,\unit{J\,mol^{-1}\,K^{-1}} is the ideal gas
      constant, and $A_{\mathrm{c}}$, $A_{\mathrm{w}}$, $Q_{\mathrm{c}}$ and
      $Q_{\mathrm{w}}$, are constant parameters corresponding to values
      measured below and above a~critical temperature threshold
      $T_{\mathrm{c}}=-10$\,\unit{{\degree}C}
      \citep[p.~72]{Paterson.Budd.1982,Cuffey.Paterson.2010}. The water
      fraction, $\omega$, is capped at a~maximum value of 0.01, above which
      no measurements are available \citep[Eq.~5.7]{Lliboutry.Duval.1985,
      Greve.1997}. Finally, $E$ is a~non-dimensional enhancement factor
      which can take different values, $E_{\text{SIA}}$, in the SIA and
      $E_{\text{SSA}}$, in the SSA.

      In all our simulations, we set constant the power-law exponent, $n=3$,
      according to \citet[p.~55--57]{Cuffey.Paterson.2010}, the Clapeyron
      constant, $\beta=7.9\times 10^{-8}$\,\unit{K\,Pa^{-1}}, according to
      \citet{Luthi.etal.2002}, the water fraction coefficient, $f=181.25$,
      according to \citet{Lliboutry.Duval.1985}, and the SSA enhancement
      factor, $E_{\text{SSA}}=1$, according to
      \citet[p.~77]{Cuffey.Paterson.2010}. These fixed parameter values are
      summarized in Table~\ref{tab:params}.

      On the other hand, we test the model sensitivity
      (Sect.~\ref{sec:sens}) to different values for the two creep
      parameters, $A_{\mathrm{c}}$ and $A_{\mathrm{w}}$, the two activation
      energies, $Q_{\mathrm{c}}$ and $Q_{\mathrm{w}}$, and the SIA
      enhancement factor, $E_{\text{SIA}}$, as follows.
%
\begin{itemize}
  \item Our \emph{default} configuration used in the control run of the
    sensitivity study and all other simulations in the paper includes
    rheological parameters,
    $A_{\mathrm{c}}$, $A_{\mathrm{w}}$, $Q_{\mathrm{c}}$ and
    $Q_{\mathrm{w}}$, derived from \citet{Paterson.Budd.1982} and given in
    \citet[Eqn.~5]{Bueler.Brown.2009}, and $E_{\text{SIA}}=1$.
  \item Our \emph{hard ice} configuration includes rheological parameters,
    $A_{\mathrm{c}}$, $A_{\mathrm{w}}$, $Q_{\mathrm{c}}$ and
    $Q_{\mathrm{w}}$, derived from \citet[p.~72 and
    76]{Cuffey.Paterson.2010}, and $E_{\text{SIA}}=1$, which correspond to
    a~stiffer rheology than that used in the control run.
  \item Our \emph{soft ice} configuration includes rheological parameters
    from \citet{Cuffey.Paterson.2010}, and $E_{\text{SIA}}=5$, the
    recommended value for ice age polar ice
    \citep[p.~77]{Cuffey.Paterson.2010}.
\end{itemize}
%
      An additional simulation using the ice rheology from
      \citet{Cuffey.Paterson.2010} and $E_{\text{SIA}}=2$, the recommended
      value for Holocene polar ice \citep[p.~77]{Cuffey.Paterson.2010} has
      been performed, but is not presented here, since its results were very
      similar to that of our default run.

      Actual parameter values for $A_{\mathrm{c}}$, $A_{\mathrm{w}}$,
      $Q_{\mathrm{c}}$, $Q_{\mathrm{w}}$ and $E_{\text{SIA}}$ used in our
      simulations are given in Table~\ref{tab:sens_params}, while the effect
      of the three different parametrisations on temperature-dependent ice
      softness, $A$, is illustrated in Fig.~\ref{fig:sens_plot_rheo}.

      Surface air
      temperature derived from the climate forcing (Sect.~\ref{sec:atm})
      provides the upper boundary condition to the ice enthalpy
      model. Temperature is computed in the ice and in the bedrock to
      a~depth of 3\,\unit{km} below the ice-bedrock interface, where it is
      conditioned by a~lower boundary geothermal heat flux of
      $q_{\mathrm{G}}=70$\,\unit{mW\,m^{-2}}. Although this uniform value
      does
      not account for the high spatial geothermal variability in the region
      \citep{Blackwell.Richards.2004}, it is, on average, representative of
      available heat flow measurements. In the low-resolution simulations,
      the vertical grid consists of 31~temperature layers in the bedrock and
      up to 51~enthalpy layers in the ice sheet, corresponding to a~vertical
      resolution of 100\,\unit{m}. The high-resolution simulations use
      61~bedrock layers and up to 101~ice layers with a~vertical resolution
      of 50\,\unit{m}.

\subsection{Basal sliding}
\label{sec:sliding}

      A~pseudo-plastic sliding law,
%
\begin{align}
&\label{eqn:pseudoplastic}
    \vec{\tau}_{\mathrm{b}} = -\tau_{\mathrm{c}} \frac{\vec{v}_{\mathrm{b}}}{{v_{\text{th}}}^q\,|\vec{v}_{\mathrm{b}}|^{1-q}} \,,
\end{align}
%
      relates the bed-parallel shear stresses, $\vec{\tau}_{\mathrm{b}}$, to
      the sliding velocity, $\vec{v}_{\mathrm{b}}$.
      The yield stress, $\tau_{\mathrm{c}}$, is modelled using the
      Mohr--Coulomb criterion,
%
\begin{align}
&\tau_{\mathrm{c}} = c_0 + N\,\tan{\phi} \,,
\end{align}
%
      where cohesion, $c_0$, is assumed to be zero. Effective pressure, $N$,
      is related to the ice overburden stress, $P_0=\rho gh$, and the
      modelled amount of
      subglacial water, using a~formula derived from laboratory experiments
      with till extracted from the base of Ice Stream B in West Antarctica
      \citep{Tulaczyk.etal.2000, Bueler.Pelt.2015},
%
\begin{align}
&\label{eqn:ntil}
&N = \delta P_0 \, 10^{(e_0/C_{\mathrm{c}}) (1 - (W/W_{\text{max}}))} \,,
\end{align}
%
      where $\delta$ sets the minimum ratio between the effective and
      overburden pressures, $e_0$ is a~reference void ratio and
      $C_{\mathrm{c}}$ is the till compressibility coefficient
      \citep{Tulaczyk.etal.2000}. The amount of water at the base, $W$,
      varies from zero to $W_{\text{max}}$, a~threshold above which
      additional melt water is assumed to drain off instantaneously.

      In all our simulations, we set constant the pseudo-plastic sliding
      exponent, $q=0.25$, and the threshold velocity,
      $v_{\text{th}}=100$\,\unit{m\,a^{-1}}, according to values used by
      \citet{Aschwanden.etal.2013}, the till cohesion, $c_0=0$, whose
      measured values are consistently negligible
      \citep[p.~268]{Tulaczyk.etal.2000, Cuffey.Paterson.2010}, the till
      reference void ratio, $e_0=0.69$, and the till compressibility
      coefficient, $C_{\mathrm{c}}=0.12$, according to the only measurements
      available to our knowledge, published by \citet{Tulaczyk.etal.2000}.
      These fixed parameter values are summarized in Table~\ref{tab:params}.

      We also use a~constant spatial distribution of the till friction
      angle, $\phi$, whose values vary from 15 to 45\unit{\degree} as
      a~piecewise-linear function of modern bed elevation, with the lowest
      value occuring below the modern sea level (0\,\unit{m} above sea level,
      \unit{m\,a.s.l.}) and the highest value occuring above the generalised
      elevation of the highest shorelines
      \citep[200\,\unit{m\,a.s.l.},][Fig.~5]{Clague.1981}. This range of
      values span over the range of measured values for glacial till of 18
      to 40\unit{\degree} \citep[p.~268]{Cuffey.Paterson.2010}. It accounts
      for frictional basal conditions associated with discontinuous till
      cover at high elevations, and for a~weakening of till associated with
      the presence of marine sediments (cf. \citealp{Martin.etal.2011};
      \citealp[Supplement]{Aschwanden.etal.2013};
      \citealp{PISM-authors.2017}).

      An additional simulation with a~constant till friction angle,
      $\phi=30$\unit{\degree}, corresponding to the average value in
      \citet[p.~268]{Cuffey.Paterson.2010}, has been performed, but is not
      presented here, since the induced variability was small.

      On the other hand, we test the model sensitivity
      (Sect.~\ref{sec:sens}) to different values for the minimum ratio
      between the effective and overburden pressures, $\delta$, and the
      maximum water height in the till, $W_{\text{max}}$, as follows.
%
\begin{itemize}
  \item Our \emph{default} configuration used in the control run of the
    sensitivity study and all other simulations in the paper includes
    $\delta=0.02$ and $W_{\text{max}}=2$\,\unit{m} as in
    \citet{Bueler.Pelt.2015}.
  \item Our \emph{soft bed} configuration use $\delta=0.01$ and
    $W_{\text{max}}=1$\,\unit{m}.
  \item Our \emph{hard bed} configuration use $\delta=0.05$ and
    $W_{\text{max}}=5$\,\unit{m}.
\end{itemize}
%
      The effect of the three different parametrisations on the effective
      pressure on the till, $N$, in response to water content, $W$, is
      illustrated in Fig.~\ref{fig:sens_plot_ntil}. All parameter choices
      are listed in Table~\ref{tab:sens_params}.

\subsection{Basal topography}
\label{sec:bedrock}

    The initial basal topography is bilinearly interpolated from the
    hole-filled, Shuttle Radar Topography Mission (SRTM) data with a~resolution
    of 30\,arc-sec \citep{Jarvis.etal.2008}. These data include post-glacial
    sediment fills and lakes surface topography. However, they were corrected
    for an estimation of present-day ice thickness according from modern
    glacier outlines, surface topography and simplified ice physics
    \citep{Huss.Farinotti.2012}.

    Basal topography responds to ice load following a~bedrock deformation model
    that includes local isostasy, elastic lithosphere flexure and viscous
    astenosphere deformation in an infinite half-space
    \citep{Lingle.Clark.1985,Bueler.etal.2007}. The astenosphere viscosity,
    $\nu_{\mathrm{m}}=2.2\times10^{20}$\,\unit{Pa\,s}, the astenosphere
    density, $\rho_{\mathrm{m}}=3300$\,\unit{kg\,m^{-3}}, and the lithosphere
    elastic rigidity, $D=1.389\e{24}$\,\unit{N\,m}, were set according to
    results from glacial isostatic adjustment modelling of deglacial rebound in
    the Alps most closely reproducing observed modern uplift rates
    \citep[Table~\ref{tab:params};][Supplementary Fig.~7]{Mey.etal.2016}. The
    latter was computed as $D=YE^3/12(1-\nu)$, where $Y=100$\,GPa is Young's
    modulus, $\nu=0.25$ is the Poisson ratio, and $E=50$\,km is the average
    effective elastic thickness of the lithosphere
    \citep[Table~\ref{tab:params};][]{Mey.etal.2016}.

\subsection{Surface mass balance}
\label{sec:surface}

      Ice surface accumulation and ablation are computed from monthly mean
      near-surface air temperature, $T_{\mathrm{m}}$, monthly standard
      deviation of near-surface air temperature, $\sigma$, and monthly
      precipitation, $P_{\mathrm{m}}$, using a~temperature-index model
      \citep[e.g.,][]{Hock.2003}. Accumulation is equal to precipitation
      when air temperatures are below 0\,\unit{{\degree}C}, and decreases to
      zero linearly with temperatures between 0 and 2\,\unit{{\degree}C}.
      Ablation is computed from PDD, defined as an integral of temperatures
      above 0\,\unit{{\degree}C} in one year.

      The PDD computation accounts for stochastic temperature variations by
      assuming a~normal temperature distribution of standard deviation
      $\sigma$ around the expected value $T_{\mathrm{m}}$. It is expressed
      by an error-function formulation \citep{Calov.Greve.2005},
\begin{align}
&\label{eqn:calovgreve}
    {\text{PDD}} = \int_{t_1}^{t_2} \mathrm{d}t
        \left[\frac{\sigma}{\sqrt{2\pi}}
                \exp\left({-\frac{T_{\mathrm{m}}^2}{2\sigma^2}}\right)
              + \frac{T_{\mathrm{m}}}{2} \, {\text{erfc}}
                \left(-\frac{T_{\mathrm{m}}}{\sqrt{2}\sigma}\right)\right] \,,
\end{align}
      which is numerically approximated using week-long sub-intervals. In
      order to account for the effects of spatial and seasonal variations of
      temperature variability \citep{Seguinot.2013}, $\sigma$ is computed
      from NARR daily temperature values from 1979 to 2000
      \citep{Mesinger.etal.2006}, including variability associated with the
      seasonal cycle, and bilinearly interpolated to the model grids
      (Fig.~\ref{fig:atm}). Degree-day factors for snow and
      ice melt are derived from mass-balance measurements on contemporary
      glaciers from the Coast Mountains and Rocky Mountains in British
      Columbia \citep[Table~\ref{tab:params};][]{Shea.etal.2009}.

\subsection{Climate forcing}
\label{sec:atm}%

      Climate forcing driving ice sheet simulations consists of
      a~present-day monthly climatology, $\{T_{\mathrm{m}0},
      P_{\mathrm{m}0}\}$, where temperatures are modified by offset time
      series, ${\Delta}T_{\text{TS}}$, and lapse-rate corrections,
      ${\Delta}T_{\text{LR}}$:
\begin{align}
&T_{\mathrm{m}}(t, x, y) = T_{\mathrm{m}0}(x, y) + {\Delta}T_{\text{TS}}(t)
                    + {\Delta}T_{\text{LR}}(t, x, y) \,, \\
&    P_{\mathrm{m}}(t, x, y) = P_{\mathrm{m}0}(x, y) \,.
\end{align}
      The present-day monthly climatology was bilinearly interpolated from
      near-surface air
      temperature and precipitation rate fields from the NARR, averaged from
      1979 to 2000. Modern climate of the North American Cordillera is
      characterised by strong geographic variations in temperature
      seasonality, timing of the maximum annual precipitation, and daily
      temperature variability (Fig.~\ref{fig:atm}). Although the ability
      of the NARR to reproduce the steep climatic gradients is limited by its
      spatial resolution of 32\,\unit{km} \citep{Jarosch.etal.2012}, it has
      been tested against observational data in our previous sensitivity
      study and identified as yielding a closer fit between the modelled LGM
      extent of the Cordilleran ice sheet and the geological evidence than
      other atmospheric reanalyses \citep{Seguinot.etal.2014}.

      Temperature offset time-series, ${\Delta}T_{\text{TS}}$, are derived
      from palaeo-temperature proxy records from the Greenland Ice Core
      Project \citep[GRIP,][]{Dansgaard.etal.1993}, the North Greenland Ice
      Core Project \citep[NGRIP,][]{Andersen.etal.2004}, the European
      Project for Ice Coring in Antarctica \citep[EPICA,][]
      {Jouzel.etal.2007}, the Vostok ice core \citep{Petit.etal.1999}, and
      Ocean Drilling Program (ODP) sites 1012 and 1020, both located off the
      coast of California \citep{Herbert.etal.2001}. Palaeo-temperature
      anomalies from the GRIP and NGRIP records were calculated from oxygen
      isotope (\chem{\delta^{18}O}) measurements using a~quadratic equation
      \citep{Johnsen.etal.1995},
\begin{align}
{\Delta}T_{\text{TS}}(t) ={~}&-11.88 [\chem{\delta^{18}O}(t)
                                -\chem{\delta^{18}O}(0)] \nonumber \\
                        &-0.1925[\chem{\delta^{18}O}(t)^2
                                 -\chem{\delta^{18}O}(0)^2] \,,
\end{align}
      while temperature reconstructions from Antarctic and oceanic cores
      were provided as such. For each proxy record used and each of the
      parameter setup used in the sensitivity tests, palaeo-temperature
      anomalies were scaled linearly (Tables~\ref{tab:records}
      and~\ref{tab:sens_params}) in order to simulate comparable ice extents
      at the LGM (Table~\ref{tab:extrema}) and realistic outlines
      (Fig.~\ref{fig:lr_maps}).

      Finally, lapse-rate corrections, ${\Delta}T_{\text{LR}}$, are computed
      as a~function of ice surface elevation, $s$, using the NARR surface
      geopotential height invariant field as a~reference topography,
      $b_{\text{ref}}$:
\begin{align}
{\Delta}T_{\text{LR}}(t, x, y) &= -\gamma [s(t, x, y)-b_{\text{ref}}] \\
                            &= -\gamma [h(t, x, y)+b(t, x, y)-b_{\text{ref}}],
\end{align}
      thus accounting for the evolution of ice thickness, ${h=s-b}$, on the
      one hand, and for differences between the basal topography of the ice
      flow model, $b$, and the NARR reference topography, $b_{\text{ref}}$,
      on the other hand. All simulations use an annual temperature lapse
      rate of $\gamma = 6\,\unit{K\,km^{-1}}$. In the rest of this paper, we
      refer to different model runs by the name of the proxy record used for
      the palaeo-temperature forcing.


% ----------------------------------------------------------------------
\section{Palaeo-climate forcing}
% ----------------------------------------------------------------------

    Table~\ref{tab:records} -- Palaeo-climate records.\\
    Fig.~\ref{fig:timeseries} -- Low-resolution time series.\\
    Fig.~\ref{fig:footprints} -- Low-resolution ice cover.

    \begin{itemize}
    \item We conducted a 5\,km-resolution sensitivity study.
    \item We used three palaeo-temperature records (Table~\ref{tab:records}).
    \item All records scaled to fit the LGM extent (Table~\ref{tab:records}).
    \item All yield glaciation during MIS~4 and 2 (Fig.~\ref{fig:timeseries}).
    \item But otherwise very different results (Fig.~\ref{fig:timeseries}).
    \item Two yield two large MIS~4 ice extent (Fig.~\ref{fig:footprints}).
    \item Only EPICA yield realistic MIS~4 (Fig.~\ref{fig:footprints}).
    \end{itemize}

    \note{An important assumption in the method is that the MIS~4 glaciation
          was less (or equally) extensive than the MIS~2, i.e. that the LGM
          outline from \citet{Ehlers.etal.2011} dates indeed from the LGM
          (MIS~2). A different assumption could be that this outline represents
          the most extensive glaciation in the last 120\,ka. If we allow the
          MIS~2 glaciation to be a bit smaller than MIS~4, simulations driven
          by GRIP and MD01-2444 also give reasonable results.}


% ----------------------------------------------------------------------
\section{Glacier dynamics}
% ----------------------------------------------------------------------

    Fig.~\ref{fig:lgmvel} -- Snapshot at 21\,ka and volume time series.\\
    Fig.~\ref{fig:timing} -- Timing of the LGM and area time series.\\
    Fig.~\ref{fig:profiles} -- Individual glacier extent profiles.

    \begin{itemize}
    \item Rapid variations of total ice volume (Fig.~\ref{fig:lgmvel}b).
    \item Two major glaciations during MIS 4 and 2 (Fig.~\ref{fig:lgmvel}b).
    \item Maximum stages flow pattern is complex (Fig.~\ref{fig:lgmvel}a).
    \item Maximum extent is farily well reproduced (Fig.~\ref{fig:lgmvel}a).
    \item Maximum extent is time-transgressive (Fig.~\ref{fig:timing}).
    \item Glaciers have different response times (Fig.~\ref{fig:timing}).
    \item Some glaciers advance many times (Fig.~\ref{fig:profiles}).
    \end{itemize}

    \idea{Matthias: The selection of figures in the main text and SOM is good.
          I find Figure 5 quite difficult to understand at first glance. I
          would combine it (already before submission) with Fig. 6. the Figure
          can then focus both on the temporal advance/retreat patterns of the 4
          lobes (four panels on the left side of figure) and the spatial
          differences in max. ice elevation (1 panel on right side of figure,
          with centerlines indicated). I would have the one with trimline
          elevation again in the Supplementary.}

    \idea{Add trimline figure in the main text now.}


% ----------------------------------------------------------------------
\section{Erosion potential}
% ----------------------------------------------------------------------

    Fig.~\ref{fig:erosion} -- Integrated erosion potential.

    \idea{Frank: I would keep this topic for another paper.}

    \begin{itemize}
    \item We calculate erosion after \citet{Herman.etal.2015}
    \item More erosion in valleys of western Alps (Fig.~\ref{fig:erosion}a).
    \item Erosion is constant during 110--15\,ka (Fig.~\ref{fig:erosion}b).
    \end{itemize}


% ----------------------------------------------------------------------
\section{Conclusions}
% ----------------------------------------------------------------------

    \begin{itemize}
    \item Summary of the previous stuff.
    \end{itemize}


% ----------------------------------------------------------------------
% References
% ----------------------------------------------------------------------

\bibliographystyle{abbrvnat}
\bibliography{../../references/references}


% ----------------------------------------------------------------------
% Figures
\clearpage
% ----------------------------------------------------------------------

    \begin{figure}
      \centerline{\includegraphics{alpcyc_hr_inputs}}
      \caption{%
        \textbf{(a)} July mean near-surface air temperature and
        \textbf{(b)} January precipitation from WorldClim
        \citep[1960--1990]{Hijmans.etal.2005}, and
        \textbf{(c)} Mordern July standard deviation of daily mean temperature
        from the ERA-Interim \citep[1979--2012]{Dee.etal.2011} from the
        reference monthly climatology used to force the surface mass balance
        (PDD) component of the ice sheet model.
        \textbf{(d)} Geothermal heat flow from applying the similarity method
        to multiple geophysical proxies \citep{Goutorbe.etal.2011} used as a
        boundary condition to the bedrock thermal model 3\,km below the
        ice-bedrock interface.}
      \label{fig:inputs}
    \end{figure}

    \begin{figure}
      \centerline{\includegraphics{alpcyc_lr_timeseries}}
      \caption{%
        \textbf{(a)} Temperature offset time-series from ice core and ocean
        records (Table~\ref{tab:records}) used as palaeo-climate forcing for
        the ice sheet model.
        \textbf{(a)} Modelled total ice volume through the last 120\,ka,
        expressed in meters of sea level equivalent (m~s.l.e.). Gray fields
        indicate Marine Oxygen Isotope Stage (MIS) boundaries for MIS~2 and
        MIS~4 according to a~global compilation of benthic \chem{\delta^{18}O}
        records \citep{Lisiecki.Raymo.2005}.
        \todo{Refine temperature offsets.}}
      \label{fig:timeseries}
    \end{figure}

    \begin{figure}
      \centerline{\includegraphics{alpcyc_lr_footprints}}
      \caption{%
        \textbf{(a--c)} Cumulative extent of modelled ice cover during MIS~2
        (29--14\,ka) using temperature time-series scaling factors
        (Table~\ref{tab:records}) adjusted to obtain model results in agreement
        with the Last Glacial Maximum (LGM) geomorphological reconstruction
        \citep[solid red line,][]{Ehlers.etal.2011}.
        \textbf{(d--f)} Cumulative extent of modelled ice cover during MIS~4
        (71--57\,ka). Only the simulation driven by the EPICA temperature
        time-series yields reasonable MIS~4 ice cover.
        \todo{Refine temperature offsets. Try to make reduced precipitation
              outlines more visible.}}
      \label{fig:footprints}
    \end{figure}

    \begin{figure}
      \centerline{\includegraphics{alpcyc_hr_lgmvel}}
      \caption{%
        \textbf{(a)} Modelled bedrock topography (grey) ice surface topography
        (200\,m contours) and ice surface velocity (blue) in the Alps
        21~thousand years (ka) before present. Modelled Last Glacial Maximum
        (LGM) ice extent (dashed orange line) and geomorphological
        reconstruction \citep[solid red line,][]{Ehlers.etal.2011}. The
        background map consists of depressed SRTM \citep{Jarvis.etal.2008}
        topography and Natural Earth Data \citep{Patterson.Kelso.2017}.
        \textbf{(b)} Temperature offset time-series from the EPICA ice core
        used as palaeo-climate forcing for the ice flow model \citep[black
        curve,][]{Jouzel.etal.2007}, and modelled total ice volume through the
        last glacial cycle (120--0\,ka), expressed in meters of sea level
        equivalent (m~s.l.e., blue curve). Gray fields indicate Marine
        Oxygen Isotope Stage (MIS) boundaries for MIS~2 and MIS~4 according to
        a~global compilation of benthic \chem{\delta^{18}O} records
        \citep{Lisiecki.Raymo.2005}.
        \note{I expect the simulation to complete in late July.}}
      \label{fig:lgmvel}
    \end{figure}

    \begin{figure}
      \centerline{\includegraphics{alpcyc_hr_timing}}
      \caption{%
        \textbf{(a)} Timing of the Last Glacial Maximum (LGM) given by the
        modelled age (colour mapping) and value (200 m contours) of maximum
        surface elevation throughout the entire simulation.
        \textbf{(b)} Temperature offset time-series from the EPICA ice core
        used as palaeo-climate forcing for the ice flow model (black curve),
        and modelled glaciated area around the LGM (coloured curve). The LGM
        is here modelled as a time-transgressive event.
        \note{I expect the simulation to complete in late July.}}
      \label{fig:timing}
    \end{figure}

    \begin{figure}
      \centerline{\includegraphics{alpcyc_hr_profiles}}
      \caption{%
        Modelled extent of glaciation along selected profiles for the Lyon,
        Solothurn, Rhine and Ivrea glaciers.
        \todo{Implement extraction of ice thickness along a more detailed
              profile, and add the other glaciers.}
        \idea{We can also move this as panel b on Fig.~\ref{fig:timing}, with
              profiles drawn on panel a. But then we should probably limit it
              to two glaciers instead of four.}
        \note{I expect the simulation to complete in late July.}}
      \label{fig:profiles}
    \end{figure}

    \begin{figure}
      \centerline{\includegraphics{alpcyc_hr_erosion}}
      \caption{%
        \textbf{(a)} Modelled total erosion integrated in time over the entire
        simulation (120--0\,ka) is highest in the deep valleys and cirques of
        the western Alps.
        \textbf{(b)} Temperature offset time-series from the EPICA ice core
        used as palaeo-climate forcing for the ice flow model (black curve),
        and modelled erosion rate integrated in space over the entire
        ice-covered area. The local erosion rate, $\dot{e}$, is computed from
        the sliding velocity, $\vec{v}_{\mathrm{b}}$, through
        $\dot{e} = K_g \cdot |\vec{v}_{\mathrm{b}}|^{l}$, with
        $l = 2.02$ and $K_g = 2.7\e{-7}\,m^{1-l}\,a^{l-1}$
        \citep{Herman.etal.2015}.
        \note{I expect the simulation to complete in late July.}}
      \label{fig:erosion}
    \end{figure}


% ----------------------------------------------------------------------
% Tables
\clearpage
% ----------------------------------------------------------------------

    \begin{table*}
      \caption{%
        Parameter values used in the ice sheet model.
        \todo{Update parameters to new configuration.}}
      \label{tab:params}
      \noindent\small\makebox[\textwidth]
      {\begin{tabular}{llrll}
        \toprule

        Not.    & Name & Value & Unit & Source \\

        \midrule
        \multicolumn{2}{l}{{Ice rheology}} \\
        \midrule

        $\rho$  & Ice density
                & 910
                & \unit{kg\,m^{-3}}
                & \citet{Aschwanden.etal.2012} \\

        $g$     & Standard gravity
                & 9.81
                & \unit{m\,s^{-2}}
                & \citet{Aschwanden.etal.2012} \\

        $n$     & Glen exponent
                & 3
                & --
                & \citet{Cuffey.Paterson.2010} \\

        $A_{\mathrm{c}}$   & Ice hardness coefficient cold
                & $2.847\e{-13}$
                & \unit{Pa^{-3}\,s^{-1}}
                & \citet{Cuffey.Paterson.2010} \\

        $A_{\mathrm{w}}$   & Ice hardness coefficient warm
                & $2.356\e{-2}$
                & \unit{Pa^{-3}\,s^{-1}}
                & \citet{Cuffey.Paterson.2010} \\

        $Q_{\mathrm{c}}$   & Flow law activation energy cold
                & $6.0\e4$
                & \unit{J\,mol^{-1}}
                & \citet{Cuffey.Paterson.2010} \\

        $Q_{\mathrm{w}}$   & Flow law activation energy warm
                & $11.5\e4$
                & \unit{J\,mol^{-1}}
                & \citet{Cuffey.Paterson.2010} \\

        $E_{\text{SIA}}$   & SIA enhancement factor
                & 2
                & --
                & \citet{Cuffey.Paterson.2010} \\

        $E_{\text{SSA}}$   & SSA enhancement factor
                & 1
                & --
                & \citet{Cuffey.Paterson.2010} \\

        $T_{\mathrm{c}}$   & Flow law critical temperature
                & 263.15
                & \unit{K}
                & \citet{Paterson.Budd.1982} \\

        $f$     & Flow law water fraction coeff.
                & 181.25
                & --
                & \citet{Lliboutry.Duval.1985} \\

        $R$     & Ideal gas constant
                & 8.31441
                & \unit{J\,mol^{-1}\,K^{-1}}
                & -- \\

        $\beta$ & Clapeyron constant
                & $7.9\e{-8}$
                & \unit{K\,Pa^{-1}}
                & \citet{Luthi.etal.2002} \\

        $c_{\mathrm{i}}$   & Ice specific heat capacity
                & 2009
                & \unit{J\,kg^{-1}\,K^{-1}}
                & \citet{Aschwanden.etal.2012} \\

        $c_{\mathrm{w}}$   & Water specific heat capacity
                & 4170
                & \unit{J\,kg^{-1}\,K^{-1}}
                & \citet{Aschwanden.etal.2012} \\

        $k$     & Ice thermal conductivity
                & 2.10
                & \unit{J\,m^{-1}\,K^{-1}\,s^{-1}}
                & \citet{Aschwanden.etal.2012} \\

        $L$     & Water latent heat of fusion
                & $3.34\e5$
                & \unit{J\,kg^{-1}\,K^{-1}}
                & \citet{Aschwanden.etal.2012} \\

        \midrule
        \multicolumn{2}{l}{{Basal sliding}} \\
        \midrule

        $q$     & Pseudo-plastic sliding exponent
                & 0.25
                & --
                & \citet{Aschwanden.etal.2013} \\

        $v_{\text{th}}$& Pseudo-plastic threshold velocity
                & 100
                & \unit{m\,a^{-1}}
                & \citet{Aschwanden.etal.2013} \\

        $c_0$   & Till cohesion
                & 0
                & Pa
                & \citet{Tulaczyk.etal.2000} \\

        $e_0$   & Till reference void ratio
                & 0.69
                & --
                & \citet{Tulaczyk.etal.2000} \\

        $C_{\mathrm{c}}$   & Till compressibility coefficient
                & 0.12
                & --
                & \citet{Tulaczyk.etal.2000} \\

        $\delta$& Minimum effective pressure ratio
                & 0.02
                & --
                & \citet{Bueler.Pelt.2015} \\

        $\phi$  & Till friction angle
                & 30
                & \degree
                & \citet{Cuffey.Paterson.2010} \\

        $W_{\text{max}}$ & Maximum till water thickness
                & 2
                & m
                & \citet{Bueler.Pelt.2015} \\

        \midrule
        \multicolumn{2}{l}{{Bedrock and lithosphere}} \\
        \midrule

        $\rho_{\mathrm{b}}$& Bedrock density
                & 3300
                & \unit{kg\,m^{-3}}
                & -- \\

        $c_{\mathrm{b}}$   & Bedrock specific heat capacity
                & 1000
                & \unit{J\,kg^{-1}\,K^{-1}}
                & -- \\

        $k_{\mathrm{b}}$   & Bedrock thermal conductivity
                & 3
                & \unit{J\,m^{-1}\,K^{-1}\,s^{-1}}
                & -- \\

        $\nu_{\mathrm{m}}$ & Astenosphere viscosity
                & $2.2\e{20}$
                & \unit{Pa\,s}
                & \citet{Mey.etal.2016} \\

        $\rho_{\mathrm{m}}$& Astenosphere density
                & 3300
                & \unit{kg\,m^{-3}}
                & \citet{Mey.etal.2016} \\

        $D$     & Lithosphere flexural rigidity
                & $1.389\e{24}$
                & \unit{N\,m}
                & \citet{Mey.etal.2016} \\

        \midrule
        \multicolumn{2}{l}{{Surface and atmosphere}} \\
        \midrule

        $T_{\mathrm{s}}$   & Temperature of snow precipitation
                & 273.15
                & \unit{K}
                & -- \\

        $T_{\mathrm{r}}$   & Temperature of rain precipitation
                & 275.15
                & \unit{K}
                & -- \\

        $F_{\mathrm{s}}$   & Degree-day factor for snow
                & $3.297\e{-3}$
                & \unit{m\,K^{-1}\,day^{-1}}
                & \citet{Huybrechts.1998} \\

        $F_{\mathrm{i}}$   & Degree-day factor for ice
                & $8.791\e{-3}$
                & \unit{m\,K^{-1}\,day^{-1}}
                & \citet{Huybrechts.1998} \\

        $R$     & Refreezing fraction
                & 0.0
                & --
                & -- \\

        $\gamma$& Air temperature lapse rate
                & $6\e{-3}$
                & \unit{K\,m{-1}}
                & -- \\

        $\omega$& Precipitation factor
                & 0.0704
                & --
                & \citet{Huybrechts.2002} \\

        \bottomrule
      \end{tabular}}
    \end{table*}

    \begin{table*}
      \caption{%
        Palaeo-temperature proxy records and scaling factors yielding
        temperature offset time-series used to force the ice sheet model
        through the last glacial cycle (Fig.~\ref{fig:timeseries}). $f$
        corresponds to the scaling factor adopted to yield Last Glacial Maximum
        ice limits in the vicinity of mapped end moraines
        (Fig.~\ref{fig:footprints}a), and $[{\Delta}T_{\textrm{TS}}]_{32}^{22}$
        refers to the resulting mean temperature anomaly during the period 32
        to~22\,\unit{ka} after scaling.
        \todo{Refine temperature offsets.}}
      \label{tab:records}
      \noindent\small\makebox[\textwidth]
      {\begin{tabular}{lccccccl}
        \toprule

        Forcing   & Latitude & Longitude & Elev. (m~a.s.l.)
                  & Proxy & $f$ & $[{\Delta}\text{TS}]_{32}^{22}$ (K)
                  & Reference\\

        \midrule

        GRIP      & \multirow{2}{*}{ 72{\degree}35$^{\prime}$\,N}   % 72.58 (decimal)
                  & \multirow{2}{*}{ 37{\degree}38$^{\prime}$\,W}   % 37.64 (decimal)
                  & \multirow{2}{*}{3238}
                  & \multirow{2}{*}{\chem{\delta^{18}O}}
                  & 0.50 & $-$8.2  % -16.4126 (before scaling)
                  & \multirow{2}{*}{\citet{Dansgaard.etal.1993}} \\

        GRIP, $\Delta P$ &&&&& 0.63 & $-$10.4 \\

        EPICA     & \multirow{2}{*}{ 75{\degree}06$^{\prime}$\,S}   % 75.1
                  & \multirow{2}{*}{123{\degree}21$^{\prime}$\,E}   % 123.35
                  & \multirow{2}{*}{3233}
                  & \multirow{2}{*}{\chem{\delta^{18}O}}
                  & 1.05 & $-$9.7  % -9.2055
                  & \multirow{2}{*}{\citet{Jouzel.etal.2007}} \\

        EPICA, $\Delta P$ &&&&& 1.33 & $-$12.2 \\

        MD01-2444 & \multirow{2}{*}{ 37{\degree}34$^{\prime}$\,N}   % 37.561
                  & \multirow{2}{*}{ 10{\degree}04$^{\prime}$\,W}   % -10.142
                  & \multirow{2}{*}{$-$2637}
                  & \multirow{2}{*}{\chem{U^{K'}_{37}}}
                  & 1.82 & $-$7.9  % -4.345625
                  & \multirow{2}{*}{\citet{Martrat.etal.2007}} \\

        MD01-2444, $\Delta P$ &&&&& 2.46 & $-$10.7 \\

        \bottomrule
      \end{tabular}}

      %\noindent\small\makebox[\textwidth]
      %{\begin{tabular}{llrlrlr}
      %  \toprule
      %
      %  Record    & \multicolumn{2}{c}{GRIP}
      %            & \multicolumn{2}{c}{EPICA}
      %            & \multicolumn{2}{c}{MD01-2444} \\
      %
      %  Precip.   & cst. & $\Delta P$
      %            & cst. & $\Delta P$
      %            & cst. & $\Delta P$ \\
      %
      %  \midrule
      %
      %  Latitude  & \multicolumn{2}{c}{ 72{\degree}35$^{\prime}$\,N}   % 72.58 (decimal)
      %            & \multicolumn{2}{c}{ 75{\degree}06$^{\prime}$\,S}   % 75.1
      %            & \multicolumn{2}{c}{ 37{\degree}34$^{\prime}$\,N} \\  % 37.561
      %
      %  Longitude & \multicolumn{2}{c}{ 37{\degree}38$^{\prime}$\,W}   % 37.64 (decimal)
      %            & \multicolumn{2}{c}{123{\degree}21$^{\prime}$\,E}   % 123.35
      %            & \multicolumn{2}{c}{ 10{\degree}04$^{\prime}$\,W} \\  % -10.142
      %
      %  Elevation & \multicolumn{2}{c}{3238\,m~a.s.l.}
      %            & \multicolumn{2}{c}{3233\,m~a.s.l.}
      %            & \multicolumn{2}{c}{$-$2637\,m~a.s.l.} \\
      %
      %  Proxy     & \multicolumn{2}{c}{\chem{\delta^{18}O}}
      %            & \multicolumn{2}{c}{\chem{\delta^{18}O}}
      %            & \multicolumn{2}{c}{\chem{U^{K'}_{37}}} \\
      %
      %  $f$       & 0.50 & 0.63
      %            & 1.05 & 1.33
      %            & 1.82 & 2.46 \\
      %
      %  $[{\Delta}\text{TS}]_{32}^{22}$
      %            & $-$8.2\,K & $-$10.4\,K     % -16.4126 (before scaling)
      %            & $-$9.7\,K & $-$12.2\,K     % -9.2055
      %            & $-$7.9\,K & $-$10.7\,K \\  % -4.345625
      %
      %  Reference & \multicolumn{2}{c}{\citet{Dansgaard.etal.1993}}
      %            & \multicolumn{2}{c}{\citet{Jouzel.etal.2007}}
      %            & \multicolumn{2}{c}{\citet{Martrat.etal.2007}} \\
      %
      %  \bottomrule
      %\end{tabular}}

    \end{table*}


% ======================================================================
\end{document}
% ======================================================================
