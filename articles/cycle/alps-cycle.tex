\documentclass{article}

\usepackage{authblk}
\usepackage[T1]{fontenc}
\usepackage[utf8]{inputenc}
\usepackage[pdftex]{xcolor}
\usepackage[pdftex]{graphicx}
\usepackage[authoryear,round]{natbib}

\graphicspath{{../../figures/}}

\definecolor{darkblue}{cmyk}{0.9,0.3,0.0,0.0}
\definecolor{darkgreen}{cmyk}{0.8,0.0,1.0,0.0}
\definecolor{darkred}{cmyk}{0.1,0.9,0.8,0.0}

\newcommand{\idea}[1]{\textcolor{darkgreen}{\emph{[\textbf{IDEA:} #1]}}}
\newcommand{\note}[1]{\textcolor{darkblue}{\emph{[\textbf{NOTE:} #1]}}}
\newcommand{\todo}[1]{\textcolor{darkred}{\emph{[\textbf{TODO:} #1]}}}
\newcommand{\aref}[0]{\textcolor{darkblue}{\textbf{[REF.]}}}

\title{Modelling last glacial cycle ice dynamics in the Alps}

\author[1]{Julien Seguinot%
           \thanks{Correspondence to seguinot@vaw.baug.ethz.ch}}
\author[1]{Guillaume Jouvet}
\author[1]{Matthias Huss}
\author[1]{Martin Funk}
\author[2]{Frank Preusser}

\affil[1]{Laboratory of Hydraulics, Hydrology and Glaciology,
          ETH Zürich, Switzerland}
\affil[2]{Institute of Earth and Environmental Sciences,
          University of Freiburg, Germany}

\newcommand{\chem}[1]{\ensuremath{\mathrm{#1}}}
\newcommand{\unit}[1]{\ensuremath{\mathrm{#1}}}

% ======================================================================
\begin{document}
% ======================================================================

\maketitle

\begin{abstract}

    The European Alps, cradle of pioneer glacial studies, are one of the
    regions where geological markers of past glaciations are most abundant and
    well-studied. Such conditions make the region ideal for testing numerical
    glacier models based on approximated ice flow physics against field-based
    reconstructions, and vice-versa.

    Here, we use the Parallel Ice Sheet Model (PISM) to model the entire last
    glacial cycle (120--0\,ka) in the Alps, with a horizontal resolution of
    1\,km. Climate forcing is derived using present-day climate data from
    WorldClim and the ERA-Interim reanalysis, and time-dependent temperature
    offsets from multiple paleo-climate proxies, among which only the EPICA ice
    core record yields glacial extent during marine oxygen isotope stages~4
    (69--62\,ka) and~2 (34--18\,ka) in agreement to geological reconstructions.

    Despite the low variability of this Antarctic-based climate forcing, our
    simulation depicts a highly dynamic ice cap, showing that alpine glaciers
    may have advanced many times over the foreland during the last glacial
    cycle. Cumulative basal sliding, a proxy for glacial erosion, is modelled
    to be highest in the deep valleys of the western Alps. Finally, the Last
    Glacial Maximum advance, often considered synchronous, is here modelled as
    a time-transgressive event, with some glacier lobes reaching their maximum
    as early as 27\,ka, and some as late as 21\,ka. Modelled ice thickness is
    about 900\,m higher than observed trimline elevations, yet our simulation
    predicts little erosion at high elevation due to cold ice conditions.

\end{abstract}

\section{Main}
\subsection{Introduction}
\subsection{Glacier dynamics}
    Fig.~\ref{fig:lgmvel} -- Snapshot at 21\,ka and volume time series.\\
    Fig.~\ref{fig:timing} -- Timing of the LGM and area time series.\\
    Fig.~\ref{fig:profiles} -- Individual glacier extent profiles.
\subsection{Erosion potential}
    Fig.~\ref{fig:erosion} -- Integrated erosion potential.

\section{Methods}
\subsection{Ice sheet model set-up}
    Tab. -- Model parameters\\
    Fig.~\ref{fig:inputs} -- Climate and geothermal model inputs.\\
\subsection{Palaeo-climate forcing}
    Tab. -- Records\\
    Fig.~\ref{fig:timeseries} -- Low-resolution time series.\\
    Fig.~\ref{fig:footprints} -- Low-resolution ice cover.

\section{Conclusion}

% ----------------------------------------------------------------------
% References
% ----------------------------------------------------------------------

\bibliographystyle{abbrvnat}
\bibliography{../../references/references}

% ----------------------------------------------------------------------
% Figures
% ----------------------------------------------------------------------

    \begin{figure}
      \centerline{\includegraphics{alpcyc_hr_lgmvel}}
      \caption{%
        \textbf{(a)} Modelled bedrock topography (grey) ice surface topography
        (200\,m contours) and ice surface velocity (blue) in the Alps
        21~thousand years (ka) before present. Modelled Last Glacial Maximum
        (LGM) ice extent (dashed orange line) and geomorphological
        reconstruction \citep[solid red line,][]{Ehlers.etal.2011}. The
        background map consists of depressed SRTM \citep{Jarvis.etal.2008}
        topography and Natural Earth Data \citep{Patterson.Kelso.2015}.
        \textbf{(b)} Temperature offset time-series from the EPICA ice core
        used as palaeo-climate forcing for the ice flow model \citep[black
        curve,][]{Jouzel.etal.2007}, and modelled total ice volume through the
        last glacial cycle (120--0\,ka), expressed in meters of sea level
        equivalent (m~s.l.e., blue curve). Gray fields indicate Marine
        Oxygen Isotope Stage (MIS) boundaries for MIS~2 and MIS~4 according to
        a~global compilation of benthic \chem{\delta^{18}O} records
        \citep{Lisiecki.Raymo.2005}.}
      \label{fig:lgmvel}
    \end{figure}

    \begin{figure}
      \centerline{\includegraphics{alpcyc_hr_timing}}
      \caption{%
        \textbf{(a)} Timing of the Last Glacial Maximum (LGM) given by the
        modelled age (colour mapping) and value (200 m contours) of maximum
        surface elevation throughout the entire simulation.
        \textbf{(b)} Temperature offset time-series from the EPICA ice core
        used as palaeo-climate forcing for the ice flow model (black curve),
        and modelled glaciated area around the LGM (coloured curve). The LGM
        is here modelled as a time-transgressive event.}
      \label{fig:timing}
    \end{figure}

    \begin{figure}
      \centerline{\includegraphics{alpcyc_hr_profiles}}
      \caption{%
        Modelled extent of glaciation along selected profiles for the Lyon,
        Solothurn, Rhine and Ivrea glaciers.
        \todo{Implement extraction of ice thickness along a more detailed
              profile, and add the other glaciers.}}
      \label{fig:profiles}
    \end{figure}

    \begin{figure}
      \centerline{\includegraphics{alpcyc_hr_erosion}}
      \caption{%
        \textbf{(a)} Modelled total erosion integrated in time over the entire
        simulation (120--0\,ka) is highest in the deep valleys and cirques of
        the western Alps.
        \textbf{(b)} Temperature offset time-series from the EPICA ice core
        used as palaeo-climate forcing for the ice flow model (black curve),
        and modelled erosion rate integrated in space over the entire
        ice-covered area. The local erosion rate, $\dot{e}$, is computed from
        the sliding velocity, $\vec{v}_{\mathrm{b}}$, through
        $\dot{e} = K_g \cdot |\vec{v}_{\mathrm{b}}|^{l}$, with
        $l = 2.02$ and $K_g = 2.7 \cdot 10^{-7}\,m^{1-l}\,a^{l-1}$
        \citep{Herman.etal.2015}.}
      \label{fig:erosion}
    \end{figure}

% ----------------------------------------------------------------------
% Supplementary tables
\setcounter{table}{0}
\renewcommand{\thetable}{S\arabic{table}}%
% ----------------------------------------------------------------------

% ----------------------------------------------------------------------
% Supplementary figures
\setcounter{figure}{0}
\renewcommand{\thefigure}{S\arabic{figure}}%
% ----------------------------------------------------------------------

    \begin{figure}
      \centerline{\includegraphics{alpcyc_hr_inputs}}
      \caption{Climate and geothermal model inputs.}
      \label{fig:inputs}
    \end{figure}

    \begin{figure}
      \centerline{\includegraphics{alpcyc_lr_timeseries}}
      \caption{Low-resolution time series.}
      \label{fig:timeseries}
    \end{figure}

    \begin{figure}
      \centerline{\includegraphics{alpcyc_lr_footprints}}
      \caption{Low-resolution ice cover.}
      \label{fig:footprints}
    \end{figure}

% ======================================================================
\end{document}
% ======================================================================
