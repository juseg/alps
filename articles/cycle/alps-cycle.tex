\documentclass[tc, manuscript]{copernicus}

\graphicspath{{../../figures/}}

\definecolor{darkblue}{cmyk}{0.9,0.3,0.0,0.0}
\definecolor{darkgreen}{cmyk}{0.8,0.0,1.0,0.0}
\definecolor{darkred}{cmyk}{0.1,0.9,0.8,0.0}
\definecolor{darkorange}{cmyk}{0.0,0.5,1.0,0.0}
\definecolor{darkpurple}{cmyk}{0.6,0.7,0.0,0.0}
\definecolor{darkbrown}{cmyk}{0.23,0.73,0.98,0.12}

\newcommand{\idea}[1]{\textcolor{darkgreen}{\emph{[\textbf{IDEA:} #1]}}}
\newcommand{\note}[1]{\textcolor{darkblue}{\emph{[\textbf{NOTE:} #1]}}}
\newcommand{\todo}[1]{\textcolor{darkred}{\emph{[\textbf{TODO:} #1]}}}
\newcommand{\aref}[0]{\textcolor{darkblue}{\textbf{[REF.]}}}

\hypersetup{colorlinks, linkcolor=darkorange,
            urlcolor=darkbrown, citecolor=darkpurple}

\title{Modelling last glacial cycle ice dynamics in the Alps}

\Author[1]{Julien}{Seguinot}
\Author[1]{Guillaume}{Jouvet}
\Author[1]{Matthias}{Huss}
\Author[1]{Martin}{Funk}
\Author[2]{Frank}{Preusser}

\affil[1]{Laboratory of Hydraulics, Hydrology and Glaciology,
          ETH Zürich, Switzerland}
\affil[2]{Institute of Earth and Environmental Sciences,
          University of Freiburg, Germany}

\runningtitle{Modelling last glacial cycle ice dynamics in the Alps}
\runningauthor{J.~Seguinot et al.}
\correspondence{J.~Seguinot (seguinot@vaw.baug.ethz.ch)}

\received{}
\pubdiscuss{}
\revised{}
\accepted{}
\published{}


% ======================================================================
\begin{document}
% ======================================================================

\firstpage{1}

\maketitle

\begin{abstract}

    The European Alps, cradle of pioneer glacial studies, are one of the
    regions where geological markers of past glaciations are most abundant and
    well-studied. Such conditions make the region ideal for testing numerical
    glacier models based on approximated ice flow physics against field-based
    reconstructions, and vice-versa.

    Here, we use the Parallel Ice Sheet Model (PISM) to model the entire last
    glacial cycle (120--0\,ka) in the Alps, with a horizontal resolution of
    1\,km. Climate forcing is derived using present-day climate data from
    WorldClim and the ERA-Interim reanalysis, and time-dependent temperature
    offsets from multiple paleo-climate proxies, among which only the EPICA ice
    core record yields glaciation during marine oxygen isotope stages~4
    (69--62\,ka) and~2 (34--18\,ka) spatially and temporally consistent with
    the geological reconstructions.

    Despite the low variability of this Antarctic-based climate forcing, our
    simulation depicts a highly dynamic ice cap, showing that alpine glaciers
    may have advanced many times over the foreland during the last glacial
    cycle. Ice flow patterns during peak glaciation are largely governed by
    subglacial topography but include occasional transfluences and
    self-sustained ice domes. Finally, the Last
    Glacial Maximum advance, often considered synchronous, is here modelled as
    a time-transgressive event, with some glacier lobes reaching their maximum
    as early as 27\,ka, and some as late as 21\,ka. Modelled ice thickness is
    about 900\,m higher than observed trimline elevations, yet our simulation
    predicts little erosion at high elevation due to cold ice conditions.

\end{abstract}

%\copyrightstatement{}


% ----------------------------------------------------------------------
\introduction
\label{sec:intro}
% ----------------------------------------------------------------------

    For nearly 300~years, montane people and early explorers of the European
    Alps learned to read the geomorphological imprint left by glaciers in the
    landscape, and to understand that glaciers had once been more extensive
    than today \citep[e.g.,][p.~21]{Windham.Martel.1744}. Contemporaneously,
    it was also observed in the Alps that glaciers move by a combination of
    meltwater-induced \emph{sliding} at the base \citep[\S532]{Saussure.1779},
    and viscous \emph{deformation} within the ice body \citep{Forbes.1846b}. As
    glaciers flow and slide across their bed, they transport rock debris and
    erode the landscape, thereby leaving geomorphological traces of their
    former presence. In the mid-nineteenth
    century, more systematic studies of glacial features showed that alpine
    glaciers extended well below their contemporary margins \citep{Venetz.1821}
    and even onto the alpine foreland \citep{Charpentier.1841}, yielding the
    idea that, under colder temperatures, expansive ice sheets had once covered
    much of Europe and North America \citep{Agassiz.1840}.

    However, this glacial theory did not gain general acceptance until the
    discovery and exploration of the two present-day ice sheets on Earth, the
    Greenland and Antarctic ice sheets, provided a modern analogue for the
    proposed European and North American ice sheets. Although it was long
    unclear whether there had been a single or multiple \emph{glaciations},
    this controversy ended with the large-scale mapping of two distinct moraine
    systems in North America \citep{Chamberlin.1894}. In the European Alps, the
    systematic classification of the glaciofluvial terraces on the northern
    foreland indicated that there had been at least four major
    glaciations in the Alps \citep{Penck.Bruckner.1909}. More recent studies
    of glaciofluvial stratigraphy in the foreland indicate up to eight
    glaciations \citep{Ivy-Ochs.etal.2008, Preusser.etal.2011}.

    More recently, palaeoclimate records extracted from deep sea sediments and
    ice cores have provided a much more detailed picture of the Earth
    environmental history \citep[e.g.,][]{Emiliani.1955,
    Shackleton.Opdyke.1973, Dansgaard.etal.1993, Augustin.etal.2004},
    indicating neither one, nor four, but rather tens of \emph{glacial cycles}.
    During the last 800000 years (800\,ka), glacial and interglacial periods
    have succeeded each other with a 100\,ka periodicity \citep{Hays.etal.1976,
    Augustin.etal.2004}. Nevertheless, this global signal is largely governed
    by the North American and Eurasian ice sheet complexes. It is thus unclear
    wether glacier advances and retreats in the Alps were in pace with global
    sea-level fluctuations.

    Besides, the landform record is typically sparse in time and, most often,
    spatially incomplete. Palaeo-ice sheets did not leave a continuous imprint
    on the lanscape, and much of this evidence has been overprinted by
    subsequent glacier re-advances and other geomorphological processes
    \citep[e.g.,][]{Kleman.1994, Kleman.etal.2006, Kleman.etal.2010}. Dating
    uncertainties typically increase with age, such that dated reconstructions
    are strongly biased towards earlier glaciations \citep{Heyman.etal.2011}.
    The European Alps have been studied in more detail than any other glacial
    region, but they are no exception to these rules. Although sparse
    geological traces indicate that the last glacial cycle may have comprised
    two or three cycles of glacier growth and decay \citep{Preusser.2004,
    Ivy-Ochs.etal.2008}, most glacial features currently left on the foreland
    present a record of the last major glaciation of the Alps, dating from the
    \emph{Last Glacial Maximum} (LGM).

    The spatial extent and thickness of Alpine glaciers during the LGM, an
    integrated footprint of thousands of years of climate history and glacier
    dynamics, have been reconstructed from moraines and trimlines across the
    mountain range \citep[e.g.,][]{Bini.etal.2009, Coutterand.2010,
    Husen.2011}. Although no evidence for cold-based glaciation has been found
    in the Alps to date, the assumption that trimlines, the upper limit of
    glacial erosion, represent the maximum ice surface elevation, which has
    been repeatedly invalidated in other glaciated regions of the globe
    \citep[e.g.,][]{Kleman.1994, Kleman.etal.2010, Fabel.etal.2012}. Ice flow
    patterns were primarily controlled by subglacial topography, but there is
    evidence for flow accross major mountain passes
    \citep[e.g.,][]{Coutterand.2010, Kelly.etal.2004, Husen.2011}, and
    self-sustained ice domes \citep{Bini.etal.2009}. Finally, the timing of the
    LGM in the Alps \citep{Ivy-Ochs.etal.2008, Monegato.etal.2017} is in good
    agreement with the maximum expansion of continental ice sheets recorded by
    the Marine Oxygen Isotope Stage (MIS) 2 (29--14\,ka)
    \citep{Lisiecki.Raymo.2005}. Regional variation between different piedmont
    lobes exist \citep[Fig.~5]{Wirsig.etal.2016}, but it is unclear wether they
    relate to climate or glacier dynamics \citep{Monegato.etal.2017} or
    uncertainties in the dating methods.

    Although the glacial history of the European Alps has been studied for
    nearly three hundred years, it thus remains incompletely known
    \begin{itemize}
      \item what climate evolution lead to the known maximum ice limits,
      \item to which extent ice flow was controlled by subglacial topography,
      \item what drove the different response of the individual lobes,
      \item how far above the trimline was the ice surface located, and
      \item how many advances occurred during the last glacial cycle.
    \end{itemize}

    Here, we use the Parallel Ice Sheet Model
    \citep[PISM,][]{PISM-authors.2017}, a numerical ice sheet model that
    approximates glacier sliding and deformation (Sect.~\ref{sec:model}), to
    model alpine glacier dynamics through the last glacial cycle (120--0\,ka),
    a period for which palaeo-temperature proxies are available, albeit non
    regional. In an attempt to analyse the long-standing questions outline
    above, we test the model sensitivity to multiple palaeo-climate forcing
    (Sect.~\ref{sec:climate}), and then explore the modelled glacier dynamics
    at high resolution for the optimal forcing (Sect.~\ref{sec:results}).


% ----------------------------------------------------------------------
\section{Ice sheet model set-up}
\label{sec:model}
% ----------------------------------------------------------------------

    Table~\ref{tab:params} -- Model parameters.\\
    Fig.~\ref{fig:inputs} -- Climate and geothermal model inputs.\\


% -- -- -- -- -- -- -- -- -- -- -- -- -- -- -- -- -- -- -- -- -- -- -- -
\subsection{Overview}
\label{sec:overview}
% -- -- -- -- -- -- -- -- -- -- -- -- -- -- -- -- -- -- -- -- -- -- -- -

    We use the Parallel Ice Sheet Model (PISM, development version~e9d2d1f), an
    open source, finite difference, shallow ice sheet model
    \citep{PISM-authors.2017}. The model requires input on basal
    topography, geothermal heat flux and climate forcing. It computes the
    evolution of ice extent and thickness over time, the thermal and dynamic
    states of the ice sheet, and the associated lithospheric response. The
    set-up used here is largely based on that used in previous studies of the
    former Cordilleran ice sheet \citep{Seguinot.2014, Seguinot.etal.2014,
    Seguinot.etal.2016}.

    \idea{Remove next paragraph?}

    Ice deformation follows temperature and water-content dependent creep
    (Sect.~\ref{sec:icedyn}). Basal sliding follows a~pseudo-plastic law where
    the yield stress accounts for till deconsolidation under high water
    pressure (Sect.~\ref{sec:sliding}). Bedrock topography is deflected
    under the ice load (Sect.~\ref{sec:bedrock}). Surface mass balance is
    computed using a~positive degree-day (PDD) model (Sect.~\ref{sec:surface}).
    Climate forcing is provided by a~monthly climatology from interpolated
    observational data \citep[WorldClim;][]{Hijmans.etal.2005} and the European
    Centre for Medium-Range Weather Forecasts Reanalysis Interim
    \citep[ERA-Interim;][]{Dee.etal.2011}, perturbed by time-dependent
    temperature offsets, temperature lapse-rate corrections, and in some cases,
    time-dependent paleo-precipitation reductions (Sect.~\ref{sec:atm},
    Table~\ref{tab:records}).

    \idea{Remove next paragraph?}

    Each simulation starts from assumed present-day ice thickness and
    equilibrium temperature distribution at 120\,\unit{ka}, and runs to the
    present. Our modelling domain of 900 by 600\,\unit{km} encompasses the
    entire Alpine range (Fig.~\ref{fig:inputs}). The simulations were run on
    two distinct grids, using a~lower horizontal resolution of 2\,\unit{km},
    and a~higher horizontal resolution of 1\,\unit{km}.

    \idea{In the following, I repeat all parameters in the text and table. Is
          this necessary? Equations are the same as in my Cordillera paper.
          Should I remove them?}


% -- -- -- -- -- -- -- -- -- -- -- -- -- -- -- -- -- -- -- -- -- -- -- -
\subsection{Ice rheology}
\label{sec:icedyn}
% -- -- -- -- -- -- -- -- -- -- -- -- -- -- -- -- -- -- -- -- -- -- -- -

    Ice sheet dynamics are typically modelled using a~combination of internal
    deformation and basal sliding. PISM is a~shallow ice sheet model, which
    implies that the balance of stresses is approximated based on their
    predominant components. The Shallow Shelf Approximation (SSA) is combined
    with the Shallow Ice Approximation (SIA) by adding velocity solutions of
    the two approximations \citep[Eqs.~7--9 and 15]{Winkelmann.etal.2011}.
    Although this heuristic approach implies errors in the transition zone
    where gravitational stresses intervene both in the SIA and SSA velocity
    computation, this hybrid scheme is computationally much more efficient than
    a fully three-dimensional model with which the simulations presented here
    would not be feasible.

    Ice deformation is governed by the constitutive law for ice
    \citep{Glen.1952, Nye.1953},
    %
    \begin{align}
      \vec{\dot{\epsilon}} = A\,\tau_{\mathrm{e}}^{n-1}\,\vec{\tau} \,.
    \end{align}
    %
    where $\vec{\dot{\epsilon}}$ is the strain-rate tensor, $\vec{\tau}$ the
    deviatoric stress tensor, and $\tau_{\mathrm{e}}$ the effective stress
    defined in our case by
    ${\tau_{\mathrm{e}}}^2=\frac{1}{2}\mathrm{tr}(\vec{\tau}^2)$. The ice
    softness coefficient, $A$, depends on ice temperature, $T$, pressure, $p$,
    and water content, $\omega$, through a~piece-wise Arrhenius-type law,
    %
    \begin{align}
    &A = E\cdot
      \begin{cases}
        A_{\mathrm{c}} \,e^\frac{-Q_{\mathrm{c}}}{RT_{\text{pa}}}
            & \text{if}\ T_   {\text{pa}}  <  T_{\mathrm{c}} \,, \\
        A_{\mathrm{w}} (1+f\omega)\,e^\frac{-Q_{\mathrm{w}}}{RT_{\text{pa}}}
            & \text{if}\ T_   {\text{pa}} \ge T_{\mathrm{c}} \,,
      \end{cases}
    \end{align}
    %
    where $T_{\text{pa}}$ is the pressure-adjusted ice temperature calculated
    using the Clapeyron relation, $T_{\text{pa}}=T-{\beta}p$.
    $R=8.31441$\,\unit{J\,mol^{-1}\,K^{-1}} is the ideal gas constant, and
    $A_{\mathrm{c}}=2.847 \times 10^{-13}$\,\unit{Pa^{-3}\,s^{-1}},
    $A_{\mathrm{w}}=2.356 \times 10^{-2}$\,\unit{Pa^{-3}\,s^{-1}},
    $Q_{\mathrm{c}}=6.0 \times 10^4$\,\unit{J\,mol^{-1}}, and
    $Q_{\mathrm{w}}=11.5 \times 10^4$\,\unit{J\,mol^{-1}} are constant parameters
    corresponding to values measured below and above a~critical temperature
    threshold $T_{\mathrm{c}}=-10$\,\unit{{\degree}C}
    \citep[p.~72]{Cuffey.Paterson.2010}. The water fraction, $\omega$, is
    capped at a~maximum value of 0.01, above which no measurements are
    available \citep[Eq.~5.7]{Lliboutry.Duval.1985, Greve.1997}. Finally, $E$
    is a~non-dimensional enhancement factor which can take different values,
    $E_{\text{SIA}}=2$, in the SIA and $E_{\text{SSA}}=1$, in the SSA, as
    recommended for Holocene polar ice \citep[p.~77]{Cuffey.Paterson.2010}.

    In all our simulations, we set constant the power-law exponent, $n=3$,
    according to \citet[p.~55--57]{Cuffey.Paterson.2010}, the Clapeyron
    constant, $\beta=7.9\times 10^{-8}$\,\unit{K\,Pa^{-1}}, according to
    \citet{Luthi.etal.2002}, and the water fraction coefficient, $f=181.25$,
    according to \citet{Lliboutry.Duval.1985}. These fixed parameter values are
    summarized in Table~\ref{tab:params}.

    Surface air temperature derived from the climate forcing
    (Sect.~\ref{sec:atm}) provides the upper boundary condition to the ice
    enthalpy model. Temperature is computed in the ice and in the bedrock to
    a~depth of 3\,\unit{km} below the ice-bedrock interface, where it is
    conditioned by a~lower boundary geothermal heat flux estimate from multiple
    geothermal proxies \citep[similarity method]{Goutorbe.etal.2011}. In the
    2\,km-resolution simulations, the vertical grid consists of 31~temperature
    layers in the bedrock and up to 126~enthalpy layers in the ice,
    corresponding to vertical resolutions of 100 and 40\,\unit{m},
    respectively. The 1\,km-resolution simulation uses 61~bedrock layers and up
    to 251~ice layers with a~vertical resolutions of 50 and 20\,\unit{m}.


% -- -- -- -- -- -- -- -- -- -- -- -- -- -- -- -- -- -- -- -- -- -- -- -
\subsection{Basal sliding}
\label{sec:sliding}
% -- -- -- -- -- -- -- -- -- -- -- -- -- -- -- -- -- -- -- -- -- -- -- -

    A~pseudo-plastic sliding law,
    %
    \begin{align}
      \vec{\tau}_{\mathrm{b}} = -\tau_{\mathrm{c}}
        \frac{\vec{v}_{\mathrm{b}}}
             {{v_{\text  {th}}}^q\,|\vec{v}_{\mathrm{b}}|^{1-q}} \,,
    \end{align}
    %
    relates the bed-parallel shear stresses, $\vec{\tau}_{\mathrm{b}}$, to the
    sliding velocity, $\vec{v}_{\mathrm{b}}$. The pseudo-plastic sliding
    exponent, $q=0.25$, and the threshold velocity,
    $v_{\text{th}}=100$\,\unit{m\,a^{-1}}, are set to values successfully used
    to model the Greenland ice sheet \citet{Aschwanden.etal.2013} The yield
    stress, $\tau_{\mathrm{c}}$, is modelled using the Mohr--Coulomb criterion,
    %
    \begin{align}
      \tau_{\mathrm{c}} = c_0 + N\,\tan{\phi} \,,
    \end{align}
    %
    where the till cohesion, $c_0=0$, was consistently measured to be
    negligible \citep[p.~268]{Tulaczyk.etal.2000, Cuffey.Paterson.2010}. We use
    a~constant till friction angle, $\phi=30$\unit{\degree}, corresponding to
    the average of values presented in \citet[p.~268]{Cuffey.Paterson.2010}.

    Effective pressure, $N$, is related to the ice overburden stress, $P_0=\rho
    gh$, and the modelled amount of subglacial water, using a~formula derived
    from laboratory experiments with till extracted from the base of Ice Stream
    B in West Antarctica \citep{Tulaczyk.etal.2000, Bueler.Pelt.2015},
    %
    \begin{align}
      N = \delta P_0 \, 10^{(e_0/C_{\mathrm{c}}) (1 - (W/W_{\text{max}}))} \,,
    \end{align}
    %
    where $\delta=0.02$ sets the minimum ratio between the effective and
    overburden pressures. Parameter values for the till reference void ratio,
    $e_0=0.69$, and the till compressibility coefficient,
    $C_{\mathrm{c}}=0.12$, were set to the only measurements available to our
    knowledge \citep{Tulaczyk.etal.2000}. The amount of water at the base, $W$,
    varies from zero to $W_{\text{max}}=2$\,m, a~threshold above which
    additional melt water is assumed to drain off instantaneously. These fixed
    parameter values are summarized in Table~\ref{tab:params}.


% -- -- -- -- -- -- -- -- -- -- -- -- -- -- -- -- -- -- -- -- -- -- -- -
\subsection{Basal topography}
\label{sec:bedrock}
% -- -- -- -- -- -- -- -- -- -- -- -- -- -- -- -- -- -- -- -- -- -- -- -

    The initial basal topography is bilinearly interpolated from the
    hole-filled, Shuttle Radar Topography Mission (SRTM) data with a~resolution
    of 30\,arc-sec \citep{Jarvis.etal.2008}. These data include post-glacial
    sediment fills and lakes surface topography. However, they were corrected
    for an estimation of present-day ice thickness according from modern
    glacier outlines, surface topography and simplified ice physics
    \citep{Huss.Farinotti.2012}.

    Basal topography responds to ice load following a~bedrock deformation model
    that includes local isostasy, elastic lithosphere flexure and viscous
    astenosphere deformation in an infinite half-space
    \citep{Lingle.Clark.1985,Bueler.etal.2007}. The astenosphere viscosity,
    $\nu_{\mathrm{m}}=2.2\times10^{20}$\,\unit{Pa\,s}, the astenosphere
    density, $\rho_{\mathrm{m}}=3300$\,\unit{kg\,m^{-3}}, and the lithosphere
    elastic rigidity, $D=1.389 \times 10^{24}$\,\unit{N\,m}, were set according to
    results from glacial isostatic adjustment modelling of deglacial rebound in
    the Alps most closely reproducing observed modern uplift rates
    \citep[Table~\ref{tab:params};][Supplementary Fig.~7]{Mey.etal.2016}. The
    latter was computed as $D=YE^3/12(1-\nu)$, where $Y=100$\,GPa is Young's
    modulus, $\nu=0.25$ is the Poisson ratio, and $E=50$\,km is the average
    effective elastic thickness of the lithosphere
    \citep[Table~\ref{tab:params};][]{Mey.etal.2016}.


% -- -- -- -- -- -- -- -- -- -- -- -- -- -- -- -- -- -- -- -- -- -- -- -
\subsection{Surface mass balance}
\label{sec:surface}
% -- -- -- -- -- -- -- -- -- -- -- -- -- -- -- -- -- -- -- -- -- -- -- -

    Ice surface accumulation and ablation are computed from monthly mean
    near-surface air temperature, $T_{\mathrm{m}}$, monthly standard deviation
    of near-surface air temperature, $\sigma$, and monthly precipitation,
    $P_{\mathrm{m}}$, using a~temperature-index model
    \citep[e.g.,][]{Hock.2003}. Accumulation is equal to precipitation when air
    temperatures are below 0\,\unit{{\degree}C}, and decreases to zero linearly
    with temperatures between 0 and 2\,\unit{{\degree}C}. Ablation is computed
    from PDD, defined as an integral of temperatures above 0\,\unit{{\degree}C}
    in one year.

    The PDD computation accounts for stochastic temperature variations by
    assuming a~normal temperature distribution of standard deviation $\sigma$
    around the expected value $T_{\mathrm{m}}$. It is expressed by an
    error-function formulation \citep{Calov.Greve.2005},
    %
    \begin{align}
      {\text{PDD}} = \int_{t_1}^{t_2} \mathrm{d}t
        \left[\frac{\sigma}{\sqrt{2\pi}}
                \exp\left({-\frac{T_{\mathrm{m}}^2}{2\sigma^2}}\right)
              + \frac{T_{\mathrm{m}}}{2} \, {\text{erfc}}
                \left(-\frac{T_{\mathrm{m}}}{\sqrt{2}\sigma}\right)\right] \,,
    \end{align}
    %
    which is numerically approximated using week-long sub-intervals. In
    order to account for the effects of spatial and seasonal variations of
    temperature variability \citep{Seguinot.2013}, $\sigma$ is computed
    from ERA-Interim daily temperature values from 1979 to 2012
    \citep{Mesinger.etal.2006}, including variability associated with the
    seasonal cycle \citep{Seguinot.2013}, and bilinearly interpolated to the
    model grids (Fig.~\ref{fig:inputs}). Degree-day factors for snow and ice
    melt are set to values used in the European Ice Sheet Modelling INiTiative
    \citep[Table~\ref{tab:params}; EISMINT,][]{Huybrechts.1998}.


% -- -- -- -- -- -- -- -- -- -- -- -- -- -- -- -- -- -- -- -- -- -- -- -
\subsection{Reference climate forcing}
\label{sec:atm}
% -- -- -- -- -- -- -- -- -- -- -- -- -- -- -- -- -- -- -- -- -- -- -- -

    Climate forcing driving ice sheet simulations consists of a~present-day
    monthly climatology, $\{T_{\mathrm{m}0}, P_{\mathrm{m}0}\}$, modified by
    temperature lapse-rate corrections, ${\Delta}T_{\text{LR}}$, temperature
    offset time series, ${\Delta}T_{\text{TS}}$, and time-dependent
    palaeo-precipitation corrections, $\Psi_{\text{PP}}$:
    %
    \begin{align}
      &T_{\mathrm{m}}(t, x, y) = T_{\mathrm{m}0}(x, y) +
                                 {\Delta}T_{\text{LR}}(t, x, y) +
                                 {\Delta}T_{\text{TS}}(t) \,, \\
      &P_{\mathrm{m}}(t, x, y) = P_{\mathrm{m}0}(x, y) \cdot
                                 {\Psi}_{\text{PP}}(t) \,, \\
    \end{align}
    %
    The present-day monthly climatology was bilinearly interpolated from
    near-surface air temperature and precipitation rate fields from
    \citep[WorldClim;][]{Hijmans.etal.2005}, representative of the period 1960
    to 1990. Modern climate of the European Alps is characterised by a
    latitudinal gradient in summer air temperatures, and a longitudinal
    gradient in winter precipitation (Fig.~\ref{fig:inputs}). WorldClim data
    were selected as an input to the ice sheet model because they incorporate
    observations from the dense weather station network of central Europe.
    Besides, last glacial cycle alpine glaciers did not extend over marine
    areas where WorldClim data is missing. Finally, WorldClim data were
    previously used as climate forcing for PISM to model the LGM extent of the
    former Cordilleran ice sheet in good agreement with geological evidence
    along the southern margin \citep{Seguinot.etal.2014} were weather station
    density is lower than in the Alps.

    The temperature lapse-rate corrections, ${\Delta}T_{\text{LR}}$, are
    computed as a~function of ice surface elevation, $s$, using the SRTM
    topography shipped with WorldClimas a~reference, $b_{\text{ref}}$:
    %
    \begin{align}
      {\Delta}T_{\text{LR}}(t, x, y) &= -\gamma [s(t, x, y)-b_{\text{ref}}] \\
                                     &= -\gamma [h(t, x, y)+
                                                 b(t, x, y)-b_{\text{ref}}],
    \end{align}
    %
    thus accounting for the evolution of ice thickness, ${h=s-b}$, on the one
    hand, and for differences between the basal topography of the ice flow
    model, $b$, and the NARR reference topography, $b_{\text{ref}}$, on the
    other hand. All simulations use an annual temperature lapse rate of
    $\gamma=6\,\unit{K\,km^{-1}}$.


% ----------------------------------------------------------------------
\section{Palaeo-climate forcing}
\label{sec:climate}
% ----------------------------------------------------------------------

    Table~\ref{tab:records} -- Palaeo-climate records.\\
    Fig.~\ref{fig:timeseries} -- Low-resolution time series.\\
    Fig.~\ref{fig:footprints} -- Low-resolution ice cover.\\

    In this section, we analyze the model sensitivity to palaeo-climate forcing
    through the last glacial cycle, using three palaeo-temperature records
    (Sect.~\ref{sec:paltemp}) and two parametrizations of palaeo-precipitation
    (Sect.~\ref{sec:palprec}), in terms of modelled evolution of total ice
    volume (Sect.~\ref{sec:timeseries}) and glaciated area during MIS~2 and~4
    (Sect.~\ref{sec:footprints}).

    These simulations use an horizontal resolution of 2\,km. The vertical grid
    consists of 31~temperature layers in the bedrock and up to 126~enthalpy
    layers in the ice, corresponding to vertical resolutions of 100 and
    40\,\unit{m}, respectively.


% -- -- -- -- -- -- -- -- -- -- -- -- -- -- -- -- -- -- -- -- -- -- -- -
\subsection{Palaeo-temperature forcing}
\label{sec:paltemp}
% -- -- -- -- -- -- -- -- -- -- -- -- -- -- -- -- -- -- -- -- -- -- -- -

    Temperature offset time-series, ${\Delta}T_{\text{TS}}$, are derived from
    palaeo-temperature proxy records from the Greenland Ice Core Project
    \citep[GRIP,][]{Dansgaard.etal.1993}, the European Project for Ice Coring
    in Antarctica \citep[EPICA,][] {Jouzel.etal.2007}, and an oceanic sediment
    core from the Iberian margin \citep[MD01-2444,][]{Martrat.etal.2007}.
    Palaeo-temperature anomalies from the GRIP record were calculated from
    oxygen isotope (\chem{\delta^{18}O}) measurements using a~quadratic
    equation \citep{Johnsen.etal.1995},
    %
    \begin{align}
      {\Delta}T_{\text{TS}}(t) ={~}&-11.88 [\chem{\delta^{18}O}(t)
                                    -\chem{\delta^{18}O}(0)] \nonumber \\
                                   &-0.1925 [\chem{\delta^{18}O}(t)^2
                                    -\chem{\delta^{18}O}(0)^2] \,,
    \end{align}
    %
    while temperature reconstructions from the EPICA and MD01-2444 records were
    provided as such. For each proxy record used and each of the parameter
    setup used in the sensitivity tests, palaeo-temperature anomalies are
    scaled linearly (Table~\ref{tab:records}) so that the cumulative glaciated
    area of the Rhine glacier piedmont lobe during Oxygen Marine Isotope Stage
    (MIS)~2 (29--14\,ka) is modelled comparably to the reconstructions
    (Fig.~\ref{fig:footprints}).


% -- -- -- -- -- -- -- -- -- -- -- -- -- -- -- -- -- -- -- -- -- -- -- -
\subsection{Palaeo-precipitation forcing}
\label{sec:palprec}
% -- -- -- -- -- -- -- -- -- -- -- -- -- -- -- -- -- -- -- -- -- -- -- -

    Finally, in some simulations, precipitation was reduced with temperature in
    order to simulate the potential rarification of atmospheric moisture in
    colder climates. This was done using an empirical relationship derived from
    observed accumulation rates and oxygen isotopes concentrations in the GRIP
    ice core \citep{Dahl-Jensen.etal.1993},
    %
    \begin{align}
      {\Psi}_{\text{PP}}(t) = \exp[\psi{\Delta}T_{\text{TS}}(t)] \,.
    \end{align}
    %
    with $\psi=0.169/2.4=0.0704$ \citep{Huybrechts.2002}. This simple
    relationship likely does not reflect the complexity of atmospheric
    circulation changes that governed moisture availability over the Alps
    during the last glacial cycle. Thus other simulations use constant
    precipitation, corresponding to $\psi=0$. In the rest of this paper, we
    refer to different model runs by the name of the proxy record used for the
    palaeo-temperature forcing. Model runs using paleo-precipitation
    corrections are labelled PP.


% -- -- -- -- -- -- -- -- -- -- -- -- -- -- -- -- -- -- -- -- -- -- -- -
\subsection{Sensitivity of ice volume evolution}
\label{sec:timeseries}
% -- -- -- -- -- -- -- -- -- -- -- -- -- -- -- -- -- -- -- -- -- -- -- -

    For the three palaeo-temperature records and the two palaeo-precipitation
    parametrizations used, the model yield significant ice volume build-up
    during MIS~4 and 2 (Fig.~\ref{fig:timeseries}). All simulations also
    yield important glaciations during MIS~5 and 3, but timing and amplitude
    varies a lot between the different forcing. All simulations overestimate
    ice cover during the Younger Dryas. This might be due partly to the 2\,km
    resolution, except for GRIP which largely overestimates Younger Dryas ice
    cover.

    Even more importantly, all simulations yield very strong ice volume
    variability which is the result of multiple glacier advance and retreats
    over the foreland, many more than two. Palaeo-precipitation reductions
    partly results in a smoother modelled ice volume time series and less
    extreme variations, but variability remains high. Ice volume variability
    is the smallest for the EPICA palaeo-temperature record, which has the
    least temperature variability.


% -- -- -- -- -- -- -- -- -- -- -- -- -- -- -- -- -- -- -- -- -- -- -- -
\subsection{Sensitivity of glaciated area}
\label{sec:footprints}
% -- -- -- -- -- -- -- -- -- -- -- -- -- -- -- -- -- -- -- -- -- -- -- -

    To select an optimal climate forcing, we look at the best known alpine
    glaciations during MIS~2 and 4. During MIS~2, all forcing yield ice extent
    of the same order of magnitude (Fig.~\ref{fig:footprints}). This is
    because the temperature records have been scaled to resul in similar
    ice covered area for the Rhine Glacier piedmont lobe. Looking at ice cover
    on other parts, all simulations tend to overestimate ice extent in the
    eastern part of the model domain, and to underestimate it in the western
    part, as compared to the reconstructed LGM margin. Regarding the western
    part, there are discussions though, as if the Lyon Lobe actually dates from
    MIS~2 or from an older glaciation. Regarding the eastern part, the MIS~2
    glaciation has been mapped in detail and dated (?), so that there is no
    doubt that our model overestimate ice cover in this region. This could
    indicate that during LGM the east-west precipitation gradient over the Alps
    was lower than today, or that temperature depression in the east was lower
    than in the west.

    There are spatial differences though: MD01-2444, and to a greater extent
    GRIP, forcings, tend to overestimate ice cover in all peripherical ranges:
    the Vosges, Black Forest, Bavarian Forest, Dinaric Alps and Pennine. This
    is because these records have a higher temperature variability so that
    after scaling they comport brief periods of cold climate, not long enough
    to develop the alpine ice cap, but long enough to build up ice on the
    peripherical ranges.

    The extent of ice cover during MIS~4 shows more sensitivity to the choice
    of palaeo-climate forcing. Using the GRIP and MD01-2444, glaciers extend
    well beyound the reach of documented moraines and glacial erratic terrain.
    Because this extent corresponds to a short-term cold climate event,
    palaeo-precipitation reductions help to greatly reduce this excessive
    modelled ice cover, yet in both cases, modelled ice extent and volume
    during MIS~4 remains higher than during MIS~2. EPICA forcing, on the other
    hand, yield a MIS~4 glaciation that is only slightly less expensive than
    the MIS~2, with only little sensitivity of the glaciated area to
    palaeo-precipitation reductions.

    Based on the above consideration on timing of the LGM ice extent and
    ice extent during MIS~4, we chose EPICA as our optimal temperature record
    for the rest of this paper. As a conservative approach in regard to quick
    ice volume fluctuations, we choose to include palaeo-precipitation
    correction in the following 1\,km-resolution simulation.

    \note{An important assumption in the method is that the MIS~4 glaciation
          was less (or equally) extensive than the MIS~2, i.e. that the LGM
          outline from \citet{Ehlers.etal.2011} dates indeed from the LGM
          (MIS~2). A different assumption could be that this outline represents
          the most extensive glaciation in the last 120\,ka. If we allow the
          MIS~2 glaciation to be a bit smaller than MIS~4, simulations driven
          by GRIP and MD01-2444 also give reasonable results.}


% ----------------------------------------------------------------------
\section{Glacier dynamics}
\label{sec:results}
% ----------------------------------------------------------------------

    Fig.~\ref{fig:lgmvel} -- Snapshot at 21\,ka and volume time series.\\
    Fig.~\ref{fig:timing} -- Timing of the LGM and area time series.\\
    Fig.~\ref{fig:trimlines} -- LGM ice thickness compared to trimlines.\\
    Fig.~\ref{fig:profiles} -- Individual glacier extent profiles.

    In this section, we compare the model output to geological evidence from
    the last glacial cycle, in terms of Last Glacial Maximum extent
    (Sect.~\ref{sec:extent}), ice flow patterns
    (Sect.~\ref{sec:flow}), timing of the Last Glacial Maximum
    (Sect.~\ref{sec:timing}), ice thickness (Sect.~\ref{sec:thickness}), and
    glacial cycle dynamics (Sect.~\ref{sec:glaciations}).

    This simulation is forced by the optimal EPICA palaeo-temperature record
    (Sect.~\ref{sec:paltemp}) and includes palaeo-precipitation reduction
    (Sect.~\ref{sec:palprec}). It uses an horizontal resolution of 1\,km. The
    vertical grid consists of 61~temperature layers in the bedrock and up to
    251~enthalpy layers in the ice, corresponding to vertical resolutions of 50
    and 20\,\unit{m}, respectively.


% -- -- -- -- -- -- -- -- -- -- -- -- -- -- -- -- -- -- -- -- -- -- -- -
\subsection{Last Glacial Maximum ice extent}
\label{sec:extent}
% -- -- -- -- -- -- -- -- -- -- -- -- -- -- -- -- -- -- -- -- -- -- -- -

    The LGM extent of alpine glaciers has been mapped with varying level of
    detail across the mountain range \citep{Penck.Bruckner.1909, Jackli.1962,
    Bini.etal.2009, Coutterand.2010, Bavec.Verbic.2011,
    Buoncristiani.Campy.2011, Husen.2011}. These maps have been compiled
    into an reconstruction covering the entire Alps \citep{Ehlers.etal.2011},
    reproduced here (Fig.~4, red line) for comparison against model results.

    The modelled total ice volume reaches a maximum of $122.5 \times 10^{3}$\,\unit{km^2},
    or 300.4\,mm of sea-level equivalent, at 24.56\,ka
    (Fig.~\ref{fig:lgmvel}b). A maximum glacierized area of
    $163.3 \times 10^{3}$\,\unit{km^2} is attained shortly aftewards at 24.57\,ka.
    (Fig.~\ref{fig:lgmvel}b). Although the modelled timing of the LGM varies
    accross the mountain range (Sect.~\ref{sec:timing}), at 24.57\,ka nearly
    all glaciers extend to within a few km from their modelled maximum stage
    (Fig.~\ref{fig:lgmvel}a, dashed orange line).

    Although the palaeo-climate forcing was adapted (Sect.~\ref{sec:climate})
    to model this maximum configuration in broad agreement with the geological
    reconstruction of the LGM extent (Fig.~\ref{fig:lgmvel}a, solid red line),
    local discrepancies remain. Most notably, reconstructed ice extent in the
    north-western Alps, including the Rhone Glacier complex, the Jura ice cap,
    and the Lyon Lobe, is largely missing from our model results. On the other
    hand, the model yields excessive glaciation in the eastern Alps, where the
    Palten, Drau and Sava glaciers are modelled to extend tens of kilometers
    beyond the mapped moraines. These discrepancies might indicate that the
    LGM climate was characterized by an east-west gradient in temperature or
    precipitation anomalies relative to present not included in the climate
    forcing, or perhaps a east-west gradient in variables not accounted for
    by the PDD model, such as cloud cover or dust deposition.

    On a more regional level, the modelled maximum extent of the Ticino, Iseo,
    Garda and Piave glaciers in the central southern Alps is constrained
    several kilometres within the mapped LGM moraines. This appears to result
    from the paleo-precipitation reduction used to force the model
    (Fig.~\ref{fig:footprints}b), which is likely unrealistic in at least this
    part of the model domain. On the other hand, the Durance glacier in the
    south-western Alps is modelled to extend several kilometers beyond the
    mapped limit, indicating that the LGM temperature depression was likely
    dampened by Mediterranean climate in this part of the model domain.

    The model reproduces peripheral ice caps where documented by geological
    evidence on the Vercors, Chartreuse and Bauge prealpine reliefs,
    the Jura Mountains, the Vosges Mountain, the Black Forest, the Bohemian
    Forest and the Dinaric Alps. An independent ice cap also covers the
    Hochwart massif during most of the simulation, yet it is engulfed by
    alpine glaciers during the LGM to become a peripheral ice dome. The LGM
    extent of peripheral ice caps is underestimated in the Vosges and Jura
    mountains, while it is overestimated for the Vercors. There is a conformity
    between the discrepancies between modelled and reconstructed LGM extent
    for the main alpine glacier complex vs peripheral ice caps, which indicates
    a climatic cause rather than an ice dynamics cause.

    \todo{add location names on Fig.~\ref{fig:lgmvel}.}


% -- -- -- -- -- -- -- -- -- -- -- -- -- -- -- -- -- -- -- -- -- -- -- -
\subsection{Ice flow patterns}
\label{sec:flow}
% -- -- -- -- -- -- -- -- -- -- -- -- -- -- -- -- -- -- -- -- -- -- -- -

    The LGM Alpine ice flow pattern was traditionally described as that of a
    network of interconnected valley glaciers that was primarily controlled by
    subglacial topography. However, the geomorphology shows that ice was thick
    enough to flow accross high mountain passes throughout the mountain range
    \citep[e.g.,][]{Onde.1938, Penck.Bruckner.1909, Jackli.1962, Husen.1985,
    Coutterand.2010, Kelly.etal.2004, Husen.2011}, and even perhaps form
    self-sustained ice domes \citep{Florineth.1998, Florineth.Schluchter.1998,
    Kelly.etal.2004, Bini.etal.2009}.

    The modelled flow pattern at 24.57\,ka is complex (Fig.~\ref{fig:lgmvel}a).
    The modelled fast-flow regions generally occur along the main river
    valleys, while ice domes and ice divides are predominantly located over
    major reliefs . Nevertheless, the model results depict occasional ice flow
    across mountain passes (transfluences) and ice divide above topographic
    lows.

    Major transfluences occur for instance across col du Mont-Cenis,
    Simplonpass, Brünigpass in the western Alps, and over Fernpass,
    Gailbergsattel and Stranigeralpe in the Eastern Alps
    (Fig.~\ref{fig:lgmvel}a). Finally, self-sustained ice domes characteristic
    of ice caps are modelled in two locations over Flüelapass and Ötztal
    (Fig.~\ref{fig:lgmvel}a).

    Transfluence across col de Montgenèvre was previously questioned
    \citep[Fig.~2]{Cossart.etal.2012}. Transfluence across col de Mont Cenis
    was previoysly described \citep[Fig.~3.18]{Onde.1938, Coutterand.2010}.
    Transfluences across Fernpass and Seefeld are known from the geomorphology
    \citep[Fig.~2.4]{Penck.Bruckner.1909, Husen.2011}. Transfluence across
    Brunigpass was already marked in the maps \citep{Jackli.1962}. There is
    numerous evidence for transfluence across Simplon Pass
    \citep{Kelly.etal.2004}. Ice crossed the Pyhrn Pass \citep{Husen.2004}. Ice
    flew across the Gailberg and Kreuzberg saddles \citep{Husen.1985}. The
    Tagliamento catchment is usually assumed to not have received tranfluences
    from the Drau catchment \citep{Monegato.etal.2007}, but this would not be
    incompatible with reconstructed ice surface elevations in this area
    \citep{Husen.1987}.

    In the Eastern Alps, modelled transfluences and crosswise divides are
    generally incompatible with geological reconstructions, indicating that
    climate deterioration was overestimated in the model input. In the Western
    Alps, however, transfluences generally occur where they have been
    documented by geologic evidence. These model results depict the LGM Alpine
    ice complex as an intermediate between ice fields and ice caps, bearing
    characteristics of both and perhaps no modern analogue.

    \todo{add location names on Fig.~\ref{fig:lgmvel}.}

%    \citep{Jouvet.etal.2017}


% -- -- -- -- -- -- -- -- -- -- -- -- -- -- -- -- -- -- -- -- -- -- -- -
\subsection{Timing of the Last Glacial Maximum}
\label{sec:timing}
% -- -- -- -- -- -- -- -- -- -- -- -- -- -- -- -- -- -- -- -- -- -- -- -

    The timing of the LGM has been documented by radiocarbon and cosmogenic
    isotope dating techniques at multiple locations around the Alps (cf.
    \citealp[Fig.~5]{Wirsig.etal.2016} for a review and the more recent
    publications by \citealp{Monegato.etal.2017, Federici.etal.2017}). These
    data indicate that Alpine glaciers reached their maximum extent between 20
    and 26\,ka (calibrated \chem{^{14}C} or \chem{^{10}Be} ages), but glacier
    lobes stayed in the foreland until as late as 22 to 17\,ka, when they
    experienced rapid retreat synchronously to lowering of the ice surface in
    the mountains (\citealp[Fig.~5]{Wirsig.etal.2016},
    \citealp[Fig.~3]{Monegato.etal.2017}).

    \note{Please edit or complete with more references if needed}

    In our simulation, the maximum areal cover is reached at 24.57\,ka
    (Fig.~\ref{fig:lgmvel}a), yet individual glacier lobes reach their
    maximum extent at different ages (Fig.~\ref{fig:timing}). For instance, the
    Dora Riparia, Ivrea, and Tagliamento glaciers reach an early maximum before
    27\,ka, the Durance, Rhone, Inn, Enns, Ticino, and Garda glaciers reach
    their maximum thickness in phase with the global Alpine areal maximum
    around 25\,ka, while the Isère, Como and Iseo glaciers reach a late maximum
    after 24\,ka (Fig.~\ref{fig:timing}). Remarkably, peripheral ice caps on
    the Vercors, Jura Mountains, Black Forest and Hochschwab reach their
    maximum extent even later, locally after 21\,ka (Fig.~\ref{fig:timing}).

    It is important to notice that these differences in timing of the LGM occur
    in the model results despite the simple climate forcing used consisting of
    homogeneous anomalies. These discrepancies between different glacier lobes
    are thus not related to climate, but are a pure product of modelled
    inherent glacier dynamics. They occur because of differences in the
    subglacial topography, in particular catchment sizes and bed roughness.

    PISM is a polythermal ice sheet model, i.e. it computes the distribution
    of englacial temperatures resulting from heat advection, vertical
    diffusion in ice and rock, and strain heat release, in three dimensions.
    For large Alpine glaciers flowing in deep valleys, such as the Rhone,
    Rhine and Garda glaciers, several thousands of years are needed to attain
    a thermodynamical equilibrium between cold ice advection from the high
    accumulation areas, upward diffusion of geothermal heat, and heat release
    from strong basal shear strain. Temperature evolution is, in part, limited
    by the slow warming of subglacial bedrock, that has been previously cooled
    downed by subfreezing air temperature before glacier advance.

    For such glaciers, this thermodynamical equilibrium is typically not yet
    attained around 25\,ka. This is why several Alpine lobes overshoot their
    balanced extent before thinning and receding towards the mountains as they
    warm towards thermodynamical equilibrium.

    Our results indicate that even for a relatively small ice sheet like that
    found in the Alps, differences in timing on the order of the millenium can
    result from complex glacier dynamics. Further differences in timing of the
    LGM caused by heterogeneous climatic anomalies, such as precipitation
    anomaly gradients, not included in our model set-up, may have
    counterbalanced or emphasize those presented here.

    \todo{add location names on Fig.~\ref{fig:lgmvel}.}


% -- -- -- -- -- -- -- -- -- -- -- -- -- -- -- -- -- -- -- -- -- -- -- -
\subsection{Ice thickness and trimlines}
\label{sec:thickness}
% -- -- -- -- -- -- -- -- -- -- -- -- -- -- -- -- -- -- -- -- -- -- -- -

    Following the general assumption that trimlines, which mark the transition
    between the steep frost-shattered ridges and the more gentle glacially eroded
    valley troughs, represent the maximum elevation of the LGM ice surface,
    the LGM ice thickness in the Alps been reconstructed in several areas
    \citep{Husen.1987, Florineth.1998, Florineth.Schluchter.1998,
    Kelly.etal.2004, Bini.etal.2009, Coutterand.2010, Cossart.etal.2012}.
    However, this assumption is challenged by geological evidence from
    Scandinavia \citep[e.g.,][]{Kleman.1994, Kleman.Borgstrom.1994,
    Stroeven.etal.2002}, the British Isles \citep[e.g.,][]{Fabel.etal.2012}, and
    North America \citep[e.g.,][]{Kleman.etal.2010}, that pre-glacial landscapes
    located well above the trimlines have been glaciated and preserved under
    under cold-based ice, sometimes for several glacial cycles.

    In several places across the Alps, the maximum elevation reached by the
    ice surface during the LGM has been reconstructed from trimlines, the upper
    limit of glacial erosion observed in the landscape. This include the entire
    western Alps, the Rhone Glacier catchment, the Rhine Glacier catchment,
    south-eastern Switzerland and all of Austria. These reconstructions are
    based on the assumption that glacial ice was significantly erosive up to
    the or almost up to the surface, which was then located at or slightly
    above the elevation of the trimlines. This assumption is supported by the
    fact that there exist a regional consistence in trimlines elevations, which
    are generally located along a surface sloping in the assumed direction of
    flow. However, in other glaciated mountain regions, including several
    sites in the Scandinavian Mountains, the Cairgorns in Scotland, and high
    plateaus in the Canadian Cordillera, that cold-based ice commonly occured
    at high elevation and preserved the pre-glacial landscape.

    In the upper Rhone valley, the maximum ice surface elevation reached
    during MIS~2 in our simulation, corrected for bedrock deformation, is
    consistently modelled to have occured several hundred metres above the
    observed trimline elevations (Fig.~\ref{fig:trimlines}a, b), with a mean
    difference of 861\,m (xx\,\%, Fig.~\ref{fig:trimlines}b). This result is
    somewhat depedent on uncertain basal sliding and ice rheological parameters
    to which the model sensitivity was not tested here. However, a previous
    sensitivity study conducted with PISM and a very similar model set-up on
    the Cordillera ice sheet show that varying ice rheological parameters
    resulted relative errors of 25\,\unit{\%} with regard to ice rheological
    parameters and 14\,\unit{\%} regarding basal sliding parameters
    \citep[Fig.~7]{Seguinot.etal.2016}.

    In addition, our model results depict a polythermal LGM Alpine ice cover.
    Due to sub-freezing temperatures applied in the climate forcing of the
    model, the entire Alpine ice field is capped by an upper layer of cold ice.
    In major Alpine valleys, such as the upper Rhone Valley, important ice
    thickness and strain heating contribute to form a layer of temperate ice
    near the glacier base, allowing basal melt, sliding and potentially
    erosion. On the other hand, on the highest mountains ice cover is to thin
    and too static to form temperate ice, resulting in cold ice down to the
    bed, preventing potential erosion.

    In the upper Rhone valley, the geographical location of observed trimlines
    show remarkable agreement with the areal extent of regions that are
    modelled to have experienced warm-based for less than 1\,ka (7\,\%) of the
    MIS~2. Although this result calls for more detailed comparisons across the
    entire Alpine range and specific sensitivity studies to relevant basal
    sliding and ice rheological parameters, the presence of an upper layer of
    cold ice during the LGM, already found in some places in a much warmer
    climate today, is inevitable. Thus we want to challenge the assumption that
    Alpine trimlines mark the maximum upper ice surface elevation of the LGM
    ice cover and call for a more accurate estimation of the thickness of the
    upper layer cold ice in the Alps.

    \todo{look for references.}


% -- -- -- -- -- -- -- -- -- -- -- -- -- -- -- -- -- -- -- -- -- -- -- -
\subsection{Glacial cycle dynamics}
\label{sec:glaciations}
% -- -- -- -- -- -- -- -- -- -- -- -- -- -- -- -- -- -- -- -- -- -- -- -

    Glacial history of the Alps prior to the LGM remains poorly constrained.
    Although the four glaciations model \citep{Penck.Bruckner.1909} has long
    been used, it is now known that glaciers advanced onto the foreland many
    more times \citep{Schluchter.1991, Preusser.etal.2011}. Sparse geological
    data indicate that the last glacial cycle may have comprised two or even
    three periods of glacier growth and decay \citep{Preusser.2004,Ivy-Ochs.etal.2008}.
    Luminescence dating from two sites in the central northern foreland
    indicate an early last glacial advance of Alpine glaciers onto the near
    foreland around $107\pm9$ to $101\pm5$\,ka \citep{Preusser.etal.2003,
    Preusser.Schluchter.2004}. Evidence for a MIS~4 glaciation is equally
    sparse, and its timing remain uncertain \citep[e.g.,][]{Preusser.etal.2003,
    Link.Preusser.2006}.

    As previously mentioned (Sect.~\ref{sec:timeseries}), independently of the
    palaeo-temperature (GRIP, EPICA, or MD01-2444) and palaeo-precipitation
    (with or without corrections) applied, all simulations presented here
    result in a high time variability in the total modelled ice volume
    (Fig.~\ref{fig:timeseries}). Unsurprisingly, this result is confirmed by
    the 1-km resolution run, using the optimal EPICA palaeo-temperature record
    (Sect.~\ref{sec:paltemp}) and conservatively includes palaeo-precipitation
    reductions (Sect.~\ref{sec:palprec}), which also results in strong total
    ice volume variability throughout the last glacial cycle
    (Fig.~\ref{fig:lgmvel}b).

    Two major glaciations occur during MIS~4 and 2 (Fig.~\ref{fig:lgmvel}b).
    However, several minor glaciations occur during MIS~5 and 3, as well as a
    minor late-glacial readvance at the onset of MIS~1
    (Fig.~\ref{fig:lgmvel}b). These episodes are the result of synchronous
    advances several Alpine glaciers well into the major valleys and sometimes
    even onto the foreland (Fig.~\ref{fig:profiles}), Supplementary animation).

    For instance the Rhine Glacier (Fig.~\ref{fig:profiles}a) reaches a lenght
    exceeding 130\,km, roughly corresponding to location of the last major
    Alpine reliefs and the beginning of the foothills, 6~times during the
    simulation, sometimes retreating almost to its present day frontal position
    (Fig.~\ref{fig:profiles}b). The Rhone Glacier, fed by several high-altitude
    accumulation areas, advanced 8\,times onto the location of modern Lake
    Geneva (Fig.~\ref{fig:profiles}c, d). The Ivrea Glacier, characterised by
    a very steep catchment, and mutliple tributaries, show even a higher
    variability and reaches close to its maximum position 6\,times throughout
    the simulation. Finally, the Isère Glacier, with its complex system of
    confluences and diffluences, needs longer time to build up and reaches the
    foreland only twice during the major glaciations of MIS~4 and 2.

    Despite the low temperature variability of the palaeo-climate forcing, and
    reduced precipitation dampening glacier response, our simulation depict
    the Alpine ice complex as highly dynamic, with dynamic of individual
    glaciers controlled by variations in subglacial topography and catchment
    geometry.

    \idea{Matthias: The selection of figures in the main text and SOM is good.
          I find Figure 5 quite difficult to understand at first glance. I
          would combine it (already before submission) with Fig. 6. the Figure
          can then focus both on the temporal advance/retreat patterns of the 4
          lobes (four panels on the left side of figure) and the spatial
          differences in max. ice elevation (1 panel on right side of figure,
          with centerlines indicated). I would have the one with trimline
          elevation again in the Supplementary.}


% ----------------------------------------------------------------------
\conclusions
% ----------------------------------------------------------------------

    This study consists in the application of the numerical ice sheet model
    PISM to ice dynamics of the last glacial cycle in the Alps, using a model
    set-up based on that previously developed and validated for the Cordilleran
    ice sheet (Sect.~\ref{sec:model}).

    Using three different palaeo-temperature forcing records (GRIP, EPICA, and
    MD01-2444, Sect.~\ref{sec:paltemp}), scaled to model the Rhine Glacier
    piedmont lobe in agreement with the mapped LGM ice margin, and two
    different palaeo-precipitation parametrizations (with and without
    precipitation reductions, Sect.~\ref{sec:palprec}), we find that only the
    EPICA palaeo-temperature record yields model results in agreement with
    geological findings, in the sense that:

    \begin{itemize}
      \item The EPICA palaeo-temperature forcing yields maximum glacierized
            area around 25\,ka, followed by a standstill of major piedmont lobe
            in the forelands until about 18\,ka, both compatible with much of
            the dating results, whereas the GRIP forcing results in early
            deglaciation of the foreland complete by 21\,ka, and the MD01-2444
            forcing results in a late LGM glaciation around 16\,ka
            (Sect.~\ref{sec:timeseries}).
      \item The EPICA palaeo-temperature forcing yields cumulative ice extent
            compatible with geological evidence during MIS~4 and 2, whereas
            both GRIP and MD01-2444 forcings result in MIS~4 glaciation well
            beyong the mapped LGM ice limits  (Sect.~\ref{sec:footprints}).
    \end{itemize}

    We then use this optimal palaeo-temperature forcing and, as a conservative
    approach, palaeo-precipitation reductions, to force a 1-km resolution
    simulation of the last glacial cycle in the Alps. A more detailed analysis
    of its output lead us to the following conclusions.

    \begin{itemize}
      \item Ice cover is generally underestimated in the Western Alps and
            overestimated in the Eastern Alps, indicating that east-west
            gradients in temperature or precipitation change, absent from our
            model forcing, probably controlled the LGM extent of ice cover in
            the Alps (Sect.~\ref{sec:extent}).
      \item The assymetric nature of LGM ice extent north and south of the Alps
            can be explained by the transient nature of the LGM extent without
            involving north-south gradients in temperature and precipitation
            change (Sect.~\ref{sec:extent}).
      \item The LGM ice flow pattern in the Alps is largely controlled by
            subglacial topography, but several transfluences and two
            self-sustained ice domes occured (Sect.~\ref{sec:flow}).
      \item The modelled LGM is a transient stage in which glaciers are not in
            equilibrium with climate. Timing potentially varies across the
            range (Sect.~\ref{sec:timing}).
      \item Ice thickness during the LGM is modelled to be much higher than in
            reconstructions. In average modeleld surface elevation is 861\,m
            above the Rhone Glacier trimlines, which may instead indicate an
            englacial thermal boundary (Sect.~\ref{sec:thickness}).
      \item Aline glaciers were potentially very dyamic. They responded quickly
            to climate fluctuation and some of them potentially advanced many
            times over the foreland during the last glacial cycle
            (Sect.~\ref{sec:glaciations})
    \end{itemize}

    These results are nevertheless limited by the simplicity of the climate
    forcing and surface mass balance model applied, uncertain ice physical
    model parameters. Through more detailed sensitivity studies, targeted to
    specific aspects of the last glacial cycle ice dynamics in the Alps, future
    modelling studies will certainly be able to quantify uncertainties
    associated with some of the above statements. However, we also hope that
    these conclusion will also serve as a basis for future studies of glacial
    geology in the Alps, and call for a more systematic aggregation and
    homogeneisation of glacial geological data to form a basis for model
    validation across the entire Alpine range.


% ----------------------------------------------------------------------
% Acknowledgements
% ----------------------------------------------------------------------

\codedataavailability{%
    PISM is available as open-source software at http://pism-docs.org/.
    Please contact the corresponding author for model output data.
    \idea{Should we make model output available somewhere?}}

\authorcontribution{%
    J.~Seguinot designed the study, ran the simulations, and wrote most of
    the manuscript. All authors contributed to the text.}

\competinginterests{%
    We have no conflict of interest in publishing these results.}

\begin{acknowledgements}

    This manuscript builds largely upon a previous study on the Cordilleran Ice
    Sheet which is part of J.~Seguinot's PhD thesis. The experimental design
    used here, as well as the general layout of this study, are in important
    proportions the work of Irina Rogozhina, Arjen~P. Stroeven, Martin Margold and
    Johan Kleman without whom this study would not have happened. As always, we
    are very thankful to Constantine Khroulev, Ed Bueler, and Andy Aschwanden
    for providing constant help and development with PISM, in particular with
    recent issues with the computation of ice temperature (Github issue
    no.~371) and with the computation of bedrock deformation (Github issues
    no.~370 and 377). We are equally thankful to Giovanni Monegato and Marc
    Lüether for very insightful discussions of preliminary model results during
    the European Geoscience Union General Assembly~2017.

    This work was supported by the Swiss National Foundation grant
    no.~200021-153179/1 to M.~Funk.
    Computer resources were provided by the Swiss National Supercomputing
    Centre (CSCS) allocation no.~s573 to J.~Seguinot.

\end{acknowledgements}


% ----------------------------------------------------------------------
% References
% ----------------------------------------------------------------------

\bibliographystyle{copernicus}
\bibliography{../../references/references}


% ----------------------------------------------------------------------
% Figures
\clearpage
% ----------------------------------------------------------------------

    \begin{figure*}
      \centerline{\includegraphics{alpcyc_hr_inputs}}
      \caption{%
        \textbf{(a)} July mean near-surface air temperature and
        \textbf{(b)} January precipitation from WorldClim
        \citep[1960--1990]{Hijmans.etal.2005}, and
        \textbf{(c)} Mordern July standard deviation of daily mean temperature
        from the ERA-Interim \citep[1979--2012]{Dee.etal.2011, Seguinot.2013}
        from the reference monthly climatology used to force the surface mass
        balance (PDD) component of the ice sheet model.
        \textbf{(d)} Geothermal heat flow from applying the similarity method
        to multiple geophysical proxies \citep{Goutorbe.etal.2011} used as a
        boundary condition to the bedrock thermal model 3\,km below the
        ice-bedrock interface.}
      \label{fig:inputs}
    \end{figure*}

    \begin{figure*}
      \centerline{\includegraphics{alpcyc_lr_timeseries}}
      \caption{%
        \textbf{(a)} Temperature offset time-series from ice core and ocean
        records (Table~\ref{tab:records}) used as palaeo-climate forcing for
        the ice sheet model.
        \textbf{(b)} Modelled total ice volume through the last 120~thousand
        years (ka) expressed in meters of sea level equivalent (m~s.l.e.). Gray
        fields indicate Marine Oxygen Isotope Stage (MIS) boundaries for MIS~2
        and MIS~4 according to a~global compilation of benthic
        \chem{\delta^{18}O} records \citep{Lisiecki.Raymo.2005}.}
      \label{fig:timeseries}
    \end{figure*}

    \begin{figure*}
      \centerline{\includegraphics{alpcyc_lr_footprints}}
      \caption{%
        \textbf{(a--c)} Cumulative extent of modelled ice cover during MIS~2
        (29--14\,ka), with (light colours) and without (dark colours)
        palaeo-precipitation corrections, and using temperature time-series
        scaling factors (Table~\ref{tab:records}) adjusted to model the area
        glaciated by the Rhine glacier piedmont lobe (black rectangle) in
        agreement with the Last Glacial Maximum (LGM) geomorphological
        reconstruction \citep[solid red line,][]{Ehlers.etal.2011}.
        \textbf{(d--f)} Cumulative extent of modelled ice cover during MIS~4
        (71--57\,ka). Only the simulation driven by the EPICA temperature
        time-series yields reasonable MIS~4 ice cover.}
      \label{fig:footprints}
    \end{figure*}

    \begin{figure*}
      \centerline{\includegraphics{alpcyc_hr_lgmvel}}
      \caption{%
        \textbf{(a)} Modelled bedrock topography (grey), ice surface topography
        (200\,m contours), and ice surface velocity (blue) in the Alps
        24.57\,ka before present, corresponding to the maximum modelled ice
        cover. Modelled Last Glacial Maximum
        (LGM) ice extent (dashed orange line) and geomorphological
        reconstruction \citep[solid red line,][]{Ehlers.etal.2011}. The
        background map consists of depressed SRTM \citep{Jarvis.etal.2008}
        topography and Natural Earth Data \citep{Patterson.Kelso.2017}.
        \textbf{(b)} Temperature offset time-series from the EPICA ice core
        used as palaeo-climate forcing for the ice flow model \citep[black
        curve,][]{Jouzel.etal.2007}, and modelled total ice volume through the
        last glacial cycle (120--0\,ka), expressed in meters of sea level
        equivalent (m~s.l.e., blue curve). Gray fields indicate Marine
        Oxygen Isotope Stage (MIS) boundaries for MIS~2 and MIS~4 according to
        a~global compilation of benthic \chem{\delta^{18}O} records
        \citep{Lisiecki.Raymo.2005}.}
      \label{fig:lgmvel}
    \end{figure*}

    \begin{figure*}
      \centerline{\includegraphics{alpcyc_hr_timing}}
      \caption{%
        \textbf{(a)} Timing of the LGM given by the modelled age of maximum ice
        thickness throughout the entire simulation (colour mapping) and
        corresponding, ice surface elevation (200\,m contours).
        \textbf{(b)} Temperature offset time-series from the EPICA ice core
        used as palaeo-climate forcing for the ice flow model (black curve),
        and modelled glacierized area during the LGM (coloured curve). The LGM
        is here modelled as a time-transgressive event.}
      \label{fig:timing}
    \end{figure*}

    \begin{figure}
      \centerline{\includegraphics{alpcyc_hr_trimlines}}
      \caption{%
        \textbf{(a)} Comparison of modelled ice surface elevation at the LGM
        (time-transgressive, corresponding to maximum ice thickness,
        Fig.~\ref{fig:timing}), compensated for bedrock deformation, against
        observed trimline elevations \citep[Table~1]{Kelly.etal.2004}. Model
        variables were bilinearly interpolated to the trimline locations.
        \textbf{(b)} Histogram of differences between modelled LGM ice surface
        elevation and trimline elevations (500\,m bands). The average
        difference is 861\,m.
        \textbf{(c)} Modelled age of maximum ice thickness at the trimline
        locations \citep[Table~1]{Kelly.etal.2004} in the upper Rhone valley
        (colour) and LGM ice surface elevation (200\,m contours). Hatches
        mark areas modelled to have experienced less than 1\,ka temperate ice
        cover during MIS~2 (29--14\,ka).
        \todo{add relative mean difference.}}
      \label{fig:trimlines}
    \end{figure}

    \begin{figure*}
      \centerline{\includegraphics{alpcyc_hr_profiles}}
      \caption{%
        \textbf{(a, c, e, g)} Approximate glacier flowlines drawn by hand
        rougly following selected glacial valley centerlines.
        \textbf{(b, d, f, h)} Evolution of modelled glacier extent in time,
        bilinearly interpolated along the corresponding profiles, showing
        numerous cycles of advance and retreat over the last glacial cycle.
        Isolated patches indicate periodic surges from tributary glaciers.}
      \label{fig:profiles}
    \end{figure*}


% ----------------------------------------------------------------------
% Tables
\clearpage
% ----------------------------------------------------------------------

    \begin{table*}
      \caption{%
        Parameter values used in the ice sheet model.}
      \label{tab:params}
      \noindent\small\makebox[\textwidth]
      {\begin{tabular}{llrll}
        \tophline

        Not.    & Name & Value & Unit & Source \\

        \middlehline
        \multicolumn{2}{l}{{Ice rheology}} \\
        \middlehline

        $\rho$  & Ice density
                & 910
                & \unit{kg\,m^{-3}}
                & \citet{Aschwanden.etal.2012} \\

        $g$     & Standard gravity
                & 9.81
                & \unit{m\,s^{-2}}
                & \citet{Aschwanden.etal.2012} \\

        $n$     & Glen exponent
                & 3
                & --
                & \citet{Cuffey.Paterson.2010} \\

        $A_{\mathrm{c}}$   & Ice hardness coefficient cold
                & $2.847 \times 10^{-13}$
                & \unit{Pa^{-3}\,s^{-1}}
                & \citet{Cuffey.Paterson.2010} \\

        $A_{\mathrm{w}}$   & Ice hardness coefficient warm
                & $2.356 \times 10^{-2}$
                & \unit{Pa^{-3}\,s^{-1}}
                & \citet{Cuffey.Paterson.2010} \\

        $Q_{\mathrm{c}}$   & Flow law activation energy cold
                & $6.0 \times 10^4$
                & \unit{J\,mol^{-1}}
                & \citet{Cuffey.Paterson.2010} \\

        $Q_{\mathrm{w}}$   & Flow law activation energy warm
                & $11.5 \times 10^4$
                & \unit{J\,mol^{-1}}
                & \citet{Cuffey.Paterson.2010} \\

        $E_{\text{SIA}}$   & SIA enhancement factor
                & 2
                & --
                & \citet{Cuffey.Paterson.2010} \\

        $E_{\text{SSA}}$   & SSA enhancement factor
                & 1
                & --
                & \citet{Cuffey.Paterson.2010} \\

        $T_{\mathrm{c}}$   & Flow law critical temperature
                & 263.15
                & \unit{K}
                & \citet{Paterson.Budd.1982} \\

        $f$     & Flow law water fraction coeff.
                & 181.25
                & --
                & \citet{Lliboutry.Duval.1985} \\

        $R$     & Ideal gas constant
                & 8.31441
                & \unit{J\,mol^{-1}\,K^{-1}}
                & -- \\

        $\beta$ & Clapeyron constant
                & $7.9 \times 10^{-8}$
                & \unit{K\,Pa^{-1}}
                & \citet{Luthi.etal.2002} \\

        $c_{\mathrm{i}}$   & Ice specific heat capacity
                & 2009
                & \unit{J\,kg^{-1}\,K^{-1}}
                & \citet{Aschwanden.etal.2012} \\

        $c_{\mathrm{w}}$   & Water specific heat capacity
                & 4170
                & \unit{J\,kg^{-1}\,K^{-1}}
                & \citet{Aschwanden.etal.2012} \\

        $k$     & Ice thermal conductivity
                & 2.10
                & \unit{J\,m^{-1}\,K^{-1}\,s^{-1}}
                & \citet{Aschwanden.etal.2012} \\

        $L$     & Water latent heat of fusion
                & $3.34\times10^5$
                & \unit{J\,kg^{-1}\,K^{-1}}
                & \citet{Aschwanden.etal.2012} \\

        \middlehline
        \multicolumn{2}{l}{{Basal sliding}} \\
        \middlehline

        $q$     & Pseudo-plastic sliding exponent
                & 0.25
                & --
                & \citet{Aschwanden.etal.2013} \\

        $v_{\text{th}}$& Pseudo-plastic threshold velocity
                & 100
                & \unit{m\,a^{-1}}
                & \citet{Aschwanden.etal.2013} \\

        $c_0$   & Till cohesion
                & 0
                & Pa
                & \citet{Tulaczyk.etal.2000} \\

        $e_0$   & Till reference void ratio
                & 0.69
                & --
                & \citet{Tulaczyk.etal.2000} \\

        $C_{\mathrm{c}}$   & Till compressibility coefficient
                & 0.12
                & --
                & \citet{Tulaczyk.etal.2000} \\

        $\delta$& Minimum effective pressure ratio
                & 0.02
                & --
                & \citet{Bueler.Pelt.2015} \\

        $\phi$  & Till friction angle
                & 30
                & \degree
                & \citet{Cuffey.Paterson.2010} \\

        $W_{\text{max}}$ & Maximum till water thickness
                & 2
                & m
                & \citet{Bueler.Pelt.2015} \\

        \middlehline
        \multicolumn{2}{l}{{Bedrock and lithosphere}} \\
        \middlehline

        $\rho_{\mathrm{b}}$& Bedrock density
                & 3300
                & \unit{kg\,m^{-3}}
                & -- \\

        $c_{\mathrm{b}}$   & Bedrock specific heat capacity
                & 1000
                & \unit{J\,kg^{-1}\,K^{-1}}
                & -- \\

        $k_{\mathrm{b}}$   & Bedrock thermal conductivity
                & 3
                & \unit{J\,m^{-1}\,K^{-1}\,s^{-1}}
                & -- \\

        $\nu_{\mathrm{m}}$ & Astenosphere viscosity
                & $2.2 \times 10^{20}$
                & \unit{Pa\,s}
                & \citet{Mey.etal.2016} \\

        $\rho_{\mathrm{m}}$& Astenosphere density
                & 3300
                & \unit{kg\,m^{-3}}
                & \citet{Mey.etal.2016} \\

        $D$     & Lithosphere flexural rigidity
                & $1.389 \times 10^{24}$
                & \unit{N\,m}
                & \citet{Mey.etal.2016} \\

        \middlehline
        \multicolumn{2}{l}{{Surface and atmosphere}} \\
        \middlehline

        $T_{\mathrm{s}}$   & Temperature of snow precipitation
                & 273.15
                & \unit{K}
                & -- \\

        $T_{\mathrm{r}}$   & Temperature of rain precipitation
                & 275.15
                & \unit{K}
                & -- \\

        $F_{\mathrm{s}}$   & Degree-day factor for snow
                & $3.297 \times 10^{-3}$
                & \unit{m\,K^{-1}\,day^{-1}}
                & \citet{Huybrechts.1998} \\

        $F_{\mathrm{i}}$   & Degree-day factor for ice
                & $8.791 \times 10^{-3}$
                & \unit{m\,K^{-1}\,day^{-1}}
                & \citet{Huybrechts.1998} \\

        $R$     & Refreezing fraction
                & 0.0
                & --
                & -- \\

        $\gamma$& Air temperature lapse rate
                & $6 \times 10^{-3}$
                & \unit{K\,m{-1}}
                & -- \\

        $\psi$  & Precipitation factor
                & 0.0704
                & --
                & \citet{Huybrechts.2002} \\

        \bottomhline
      \end{tabular}}
    \end{table*}

    \begin{table*}
      \caption{%
        Palaeo-temperature proxy records and scaling factors yielding
        temperature offset time-series used to force the ice sheet model
        through the last glacial cycle (Fig.~\ref{fig:timeseries}). $f$
        corresponds to the scaling factor adopted to yield Last Glacial Maximum
        ice limits in the vicinity of mapped end moraines
        (Fig.~\ref{fig:footprints}a), and $[{\Delta}T_{\textrm{TS}}]_{32}^{22}$
        refers to the resulting mean temperature anomaly during the period 32
        to~22\,\unit{ka} after scaling.}
      \label{tab:records}
      \noindent\small\makebox[\textwidth]
      {\begin{tabular}{lccccccl}
        \tophline

        Forcing   & Latitude & Longitude & Elev. (m~a.s.l.)
                  & Proxy & $f$ & $[{\Delta}\text{TS}]_{32}^{22}$ (K)
                  & Reference\\

        \middlehline

        GRIP      & \multirow{2}{*}{ 72{\degree}35$^{\prime}$\,N}   % 72.58 (decimal)
                  & \multirow{2}{*}{ 37{\degree}38$^{\prime}$\,W}   % 37.64 (decimal)
                  & \multirow{2}{*}{3238}
                  & \multirow{2}{*}{\chem{\delta^{18}O}}
                  & 0.50 & $-$8.2  % -16.4126 (before scaling)
                  & \multirow{2}{*}{\citet{Dansgaard.etal.1993}} \\

        GRIP, $\Delta P$ &&&&& 0.63 & $-$10.4 \\

        EPICA     & \multirow{2}{*}{ 75{\degree}06$^{\prime}$\,S}   % 75.1
                  & \multirow{2}{*}{123{\degree}21$^{\prime}$\,E}   % 123.35
                  & \multirow{2}{*}{3233}
                  & \multirow{2}{*}{\chem{\delta^{18}O}}
                  & 1.05 & $-$9.7  % -9.2055
                  & \multirow{2}{*}{\citet{Jouzel.etal.2007}} \\

        EPICA, $\Delta P$ &&&&& 1.33 & $-$12.2 \\

        MD01-2444 & \multirow{2}{*}{ 37{\degree}34$^{\prime}$\,N}   % 37.561
                  & \multirow{2}{*}{ 10{\degree}04$^{\prime}$\,W}   % -10.142
                  & \multirow{2}{*}{$-$2637}
                  & \multirow{2}{*}{\chem{U^{K'}_{37}}}
                  & 1.84 & $-$8.0  % -4.345625
                  & \multirow{2}{*}{\citet{Martrat.etal.2007}} \\

        MD01-2444, $\Delta P$ &&&&& 2.44 & $-$10.6 \\

        \bottomhline
      \end{tabular}}

      %\noindent\small\makebox[\textwidth]
      %{\begin{tabular}{llrlrlr}
      %  \tophline
      %
      %  Record    & \multicolumn{2}{c}{GRIP}
      %            & \multicolumn{2}{c}{EPICA}
      %            & \multicolumn{2}{c}{MD01-2444} \\
      %
      %  Precip.   & cst. & $\Delta P$
      %            & cst. & $\Delta P$
      %            & cst. & $\Delta P$ \\
      %
      %  \middlehline
      %
      %  Latitude  & \multicolumn{2}{c}{ 72{\degree}35$^{\prime}$\,N}   % 72.58 (decimal)
      %            & \multicolumn{2}{c}{ 75{\degree}06$^{\prime}$\,S}   % 75.1
      %            & \multicolumn{2}{c}{ 37{\degree}34$^{\prime}$\,N} \\  % 37.561
      %
      %  Longitude & \multicolumn{2}{c}{ 37{\degree}38$^{\prime}$\,W}   % 37.64 (decimal)
      %            & \multicolumn{2}{c}{123{\degree}21$^{\prime}$\,E}   % 123.35
      %            & \multicolumn{2}{c}{ 10{\degree}04$^{\prime}$\,W} \\  % -10.142
      %
      %  Elevation & \multicolumn{2}{c}{3238\,m~a.s.l.}
      %            & \multicolumn{2}{c}{3233\,m~a.s.l.}
      %            & \multicolumn{2}{c}{$-$2637\,m~a.s.l.} \\
      %
      %  Proxy     & \multicolumn{2}{c}{\chem{\delta^{18}O}}
      %            & \multicolumn{2}{c}{\chem{\delta^{18}O}}
      %            & \multicolumn{2}{c}{\chem{U^{K'}_{37}}} \\
      %
      %  $f$       & 0.50 & 0.63
      %            & 1.05 & 1.33
      %            & 1.84 & 2.44 \\
      %
      %  $[{\Delta}\text{TS}]_{32}^{22}$
      %            & $-$8.2\,K & $-$10.4\,K     % -16.4126 (before scaling)
      %            & $-$9.7\,K & $-$12.2\,K     % -9.2055
      %            & $-$8.0\,K & $-$10.6\,K \\  % -4.345625
      %
      %  Reference & \multicolumn{2}{c}{\citet{Dansgaard.etal.1993}}
      %            & \multicolumn{2}{c}{\citet{Jouzel.etal.2007}}
      %            & \multicolumn{2}{c}{\citet{Martrat.etal.2007}} \\
      %
      %  \bottomhline
      %\end{tabular}}

    \end{table*}


% ======================================================================
\end{document}
% ======================================================================
