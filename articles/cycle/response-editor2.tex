% response-editor.tex
% ----------------------------------------------------------------------
% response-header.tex
% ----------------------------------------------------------------------

% Base class and packages
\documentclass[11pt]{article}

% Included in online comment header
\usepackage[pdftex]{graphicx}
\usepackage[pdftex]{color}
\usepackage{amssymb}
%\usepackage{times}

% Additional packages
\usepackage[T1]{fontenc}
\usepackage{geometry}
\usepackage[hidelinks]{hyperref}
\usepackage{natbib}

% Graphic path of main manuscript
\graphicspath{{../../figures/}}

% Replacements for Copernicus commands
\newcommand{\unit}[1]{\ensuremath{\mathrm{#1}}}
\newcommand{\chem}[1]{\ensuremath{\mathrm{#1}}}
\newcommand{\urlprefix}[0]{}

% Default font and spacing
\renewcommand\familydefault{\sfdefault}
\setlength{\parskip}{1.2ex}
\setlength{\parindent}{0em}
\linespread{1.0}

% Color defined in comment template
\definecolor{journalname}{rgb}{0.34,0.59,0.82}

% Personal colours
\definecolor{darkblue}{cmyk}{0.9,0.3,0.0,0.0}
\definecolor{darkgreen}{cmyk}{0.8,0.0,1.0,0.0}
\definecolor{darkred}{cmyk}{0.1,0.9,0.8,0.0}
\definecolor{darkorange}{cmyk}{0.0,0.5,1.0,0.0}
\definecolor{darkpurple}{cmyk}{0.6,0.7,0.0,0.0}
\definecolor{darkbrown}{cmyk}{0.23,0.73,0.98,0.12}

% Personal commands not used in final version
\newcommand{\todo}[1]{\textcolor{darkred}{\emph{[\textbf{TODO:} #1]}}}
\newcommand{\idea}[1]{\textcolor{darkgreen}{\emph{[\textbf{IDEA:} #1]}}}
\newcommand{\note}[1]{\textcolor{darkblue}{\emph{[\textbf{NOTE:} #1]}}}
\newcommand{\aref}[0]{\textcolor{darkblue}{\textbf{[REF.]}}}

% Redefine title and section heads
\makeatletter
\renewcommand{\familydefault}{\sfdefault}
\renewcommand{\maketitle}{\noindent\textbf{\@title}\\\@author\\\@date\\[3ex]}
\renewcommand\section{\@startsection{section}{1}{\z@}{-3ex}{2ex}%
                                    {\normalfont\large\bfseries}}
\renewcommand\subsection{\@startsection{subsection}{2}{\z@}{-3ex}{2ex}%
                                       {\normalfont\bfseries}}
\makeatother


\title{Authors' response to the Editor}
\author{J. Seguinot, on behalf of all authors.}
%\date{}

\begin{document}
\thispagestyle{empty}
\maketitle
\bigskip

% ----------------------------------------------------------------------
% Interactive comment text begins
% ----------------------------------------------------------------------

    Dear Andreas Vieli,

    We apologize for delays accumulated during the peer review of our
    manuscript. We believe that we have addressed all points raised by the
    reviews and hereby submit our revised version to \emph{The Cryosphere}.
    Please find hereafter a short summary and a marked-up file listing
    changes made to the manuscript since its first publication in \emph{The
    Cryosphere Discussion}. Please refer to our public reponses for more
    detail explanations.

    \begin{itemize}

        \item \textbf{Sect.~2 (Ice sheet model set-up):}
        We have made explicit the lack of measurements on ice softness for
        water contents above 1\,\%, clarified that subglacial water is not
        transported, explained the effect of erroneous elastic thickness, and
        justified our value for the air temperature lapse-rate.

        \item \textbf{Sect.~3 (Palaeo-climate forcing):}
        We have provided a short review on available palaeo-climate proxy
        records in or near the Alps, summarized the state of knowledge on past
        precipitation changes, and pointed out inconsistencies between our
        reference climate forcing and other data.

        \item \textbf{Sect.~4 (Results and discussion):}
        We have pointed out inconsistent LGM ice extent reconstructions in the
        south-western Alps, removed the discussion on self-sustained ice domes,
        and discussed uncertainties regarding our conclusion on trimlines.
        \emph{New references were added following private comments from
              Christopher~Carcaillet and Wilfried~Haeberli}.

        \item \textbf{Sect.~5 (Conclusions):}
        We have added a word of caution regarding our conclusion on EPICA being
        the optimal forcing, listed major sources of uncertainties, and
        emphasized the need for more realistic climate forcing in future
        studies.

        \item \textbf{Fig.~1:}
        We have removed self-sustained ice domes and updated our somewhat
        arbitrary choice of transfluence locations.

        \item \textbf{Fig.~6:}
        We have corrected histogram data and replaced areas modelled to have
        experienced temperate ice for less than 1\,ka by frozen-based areas
        at the time of maximum ice thicknes.

        \item \textbf{Various:}
        \emph{We found that thicker 200\,m contour lines were disturbing other
              figure elements and eventually decided to keep original thickness
              inconsistently with our response to referee~\#2.}
        We have implemented numerous suggestions for text improvements.

    \end{itemize}

    We thank you again very much for your editorial work on our manuscript.

\end{document}
