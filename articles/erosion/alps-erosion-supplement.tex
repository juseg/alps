% Copyright (c) 2021, Julien Seguinot (juseg.github.io), Ian Delaney
% Creative Commons Attribution-ShareAlike 4.0 International License
% (CC BY-SA 4.0, http://creativecommons.org/licenses/by-sa/4.0/)

% Alps erosion paper supplement
% =============================

\documentclass[esurf]{copernicus}

% figures directory
\graphicspath{{../../figures/}}

% document properties
\title{Last glacial cycle glacier erosion potential in the Alps
       \large \newline Supplementary figures}
\Author[1]{Julien}{Seguinot}
\Author[2]{Ian}{Delaney}
\affil[1]{Independent scholar, Anafi, Greece}
\affil[2]{Institute of Earth Surface Dynamics, University of Lausanne, Switzerland}
\runningtitle{Last glacial cycle glacier erosion potential in the Alps}
\runningauthor{J.~Seguinot and I.~Delaney}


% ======================================================================
\begin{document}
% ======================================================================

\nolinenumbers
\onecolumn

    \maketitle

    This document contains copies of the main text Figs. 1--5, and their
    equivalent using the other tested erosion laws. Note that colour levels and
    vertical axes were adapted to the magnitude of the results. Potential
    erosion rates aggregated in time, space and elevation bands, and used to
    plot these figures were made available in a long-term online archive
    \citep[\doi{10.5281/zenodo.4495419}]{Seguinot.2021}.

    \listoffigures

% -- -- -- -- -- -- -- -- -- -- -- -- -- -- -- -- -- -- -- -- -- -- -- -
% Figures
% -- -- -- -- -- -- -- -- -- -- -- -- -- -- -- -- -- -- -- -- -- -- -- -

    \renewcommand\thefigure{S\arabic{figure}}

    \begin{figure*}
      \includegraphics{alpero_cumulative}
      \caption[%
        Cumulative erosion potential and annual erosion volume after
        \citet{Koppes.etal.2015}, same as main text Fig. 1.
        ]{%
        \textbf{(a)} Modelled cumulative (time-integrated) glacial erosion
          potential over the last glacial cycle and geomorphological
          reconstruction of Last Glacial Maximum Alpine glacier extent for
          comparison \citep[solid blue line,][]{Ehlers.etal.2011}.
          The background maps consists of the initial basal topography from
          SRTM \citep{Jarvis.etal.2008} and Natural Earth Data
          \citep{Patterson.Kelso.2017}.
        \textbf{(b)} Modelled total ice volume in centimetres of sea-level
          equivalent (cm~s.l.e., black), annual (domain-integrated) potential
          erosion volume (light brown) and its 1-ka running mean (dark brown).
          Shaded gray areas indicate the timing for MIS~2 and~4
          \citep{Lisiecki.Raymo.2005}. Hatches mark periods with ice volume
          below 3\,cm~s.l.e. where glacier sliding may be affected by
          stress-balance approximations and model horizontal resolution.
        After \citet{Koppes.etal.2015}, same as main text Fig. 1.}
      \label{fig:cumulative}
    \end{figure*}

    \begin{figure*}
      \includegraphics{alpero_cumulative_her2015}
      \caption{%
        Cumulative erosion potential and annual erosion volume after
        \citet{Herman.etal.2015}.}
    \end{figure*}

    \begin{figure*}
      \includegraphics{alpero_cumulative_hum1994}
      \caption{%
        Cumulative erosion potential and annual erosion volume after
        \citet{Humphrey.Raymond.1994}}
    \end{figure*}

    \begin{figure*}
      \includegraphics{alpero_cumulative_coo2020}
      \caption{%
        Cumulative erosion potential and annual erosion volume after
        \citet{Cook.etal.2020}}
    \end{figure*}

    \begin{figure*}
      \begin{minipage}[t]{85mm}
      \includegraphics{alpero_evolution}
      \caption[%
        Potential erosion volume in relation to ice volume after
        \citet{Koppes.etal.2015}, same as main text Fig. 2.
      ]{%
        Modelled annual (domain-integrated) potential erosion volume (light
        curves) and its 1-ka running mean (dark curves) in relation to the
        modelled total ice volume in centimetres of sea-level equivalent. Blue
        indicates increasing ice volume and brown decreasing ice volume.
        After \citet{Koppes.etal.2015}, same as main text Fig. 2.}
      \end{minipage}
      \hfill
      \begin{minipage}[t]{85mm}
      \includegraphics{alpero_evolution_her2015}
      \caption{%
        Potential erosion volume in relation to ice volume after
        \citet{Herman.etal.2015}.}
      \end{minipage}
    \end{figure*}

    \begin{figure*}
      \begin{minipage}[t]{85mm}
      \includegraphics{alpero_evolution_hum1994}
      \caption{%
        Potential erosion volume in relation to ice volume after
        \citet{Humphrey.Raymond.1994}.}
      \end{minipage}
      \hfill
      \begin{minipage}[t]{85mm}
      \includegraphics{alpero_evolution_coo2020}
      \caption{%
        Potential erosion volume in relation to ice volume after
        \citet{Cook.etal.2020}.}
      \end{minipage}
    \end{figure*}

    \begin{figure*}
      \centerline{\includegraphics{alpero_transects}}
      \caption[%
        Rhine Glacier potential erosion rate over time after
        \citet{Koppes.etal.2015}, same as main text Fig. 3.
      ]{%
        \textbf{(a--d)} Modelled instantaneous potential erosion rate of the
          Rhine Glacier for selected glacier advance and retreat ages, and the
          final model state for topographic reference.
        \textbf{(e)} Interpolated instantaneous potential erosion rate along a
          Rhine Glacier transect (upper panels dashed line) for the entire last
          glacial cycle.
        After \citet{Koppes.etal.2015}, same as main text Fig. 3.}
    \end{figure*}

    \begin{figure*}
      \includegraphics{alpero_transects_her2015}
      \caption{%
        Rhine Glacier potential erosion rate over time after
        \citet{Herman.etal.2015}.}
    \end{figure*}

    \begin{figure*}
      \includegraphics{alpero_transects_hum1994}
      \caption{%
        Rhine Glacier potential erosion rate over time after
        \citet{Humphrey.Raymond.1994}.}
    \end{figure*}

    \begin{figure*}
      \includegraphics{alpero_transects_coo2020}
      \caption{%
        Rhine Glacier potential erosion rate over time after
        \citet{Cook.etal.2020}.}
    \end{figure*}

    \begin{figure*}
      \centerline{\includegraphics{alpero_hypsogram}}
      \caption[
        Potential erosion rate per altitude band after
        \citet{Koppes.etal.2015}, same as main text Fig. 4.
      ]{%
        \textbf{(a)} Potential erosion rate ``hypsogram'', showing the
          geometric mean of modelled rates in 100-m elevation bands
          across the entire model domain and its time evolution. Hatches
          indicate elevation bands with fewer than a hundred grid cells
          (100\,$km^2$).
        \textbf{(b)} Distribution of model domain bedrock topography (grey
          bars) and glaciated topography (blue bars) in 100-m elevation bands,
          and the corresponding cumulative potential erosion volume (brown
          line).
        \textbf{(c)} Same as Fig.~\ref{fig:cumulative}b.
        After \citet{Koppes.etal.2015}, same as main text Fig. 4.}
    \end{figure*}

    \begin{figure*}
      \includegraphics{alpero_hypsogram_her2015}
      \caption{%
        Potential erosion rate per altitude band after
        \citet{Herman.etal.2015}.}
    \end{figure*}

    \begin{figure*}
      \includegraphics{alpero_hypsogram_hum1994}
      \caption{%
        Potential erosion rate per altitude band after
        \citet{Humphrey.Raymond.1994}.}
    \end{figure*}

    \begin{figure*}
      \includegraphics{alpero_hypsogram_coo2020}
      \caption{%
        Potential erosion rate per altitude band after
        \citet{Cook.etal.2020}.}
    \end{figure*}

    \begin{figure*}
      \centerline{\includegraphics{alpero_sensitivity}}
      \caption[
        Climate sensitivity of cumulative erosion potential after
        \citet{Koppes.etal.2015}, same as main text Fig. 5.
      ]{%
        Modelled cumulative glacial erosion potential over the last glacial
        cycle using three different paleo-temperature histories from
        \textbf{(a, d)} the Greenland Ice Core Project
        \citep[GRIP;][]{Dansgaard.etal.1993}, \textbf{(b, e)} the European
        Project for Ice Coring in Antarctica \citep[our default,
        EPICA;][]{Jouzel.etal.2007}, and \textbf{(c, f)} an oceanic sediment
        core from the Iberian margin \citep[MD01-2444;][]{Martrat.etal.2007},
        and both \textbf{(a--c)} without and \textbf{(d--f)} with
        paleo-precipitation reductions \citep[cf.][]{Seguinot.etal.2018}.
        After \citet{Koppes.etal.2015}, same as main text Fig. 5.}
    \end{figure*}

    \begin{figure*}
      \includegraphics{alpero_sensitivity_her2015}
      \caption{%
        Climate sensitivity of cumulative erosion potential after
        \citet{Herman.etal.2015}.}
    \end{figure*}

    \begin{figure*}
      \includegraphics{alpero_sensitivity_hum1994}
      \caption{%
        Climate sensitivity of cumulative erosion potential after
        \citet{Humphrey.Raymond.1994}}
    \end{figure*}

    \begin{figure*}
      \includegraphics{alpero_sensitivity_coo2020}
      \caption{%
        Climate sensitivity of cumulative erosion potential after
        \citet{Cook.etal.2020}}
    \end{figure*}

% ----------------------------------------------------------------------
% References
% ----------------------------------------------------------------------

\clearpage
\bibliographystyle{copernicus}
\bibliography{../../../references/references}


% ======================================================================
\end{document}
% ======================================================================
