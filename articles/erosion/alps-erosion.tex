\documentclass[utf8]{article}

%\usepackage{doi}
\usepackage{authblk}
\usepackage[T1]{fontenc}
\usepackage[utf8]{inputenc}
%\usepackage[pdftex]{xcolor}
%\usepackage[pdftex]{graphicx}
%\usepackage[authoryear,round]{natbib}

% review mode
\usepackage{geometry}
\usepackage{lineno}
\linenumbers
\linespread{1.5}

%\graphicspath{{../../figures/}}

\title{Last glacial cycle glacier erosion potential in the Alps}

\author[1]{Julien Seguinot}
\author[2]{Susan Ivy-Ochs}

\affil[1]{Laboratory of Hydraulics, Hydrology and Glaciology,
          ETH Zürich, Switzerland}
\affil[2]{Laboratory of Ion Beam Physics, ETH Zürich, Switzerland}


% ======================================================================
\begin{document}
% ======================================================================

\maketitle

\begin{abstract}

    The glacial landscape of the Alps has fascinated generations of explorers,
    artists, mountaineers and scientists with its diversity, including
    erosional features of all scales from high-mountain cirques, to steep
    glacial valleys and large overdeepenings. Using previous glacier modelling
    results, and modern observations of bedrock erosion under glaciers, we
    infer a distribution of potential glacier erosion in the Alps over the last
    glacial cycle from 120\,000 years ago to the present.
    %
    Depsite large uncertainties related to the climate history of the Alps and
    glacier erosion processes, the resulting modelled patterns of glacier
    erosion shows persistent features (hopefully). The cumulative imprint of
    the last glacial cycle shows peak erosion at the mouth of the large Alpine
    valleys where glaciers are modelled to have flown with the highest
    velocity. The modelled erosion pattern varies significantly through the
    glacial cycle, but surprisingly the total rate of glacier erosion is
    modelled to be relatively stable during the entire glacial cycle.  While
    glacial maxima lead to high erosion rates at low elevations, the
    high-elevation areas are typically preserved under cold-based ice during
    these periods, but are modelled to have experience mode intense erosion
    during periods of intermediate glaciation extent.
    %
    This result indicate that different landscapes of the same mountain range
    most likely correspond to different tmie periods, and explains the
    diversity of glacial landscapes in the Alps.

\end{abstract}

% ----------------------------------------------------------------------
%\section{Introduction}
% ----------------------------------------------------------------------

% -- -- -- -- -- -- -- -- -- -- -- -- -- -- -- -- -- -- -- -- -- -- -- -
%\subsection{Erosion}
% -- -- -- -- -- -- -- -- -- -- -- -- -- -- -- -- -- -- -- -- -- -- -- -

% ----------------------------------------------------------------------
% Acknowledgements
% ----------------------------------------------------------------------

%\paragraph{Acknowledgements}
%\paragraph{Author contributions}
%\paragraph{Conflict of interest}
%\paragraph{Contribution to the field}
%\paragraph{Data availability}


% ----------------------------------------------------------------------
% References
% ----------------------------------------------------------------------

%\bibliographystyle{frontiersinSCNS_ENG_HUMS}
%\bibliography{../../references/references}


% ----------------------------------------------------------------------
% Figures
%\clearpage
% ----------------------------------------------------------------------

%    \begin{figure}
%      \centerline{\includegraphics{alpero_glerosion}}
%      \caption{%
%          Modelled glacial erosion over the last glacial cycle.}
%      \label{fig:glerosion}
%    \end{figure}

% ----------------------------------------------------------------------
% Tables
%\clearpage
% ----------------------------------------------------------------------


% ======================================================================
\end{document}
% ======================================================================
