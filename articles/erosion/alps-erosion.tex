\documentclass[utf8]{article}

%\usepackage{doi}
\usepackage{authblk}
\usepackage[T1]{fontenc}
\usepackage[utf8]{inputenc}
\usepackage[pdftex]{xcolor}
\usepackage[pdftex]{graphicx}
\usepackage[authoryear,round]{natbib}

% review mode
\usepackage{geometry}
\usepackage{lineno}
\linenumbers
\linespread{1.5}

% custom colours
\definecolor{darkblue}{cmyk}{0.9,0.3,0.0,0.0}
\definecolor{darkgreen}{cmyk}{0.8,0.0,1.0,0.0}
\definecolor{darkred}{cmyk}{0.1,0.9,0.8,0.0}
\definecolor{darkorange}{cmyk}{0.0,0.5,1.0,0.0}
\definecolor{darkpurple}{cmyk}{0.6,0.7,0.0,0.0}
\definecolor{darkbrown}{cmyk}{0.23,0.73,0.98,0.12}

% custom commands
\newcommand{\idea}[1]{\textcolor{darkgreen}{\emph{[\textbf{IDEA:} #1]}}}
\newcommand{\note}[1]{\textcolor{darkblue}{\emph{[\textbf{NOTE:} #1]}}}
\newcommand{\todo}[1]{\textcolor{darkred}{\emph{[\textbf{TODO:} #1]}}}
\newcommand{\aref}[0]{\textcolor{darkblue}{\textbf{[REF.]}}}

\graphicspath{{../../figures/}}

\title{Last glacial cycle glacier erosion potential in the Alps}
\author{Julien Seguinot}
\affil{Anafi, Greece}


% ======================================================================
\begin{document}
% ======================================================================

\maketitle

\begin{abstract}

    The glacial landscape of the Alps has fascinated generations of explorers,
    artists, mountaineers and scientists with its diversity, including
    erosional features of all scales from high-mountain cirques, to steep
    glacial valleys and large overdeepenings. Using previous glacier modelling
    results, and modern observations of bedrock erosion under glaciers, we
    infer a distribution of potential glacier erosion in the Alps over the last
    glacial cycle from 120\,000 years ago to the present.
    %
    Depsite large uncertainties related to the climate history of the Alps and
    glacier erosion processes, the resulting modelled patterns of glacier
    erosion shows persistent features (hopefully). The cumulative imprint of
    the last glacial cycle shows peak erosion at the mouth of the large Alpine
    valleys where glaciers are modelled to have flown with the highest
    velocity. The modelled erosion pattern varies significantly through the
    glacial cycle, but surprisingly the total rate of glacier erosion is
    modelled to be relatively stable during the entire glacial cycle.  While
    glacial maxima lead to high erosion rates at low elevations, the
    high-elevation areas are typically preserved under cold-based ice during
    these periods, but are modelled to have experience mode intense erosion
    during periods of intermediate glaciation extent.
    %
    This result indicate that different landscapes of the same mountain range
    most likely correspond to different time periods, and explains the
    diversity of glacial landscapes in the Alps.

\end{abstract}

% ----------------------------------------------------------------------
\section{Introduction}
% ----------------------------------------------------------------------

    The glacial erosion landscape of the Alps has fascinated generations of
    explorers, artists, and scientists for centuries. Indeed, its cultural
    impact is so far-reaching that in English, a non-Alpine language, the
    adjective ``alpine'' with non-capital ``a'' is commonly used to describe
    an Alpine-like, glacially modified, mountain landscape outside the Alps,
    while the proper noun ``Alps'' has been reused to nick-name glacier eroded
    mountain ranges of New Zealand, Japan, and others.

    Some mountain ranges are predominantly characterised by cirque glaciation
    (e.g. Uinta Mountains), glacial valleys (e.g. Putorana Plateau), or
    large-scale overdeepenings (e.g. Patagonia). But other regions, including
    the Alps, present a higher variety glacial erosional landforms, thereby
    bearing a different history of glacial erosion (Fig.~\ref{fig:landscape}).
    [The degree of glacial modification of a landscape has sometimes been used
    as a proxy for the duration of past glacier cover \aref, including for
    instance in geomorphological mapping \aref.] This paper challenges this
    assumption.

    While it has long been understood that glaciers leave a specific erosional
    imprint on the landscape, observing and quantifying the long-term erosion
    and sedimentation processes has been a challenge, not least due to the
    inaccessibility of glacier beds and the slow and stochastic nature of
    glacial erosion processes. More recent studies have moved away from process
    understanding in attempts to determine empirical erosion laws for an entire
    glacier or set of glaciers. \citet{Herman.etal.2015} collected sediment
    samples for \todo{X} months at the snout of Franz-Josef Glacier and mapped
    their chemical composition to different geological zones, and hence
    different glacier velocity, to construct a relationship between the
    reconstructed glacier basal velocity and the measured erosion rate.
    \citet{Koppes.etal.2015} \todo{check what they did}. \citet{Cook.etal.2020}
    \todo{check what they did}.

    These studies show that, while glacier and ice sheet velocities vary by
    several orders of magnitude, their relationship to erosion can be
    non-linear, yielding an even larger spectrum of variation for glacier
    erosion rates. Besides, cold-based glaciers have been shown to preserve
    landscape, but it remains unclear how sharp the erosional transition to
    cold-base ice is, and why some glaciated regions show traces of cold-base
    glaciations and others not.

    Here, a non-linear erosion law is applied to previously published model
    results and the patterns of modelled erosion rate and cumulative
    last glacial cycle erosion potential are analysed. The resulting
    uncertainties are expectantly enormous, yet the results are hoped to
    provide insights on the diversity of the Alpine glacial erosion landscape.

    \note{OK, maybe I need to do a literature review...}

% ----------------------------------------------------------------------
\section{Methods}
% ----------------------------------------------------------------------

% -- -- -- -- -- -- -- -- -- -- -- -- -- -- -- -- -- -- -- -- -- -- -- -
\subsection{Ice sheet modelling}
% -- -- -- -- -- -- -- -- -- -- -- -- -- -- -- -- -- -- -- -- -- -- -- -

    Summary of PISM physics and boundary conditions.

% -- -- -- -- -- -- -- -- -- -- -- -- -- -- -- -- -- -- -- -- -- -- -- -
\subsection{Erosion law}
% -- -- -- -- -- -- -- -- -- -- -- -- -- -- -- -- -- -- -- -- -- -- -- -

    We follow the near-quadratic erosion law of \citep{Herman.etal.2015}.

    We refer to erosion potential as the time-integrated erosion rate.

% ----------------------------------------------------------------------
\section{Results}
% ----------------------------------------------------------------------

% -- -- -- -- -- -- -- -- -- -- -- -- -- -- -- -- -- -- -- -- -- -- -- -
\subsection{Cumulative erosion}
% -- -- -- -- -- -- -- -- -- -- -- -- -- -- -- -- -- -- -- -- -- -- -- -

    Modelled cumulative erosion potential vary by several orders of magnitude.
    There is a very strong localization in valleys with a large catchment and
    valley heads, where the results are limited by model resolution.

    Fig.~\ref{fig:cumulative} -- Cumulative erosion.

% -- -- -- -- -- -- -- -- -- -- -- -- -- -- -- -- -- -- -- -- -- -- -- -
\subsection{Temporal evolution}
% -- -- -- -- -- -- -- -- -- -- -- -- -- -- -- -- -- -- -- -- -- -- -- -

    Above a certain level of glaciation, erosion rates appear uncorrelated with
    the ice volume. Volumetric erosion rates during MIS~4 and MIS~2 are not
    higher than other parts of the last glacial cycle. However, there is a
    tendency for higher erosion rate during periods of ice retreat.

    Fig.~\ref{fig:evolution} -- Erosion rate and ice volume.

    Along the Rhine Glacier transect, stages of extensive glaciation see
    erosion rates localized in the lower part of the catchment. Erosion rates
    propagate inward during periods of ice retreat.

    Fig.~\ref{fig:transects} -- Erosion rate along transects.

    These results can be generalized to the entire model domain by using
    elevation as a proxy for along-flow distance. During periods of larger
    extent, erosion rates shift to lower elevation, leaving upper grounds
    untouched. During deglaciation, erosion propagates inland.

    Fig.~\ref{fig:hypsogram} -- Erosion rate and elevation.

% ----------------------------------------------------------------------
\section{Discussion}
% ----------------------------------------------------------------------

% -- -- -- -- -- -- -- -- -- -- -- -- -- -- -- -- -- -- -- -- -- -- -- -
\subsection{Results interpretation}
% -- -- -- -- -- -- -- -- -- -- -- -- -- -- -- -- -- -- -- -- -- -- -- -

    It is often believed that glaciers are more erosive during deglaciations
    because of higher water availability. However, surface meltwater is absent
    from the model. Higher erosion rates can be explained by enhanced ice
    dynamics after longer-term ice cover.

    The results also indicate that if a square-rule erosion law is valid, much
    of the intra-Alpine erosional landscape do not date from stages of major
    glaciation.

% -- -- -- -- -- -- -- -- -- -- -- -- -- -- -- -- -- -- -- -- -- -- -- -
\subsection{Sensitivity to climate}
% -- -- -- -- -- -- -- -- -- -- -- -- -- -- -- -- -- -- -- -- -- -- -- -

    The modelled erosion potential depends significantly on the choice of
    pleoclimate forcing applied, but the general patterns remain similar.
    Higher precipitation yields higher erosion. The results also appear to be
    sensitive to model resolution, primarily in the mountains interior, where
    discretized slopes depend on the grid size.

    Fig.~\ref{fig:sensitivity} -- Sensitivity to climate.

% -- -- -- -- -- -- -- -- -- -- -- -- -- -- -- -- -- -- -- -- -- -- -- -
\subsection{Erosion uncertainties}
% -- -- -- -- -- -- -- -- -- -- -- -- -- -- -- -- -- -- -- -- -- -- -- -

    The cumulative erosion potential modelled by the Franz-Joseph Glacier
    erosion law \citep{Herman.etal.2015} appear overestimated. However, the
    error bars are large.

    The erosion law by \citet{Cook.etal.2020} results in similarly high values
    but a flatter (less localized) pattern of cumulative erosion potential,
    with lower values in the troughs and more erosion in the mountains. The two
    erosion laws may appear incompatible, and there would be a lot of reasons
    for them to be incompatible. But as pointed by \citet{Cook.etal.2020}, they
    are not necessarily incompatible. The erosion's feedback onto ice dynamics
    may yield to a delocalization of high-velocity flow in the long run.

    The erosion law by \citet{Koppes.etal.2015} yields a similar pattern but
    lower values. This law is for tidewater glaciers, which flow much faster
    than mountain glaciers. However, we can imagine that Alpine paleoglaciers
    flowing on a thick sediment bed (and sometimes into proglacial lakes, not
    included), behave closer to tidewater glaciers than the Franz-Josef steep
    mountain glacier.

    Fig.~\ref{fig:uncertainty} -- Erosion uncertainties

% ----------------------------------------------------------------------
\section{Conclusions}
% ----------------------------------------------------------------------

    The modelled erosion rates are very highly uncertain. The results are
    limited as there is no feedback of erosion onto ice dynamics, and
    erosion by rivers is not included. But if a square-law is used:
    \begin{itemize}
      \item Cumulative erosion potential is strongly localized in regions
        of fast past glacier flow.
      \item The total erosion rate is not correlated with the ice volume.
      \item During major glaciations, erosion rates are high at low elevation,
        but the mountain's interior is preserved.
      \item Higher-elevation glacial erosional landforms were formed during
        stages of intermediate glaciation.
      \item Erosion increases during deglaciations, regardless of surface
        meltwater availability.
    \end{itemize}

% ----------------------------------------------------------------------
% Acknowledgements
% ----------------------------------------------------------------------

%\paragraph{Acknowledgements}
%\paragraph{Author contributions}
%\paragraph{Conflict of interest}
%\paragraph{Contribution to the field}
%\paragraph{Data availability}


% ----------------------------------------------------------------------
% References
% ----------------------------------------------------------------------

\bibliographystyle{abbrvnat}
\bibliography{../../../references/references}


% ----------------------------------------------------------------------
% Figures
\clearpage
% ----------------------------------------------------------------------

    \begin{figure*}
      \centerline{\includegraphics{alpero_landscape}}
      \caption{%
        Alpine glacial erosion landscape diversity.
        \textbf{(a)} Piedmont overdeepening of Lake Constance, ca.~10x50\,km.
        \textbf{(b)} Glacial trough of Lauterbrunnental, ca.~1x10\,km.
        \textbf{(c)} Mountain cirque of Chüebodengletscher, ca.~1x1\,km.}
      \label{fig:landscape}
    \end{figure*}

    \begin{figure*}
      \centerline{\includegraphics{alpero_cumulative}}
      \caption{%
        \textbf{(a)} Modelled cumulative (time-integrated) glacial erosion
          potential over the last glacial cycle.
        \textbf{(b)} Modelled total ice volume in centimetres of sea-level
          equivalent (cm~s.l.e., black), volumic (domain-integrated) erosion
          rate (light brown) and 100-a running mean (dark brown). Shaded gray
          areas indicate the timing for MIS~2 and~4
          \citep{Lisiecki.Raymo.2005}.} \label{fig:cumulative}
    \end{figure*}

    \begin{figure}
      \centerline{\includegraphics{alpero_evolution}}
      \caption{%
        Modelled volumic (domain-integrated) erosion rate (light brown) and 100-a
        running mean (dark brown) in relation to modelled total ice volume in
        centimetres of sea-level equivalent (black).}
      \label{fig:evolution}
    \end{figure}

    \begin{figure*}
      \centerline{\includegraphics{alpero_transects}}
      \caption{%
        \textbf{(a, b, c)} Modelled instantaneous erosion rate of the Rhine
          Glacier for selected post-Last Glacial Maximum ages.
        \textbf{(d)} Interpolated instantaneous erosion rate along a Rhine
          Glacier transect for the entire last glacial cycle (upper panels
          dashed line).}
      \label{fig:transects}
    \end{figure*}

    \begin{figure*}
      \centerline{\includegraphics{alpero_hypsogram}}
      \caption{%
        \textbf{(a)} Modelled erosion rate ``hypsogram'', showing the geometric
          mean of (non-zero) modelled erosion rates in 10-m elevation bands
          across the entire model domain and its time evolution.
        \textbf{(b)} Same as Fig.~\ref{fig:cumulative}b.}
      \label{fig:hypsogram}
    \end{figure*}

    \begin{figure*}
      \centerline{\includegraphics{alpero_sensitivity}}
      \caption{%
        Modelled cumulative glacial erosion potential over the last glacial
        cycle without \textbf{(a, b, c)} and with \textbf{(d, e, f)}
        paleo-precipitation corrections, and using three different
        palaeo-temperature histories \citep[see][]{Seguinot.etal.2018}.}
      \label{fig:sensitivity}
    \end{figure*}

    \begin{figure*}
      \centerline{\includegraphics{alpero_uncertainty}}
      \caption{%
        Modelled cumulative (time-integrated) glacial erosion potential for
        three different erosion laws published by
        \textbf{(a)} \citet{Cook.etal.2020},
          $1.665 \times 10^{-1} \textbf{u}_\mathrm{b} ^{0.6459}$,
        \textbf{(b)} \citet{Herman.etal.2015},
          $2.7 \times 10^{-7} \textbf{u}_\mathrm{b} ^{2.02}$
          (same as Fig.~\ref{fig:cumulative}a), amd
        \textbf{(c)} \citet{Koppes.etal.2015},
          $5.2 \times 10^{-8} \textbf{u}_\mathrm{b} ^{2.34}$.
        \textbf{(d)} Corresponding erosion power laws, and
        \textbf{(e)} modelled cumulative (time-integrated) glacial erosion
          potential along a Rhine Glacier transect (upper panels dashed line).}
      \label{fig:uncertainty}
    \end{figure*}

% ----------------------------------------------------------------------
% Tables
%\clearpage
% ----------------------------------------------------------------------


% ======================================================================
\end{document}
% ======================================================================
