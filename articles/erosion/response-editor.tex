% Copyright (c) 2021, Julien Seguinot (juseg.github.io), Ian Delaney
% Creative Commons Attribution-ShareAlike 4.0 International License
% (CC BY-SA 4.0, http://creativecommons.org/licenses/by-sa/4.0/) -->

% response-header.tex
% ----------------------------------------------------------------------

% Base class and packages
\documentclass[11pt]{article}

% Included in online comment header
\usepackage[pdftex]{graphicx}
\usepackage[pdftex]{color}
\usepackage{amssymb}
%\usepackage{times}

% Additional packages
\usepackage[T1]{fontenc}
\usepackage{geometry}
\usepackage[hidelinks]{hyperref}
\usepackage{natbib}

% Graphic path of main manuscript
\graphicspath{{../../figures/}}

% Replacements for Copernicus commands
\newcommand{\unit}[1]{\ensuremath{\mathrm{#1}}}
\newcommand{\chem}[1]{\ensuremath{\mathrm{#1}}}
\newcommand{\urlprefix}[0]{}

% Default font and spacing
\renewcommand\familydefault{\sfdefault}
\setlength{\parskip}{1.2ex}
\setlength{\parindent}{0em}
\linespread{1.0}

% Color defined in comment template
\definecolor{journalname}{rgb}{0.34,0.59,0.82}

% Personal colours
\definecolor{darkblue}{cmyk}{0.9,0.3,0.0,0.0}
\definecolor{darkgreen}{cmyk}{0.8,0.0,1.0,0.0}
\definecolor{darkred}{cmyk}{0.1,0.9,0.8,0.0}
\definecolor{darkorange}{cmyk}{0.0,0.5,1.0,0.0}
\definecolor{darkpurple}{cmyk}{0.6,0.7,0.0,0.0}
\definecolor{darkbrown}{cmyk}{0.23,0.73,0.98,0.12}

% Personal commands not used in final version
\newcommand{\todo}[1]{\textcolor{darkred}{\emph{[\textbf{TODO:} #1]}}}
\newcommand{\idea}[1]{\textcolor{darkgreen}{\emph{[\textbf{IDEA:} #1]}}}
\newcommand{\note}[1]{\textcolor{darkblue}{\emph{[\textbf{NOTE:} #1]}}}
\newcommand{\aref}[0]{\textcolor{darkblue}{\textbf{[REF.]}}}

% Redefine title and section heads
\makeatletter
\renewcommand{\familydefault}{\sfdefault}
\renewcommand{\maketitle}{\noindent\textbf{\@title}\\\@author\\\@date\\[3ex]}
\renewcommand\section{\@startsection{section}{1}{\z@}{-3ex}{2ex}%
                                    {\normalfont\large\bfseries}}
\renewcommand\subsection{\@startsection{subsection}{2}{\z@}{-3ex}{2ex}%
                                       {\normalfont\bfseries}}
\makeatother


\title{Authors' response to the Editor}
\author{J. Seguinot, on behalf of all authors.}
%\date{}

\begin{document}
\thispagestyle{empty}
\maketitle
\bigskip

    Dear Andreas Lang,

    We apologize for the delays during the peer review of our manuscript. We
    believe that we have addressed all points raised by the reviews and hereby
    submit our revised version to \emph{Earth Surface Dynamics}.

    Please find hereafter a short summary (and attached, a marked-up file)
    listing changes
    made to the manuscript since its first publication as a preprint. If
    possible, we ask you to please refer to our public responses for more
    detailed point-by-point lists of changes.

    \begin{itemize}

        \item \textbf{Sect. 2 (Methods):}
        We made explicit the input topography and temperature records used in
        the parent study (Seguinot et~al., 2018), clarified simplifications
        made when computing erosion rates from basal velocities, and,
        accordingly, refer to the resulting quantities as ``erosion potential''
        in the entire manuscript.

        \item \textbf{Sect. 4 (Discussion):}
        We added sentences and whole paragraphs to discuss the systematic
        eastwards bias of ice cover (and thus erosion potential), the magnitude
        of our computed erosion potential versus expectable glacial cycle
        erosion, the uncertainties on basal sliding, and model limitations in
        high mountains due to resolution and approximated physics.

        \item \textbf{Figures:}
        We added scale bars on all map figures, and hatched regions on Figs.~1,
        2, 4 and~6 to indicate low confidence in our results for small
        glaciers. We added the field-based LGM outline on Fig.~1 (ex-Fig.~2),
        used different line styles to distinguish glacier advance and retreat
        on Fig.~2 (ex-Fig.~3), added glacier-advance (36 ka) and ice-free
        (0\,ka) erosion maps of the Rhine Glacier on Fig.~3 (ex-Fig.~4),
        plotted glaciated hypsometry and cumulative erosion per elevation band
        on Fig.~4 (ex-Fig.~5), made style changes to improve readability of
        Figs.~6 and 7 (ex-Figs.~7 and~8), moved Fig.~8 (ex-Fig.~1) into the
        discussion and replaced one of the photos.

        \item \textbf{Supplementary figures:}
        This was not mentioned in our responses. Following the reviews, and
        if this does not result in additional page charges, we would like to
        amend a supplementary document containing copies of Figs.~1--5, and
        their equivalent as plotted with alternative erosion laws (more
        briefly presented in the manuscript Fig.~6). Comments from both
        reviewers highlight the difficulty of labelling any of the tested
        erosion laws more ``realistic'' (this wording was removed), and the
        supplementary figures may prove useful alongside future developments of
        empirical glacier erosion laws.

        \item \textbf{Text improvements}
        were made throughout the text and figure captions following suggestions
        from the reviewers.

    \end{itemize}

    We thank you again very much for your editorial work on our manuscript.

\end{document}
